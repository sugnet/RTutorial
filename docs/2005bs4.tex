% Options for packages loaded elsewhere
\PassOptionsToPackage{unicode}{hyperref}
\PassOptionsToPackage{hyphens}{url}
\documentclass[
]{book}
\usepackage{xcolor}
\usepackage{amsmath,amssymb}
\setcounter{secnumdepth}{5}
\usepackage{iftex}
\ifPDFTeX
  \usepackage[T1]{fontenc}
  \usepackage[utf8]{inputenc}
  \usepackage{textcomp} % provide euro and other symbols
\else % if luatex or xetex
  \usepackage{unicode-math} % this also loads fontspec
  \defaultfontfeatures{Scale=MatchLowercase}
  \defaultfontfeatures[\rmfamily]{Ligatures=TeX,Scale=1}
\fi
\usepackage{lmodern}
\ifPDFTeX\else
  % xetex/luatex font selection
\fi
% Use upquote if available, for straight quotes in verbatim environments
\IfFileExists{upquote.sty}{\usepackage{upquote}}{}
\IfFileExists{microtype.sty}{% use microtype if available
  \usepackage[]{microtype}
  \UseMicrotypeSet[protrusion]{basicmath} % disable protrusion for tt fonts
}{}
\makeatletter
\@ifundefined{KOMAClassName}{% if non-KOMA class
  \IfFileExists{parskip.sty}{%
    \usepackage{parskip}
  }{% else
    \setlength{\parindent}{0pt}
    \setlength{\parskip}{6pt plus 2pt minus 1pt}}
}{% if KOMA class
  \KOMAoptions{parskip=half}}
\makeatother
\usepackage{color}
\usepackage{fancyvrb}
\newcommand{\VerbBar}{|}
\newcommand{\VERB}{\Verb[commandchars=\\\{\}]}
\DefineVerbatimEnvironment{Highlighting}{Verbatim}{commandchars=\\\{\}}
% Add ',fontsize=\small' for more characters per line
\usepackage{framed}
\definecolor{shadecolor}{RGB}{248,248,248}
\newenvironment{Shaded}{\begin{snugshade}}{\end{snugshade}}
\newcommand{\AlertTok}[1]{\textcolor[rgb]{0.94,0.16,0.16}{#1}}
\newcommand{\AnnotationTok}[1]{\textcolor[rgb]{0.56,0.35,0.01}{\textbf{\textit{#1}}}}
\newcommand{\AttributeTok}[1]{\textcolor[rgb]{0.13,0.29,0.53}{#1}}
\newcommand{\BaseNTok}[1]{\textcolor[rgb]{0.00,0.00,0.81}{#1}}
\newcommand{\BuiltInTok}[1]{#1}
\newcommand{\CharTok}[1]{\textcolor[rgb]{0.31,0.60,0.02}{#1}}
\newcommand{\CommentTok}[1]{\textcolor[rgb]{0.56,0.35,0.01}{\textit{#1}}}
\newcommand{\CommentVarTok}[1]{\textcolor[rgb]{0.56,0.35,0.01}{\textbf{\textit{#1}}}}
\newcommand{\ConstantTok}[1]{\textcolor[rgb]{0.56,0.35,0.01}{#1}}
\newcommand{\ControlFlowTok}[1]{\textcolor[rgb]{0.13,0.29,0.53}{\textbf{#1}}}
\newcommand{\DataTypeTok}[1]{\textcolor[rgb]{0.13,0.29,0.53}{#1}}
\newcommand{\DecValTok}[1]{\textcolor[rgb]{0.00,0.00,0.81}{#1}}
\newcommand{\DocumentationTok}[1]{\textcolor[rgb]{0.56,0.35,0.01}{\textbf{\textit{#1}}}}
\newcommand{\ErrorTok}[1]{\textcolor[rgb]{0.64,0.00,0.00}{\textbf{#1}}}
\newcommand{\ExtensionTok}[1]{#1}
\newcommand{\FloatTok}[1]{\textcolor[rgb]{0.00,0.00,0.81}{#1}}
\newcommand{\FunctionTok}[1]{\textcolor[rgb]{0.13,0.29,0.53}{\textbf{#1}}}
\newcommand{\ImportTok}[1]{#1}
\newcommand{\InformationTok}[1]{\textcolor[rgb]{0.56,0.35,0.01}{\textbf{\textit{#1}}}}
\newcommand{\KeywordTok}[1]{\textcolor[rgb]{0.13,0.29,0.53}{\textbf{#1}}}
\newcommand{\NormalTok}[1]{#1}
\newcommand{\OperatorTok}[1]{\textcolor[rgb]{0.81,0.36,0.00}{\textbf{#1}}}
\newcommand{\OtherTok}[1]{\textcolor[rgb]{0.56,0.35,0.01}{#1}}
\newcommand{\PreprocessorTok}[1]{\textcolor[rgb]{0.56,0.35,0.01}{\textit{#1}}}
\newcommand{\RegionMarkerTok}[1]{#1}
\newcommand{\SpecialCharTok}[1]{\textcolor[rgb]{0.81,0.36,0.00}{\textbf{#1}}}
\newcommand{\SpecialStringTok}[1]{\textcolor[rgb]{0.31,0.60,0.02}{#1}}
\newcommand{\StringTok}[1]{\textcolor[rgb]{0.31,0.60,0.02}{#1}}
\newcommand{\VariableTok}[1]{\textcolor[rgb]{0.00,0.00,0.00}{#1}}
\newcommand{\VerbatimStringTok}[1]{\textcolor[rgb]{0.31,0.60,0.02}{#1}}
\newcommand{\WarningTok}[1]{\textcolor[rgb]{0.56,0.35,0.01}{\textbf{\textit{#1}}}}
\usepackage{longtable,booktabs,array}
\usepackage{calc} % for calculating minipage widths
% Correct order of tables after \paragraph or \subparagraph
\usepackage{etoolbox}
\makeatletter
\patchcmd\longtable{\par}{\if@noskipsec\mbox{}\fi\par}{}{}
\makeatother
% Allow footnotes in longtable head/foot
\IfFileExists{footnotehyper.sty}{\usepackage{footnotehyper}}{\usepackage{footnote}}
\makesavenoteenv{longtable}
\usepackage{graphicx}
\makeatletter
\newsavebox\pandoc@box
\newcommand*\pandocbounded[1]{% scales image to fit in text height/width
  \sbox\pandoc@box{#1}%
  \Gscale@div\@tempa{\textheight}{\dimexpr\ht\pandoc@box+\dp\pandoc@box\relax}%
  \Gscale@div\@tempb{\linewidth}{\wd\pandoc@box}%
  \ifdim\@tempb\p@<\@tempa\p@\let\@tempa\@tempb\fi% select the smaller of both
  \ifdim\@tempa\p@<\p@\scalebox{\@tempa}{\usebox\pandoc@box}%
  \else\usebox{\pandoc@box}%
  \fi%
}
% Set default figure placement to htbp
\def\fps@figure{htbp}
\makeatother
\setlength{\emergencystretch}{3em} % prevent overfull lines
\providecommand{\tightlist}{%
  \setlength{\itemsep}{0pt}\setlength{\parskip}{0pt}}
\usepackage[]{natbib}
\bibliographystyle{apalike}
\usepackage{booktabs}
\usepackage{bookmark}
\IfFileExists{xurl.sty}{\usepackage{xurl}}{} % add URL line breaks if available
\urlstyle{same}
\hypersetup{
  pdftitle={Introduction into R Applications and Programming: A Tutorial},
  pdfauthor={Niël J le Roux and Sugnet Lubbe},
  hidelinks,
  pdfcreator={LaTeX via pandoc}}

\title{Introduction into R Applications and Programming: A Tutorial}
\author{Niël J le Roux and Sugnet Lubbe}
\date{2025}

\begin{document}
\maketitle

{
\setcounter{tocdepth}{1}
\tableofcontents
}
\chapter*{Preface}\label{preface}
\addcontentsline{toc}{chapter}{Preface}

\includegraphics[width=1\linewidth,height=\textheight,keepaspectratio]{frontcover.jpg}

This book is an updated version of \citep{StepbyStep}.

\section*{Preface to A Step-by-Step R Tutorial (2013)}\label{preface-to-a-step-by-step-r-tutorial-2013}
\addcontentsline{toc}{section}{Preface to A Step-by-Step R Tutorial (2013)}

The R system is an open-source software project for analyzing data and constructing graphics. It provides a general computer language for performing tasks like organizing data, statistical analyses, simulation studies, model fitting, building of complex graphics and many more.

Central to the R system is the high-level R computer language. Its roots date back to the birth of the computer language S on May 5, 1976 at Bell Labs, Murray Hill, New Jersey \citep{Chambers2008}. In its early days S underwent several revisions and extensions mainly for implementation on the UNIX operating system. Eventually an enhanced version of S was licensed under the name S-PLUS and became available for the Windows operating system under the name S-PLUS for Windows. The earlier versions of R adhered to the principles of functional programming and with the release of version S3 in the middle eighties its building blocks were dynamically generated, self-describing objects. The publication The New S Language \citep{BeckerChambersWilks1988} provides a detailed description of S3. The next major development of S was the release of Statistical Models in S \citep{ChambersHastie1993} which involved the merging of the functional style of S with object-oriented programming concepts of classes and methods. However, S3 has only limited formal support for classes and methods. The introduction of S4 objects \citep{Chambers1998} introduced a new class and method system but retains S3 compatibility. In the meantime several versions of S-PLUS based upon S3 at first and later on S4 were released in the commercial market.

The R language itself was introduced in a paper published by Ross Ihaka and Robert Gentleman of Auckland, New Zealand in 1996 \citep{IhakaGentleman1996}. This proposal was to a large extent compatible with S but included features from the Lisp/Scheme family of languages. An important aspect of R was its availability as an open-source system.

Both R and S-PLUS can be considered to be clones of the same underlying S. That means that if you are able to program in the one you can quite easily program in the other but be warned: there are also fundamental differences between the two systems.

In the first two decades of the twenty-first century interest in R has exceeded all possible expectations. Apart from a well-maintained core system with new releases every few months there are currently literally thousands of researchers contributing add-on packages on cutting-edge developments in statistics and data analysis.

This book is a tutorial with a twofold aim; learning the basics of the R system and how to program efficiently in R. It is the result of an introductory course in S-PLUS taught at the University of Stellenbosch since 1995. The initial course was based on the book An Introduction to S and S-Plus \citep{Spector1994}. Since 2002 increasingly more emphasis was put on R to such an extent that it is currently exclusively devoted to R. This change necessitated the preparation of class notes for a ten-day (eight hours a day) tutorial course in R. The result is A Step-by-Step R Tutorial: An introduction into R applications and programming.

\section*{Preface to A Step-by-Step R Tutorial (2021)}\label{preface-to-a-step-by-step-r-tutorial-2021}
\addcontentsline{toc}{section}{Preface to A Step-by-Step R Tutorial (2021)}

Since the first publication of A Step-by-Step R Tutorial: An introduction into R applications and programming the R system has experienced a dramatic evolutionary process. This edition still maintains the twofold aim of the first edition while adapting its contents to the needs of the modernization that has been happening within the R system itself. Deprecated or outdated material has been omitted and new developments included. What follows is a brief description of these changes.

Chapter 1 contains a new section explaining how to use R Markdown for creating PDF and HTML documents from R output. Chapters 2, 3, 4 and 5 see only minor changes. In Chapter 6 changes are made in the data sets used as well as in some exercises being borrowed from later chapters in the first edition. In Chapter 7, `Writing R Functions', a notable reference is made to the \texttt{Rcpp} package for the inclusion of C++ code into R. This package allows compiled code to be included considerably easier and more robust. Vectorized programming and mapping functions are enhanced in Chapter 8 by a discussion of the function \texttt{mapply()}. A major addition is a discussion in section 8.14 for writing user-friendly applications using the package \texttt{shiny}. This replaces the usage of the function \texttt{menu()}. An exercise to create a simple shiny App is also included.

In the first part of Chapter 9, `Reading data files into R, formatting and printing', methods for reading Microsoft Excel files have been updated; functions like \texttt{readRDS()} and \texttt{writeRDS()} for transporting R objects are introduced; and the \texttt{clipr} package is discussed. A major addition to this chapter is the section devoted to the functionality provided by the \texttt{tidyverse} collection of R packages for data manipulation and exploration; \texttt{tibbles} are discussed in detail as well as the pipe operator \texttt{\%\textgreater{}\%}, tidy data is illustrated and the data manipulation functions of \texttt{dplyr} illustrated in detail.

Chapter 10, `R graphics: Round II', has been considerably extended by the inclusion of a section on how to specify colours; a rewritten section on quantile plots and inclusion of material previously in Chapter 11. There is now a section on density estimation, which includes a discussion of density histograms and average shifted histograms. In the new section 10.14 the package \texttt{ggplot2} is discussed with many examples of its capabilities.

The chapter on `Modelling in R' (Chapter 11) and the extensive discussion of the Analysis of Variance and Covariance (Chapter 12) in the previous edition have been rewritten completely and consolidated into a new Chapter 11. The final chapter is now Chapter 12, `Introduction to Optimization'. Apart from a new data set the material is similar to that in Chapter 13 of the previous edition.

\chapter{Introducing the R System}\label{intro}

\section{Introduction}\label{introduction}

This chapter introduces the R system to the new R user. The Windows operating system is emphasized but most of the material covered also applies to other operating systems after allowing for the requirements of the particular operating system in use. Users with some experience with R should quickly glance through this chapter making sure they have mastered all topics covered here before proceeding with the main tutorial starting with Chapter 2.

In the computer age statistics has become inseparable from being able to write computer programs. Therefore, let us start with a reminder of the Fundamental Goal of S:

\textbf{\emph{Conversion of an idea into useful software}}

The challenge is to pursue this goal keeping in mind the Mission of R \citep{Chambers2008}:

\textbf{\emph{\ldots{} to enable the best and most thorough exploration of data possible}}

and its Prime Directive \citep{Chambers2008}:

\textbf{\emph{\ldots{} places and obligation on all creators of software to program in such a way that the computations can be understood and trusted}}.

\section{Downloading the R system}\label{downloading-the-r-system}

\href{http://www.R-project.org}{Website for downloading R}.

To download R to your own computer: Navigate to \emph{\ldots/bin/windows/base} and save the file \emph{R-x.y.z.-win.exe} on your computer. Click this file to start the installation procedure and select the defaults unless you have a good reason not to do so. If you select `Create desktop icon' during the installation phase, an icon similar to the one below should appear on the desktop. Alternatively, you can find R under \emph{All Applications}.

\pandocbounded{\includegraphics[keepaspectratio]{pics/Rlogo.jpg}}

The core R system that is installed includes several \emph{{packages}}. Apart from these installed packages several thousands of dedicated \emph{{contributed packages}} are available to be downloaded by users in need of any of them.

\section{A quick sample R session}\label{QuickSample}

Click the R icon created on your desktop to open the \emph{{Commands Window}} or \emph{{Console}}. Notice the R prompt \texttt{\textgreater{}} waiting for some instruction from the user.

\begin{enumerate}
\def\labelenumi{(\alph{enumi})}
\tightlist
\item
  At the R prompt \texttt{\textgreater{}} enter \texttt{5\ –\ 8}. We will follow the following convention to write instructions:
\end{enumerate}

\begin{Shaded}
\begin{Highlighting}[]
\DecValTok{5} \SpecialCharTok{{-}} \DecValTok{8}
\CommentTok{\#\textgreater{} [1] {-}3}
\end{Highlighting}
\end{Shaded}

\begin{enumerate}
\def\labelenumi{(\alph{enumi})}
\setcounter{enumi}{1}
\tightlist
\item
  Repeat (a) but enter only 5 -- and see what happens:
\end{enumerate}

\begin{Shaded}
\begin{Highlighting}[]
\NormalTok{\textgreater{} 5 {-}}
\NormalTok{\textgreater{} +}
\NormalTok{\textgreater{} +}
\end{Highlighting}
\end{Shaded}

The above \texttt{+} is the secondary R prompt. It indicates that an instruction is unfinished. Either respond by completing the instruction or press the {Esc} key to start all over again from the primary prompt.

\begin{enumerate}
\def\labelenumi{(\alph{enumi})}
\setcounter{enumi}{2}
\tightlist
\item
  Enter
\end{enumerate}

\begin{Shaded}
\begin{Highlighting}[]
\NormalTok{xx }\OtherTok{\textless{}{-}} \DecValTok{1}\SpecialCharTok{:}\DecValTok{10}
\end{Highlighting}
\end{Shaded}

This instruction creates an R object with name (or label) \texttt{xx} containing the vector
(1, 2, 3, 4, 5, 6, 7, 8, 10).

\begin{enumerate}
\def\labelenumi{(\alph{enumi})}
\setcounter{enumi}{3}
\tightlist
\item
  Enter
\end{enumerate}

\begin{Shaded}
\begin{Highlighting}[]
\NormalTok{yy }\OtherTok{\textless{}{-}} \FunctionTok{rnorm}\NormalTok{(}\AttributeTok{n =} \DecValTok{20}\NormalTok{, }\AttributeTok{mean =} \DecValTok{50}\NormalTok{, }\AttributeTok{sd =} \DecValTok{15}\NormalTok{) }
\end{Highlighting}
\end{Shaded}

This instruction creates an R object with name \texttt{yy} containing a random sample of 20 values from a normal distribution with a mean of 50 and a standard deviation of 15.

\begin{enumerate}
\def\labelenumi{(\alph{enumi})}
\setcounter{enumi}{4}
\tightlist
\item
  Enter
\end{enumerate}

\begin{Shaded}
\begin{Highlighting}[]
\NormalTok{xx}
\CommentTok{\#\textgreater{}  [1]  1  2  3  4  5  6  7  8  9 10}
\end{Highlighting}
\end{Shaded}

The above example shows that {when the name of an R object is entered at the prompt, R will respond by displaying the contents of the object}.

\begin{enumerate}
\def\labelenumi{(\alph{enumi})}
\setcounter{enumi}{5}
\item
  Obtain a representation of the contents of the object \texttt{yy} created in (d).
\item
  A program in R is called a \emph{{function}}. Any function in R is also an R \emph{{object}} and therefore has a name (or label). It follows from (e) that if the name of a function is entered at the prompt, R will respond by displaying the contents of the function.

  How then can an R function be executed i.e.~how can an R function be called? Apart from its name an R function has a list of arguments enclosed within parentheses. An R function is called by entering its name followed by a list of arguments enclosed within parentheses. As an example, let us calculate the mean of the object \texttt{yy} created above by calling the function \texttt{mean}:
\end{enumerate}

\begin{Shaded}
\begin{Highlighting}[]
\FunctionTok{mean}\NormalTok{(yy) }
\CommentTok{\#\textgreater{} [1] 48.93751}
\end{Highlighting}
\end{Shaded}

Note that the prompt appear followed by the mean of object \texttt{yy}.

\begin{enumerate}
\def\labelenumi{(\alph{enumi})}
\setcounter{enumi}{7}
\item
  Objects created during an R session in the workspace are stored in a database {.RData} in the current folder. A listing of all the objects in a database can be obtained by calling the functions \texttt{ls()} or \texttt{objects()}. Now, first enter, at the R prompt, the instruction \texttt{objects} (or \texttt{ls}) and then the instruction \texttt{objects()} (or \texttt{ls()}). Explain what has happened.
\item
  Objects can be removed by the following instruction: \texttt{rm(name1,\ name2,\ ...\ )}.
\item
  Apart from the \emph{{console}} there are several other types of windows available in R e.g.~graphs are displayed in graph windows. To illustrate, enter the following instructions at the R prompt in the console or commands window:
\end{enumerate}

\begin{Shaded}
\begin{Highlighting}[]
\NormalTok{gr.data }\OtherTok{\textless{}{-}} \FunctionTok{rnorm}\NormalTok{(}\DecValTok{1000}\NormalTok{) }
\FunctionTok{hist}\NormalTok{(gr.data)}
\end{Highlighting}
\end{Shaded}

\pandocbounded{\includegraphics[keepaspectratio]{01-intro_files/figure-latex/unnamed-chunk-7-1.pdf}}

These instructions have resulted in the opening of a graph window containing the required histogram and the user can switch from the console to the graph window and back again to the console.

\begin{enumerate}
\def\labelenumi{(\alph{enumi})}
\setcounter{enumi}{10}
\tightlist
\item
  The R session can be terminated by closing the window or entering \texttt{q()} at the R prompt. Either way the user is prompted to save the workspace. If the user chooses not to save, all objects created during the session are lost.
\end{enumerate}

\section{Working with RStudio}\label{working-with-rstudio}

Many users of R prefer working with \textbf{{RStudio}}. RStudio is a free and open source integrated development environment for R which works with the standard version of R available from CRAN. It can be downloaded from the \href{www.rstudio.com}{RStudio} home page to be run from your desktop (Windows, Mac or Linux). Full details about the functionality of RStudio are available from its home page. Here, only a brief introduction to RStudio is given.

When RStudio is installed on your computer the following icon is created on the desktop:

\pandocbounded{\includegraphics[keepaspectratio]{pics/RStudio.jpg}}

Clicking the above icon open the RStudio development environment as shown in Figure \ref{fig:RStudioLayout}. In order to open any R workspace with RStudio drag the corresponding {.RData} file to the above RStudio icon and drop it as soon as `Open with RStudio' becomes visible.

\begin{figure}
\centering
\pandocbounded{\includegraphics[keepaspectratio]{pics/RStudio_layout.jpg}}
\caption{\label{fig:RStudioLayout}The RStudio development environment for R.}
\end{figure}

The bottom left-hand panel is the familiar R console.

The bottom right-hand panel is used for :
(a) a listing of the files in the folder where the {workspace} ({.RData}) for the active project is kept
(b) a listing of all installed packages available to be attached to the search path as well as menus for installing and updating packages
(c) the graph windows (if any)
(d) the Help facilities.

The top left-hand panel can be used for creating and managing script files (see \ref{script}) while the top right-hand panel provides information on the objects in the current folder as well as the history of previous commands given in the console.

\section{R: an interpretive computer language}\label{r-an-interpretive-computer-language}

Essentially, in an interpretive language instructions are given one by one. Each instruction is then evaluated or interpreted in turn by an internal program called an \emph{{interpreter}} or \emph{{evaluator}} and some immediate action is taken. For example, the instruction given in \ref{QuickSample}(a) is evaluated by the R evaluator resulting in the answer \texttt{–3} being returned. On the other hand, in \ref{QuickSample}(b) the evaluator found the instruction to be incomplete and therefore asked for more information.

An advantage of an interpretive language is that intermediate results can be obtained quickly without having first to wait for a complete program to finish as is the case with a compiler language. In the latter case a complete program is translated (or compiled) by a program called a compiler. The compiled program can then be converted to a standalone application that can be called by other programs to perform a complete task. In general compiler languages handle computer memory relatively more efficiently and calculations are executed more speedily.
Communication with the R evaluator takes place through a set of instructions called \emph{{escape sequences}}. These escape sequences take the form of a backslash preceding a character. Examples of such escape sequences are:

\texttt{\textbackslash{}n} new line

\texttt{\textbackslash{}r} carriage return

\texttt{\textbackslash{}t} go to next tab stop

\texttt{\textbackslash{}b} backspace

\texttt{\textbackslash{}a} bell

\texttt{\textbackslash{}f} form feed

\texttt{\textbackslash{}v} vertical tab

A consequence of the above role of the backslash in R is that a single backslash in a filename will not be properly recognized. Therefore, when referring in R to the following file path \emph{{``c:\textbackslash My Documents\textbackslash myFile.txt''}} all backslashes must be entered as double backslashes i.e.~\texttt{"c:\textbackslash{}\textbackslash{}My\ Documents\textbackslash{}\textbackslash{}myFile.txt"} or as \texttt{"c:/My\ Documents/myFile.txt"}.

\subsection{Exercise}\label{exercise}

The \texttt{cat()} function can be used to write a text message to the console. Initialize a new R session and investigate the results of the following R instructions:

\begin{Shaded}
\begin{Highlighting}[]
\FunctionTok{cat}\NormalTok{(}\StringTok{"aaa bbb"}\NormalTok{)}
\FunctionTok{cat}\NormalTok{(}\StringTok{"aaa bbb }\SpecialCharTok{\textbackslash{}n}\StringTok{"}\NormalTok{)}
\FunctionTok{cat}\NormalTok{(}\StringTok{"aaa }\SpecialCharTok{\textbackslash{}n}\StringTok{ bbb }\SpecialCharTok{\textbackslash{}n}\StringTok{"}\NormalTok{)}
\FunctionTok{cat}\NormalTok{(}\StringTok{"aaa }\SpecialCharTok{\textbackslash{}n}\StringTok{bbb }\SpecialCharTok{\textbackslash{}n}\StringTok{"}\NormalTok{)}
\FunctionTok{cat}\NormalTok{(}\StringTok{"aaa }\SpecialCharTok{\textbackslash{}t\textbackslash{}t}\StringTok{ bbb }\SpecialCharTok{\textbackslash{}n}\StringTok{"}\NormalTok{) }
\FunctionTok{cat}\NormalTok{(}\StringTok{"aaa}\SpecialCharTok{\textbackslash{}b\textbackslash{}b\textbackslash{}b}\StringTok{bbb }\SpecialCharTok{\textbackslash{}n}\StringTok{"}\NormalTok{) }
\FunctionTok{cat}\NormalTok{(}\StringTok{"aaa }\SpecialCharTok{\textbackslash{}n\textbackslash{}a}\StringTok{ bbb }\SpecialCharTok{\textbackslash{}a\textbackslash{}n}\StringTok{"}\NormalTok{) }
\FunctionTok{cat}\NormalTok{(}\StringTok{"1}\SpecialCharTok{\textbackslash{}a\textbackslash{}n}\StringTok{"}\NormalTok{); }\FunctionTok{cat}\NormalTok{(}\StringTok{"2}\SpecialCharTok{\textbackslash{}a\textbackslash{}n}\StringTok{"}\NormalTok{)}
\end{Highlighting}
\end{Shaded}

What is the purpose of the semi-colon in the line above?

Could you distinguish the two soundings of the bell? Try the following:

\begin{Shaded}
\begin{Highlighting}[]
\FunctionTok{cat}\NormalTok{(}\StringTok{"1}\SpecialCharTok{\textbackslash{}a\textbackslash{}n}\StringTok{"}\NormalTok{); }\FunctionTok{Sys.sleep}\NormalTok{(}\DecValTok{2}\NormalTok{); }\FunctionTok{cat}\NormalTok{(}\StringTok{"2}\SpecialCharTok{\textbackslash{}a\textbackslash{}n}\StringTok{"}\NormalTok{) }
\end{Highlighting}
\end{Shaded}

Could you now distinguish the two soundings of the bell?

What is the purpose of the \texttt{Sys.sleep()} instruction?

\subsection{Exercise}\label{exercise-1}

Write R code to achieve the following output:

\texttt{My\ name\ is:}

Bell sounds once.

Your name appears on a new line.

Two distinct sounds of the bell are heard and

\texttt{Thank\ you} is visible on a new line.

The cursor appears on a new line.

\section{Accessing the Help functionality}\label{accessing-the-help-functionality}

\begin{enumerate}
\def\labelenumi{(\alph{enumi})}
\tightlist
\item
  Use
\end{enumerate}

\begin{Shaded}
\begin{Highlighting}[]
\NormalTok{?mean}
\end{Highlighting}
\end{Shaded}

to obtain help on the usage of the R function \texttt{mean()}.

\begin{enumerate}
\def\labelenumi{(\alph{enumi})}
\setcounter{enumi}{1}
\tightlist
\item
  Find out what is the difference between the instructions
\end{enumerate}

\begin{Shaded}
\begin{Highlighting}[]
\NormalTok{?mean}
\end{Highlighting}
\end{Shaded}

and

\begin{Shaded}
\begin{Highlighting}[]
\NormalTok{??mean}
\end{Highlighting}
\end{Shaded}

\begin{enumerate}
\def\labelenumi{(\alph{enumi})}
\setcounter{enumi}{2}
\tightlist
\item
  What help is available via the instruction
\end{enumerate}

\begin{Shaded}
\begin{Highlighting}[]
\FunctionTok{help.start}\NormalTok{()}
\end{Highlighting}
\end{Shaded}

\begin{enumerate}
\def\labelenumi{(\alph{enumi})}
\setcounter{enumi}{3}
\tightlist
\item
  Use
\end{enumerate}

\begin{Shaded}
\begin{Highlighting}[]
\NormalTok{?}\FunctionTok{help.search}\NormalTok{()}
\end{Highlighting}
\end{Shaded}

to find out how to obtain help using the R function \texttt{help.search(xx)}. Note: For hep on an operator or reserved word quotes are needed, e.g.

\begin{Shaded}
\begin{Highlighting}[]
\NormalTok{?matrix}
\end{Highlighting}
\end{Shaded}

but

\begin{Shaded}
\begin{Highlighting}[]
\NormalTok{?}\StringTok{"?"}
\end{Highlighting}
\end{Shaded}

or

\begin{Shaded}
\begin{Highlighting}[]
\NormalTok{?}\StringTok{"for"}
\end{Highlighting}
\end{Shaded}

\section{More R basics}\label{MoreBasics}

\begin{enumerate}
\def\labelenumi{(\alph{enumi})}
\item
  R as an \emph{{interactive}} language allows for fast acquisition of results.
\item
  R is a \emph{{functional}} language in two important senses: In a more technical sense it means the R model of computation relies more on \emph{{function evaluation}} than by procedural computations and changes of state. The second sense refers to the way how users communicate to R namely almost entirely through \emph{{function calls}}.
\item
  R as an \emph{{object-oriented}} language refers in a technical sense to the S4 or S5 type of objects with their associated classes and methods as mentioned in the Preface. In a less technical sense it means that everything in R is an object.
\item
  R objects will be studied in detail in later chapters. What is important for now, is the following:
\end{enumerate}

\begin{itemize}
\tightlist
\item
  Everything in R is an object.
\item
  There are different types of objects e.g.~function objects, data objects, graphics objects, character objects, numeric objects.
\item
  Usually objects are stored in the current folder called the \emph{{Global environment}}; recognized by R under the name \texttt{.GlobalEnv} and available in the file system under the name \emph{{.RData}}.
\item
  Objects are created from the console by \emph{{assignment}} through the instruction
\end{itemize}

\begin{Shaded}
\begin{Highlighting}[]
\NormalTok{name }\OtherTok{\textless{}{-}}\NormalTok{ object}
\end{Highlighting}
\end{Shaded}

or

\begin{Shaded}
\begin{Highlighting}[]
\NormalTok{object }\OtherTok{\textless{}{-}}\NormalTok{ name}
\end{Highlighting}
\end{Shaded}

\begin{itemize}
\tightlist
\item
  In R names are \emph{{case sensitive}} i.e.~peter and Peter are two different objects.
\item
  Objects created by assignment during an R session are stored permanently in the Global environment (working directory) unless the user chooses not to save when terminating an R session.
\item
  Care must be exercised when creating a new object by assignment: if an object with the name my.object already exists in the Global environment and a new object is created by assigning it to the name my.object then the old my.object is over-written and it is replaced by the new object \emph{{without any warning}}.
\item
  Remember the way the R evaluator operates: if an object name is given at the R prompt the R evaluator responds by displaying the content of the object. Review the difference between the instructions
\end{itemize}

\begin{Shaded}
\begin{Highlighting}[]
\NormalTok{q}
\end{Highlighting}
\end{Shaded}

and

\begin{Shaded}
\begin{Highlighting}[]
\FunctionTok{q}\NormalTok{()}
\end{Highlighting}
\end{Shaded}

\begin{enumerate}
\def\labelenumi{(\alph{enumi})}
\setcounter{enumi}{4}
\tightlist
\item
  The symbol \texttt{\#} marks a comment. Everything following a \texttt{\#} on a line is ignored by the R evaluator. Check for example the result of the instruction
\end{enumerate}

\begin{Shaded}
\begin{Highlighting}[]
\DecValTok{5}\SpecialCharTok{+}\DecValTok{8} \CommentTok{\# +12}
\CommentTok{\#\textgreater{} [1] 13}
\end{Highlighting}
\end{Shaded}

\begin{enumerate}
\def\labelenumi{(\alph{enumi})}
\setcounter{enumi}{5}
\item
  Usage of the symbols \texttt{\textless{}-}, \texttt{=} and \texttt{==}. The symbol \texttt{\textless{}-} is used for assigning the object on its right-hand side to a name (label) on its left-hand side; the equality sign \texttt{=} is used for specifying the arguments of functions while the double equality symbol \texttt{==} is used for comparison purposes. In earlier versions of R these rules were strictly applied by the R evaluator. However, in recent versions of R the evaluator allows the equality sign also in the case for assigning an object to a name. We believe that reserving the equality sign only for argument specifications in functions leads to more clarity when writing complex functions and therefore we discourage its usage for creating objects by assignment. In this book creating objects by assignment will be exclusively carried out with the assignment symbol \texttt{\textless{}-}.
\item
  The symbol \texttt{-\textgreater{}} assigns the object on its left-hand side to the name (label) on its right-hand side.
\item
  Working with packages: The core installation includes several packages. To see them issue the command \texttt{search()} from the R prompt in the console. Notice that the first object in the search list is \texttt{.GlobalEnv}. This is followed by other objects. Packages are recognized by the string package followed by a colon and the name of the package. In order for a package to be used the following steps must be followed: if the package has been \emph{{installed}} previously it needs only to be \emph{{loaded}} into the search path using the command \texttt{library(packagename)} from the R prompt. This will load the package by default in the second position on the search path. If the package has not been installed previously it must first be installed. This is most easily done using the top menu Packages. The command \texttt{require(packagename)} appears to be identical to \texttt{library(packagename)}. The function \texttt{require()} is designed for use inside other functions as it gives a warning, rather than an error, if the package does not exist.
\item
  More on the help (\texttt{?}) facility: Table \ref{tab:HelpQueries} contains details about help available for some special keywords.
\end{enumerate}

\begin{longtable}[]{@{}
  >{\raggedright\arraybackslash}p{(\linewidth - 2\tabcolsep) * \real{0.2857}}
  >{\raggedright\arraybackslash}p{(\linewidth - 2\tabcolsep) * \real{0.7143}}@{}}
\caption{\label{tab:HelpQueries} Some useful keywords available for help queries.}\tabularnewline
\toprule\noalign{}
\begin{minipage}[b]{\linewidth}\raggedright
\emph{{Help query}}
\end{minipage} & \begin{minipage}[b]{\linewidth}\raggedright
\emph{{Explanation}}
\end{minipage} \\
\midrule\noalign{}
\endfirsthead
\toprule\noalign{}
\begin{minipage}[b]{\linewidth}\raggedright
\emph{{Help query}}
\end{minipage} & \begin{minipage}[b]{\linewidth}\raggedright
\emph{{Explanation}}
\end{minipage} \\
\midrule\noalign{}
\endhead
\bottomrule\noalign{}
\endlastfoot
\texttt{?Arithmetic} & Unary and binary operators to perform arithmetic on numeric and complex vectors \\
\texttt{?Comparison} & Binary operators for comparison of values in vectors \\
\texttt{?Control} & The basic constructs for control of the flow in R instructions \\
\texttt{?dotsMethods} & The use of the special operator \texttt{...} \\
\texttt{?Extract} & Operators to extract or replace parts of vectors, matrices, arrays and lists \\
\texttt{?Logic} & Logical operators for operating on logical and numeric vectors \\
\texttt{?.Machine} & Information on the variable \texttt{.Machine} holding information on the numerical characteristics of the machine R is running on \\
\texttt{?NumericConstants} & How R parses numeric constants including \texttt{Inf}, \texttt{NaN}, \texttt{NA} \\
\texttt{?options} & Allow the user to set and examine a variety of global options which affect the way in which R computes and displays its results \\
\texttt{?Paren} & Parentheses and braces in R \\
\texttt{?Quotes} & Single and double quotation marks. Back quote (backtick) and backslash for starting an escape sequence \\
\texttt{?Reserved} & Description of reserved words in R \\
\texttt{?Special} & Special mathematical functions related to the beta and gamma functions including permutations and combinations \\
\texttt{?Syntax} & Outlines R syntax and gives the precedence of operators \\
\end{longtable}

\section{Regular expressions in R: the basics}\label{regular-expressions-in-r-the-basics}

It follows from \ref{MoreBasics}(d) that care must be taken when objects are assigned to names. Furthermore, the Global environment or any other R database may easily contain hundreds of objects. Therefore, a frequent task is to search for patterns in the names of objects e.g.~searching for all object names starting with ``Figure'' or ending in ``.dat''. The R function \texttt{objects()} or \texttt{ls()} has arguments \texttt{pos} and \texttt{pattern} for specifying the position of a database to search and a pattern of characters appearing in a name (or string), respectively. The pattern argument can be given any \emph{{regular expression}}. Regular expressions provide a method of expressing patterns in character values and are used to perform various tasks in R. Here we are only considering the task of extracting certain specified objects in a database using the pattern argument of \texttt{objects()} or \texttt{ls()}.

The syntax of regular expressions follows different rules to the syntax of ordinary R instructions. Moreover its syntax differs depending on the particular implementation a program uses. By default, R uses a set of regular expressions similar to those used by UNIX utilities, but function arguments are available for changing the default e.g.~by setting argument \texttt{perl\ =\ TRUE}.

Regular expressions consist of three components: \emph{{single characters}}, \emph{{character classes}} and \emph{{modifiers}} operating on single characters and character classes.

Character classes are formed by using square brackets surrounding a set of characters to be matched e.g.~\texttt{{[}abc123{]}}, \texttt{{[}a-z{]}}, \texttt{{[}a-zA-Z{]}}, \texttt{{[}0-9a-z{]}}. Note the usage of the dash to indicate a range of values.

The modifiers operating on characters or character classes are summarized in Table \ref{tab:RegExprMod}.

\begin{longtable}[]{@{}
  >{\raggedright\arraybackslash}p{(\linewidth - 2\tabcolsep) * \real{0.2857}}
  >{\raggedright\arraybackslash}p{(\linewidth - 2\tabcolsep) * \real{0.7143}}@{}}
\caption{\label{tab:RegExprMod} Modifiers for regular expressions.}\tabularnewline
\toprule\noalign{}
\begin{minipage}[b]{\linewidth}\raggedright
\emph{{Modifier}}
\end{minipage} & \begin{minipage}[b]{\linewidth}\raggedright
\emph{{Operation}}
\end{minipage} \\
\midrule\noalign{}
\endfirsthead
\toprule\noalign{}
\begin{minipage}[b]{\linewidth}\raggedright
\emph{{Modifier}}
\end{minipage} & \begin{minipage}[b]{\linewidth}\raggedright
\emph{{Operation}}
\end{minipage} \\
\midrule\noalign{}
\endhead
\bottomrule\noalign{}
\endlastfoot
\texttt{\^{}} & Expression anchors at beginning of target string \\
\texttt{\$} & Expression anchors at end of target string \\
\texttt{.} & Any single character except newline is matched \\
\texttt{\textbar{}} & Alternative patterns are separated \\
\texttt{(\ )} & Patterns are grouped together \\
\texttt{*} & Zero or more occurrences of preceding entity are matched \\
\texttt{?} & Zero or one occurrences of preceding entity are matched \\
\texttt{+} & One or more occurrences of preceding entity are matched \\
\texttt{\{n\}} & Exactly n occurrences of preceding entity are matched \\
\texttt{\{n,\}} & At least n occurrences of preceding entity are matched \\
\texttt{\{n,\ m\}} & At least n and at most m occurrences of preceding entity are matched \\
\end{longtable}

Because of their role as modifiers or in forming character classes the following characters must be preceded by a backslash when their literal meaning is needed:

\begin{verbatim}
[  ]  {  }  (  )  ^  $  .  |  *  +  \
\end{verbatim}

Note that in R this means that whenever one of the above characters needs to be escaped in a regular expression it must be preceded by double backslashes. Table \ref{tab:RegExpr} contains some examples of regular expressions.

\begin{longtable}[]{@{}
  >{\raggedright\arraybackslash}p{(\linewidth - 2\tabcolsep) * \real{0.2857}}
  >{\raggedright\arraybackslash}p{(\linewidth - 2\tabcolsep) * \real{0.7143}}@{}}
\caption{\label{tab:RegExpr} Examples of regular expressions.}\tabularnewline
\toprule\noalign{}
\begin{minipage}[b]{\linewidth}\raggedright
\emph{{Regular expression}}
\end{minipage} & \begin{minipage}[b]{\linewidth}\raggedright
\emph{{Meaning}}
\end{minipage} \\
\midrule\noalign{}
\endfirsthead
\toprule\noalign{}
\begin{minipage}[b]{\linewidth}\raggedright
\emph{{Regular expression}}
\end{minipage} & \begin{minipage}[b]{\linewidth}\raggedright
\emph{{Meaning}}
\end{minipage} \\
\midrule\noalign{}
\endhead
\bottomrule\noalign{}
\endlastfoot
\texttt{"{[}a-z{]}{[}a-z{]}{[}0-9{]}"} & Matches a string consisting of two lower case letters followed by a digit \\
\texttt{"{[}a-z{]}{[}a-z{]}{[}0-9{]}\$"} & Matches a string ending in two lower case letters followed by a digit \\
\texttt{"\^{}{[}a-zA-Z{]}+\textbackslash{}\textbackslash{}."} & Matches a string beginning with any number of lower or upper case letters followed by a period \\
\texttt{"(ab)\{2\}(34)\{2\}\$"} & Matches a string ending in \texttt{abab3434} \\
\end{longtable}

\subsection{Exercise}\label{exercise-2}

Initialize an R session

\begin{enumerate}
\def\labelenumi{(\alph{enumi})}
\tightlist
\item
  Attach the MASS package in the second (the default) position on the search path by issuing the command
\end{enumerate}

\begin{Shaded}
\begin{Highlighting}[]
\FunctionTok{library}\NormalTok{(MASS)}
\end{Highlighting}
\end{Shaded}

\begin{enumerate}
\def\labelenumi{(\alph{enumi})}
\setcounter{enumi}{1}
\tightlist
\item
  Get a listing of all the objects in package MASS by requesting
\end{enumerate}

\begin{Shaded}
\begin{Highlighting}[]
\FunctionTok{objects}\NormalTok{(}\AttributeTok{pos=}\DecValTok{2}\NormalTok{)}
\end{Highlighting}
\end{Shaded}

\begin{enumerate}
\def\labelenumi{(\alph{enumi})}
\setcounter{enumi}{2}
\tightlist
\item
  Explain the difference between \texttt{objects(pos=2,\ pat=".")} and \texttt{objects(pos=2,\ patt="\textbackslash{}\textbackslash{}.")}.
\item
  Obtain a listing of all objects with names starting with three letters followed by a digit.
\item
  Obtain a listing of all objects with names ending with three letters followed by a digit.
\item
  Obtain a listing of all objects with names ending in a period followed by exactly three or four letters.
\end{enumerate}

\section{From single instructions to sets of instructions: introducing R functions}\label{FunctionIntro}

Consider the following problem: the R data set \texttt{sleep} contains the extra hours of sleep of 20 patients after a drug treatment. Suppose this data set can be considered a sample from a normal population. A 95\% confidence interval is required for the mean extra hours of sleep. It is known that the confidence interval is given by \(\left[ \mathbf{\bar{x}}- \left( \frac{s}{\sqrt(n)} \right) t_{n-1,0.025}; \mathbf{\bar{x}}+ \left( \frac{s}{\sqrt(n)} \right) t_{n-1,0.025} \right]\). This problem can be solved by entering the following instructions one by one:

\begin{Shaded}
\begin{Highlighting}[]
\NormalTok{sleep.data }\OtherTok{\textless{}{-}}\NormalTok{ sleep[ ,}\DecValTok{1}\NormalTok{]   }
\NormalTok{sleep.mean }\OtherTok{\textless{}{-}} \FunctionTok{mean}\NormalTok{(sleep.data)   }
\NormalTok{sleep.sd }\OtherTok{\textless{}{-}} \FunctionTok{sd}\NormalTok{(sleep.data)    }
\NormalTok{t.perc }\OtherTok{\textless{}{-}} \FunctionTok{qt}\NormalTok{(}\FloatTok{0.975}\NormalTok{,}\DecValTok{19}\NormalTok{) }
\NormalTok{left.boundary }\OtherTok{\textless{}{-}}\NormalTok{ sleep.mean }\SpecialCharTok{{-}}\NormalTok{ (sleep.sd}\SpecialCharTok{/}\FunctionTok{sqrt}\NormalTok{(}\FunctionTok{length}\NormalTok{(sleep.data)))}\SpecialCharTok{*}\NormalTok{t.perc }
\NormalTok{right.boundary }\OtherTok{\textless{}{-}}\NormalTok{ sleep.mean }\SpecialCharTok{+}\NormalTok{ (sleep.sd}\SpecialCharTok{/}\FunctionTok{sqrt}\NormalTok{(}\FunctionTok{length}\NormalTok{(sleep.data)))}\SpecialCharTok{*}\NormalTok{t.perc}
\FunctionTok{cat}\NormalTok{ (}\StringTok{"["}\NormalTok{, left.boundary, }\StringTok{";"}\NormalTok{, right.boundary, }\StringTok{"]}\SpecialCharTok{\textbackslash{}n}\StringTok{"}\NormalTok{)}
\CommentTok{\#\textgreater{} [ 0.5955845 ; 2.484416 ]}
\end{Highlighting}
\end{Shaded}

In situations like the above, the problem can be addressed using a \emph{{script file}} or writing a \emph{{function}}. We are going to introduce two methods for writing functions in R:

\begin{enumerate}
\def\labelenumi{(\roman{enumi})}
\tightlist
\item
  using a script file and
\item
  using the function \texttt{fix()}.
\end{enumerate}

\subsection{Writing an R function using a script file}\label{script}

\begin{enumerate}
\def\labelenumi{(\alph{enumi})}
\tightlist
\item
  From the R top menu select \emph{File; New script}. A script window will open with a simultaneous change in the menu bar.
\item
  Type the instructions in the script window.
\item
  Select all the typed text and run the script by clicking the run icon (or Ctrl+R).
\item
  Note what is shown in the R console window.
\item
  Script files are ordinary text files. They can be saved, edited and opened using any text editor.
\item
  By convention R script files have the extension {xxxx.r}.
\item
  Next, change the spelling in the last two lines from \texttt{right.boundary} to \texttt{Right.boundary}. Select all the text and run the script. Check the output appearing on the console.
\item
  Script windows can also be used for creating an R function.
\item
  Create an R function by changing the text as shown below.
\end{enumerate}

\begin{Shaded}
\begin{Highlighting}[]
\NormalTok{conf.int }\OtherTok{\textless{}{-}} \ControlFlowTok{function}\NormalTok{ (}\AttributeTok{x =}\NormalTok{ sleep[,}\DecValTok{1}\NormalTok{])}
\NormalTok{\{}
\NormalTok{  x.mean }\OtherTok{\textless{}{-}} \FunctionTok{mean}\NormalTok{(x)   }
\NormalTok{  x.sd }\OtherTok{\textless{}{-}} \FunctionTok{sd}\NormalTok{(x)    }
\NormalTok{  t.perc }\OtherTok{\textless{}{-}} \FunctionTok{qt}\NormalTok{(}\FloatTok{0.975}\NormalTok{,}\DecValTok{19}\NormalTok{) }
\NormalTok{  left.boundary }\OtherTok{\textless{}{-}}\NormalTok{ x.mean }\SpecialCharTok{{-}}\NormalTok{ (x.sd}\SpecialCharTok{/}\FunctionTok{sqrt}\NormalTok{(}\FunctionTok{length}\NormalTok{(x)))}\SpecialCharTok{*}\NormalTok{t.perc }
\NormalTok{  right.boundary }\OtherTok{\textless{}{-}}\NormalTok{ x.mean }\SpecialCharTok{+}\NormalTok{ (x.sd}\SpecialCharTok{/}\FunctionTok{sqrt}\NormalTok{(}\FunctionTok{length}\NormalTok{(x)))}\SpecialCharTok{*}\NormalTok{t.perc}
  \FunctionTok{list}\NormalTok{ (}\AttributeTok{lower =}\NormalTok{ left.boundary, }\AttributeTok{upper =}\NormalTok{ right.boundary)  }
\NormalTok{\}}
\end{Highlighting}
\end{Shaded}

\begin{enumerate}
\def\labelenumi{(\alph{enumi})}
\setcounter{enumi}{9}
\tightlist
\item
  Select the text and notice what happens in the R commands window (the console).
\item
  Give the instruction \texttt{objects()} at the R prompt. What has happened?
\item
  You can now run the function from the commands window (the console) by typing:
\end{enumerate}

\begin{Shaded}
\begin{Highlighting}[]
\FunctionTok{conf.int}\NormalTok{ (}\AttributeTok{x =}\NormalTok{ sleep[,}\DecValTok{1}\NormalTok{])}
\CommentTok{\#\textgreater{} $lower}
\CommentTok{\#\textgreater{} [1] 0.5955845}
\CommentTok{\#\textgreater{} }
\CommentTok{\#\textgreater{} $upper}
\CommentTok{\#\textgreater{} [1] 2.484416}
\end{Highlighting}
\end{Shaded}

\begin{enumerate}
\def\labelenumi{(\alph{enumi})}
\setcounter{enumi}{11}
\tightlist
\item
  If you want to create and run the function \texttt{conf.int} in a script window then add the instruction \texttt{conf.func\ (x\ =\ sleep{[},1{]})} as the last line in the script window. Now, select only this line and run it. Check the R console.
\item
  What will happen if a syntax error is made in the script window? Change the code in the script file as follows, deliberately deleting the last closing parenthesis in the last line of the function.
\end{enumerate}

\begin{Shaded}
\begin{Highlighting}[]
\NormalTok{conf.int }\OtherTok{\textless{}{-}} \ControlFlowTok{function}\NormalTok{ (}\AttributeTok{x =}\NormalTok{ sleep[,}\DecValTok{1}\NormalTok{])}
\NormalTok{\{}
\NormalTok{  x.mean }\OtherTok{\textless{}{-}} \FunctionTok{mean}\NormalTok{(x)   }
\NormalTok{  x.sd }\OtherTok{\textless{}{-}} \FunctionTok{sd}\NormalTok{(x)    }
\NormalTok{  t.perc }\OtherTok{\textless{}{-}} \FunctionTok{qt}\NormalTok{(}\FloatTok{0.975}\NormalTok{,}\DecValTok{19}\NormalTok{) }
\NormalTok{  left.boundary }\OtherTok{\textless{}{-}}\NormalTok{ x.mean }\SpecialCharTok{{-}}\NormalTok{ (x.sd}\SpecialCharTok{/}\FunctionTok{sqrt}\NormalTok{(}\FunctionTok{length}\NormalTok{(x)))}\SpecialCharTok{*}\NormalTok{t.perc }
\NormalTok{  right.boundary }\OtherTok{\textless{}{-}}\NormalTok{ x.mean }\SpecialCharTok{+}\NormalTok{ (x.sd}\SpecialCharTok{/}\FunctionTok{sqrt}\NormalTok{(}\FunctionTok{length}\NormalTok{(x)))}\SpecialCharTok{*}\NormalTok{t.perc}
  \FunctionTok{list}\NormalTok{ (}\AttributeTok{lower =}\NormalTok{ left.boundary, }\AttributeTok{upper =}\NormalTok{ right.boundary}
\ErrorTok{\}}
\FunctionTok{conf.int}\NormalTok{ (}\AttributeTok{x =}\NormalTok{ sleep[,}\DecValTok{1}\NormalTok{])}
\end{Highlighting}
\end{Shaded}

\begin{enumerate}
\def\labelenumi{(\alph{enumi})}
\setcounter{enumi}{13}
\tightlist
\item
  Select \emph{only the final line} and run it. Check the R console. No problem, the function executed correctly. This is because the code for \texttt{conf.int} in the script file was changed, but the updated object was not created by running it in the console.
\item
  Select \emph{all the code} in the script and run it. Check the R console. Discuss.
\end{enumerate}

\subsection{\texorpdfstring{Writing an R function using \texttt{fix()}}{Writing an R function using fix()}}\label{writing-an-r-function-using-fix}

When using \texttt{fix()} the built-in \emph{{R text editor}} can be used when using script files but in the windows environment {notepad} or preferably \href{www.notepad-plus-plus.org/download/}{notepad++} or \href{https://tinn-r.org/en/}{Tinn-R} is preferred.

The following instruction is necessary for changing the default editor to be used with \texttt{fix()}:

\begin{Shaded}
\begin{Highlighting}[]
\FunctionTok{options}\NormalTok{(}\AttributeTok{editor =} \StringTok{"notepad"}\NormalTok{)}
\end{Highlighting}
\end{Shaded}

or

\begin{Shaded}
\begin{Highlighting}[]
\FunctionTok{options}\NormalTok{(}\AttributeTok{editor =} \StringTok{"full path to the relevant exe file"}\NormalTok{)}
\end{Highlighting}
\end{Shaded}

\begin{enumerate}
\def\labelenumi{(\alph{enumi})}
\tightlist
\item
  Enter \texttt{fix\ (my.func)} at the R prompt. A text editor will open. Type the instructions as shown below.
\end{enumerate}

\begin{Shaded}
\begin{Highlighting}[]
\NormalTok{function (x = sleep[,1])}
\NormalTok{\{}
\NormalTok{  x.mean \textless{}{-} mean(x)\textasciigrave{}}
\NormalTok{  x.sd \textless{}{-} sd(x)}
\NormalTok{  t.perc \textless{}{-} qt(0.975,19)}
\NormalTok{  left.boundary \textless{}{-} x.mean {-} (x.sd/sqrt(length(x)))*t.perc}
\NormalTok{  right.boundary \textless{}{-} x.mean + (x.sd/sqrt(length(x)))*t.perc}
\NormalTok{  list (lower = left.boundary, upper = right.boundary)}
\NormalTok{\}}
\end{Highlighting}
\end{Shaded}

Close the window. Check what happens in the R console.

You can now run the function from the commands window (the console) similar to in \ref{script}(l), but changing the name of the function from \texttt{conf.int} to \texttt{my.func}.

\begin{enumerate}
\def\labelenumi{(\alph{enumi})}
\setcounter{enumi}{1}
\item
  What will happen if a syntax error is made when using fix? At the R prompt type fix (my.func). Make a deliberate syntax error, e.g.~delete the last closing brace. Close the text editor window. What happens in the console? What is to be done to correct the mistake?
\item
  Carefully study the message in the R console when a syntax error occurred in a function created by \texttt{fix()}:
\end{enumerate}

\begin{Shaded}
\begin{Highlighting}[]
\NormalTok{\textgreater{} Error in edit(name, file, title, editor) :}
\NormalTok{    unexpected \textquotesingle{}yyy\textquotesingle{} occurred on line xx}
\NormalTok{    use a command like}
\NormalTok{    x \textless{}{-} edit()}
\NormalTok{    to recover}
\end{Highlighting}
\end{Shaded}

\begin{enumerate}
\def\labelenumi{(\alph{enumi})}
\setcounter{enumi}{3}
\tightlist
\item
  The following is the correct way to respond to the above message from the R evaluator:
\end{enumerate}

\begin{Shaded}
\begin{Highlighting}[]
\NormalTok{my.func }\OtherTok{\textless{}{-}} \FunctionTok{edit}\NormalTok{()}
\end{Highlighting}
\end{Shaded}

If you simply use \texttt{fix(my.func)} at this point, the R and the editor will revert to the version of the function \emph{before} the previous edit.

\textbf{WARNING}

Before writing a function for solving any problem: make sure the problem is understood exactly; make 100\% sure the relevant statistical theory is understood correctly. Failure to do so is careless and dangerous!

\section{R Projects}\label{r-projects}

The different windows in R are the Data window, Script window, Graph window and Menus and Dialog windows. The current {workspace} in R is \texttt{.GlobalEnv}. The function \texttt{getwd()} is used to obtain the path to the current folder's {.Rdata} and {.Rhistory}.

\emph{Note}: In order to see the files {.Rdata} and {.Rhistory} being displayed as such, it may be necessary to turn off the option ``Hide extensions for known file types'' in Windows Explorer.

It is important to make provision for different {workspaces} associated with different \emph{{projects}}. In R, different \emph{{.Rdata}} files in different folders would separate different projects. There is however much to gain in using Projects in RStudio.

\subsection{Creating a project in RStudio}\label{creating-a-project-in-rstudio}

From the top menu, select \emph{File, New Project}. Follow the prompts to create a new project, either in an existing folder or creating a new folder for your project, say {MyProject}.

\begin{enumerate}
\def\labelenumi{(\alph{enumi})}
\tightlist
\item
  Navigate to the folder {MyProject} in Windows Explorer.
\item
  Notice a file {MyProject.Rproj} has been created in the folder.
\item
  By double-clicking on this file you open the project in RStudio. The advantages of opening the project this way are:
\end{enumerate}

\begin{itemize}
\tightlist
\item
  your {workspace} from the file {MyProject.Rdata} is automatically loaded
\item
  by placing any related files like data set in the folder {MyProject} or a subfolder, say {MyProject\textbackslash data} means that in your code you only have to use relative folder references, i.e.~refer to {MyProject\textbackslash mydata.xlsx} or {MyProject\textbackslash data\textbackslash mydata.xlsx} instead of something like {c:\textbackslash users\textbackslash myname\textbackslash Documents\textbackslash MyProject\textbackslash data\textbackslash mydata.xlsx}.
\item
  the major advantage of relative references is that it is not specific to the computer and makes porting between devices possible
\item
  sharing your project with a collaborator will simply entail copying the entire contents of the {MyProject} folder.
\end{itemize}

\section{A note on computations by a computer}\label{a-note-on-computations-by-a-computer}

When writing R functions it is important to keep in mind that the way computations are performed by a computer are not always according to the rules of algebra. Two important occurrences are given below.

\begin{itemize}
\item
  In mathematics the following statement is incorrect: \texttt{x\ =\ x\ +\ k} for \(k \neq 0\) but in computer programming the statement \texttt{x\ =\ x\ +\ k} is legitimate and it means \texttt{x} is replaced by \texttt{x\ +\ k}.
\item
  In general, the treatment of integers and real numbers for which R uses floating point representation happens at a fundamental level over which R has no control. Real numbers cannot necessarily be exactly represented in a computer -- some can only be approximated. Furthermore, there are limitations to the minimum and maximum numbers that can be represented in a computer. This might lead to what is known as \emph{{underflow}} or \emph{{overflow}}. A more detailed discussion appears in a later chapter.
\end{itemize}

Open an R session and issue the command

\begin{Shaded}
\begin{Highlighting}[]
\NormalTok{.Machine}
\end{Highlighting}
\end{Shaded}

for details about the numerical environment of your computer.

\section{Built-in data sets in R}\label{built-in-data-sets-in-r}

R contains several built-in data sets collected in the package \texttt{datasets}. This package is automatically attached to the search path. Type \texttt{?datasets} at the R prompt for details. Apart from these data sets several other data sets from other packages are also used in this book.

\section{\texorpdfstring{The use of \texttt{.First()} and \texttt{.Last()}}{The use of .First() and .Last()}}\label{the-use-of-.first-and-.last}

The function \texttt{.First()} is executed at the beginning of every R session. \emph{{This only works in R and not in RStudio}}.

Instead of having to specify

\begin{Shaded}
\begin{Highlighting}[]
\FunctionTok{options}\NormalTok{(}\AttributeTok{editor =} \StringTok{"notepad"}\NormalTok{)}
\end{Highlighting}
\end{Shaded}

each time an R session is initialized, create the following function and save in the {.Rdata} before exiting R.

\begin{Shaded}
\begin{Highlighting}[]
\NormalTok{.First }\OtherTok{\textless{}{-}} \ControlFlowTok{function}\NormalTok{() \{ }\FunctionTok{options}\NormalTok{(}\AttributeTok{editor =} \StringTok{"notepad"}\NormalTok{) \}}
\end{Highlighting}
\end{Shaded}

to ensures that Notepad is the text editor during any subsequent session.

Similar to \texttt{.First()} the function \texttt{.Last()} can be created for execution at the end of an R session.

\subsection{\texorpdfstring{Security: an example of the usage of \texttt{.First()}}{Security: an example of the usage of .First()}}\label{security-an-example-of-the-usage-of-.first}

The \texttt{.First()} facility can be used to prevent access to a R workspace by setting a password protection. This can be done as follows:

Create a new workspace for running the example on security. In this workspace create the following R function

\begin{Shaded}
\begin{Highlighting}[]
\NormalTok{password }\OtherTok{\textless{}{-}} \ControlFlowTok{function}\NormalTok{()        }\CommentTok{\# Note the structure of a function}
\NormalTok{\{ }\FunctionTok{cat}\NormalTok{(}\StringTok{"Password? }\SpecialCharTok{\textbackslash{}n}\StringTok{"}\NormalTok{)}
\NormalTok{  password }\OtherTok{\textless{}{-}} \FunctionTok{readline}\NormalTok{()      }\CommentTok{\# What is the usage of readline()? }
  \ControlFlowTok{if}\NormalTok{ (password }\SpecialCharTok{!=} \StringTok{"PASSWORD"}\NormalTok{) }
    \FunctionTok{q}\NormalTok{(}\AttributeTok{save=}\StringTok{"no"}\NormalTok{)              }\CommentTok{\# The meaning of !=  is "not equal to"}
  \ControlFlowTok{else}\NormalTok{ (}\FunctionTok{cat}\NormalTok{(}\StringTok{"You can proceed }\SpecialCharTok{\textbackslash{}n}\StringTok{"}\NormalTok{))}
\NormalTok{\}               }
\end{Highlighting}
\end{Shaded}

Now create the function:

\begin{Shaded}
\begin{Highlighting}[]
\NormalTok{.First }\OtherTok{\textless{}{-}} \ControlFlowTok{function}\NormalTok{()}
\NormalTok{\{   }\CommentTok{\#  What must you be careful of?}
   \FunctionTok{password}\NormalTok{()}
\NormalTok{\}}
\end{Highlighting}
\end{Shaded}

\begin{itemize}
\tightlist
\item
  Terminate your R session and open it again.
\item
  Discuss the construction and usage of the above functions.
\item
  Can you break the above security?
\item
  Can you make changes to the above security to make it more safe?
\end{itemize}

\section{Options}\label{options}

Study the result of the instruction \texttt{\textgreater{}\ options()} in R.

\section{Creating PDF and HTML documents from R output: R Markdown}\label{creating-pdf-and-html-documents-from-r-output-r-markdown}

The R package \texttt{knitr} is used to obtain reproducible results from R code in the form of PDF or HTML documents. In addition to \texttt{knitr}, R \textbf{{Markdown}} can be used to create HTML, PDF or even MS Word documents. Markdown is a so-called markup language with plain-text-formating syntax. An R Markdown document is written in markdown and contains chunks of embedded R code. Although the \texttt{render()} function in the package \texttt{rmarkdown} can be used (similar to the \texttt{knit()} function from the package \texttt{knitr}), to create the output document from the R Markdown {.Rmd} file, R Markdown is typically used in conjunction with {RStudio}. In the top menu, select \emph{File, New File, R Markdown\ldots{}} to open the {example.Rmd} file providing the user with the structure of an R Markdown file. For our illustration, we will select the output format as HTML.

Edit the {example.Rmd} file to contain the following:

\begin{Shaded}
\begin{Highlighting}[]
\NormalTok{{-}{-}{-}}
\NormalTok{title: "An Illustration of Some Capabilities of R Markdown"}
\NormalTok{author: "Niel le Roux and Sugnet Lubbe"}
\NormalTok{date: "22/01/2021"}
\NormalTok{output: html\_document}
\NormalTok{{-}{-}{-}}

\NormalTok{\textasciigrave{}\textasciigrave{}\textasciigrave{}\{r setup, include=FALSE\}}
\NormalTok{knitr::opts\_chunk$set(echo = TRUE)}
\NormalTok{\textasciigrave{}\textasciigrave{}\textasciigrave{}}

\NormalTok{\#\# Short description}

\NormalTok{Code chunks in .Rmd files are delimited with \textasciigrave{} \textasciigrave{}\textasciigrave{}\textasciigrave{}\{r\} \textasciigrave{} at the top where a chunk }
\NormalTok{label and any chunk options can appear and  \textasciigrave{} \textasciigrave{}\textasciigrave{}\textasciigrave{} \textasciigrave{} at the end. In{-}line R code }
\NormalTok{chunks are indicated with single \textasciigrave{} \textasciigrave{}r \textasciigrave{} on either side.}

\NormalTok{*****}

\NormalTok{Here is an example containing several chunks of code. Note that in the first }
\NormalTok{chunk R code is not shown due to the option \textasciigrave{}echo = FALSE\textasciigrave{}. In the remaining }
\NormalTok{chunks R code is shown due to the option above \textquotesingle{}echo = TRUE\textquotesingle{}.}

\NormalTok{\_Note R code not shown for this chunk.\_}

\NormalTok{\textasciigrave{}\textasciigrave{}\textasciigrave{}\{r y, echo=FALSE\}}
\NormalTok{y \textless{}{-} 1}
\NormalTok{y}
\NormalTok{\textasciigrave{}\textasciigrave{}\textasciigrave{}}

\NormalTok{\textasciigrave{}\textasciigrave{}\textasciigrave{}\{r rnorm\}}
\NormalTok{require(lattice)}
\NormalTok{set.seed(123)}
\NormalTok{x \textless{}{-} rnorm(1000, 20, 5)}
\NormalTok{\textasciigrave{}\textasciigrave{}\textasciigrave{}}

\NormalTok{We analyse data drawn from $\textbackslash{}mathcal\{N\}(20,25)$. The mean is }
\NormalTok{\textasciigrave{}r round(mean(x),3)\textasciigrave{}. The following code shows the distribution via a histogram}

\NormalTok{\textasciigrave{}\textasciigrave{}\textasciigrave{}\{r histexample\}}
\NormalTok{  hist(x)}
\NormalTok{\textasciigrave{}\textasciigrave{}\textasciigrave{}}

\NormalTok{and the code below via a boxplot.}

\NormalTok{\textasciigrave{}\textasciigrave{}\textasciigrave{}\{r boxexample\}}
\NormalTok{  boxplot(x)}
\NormalTok{\textasciigrave{}\textasciigrave{}\textasciigrave{}}

\NormalTok{The first element of \textbackslash{}texttt\{x\} is \textasciigrave{}r x[1]\textasciigrave{}. Note the usage of \textasciigrave{} \textbackslash{}texttt\{x\} \textasciigrave{} }
\NormalTok{above.}

\NormalTok{*two plots side by side (option fig.show=\textquotesingle{}hold\textquotesingle{})*}

\NormalTok{\textasciigrave{}\textasciigrave{}\textasciigrave{}\{r side{-}by{-}side, fig.show=\textquotesingle{}hold\textquotesingle{}, out.width="50\%"\}}
\NormalTok{  par(mar=c(4,4,0.1,0.1), cex.lab=0.95, cex.axis=0.9, mgp=c(2,0.7,0), }
\NormalTok{      tcl={-}0.3, las=1)}
\NormalTok{  boxplot(x)}
\NormalTok{  hist(x,main="")}
\NormalTok{\textasciigrave{}\textasciigrave{}\textasciigrave{}}

\NormalTok{\textasciigrave{}\textasciigrave{}\textasciigrave{}\{r linear\_model\}}
\NormalTok{  n \textless{}{-} 10}
\NormalTok{  x \textless{}{-} rnorm(n)}
\NormalTok{  y \textless{}{-} 2*x + rnorm(n)}
\NormalTok{  out \textless{}{-} lm(y \textasciitilde{} x)}
\NormalTok{  summary(out)$coef}
\NormalTok{\textasciigrave{}\textasciigrave{}\textasciigrave{}}
\end{Highlighting}
\end{Shaded}

At the top of the text editor, click on \emph{Knit} to create the HTML document. Note that with the down arrow, options \emph{Knit to PDF} and \emph{Knit to Word} can also be chosen. The output format is also specified in line 5 of the text file with \texttt{output:\ html\_document}. Had we chosen PDF as output format, it would be \texttt{output:\ pdf\_document}. Typically, R Markdown is used for reporting, directly incorporating the R code and output. For more formal documents with Figure and Table caption references, tables of content, etc. the R package \texttt{bookdown} should be used. Install the package and replace the output statement with \texttt{output:bookdown::pdf\_document2}. For more information on the use of bookdown, \href{https://bookdown.org/}{click here}.

\section{Command line editing}\label{command-line-editing}

Commands given in an R session are stored together with commands given in previous sessions in a file {.History} in the same folder as the {.RData} file. In an R session previous commands can be retrieved at the R prompt by pressing the \emph{up} and \emph{down} arrow keys. A previous command can then be edited using the \emph{backspace}, \emph{delete}, \emph{home}, \emph{end} keys as well as the shortcuts for \emph{copy} and \emph{paste}.

\chapter{Managing objects}\label{objects}

After completing the introductory chapter you now know how to

\begin{itemize}
\tightlist
\item
  initialize an R session;
\item
  save your workspace;
\item
  open an existing project;
\item
  execute simple tasks in R to obtain numerical, text or graphical results;
\item
  obtain help.
\end{itemize}

You know also that everything in R can be considered as some kind of an object. In this chapter the focus is on what properties the different objects have and how to manage objects in the workspace.

\section{Instructions and objects in R}\label{instructions-and-objects-in-r}

\subsection{General}\label{general}

Recall that

\begin{itemize}
\item
  instructions are separated by a semi-colon or start on new lines;
\item
  the \texttt{\#} symbol marks the rest of the line as comments;
\item
  the default R (primary) prompt is \texttt{\textgreater{}}; the secondary default prompt is \texttt{+};
\item
  use of \texttt{\textless{}-} to create objects. (The equality sign (\texttt{=}) will also be accepted. However, avoid this practice and use

  \begin{itemize}
  \tightlist
  \item
    \texttt{=} only for function arguments;
  \item
    \texttt{\textless{}-} for assignment;
  \item
    \texttt{==} for comparison / control structures);
  \end{itemize}
\item
  the use of \texttt{-\textgreater{}} for assigning left-hand side to the name on right-hand side.
\item
  the use of function \texttt{assign()} for assigning names to objects. (to be discussed in detail in Chapter \ref{operators})
\end{itemize}

\paragraph*{Examples}\label{examples}
\addcontentsline{toc}{paragraph}{Examples}

\begin{Shaded}
\begin{Highlighting}[]
\NormalTok{aa }\OtherTok{\textless{}{-}} \DecValTok{1}\SpecialCharTok{:}\DecValTok{10}
\end{Highlighting}
\end{Shaded}

Assigning numeric vector to name ``aa''. Assignment takes place in global environment.

\begin{Shaded}
\begin{Highlighting}[]
\NormalTok{Aa }\OtherTok{\textless{}{-}} \FunctionTok{seq}\NormalTok{(}\AttributeTok{from =} \DecValTok{1}\NormalTok{,}\AttributeTok{to =} \DecValTok{10}\NormalTok{,}\AttributeTok{by =} \FloatTok{0.01}\NormalTok{); yy }\OtherTok{\textless{}{-}} \FunctionTok{c}\NormalTok{(}\StringTok{"a"}\NormalTok{,}\StringTok{"b"}\NormalTok{,}\StringTok{"c"}\NormalTok{)}
\FunctionTok{c}\NormalTok{(}\StringTok{"a"}\NormalTok{,}\StringTok{"b"}\NormalTok{,}\StringTok{"c"}\NormalTok{) }\OtherTok{{-}\textgreater{}}\NormalTok{ bb }
\end{Highlighting}
\end{Shaded}

Assigning character vector to name ``bb''.

\begin{Shaded}
\begin{Highlighting}[]
\FunctionTok{assign}\NormalTok{(}\StringTok{"aa"}\NormalTok{, }\FunctionTok{rnorm}\NormalTok{(}\DecValTok{10}\NormalTok{), }\AttributeTok{pos =} \DecValTok{1}\NormalTok{)}
\end{Highlighting}
\end{Shaded}

Note the use of the argument \texttt{pos}, '' '' or ' ' are used for characters. Be careful when mixing single quotes and double quotes. See below.

\begin{Shaded}
\begin{Highlighting}[]
\FunctionTok{c}\NormalTok{(}\StringTok{"u"}\NormalTok{,}\StringTok{\textquotesingle{}v\textquotesingle{}}\NormalTok{,}\StringTok{"\textquotesingle{}w\textquotesingle{}"}\NormalTok{,}\StringTok{""}\NormalTok{x}\StringTok{""}\NormalTok{,}\StringTok{\textquotesingle{}"y"\textquotesingle{}}\NormalTok{,}\StringTok{\textquotesingle{}\textquotesingle{}}\NormalTok{z}\StringTok{\textquotesingle{}\textquotesingle{}}\NormalTok{) }\OtherTok{{-}\textgreater{}}\NormalTok{ cc}
\CommentTok{\#\textgreater{} Error in parse(text = input): \textless{}text\textgreater{}:1:19: unexpected symbol}
\CommentTok{\#\textgreater{} 1: c("u",\textquotesingle{}v\textquotesingle{},"\textquotesingle{}w\textquotesingle{}",""x}
\CommentTok{\#\textgreater{}                       \^{}}
\end{Highlighting}
\end{Shaded}

\begin{Shaded}
\begin{Highlighting}[]
\FunctionTok{c}\NormalTok{(}\StringTok{"u"}\NormalTok{,}\StringTok{\textquotesingle{}v\textquotesingle{}}\NormalTok{,}\StringTok{"\textquotesingle{}w\textquotesingle{}"}\NormalTok{,}\StringTok{\textquotesingle{}"x"\textquotesingle{}}\NormalTok{,}\StringTok{\textquotesingle{}"y"\textquotesingle{}}\NormalTok{,}\StringTok{\textquotesingle{}\textquotesingle{}}\NormalTok{z}\StringTok{\textquotesingle{}\textquotesingle{}}\NormalTok{) }\OtherTok{{-}\textgreater{}}\NormalTok{ cc}
\CommentTok{\#\textgreater{} Error in parse(text = input): \textless{}text\textgreater{}:1:31: unexpected symbol}
\CommentTok{\#\textgreater{} 1: c("u",\textquotesingle{}v\textquotesingle{},"\textquotesingle{}w\textquotesingle{}",\textquotesingle{}"x"\textquotesingle{},\textquotesingle{}"y"\textquotesingle{},\textquotesingle{}\textquotesingle{}z}
\CommentTok{\#\textgreater{}                                   \^{}}
\end{Highlighting}
\end{Shaded}

\begin{Shaded}
\begin{Highlighting}[]
\FunctionTok{c}\NormalTok{(}\StringTok{"u"}\NormalTok{,}\StringTok{\textquotesingle{}v\textquotesingle{}}\NormalTok{,}\StringTok{"\textquotesingle{}w\textquotesingle{}"}\NormalTok{,}\StringTok{\textquotesingle{}"x"\textquotesingle{}}\NormalTok{,}\StringTok{\textquotesingle{}"y"\textquotesingle{}}\NormalTok{,}\StringTok{\textquotesingle{}z\textquotesingle{}}\NormalTok{) }\OtherTok{{-}\textgreater{}}\NormalTok{ cc }
\NormalTok{cc}
\CommentTok{\#\textgreater{} [1] "u"     "v"     "\textquotesingle{}w\textquotesingle{}"   "\textbackslash{}"x\textbackslash{}"" "\textbackslash{}"y\textbackslash{}"" "z"}
\end{Highlighting}
\end{Shaded}

\begin{itemize}
\tightlist
\item
  Explain error message above.
\item
  Explain backslash above.
\end{itemize}

\begin{Shaded}
\begin{Highlighting}[]
\FunctionTok{objects}\NormalTok{()}
\CommentTok{\#\textgreater{} [1] "aa" "Aa" "bb" "cc" "yy"}
\NormalTok{aa}
\CommentTok{\#\textgreater{}  [1] {-}0.23011972  0.43608710 {-}0.60065975  0.36947189}
\CommentTok{\#\textgreater{}  [5] {-}1.31056587  3.25775913 {-}0.90372152  0.53345207}
\CommentTok{\#\textgreater{}  [9]  0.06807774  0.43262019}
\NormalTok{bb}
\CommentTok{\#\textgreater{} [1] "a" "b" "c"}
\FunctionTok{objects}\NormalTok{()[}\DecValTok{3}\NormalTok{]}
\CommentTok{\#\textgreater{} [1] "bb"}
\FunctionTok{parse}\NormalTok{(}\AttributeTok{text=}\FunctionTok{objects}\NormalTok{()[}\DecValTok{3}\NormalTok{])}
\CommentTok{\#\textgreater{} expression(bb)}
\FunctionTok{eval}\NormalTok{(}\FunctionTok{parse}\NormalTok{(}\AttributeTok{text=}\FunctionTok{objects}\NormalTok{()[}\DecValTok{3}\NormalTok{]))}
\CommentTok{\#\textgreater{} [1] "a" "b" "c"}
\FunctionTok{rm}\NormalTok{(a,b)}
\CommentTok{\#\textgreater{} Warning in rm(a, b): object \textquotesingle{}a\textquotesingle{} not found}
\CommentTok{\#\textgreater{} Warning in rm(a, b): object \textquotesingle{}b\textquotesingle{} not found}
\FunctionTok{rm}\NormalTok{(aa,bb)}
\FunctionTok{objects}\NormalTok{()}
\CommentTok{\#\textgreater{} [1] "Aa" "cc" "yy"}
\FunctionTok{rm}\NormalTok{(}\StringTok{"cc"}\NormalTok{)}
\FunctionTok{objects}\NormalTok{()}
\CommentTok{\#\textgreater{} [1] "Aa" "yy"}
\end{Highlighting}
\end{Shaded}

\subsection{Objects in R}\label{objects-in-r}

\begin{enumerate}
\def\labelenumi{(\alph{enumi})}
\item
  Everything is an object but there are many different types of objects.
\item
  Study and also take note of the following \emph{{naming conventions}}:
\end{enumerate}

\begin{itemize}
\tightlist
\item
  Allowed are upper or lower case letters, numbers 0 -- 9, full stop(s) and underscore(s).
\item
  Must not begin with a number.
\item
  R is case sensitive i.e.~\texttt{John} and \texttt{john} refer to different objects.
\item
  Use full stops (periods) or underscores to break up a name into meaningful words.
\item
  Avoid \texttt{c}, \texttt{s}, \texttt{t}, \texttt{C}, \texttt{F}, \texttt{T}, \texttt{diff} as well as other reserved words for naming an object.
\end{itemize}

\begin{enumerate}
\def\labelenumi{(\alph{enumi})}
\setcounter{enumi}{2}
\tightlist
\item
  The use of the functions \texttt{conflicts()} and \texttt{find()} when naming objects. The instruction \texttt{conflicts\ (detail\ =\ TRUE)} outputs details on whether and where objects with identical names exist on the search path e.g.
\end{enumerate}

\begin{Shaded}
\begin{Highlighting}[]
\FunctionTok{conflicts}\NormalTok{(}\AttributeTok{detail=}\ConstantTok{TRUE}\NormalTok{)}
\CommentTok{\#\textgreater{} $\textasciigrave{}package:graphics\textasciigrave{}}
\CommentTok{\#\textgreater{} [1] "plot"}
\CommentTok{\#\textgreater{} }
\CommentTok{\#\textgreater{} $\textasciigrave{}package:methods\textasciigrave{}}
\CommentTok{\#\textgreater{} [1] "body\textless{}{-}"    "kronecker"}
\CommentTok{\#\textgreater{} }
\CommentTok{\#\textgreater{} $\textasciigrave{}package:base\textasciigrave{}}
\CommentTok{\#\textgreater{} [1] "body\textless{}{-}"    "kronecker" "plot"}
\end{Highlighting}
\end{Shaded}

The instruction \texttt{find\ ("object")} outputs details on whether and where objects with the name object exist on the search path e.g.

\begin{Shaded}
\begin{Highlighting}[]
\FunctionTok{find}\NormalTok{(}\StringTok{"kronecker"}\NormalTok{)}
\CommentTok{\#\textgreater{} [1] "package:methods" "package:base"}
\end{Highlighting}
\end{Shaded}

\begin{enumerate}
\def\labelenumi{(\alph{enumi})}
\setcounter{enumi}{3}
\tightlist
\item
  Objects can possess several attributes e.g.~
\end{enumerate}

\begin{itemize}
\tightlist
\item
  mode (The way an object is internally stored)
\item
  length
\item
  names
\item
  dim
\item
  class
\end{itemize}

\subsubsection*{Examples}\label{examples-1}
\addcontentsline{toc}{subsubsection}{Examples}

\begin{Shaded}
\begin{Highlighting}[]
\NormalTok{a }\OtherTok{\textless{}{-}} \DecValTok{1}\SpecialCharTok{:}\DecValTok{10}
\FunctionTok{class}\NormalTok{(a)}
\CommentTok{\#\textgreater{} [1] "integer"}
\NormalTok{b }\OtherTok{\textless{}{-}} \FunctionTok{factor}\NormalTok{(}\FunctionTok{c}\NormalTok{(}\StringTok{"a"}\NormalTok{,}\StringTok{"b"}\NormalTok{,}\StringTok{"c"}\NormalTok{))}
\FunctionTok{class}\NormalTok{(b)}
\CommentTok{\#\textgreater{} [1] "factor"}
\NormalTok{b}
\CommentTok{\#\textgreater{} [1] a b c}
\CommentTok{\#\textgreater{} Levels: a b c}
\FunctionTok{mode}\NormalTok{(a)}
\CommentTok{\#\textgreater{} [1] "numeric"}
\FunctionTok{mode}\NormalTok{(b)}
\CommentTok{\#\textgreater{} [1] "numeric"}
\FunctionTok{length}\NormalTok{(a)}
\CommentTok{\#\textgreater{} [1] 10}
\FunctionTok{length}\NormalTok{(b)}
\CommentTok{\#\textgreater{} [1] 3}
\FunctionTok{dim}\NormalTok{(a)}
\CommentTok{\#\textgreater{} NULL}
\NormalTok{mat }\OtherTok{\textless{}{-}} \FunctionTok{matrix}\NormalTok{(}\DecValTok{1}\SpecialCharTok{:}\DecValTok{12}\NormalTok{,}\AttributeTok{nrow=}\DecValTok{4}\NormalTok{)}
\NormalTok{mat}
\CommentTok{\#\textgreater{}      [,1] [,2] [,3]}
\CommentTok{\#\textgreater{} [1,]    1    5    9}
\CommentTok{\#\textgreater{} [2,]    2    6   10}
\CommentTok{\#\textgreater{} [3,]    3    7   11}
\CommentTok{\#\textgreater{} [4,]    4    8   12}
\FunctionTok{dim}\NormalTok{(mat)}
\CommentTok{\#\textgreater{} [1] 4 3}
\FunctionTok{mode}\NormalTok{(mat)}
\CommentTok{\#\textgreater{} [1] "numeric"}
\NormalTok{logic }\OtherTok{\textless{}{-}} \FunctionTok{c}\NormalTok{(}\ConstantTok{TRUE}\NormalTok{,}\ConstantTok{TRUE}\NormalTok{,}\ConstantTok{FALSE}\NormalTok{,}\ConstantTok{TRUE}\NormalTok{)}
\FunctionTok{mode}\NormalTok{(logic)}
\CommentTok{\#\textgreater{} [1] "logical"}
\FunctionTok{class}\NormalTok{(logic)}
\CommentTok{\#\textgreater{} [1] "logical"}
\end{Highlighting}
\end{Shaded}

Levels show that it is a categorical variable (object).

Mode \texttt{numeric} tells us that the categorical variable (object) \texttt{b} is internally stored as a set of numeric codes.

\begin{enumerate}
\def\labelenumi{(\alph{enumi})}
\setcounter{enumi}{4}
\tightlist
\item
  Special attention is given to the class and mode of integers. An object of type integer is stored internally more effectively than an integer represented in double format.
\end{enumerate}

\begin{Shaded}
\begin{Highlighting}[]
\NormalTok{x }\OtherTok{\textless{}{-}} \DecValTok{5}
\NormalTok{y }\OtherTok{\textless{}{-}} \DecValTok{5}\DataTypeTok{L}
\FunctionTok{typeof}\NormalTok{(x)}
\CommentTok{\#\textgreater{} [1] "double"}
\FunctionTok{typeof}\NormalTok{(y)}
\CommentTok{\#\textgreater{} [1] "integer"}
\FunctionTok{class}\NormalTok{(x)}
\CommentTok{\#\textgreater{} [1] "numeric"}
\FunctionTok{class}\NormalTok{(y)}
\CommentTok{\#\textgreater{} [1] "integer"}
\FunctionTok{mode}\NormalTok{(x)}
\CommentTok{\#\textgreater{} [1] "numeric"}
\FunctionTok{mode}\NormalTok{(y)}
\CommentTok{\#\textgreater{} [1] "numeric"}
\end{Highlighting}
\end{Shaded}

\begin{enumerate}
\def\labelenumi{(\alph{enumi})}
\setcounter{enumi}{5}
\item
  Objects in R are \emph{{vectors}}, \emph{{functions}} or \emph{{lists}}. There are no scalars - instead vectors of length one are used. In addition to the above three types, there are several other types of objects.
\item
  Objects that are created during a session are permanently stored in the {.RData} file in the folder containing the workspace (unless not saved at termination).
\item
  Objects that are created within a function exist only for as long as the function is being executed.
\item
  Use of \texttt{rm()} and \texttt{rm(list\ =\ ListOfNames)} to remove objects from the workspace.
\item
  Use of \texttt{objects()} or equivalently \texttt{ls()} to obtain a list of object names in a data base (by default the workspace). Note the optional arguments \texttt{pos}, \texttt{all.names} and \texttt{pattern} to specify which database to be considered and what object names to include.
\item
  How can an object be printed to the screen?
\item
  \emph{{Warning:}} If a new object is assigned to a name that already exists in the working directory the old object is overwritten without warning and it cannot be retrieved again.
\end{enumerate}

\subsection{Data in R}\label{data-in-r}

\begin{enumerate}
\def\labelenumi{(\alph{enumi})}
\item
  R has several built-in data sets. Use \texttt{?datasets} and/or \texttt{library(help=\ "datasets")} for details. Note that the two instructions return different information.
\item
  Study the help file of \texttt{c()}.
\item
  Study the help file of \texttt{scan()}.
\item
  Study the help files of \texttt{read.table()} and \texttt{read.csv()}. Care must be taken with data containing characters (text) and categorical variables. Reading data into R will be discussed in detail in Chapter \ref{data}.
\end{enumerate}

\subsection{Generation of data}\label{generation-of-data}

Study the operators and functions \texttt{:}, \texttt{seq()}, \texttt{rep()}, \texttt{rev()}, \texttt{rnorm()}, \texttt{runif()} with the following instructions:

\begin{Shaded}
\begin{Highlighting}[]
\DecValTok{1}\SpecialCharTok{:}\DecValTok{10}
\DecValTok{8}\SpecialCharTok{:}\DecValTok{3}
\FunctionTok{seq}\NormalTok{(}\AttributeTok{from=}\DecValTok{1}\NormalTok{, }\AttributeTok{to=}\DecValTok{10}\NormalTok{, }\AttributeTok{length=}\DecValTok{10}\NormalTok{)}
\FunctionTok{seq}\NormalTok{(}\AttributeTok{from=}\DecValTok{2}\NormalTok{, }\AttributeTok{to=}\DecValTok{10}\NormalTok{, }\AttributeTok{length=}\DecValTok{5}\NormalTok{)}
\FunctionTok{rev}\NormalTok{(}\DecValTok{10}\SpecialCharTok{:}\DecValTok{1}\NormalTok{)}
\FunctionTok{rnorm}\NormalTok{ (}\DecValTok{20}\NormalTok{, }\AttributeTok{mean=}\DecValTok{50}\NormalTok{, }\AttributeTok{sd=}\DecValTok{5}\NormalTok{)}
\FunctionTok{runif}\NormalTok{ (}\DecValTok{10}\NormalTok{, }\AttributeTok{min=}\DecValTok{1}\NormalTok{, }\AttributeTok{max=}\DecValTok{3}\NormalTok{)}
\end{Highlighting}
\end{Shaded}

The function \texttt{rmvnorm()} for generating multivariate normal samples is in the \texttt{mvtnorm} R package. This package must first be loaded by using the instruction

\begin{Shaded}
\begin{Highlighting}[]
\FunctionTok{library}\NormalTok{(mvtnorm)}
\end{Highlighting}
\end{Shaded}

Alternatively, for generating multivariate normally data there is also a function \texttt{mvrnorm()} in R package \texttt{MASS}.

\section{Introduction to functions in R}\label{introduction-to-functions-in-r}

We introduced R functions in section \ref{FunctionIntro}. The basic structure of an R function is as follows:

\begin{Shaded}
\begin{Highlighting}[]
\NormalTok{func.name }\OtherTok{\textless{}{-}} \ControlFlowTok{function}\NormalTok{(list of arguments)}
\NormalTok{\{}
  \CommentTok{\# R code}
\NormalTok{\}}
\end{Highlighting}
\end{Shaded}

When the function \texttt{func.name()} is called, the code in \texttt{\{\ \}} is executed.

The arguments of a function can be inspected by using the command

\begin{Shaded}
\begin{Highlighting}[]
\FunctionTok{args}\NormalTok{(name of }\ControlFlowTok{function}\NormalTok{)}
\end{Highlighting}
\end{Shaded}

The function \texttt{str(x)} provides information on the object \texttt{x}. If \texttt{x} is a function its output is similar to that of \texttt{args()}. Default values are given to function arguments using the construction (\texttt{argument\ name\ =\ value}). It is good programming practice to make extensively use of comments to describe arguments and / or what a particular chunk of code does.
What is the usage of the following function:

\begin{Shaded}
\begin{Highlighting}[]
\NormalTok{cube }\OtherTok{\textless{}{-}} \ControlFlowTok{function}\NormalTok{(a) a}\SpecialCharTok{\^{}}\DecValTok{3}
\end{Highlighting}
\end{Shaded}

In the above function the argument a is called a \emph{{dummy argument}}. What will happen to an object \texttt{a} in the working directory?

Functions are called by replacing the \emph{{formal arguments}} by the \emph{{actual arguments}}. This can be done \emph{{by position}} or \emph{{by name}}. \emph{Hint}: It is less error prone to call functions using named arguments. Create the following function

\begin{Shaded}
\begin{Highlighting}[]
\NormalTok{Demofunc }\OtherTok{\textless{}{-}} \ControlFlowTok{function}\NormalTok{(}\AttributeTok{vec =} \DecValTok{1}\SpecialCharTok{:}\DecValTok{10}\NormalTok{, m,k)}
\NormalTok{ \{ }\CommentTok{\# Function to subtract a specified constant from}
   \CommentTok{\# each element of a given vector and after subtraction}
   \CommentTok{\# divide each element by a second specified constant.}
   \CommentTok{\# The result of the above transformation is returned.}
\NormalTok{ (vec }\SpecialCharTok{{-}}\NormalTok{ m)}\SpecialCharTok{/}\NormalTok{ k }
\NormalTok{\}}
\end{Highlighting}
\end{Shaded}

Execute the following function calls and explain the output

\begin{Shaded}
\begin{Highlighting}[]
\FunctionTok{Demofunc}\NormalTok{(}\DecValTok{3}\NormalTok{, }\DecValTok{2}\NormalTok{, }\DecValTok{5}\NormalTok{)}
\CommentTok{\#\textgreater{} [1] 0.2}
\FunctionTok{Demofunc}\NormalTok{(}\DecValTok{2}\NormalTok{,}\DecValTok{5}\NormalTok{)}
\CommentTok{\#\textgreater{} Error in Demofunc(2, 5): argument "k" is missing, with no default}
\FunctionTok{Demofunc}\NormalTok{(}\AttributeTok{m =} \DecValTok{2}\NormalTok{, }\AttributeTok{k =} \DecValTok{5}\NormalTok{)}
\CommentTok{\#\textgreater{}  [1] {-}0.2  0.0  0.2  0.4  0.6  0.8  1.0  1.2  1.4  1.6}
\FunctionTok{Demofunc}\NormalTok{(}\AttributeTok{m =} \DecValTok{2}\NormalTok{, }\AttributeTok{k =} \DecValTok{5}\NormalTok{, }\AttributeTok{vec =} \DecValTok{1}\SpecialCharTok{:}\DecValTok{100}\NormalTok{)}
\CommentTok{\#\textgreater{}   [1] {-}0.2  0.0  0.2  0.4  0.6  0.8  1.0  1.2  1.4  1.6  1.8}
\CommentTok{\#\textgreater{}  [12]  2.0  2.2  2.4  2.6  2.8  3.0  3.2  3.4  3.6  3.8  4.0}
\CommentTok{\#\textgreater{}  [23]  4.2  4.4  4.6  4.8  5.0  5.2  5.4  5.6  5.8  6.0  6.2}
\CommentTok{\#\textgreater{}  [34]  6.4  6.6  6.8  7.0  7.2  7.4  7.6  7.8  8.0  8.2  8.4}
\CommentTok{\#\textgreater{}  [45]  8.6  8.8  9.0  9.2  9.4  9.6  9.8 10.0 10.2 10.4 10.6}
\CommentTok{\#\textgreater{}  [56] 10.8 11.0 11.2 11.4 11.6 11.8 12.0 12.2 12.4 12.6 12.8}
\CommentTok{\#\textgreater{}  [67] 13.0 13.2 13.4 13.6 13.8 14.0 14.2 14.4 14.6 14.8 15.0}
\CommentTok{\#\textgreater{}  [78] 15.2 15.4 15.6 15.8 16.0 16.2 16.4 16.6 16.8 17.0 17.2}
\CommentTok{\#\textgreater{}  [89] 17.4 17.6 17.8 18.0 18.2 18.4 18.6 18.8 19.0 19.2 19.4}
\CommentTok{\#\textgreater{} [100] 19.6}
\end{Highlighting}
\end{Shaded}

Note the use of \texttt{prompt()} and \texttt{package.skeleton()} to provide a new function with a help-file.

The final expression in an R function is automatically returned when the function completes execution.

\begin{Shaded}
\begin{Highlighting}[]
\NormalTok{my.func }\OtherTok{\textless{}{-}} \ControlFlowTok{function}\NormalTok{(}\AttributeTok{a=}\DecValTok{5}\NormalTok{) }
\NormalTok{\{  a}\SpecialCharTok{+}\DecValTok{2}
\NormalTok{\}}
\FunctionTok{my.func}\NormalTok{()}
\CommentTok{\#\textgreater{} [1] 7}
\end{Highlighting}
\end{Shaded}

When a function consists of a single line, it can be written more succinctly

\begin{Shaded}
\begin{Highlighting}[]
\NormalTok{my.func }\OtherTok{\textless{}{-}} \ControlFlowTok{function}\NormalTok{(}\AttributeTok{a=}\DecValTok{5}\NormalTok{) \{  a}\SpecialCharTok{+}\DecValTok{2}\NormalTok{  \}}
\FunctionTok{my.func}\NormalTok{()}
\CommentTok{\#\textgreater{} [1] 7}
\end{Highlighting}
\end{Shaded}

or even without the \texttt{\{\ \}}:

\begin{Shaded}
\begin{Highlighting}[]
\NormalTok{my.func }\OtherTok{\textless{}{-}} \ControlFlowTok{function}\NormalTok{(}\AttributeTok{a=}\DecValTok{5}\NormalTok{) a}\SpecialCharTok{+}\DecValTok{2}
\FunctionTok{my.func}\NormalTok{()}
\CommentTok{\#\textgreater{} [1] 7}
\end{Highlighting}
\end{Shaded}

In general, functions will consist of more lines of code and often multiple outputs are returned. If only a single output object needs to be returned, the object can be created in the last line of the code

\begin{Shaded}
\begin{Highlighting}[]
\NormalTok{my.func }\OtherTok{\textless{}{-}} \ControlFlowTok{function}\NormalTok{(}\AttributeTok{a=}\DecValTok{5}\NormalTok{)}
\NormalTok{  \{  number }\OtherTok{\textless{}{-}}\NormalTok{ (a}\SpecialCharTok{+}\DecValTok{3}\NormalTok{)}\SpecialCharTok{\^{}}\DecValTok{2}
\NormalTok{     number}\SpecialCharTok{/}\NormalTok{a}
\NormalTok{  \}}
\FunctionTok{my.func}\NormalTok{()}
\CommentTok{\#\textgreater{} [1] 12.8}
\end{Highlighting}
\end{Shaded}

or with a \texttt{return()} statement:

\begin{Shaded}
\begin{Highlighting}[]
\NormalTok{my.func }\OtherTok{\textless{}{-}} \ControlFlowTok{function}\NormalTok{(}\AttributeTok{a=}\DecValTok{5}\NormalTok{)}
\NormalTok{  \{  number }\OtherTok{\textless{}{-}}\NormalTok{ (a}\SpecialCharTok{+}\DecValTok{3}\NormalTok{)}\SpecialCharTok{\^{}}\DecValTok{2}
     \FunctionTok{return}\NormalTok{(number}\SpecialCharTok{/}\NormalTok{a)}
\NormalTok{  \}}
\FunctionTok{my.func}\NormalTok{()}
\CommentTok{\#\textgreater{} [1] 12.8}
\end{Highlighting}
\end{Shaded}

In general, all the outputs are combined and returned as a \texttt{list}. The final expression in the function creates the list object:

\begin{Shaded}
\begin{Highlighting}[]
\NormalTok{my.func }\OtherTok{\textless{}{-}} \ControlFlowTok{function}\NormalTok{(}\AttributeTok{a=}\DecValTok{5}\NormalTok{)}
\NormalTok{  \{  number }\OtherTok{\textless{}{-}}\NormalTok{ (a}\SpecialCharTok{+}\DecValTok{3}\NormalTok{)}\SpecialCharTok{\^{}}\DecValTok{2}
     \FunctionTok{list}\NormalTok{(number}\SpecialCharTok{/}\NormalTok{a)}
\NormalTok{  \}}
\FunctionTok{my.func}\NormalTok{()}
\CommentTok{\#\textgreater{} [[1]]}
\CommentTok{\#\textgreater{} [1] 12.8}
\end{Highlighting}
\end{Shaded}

To return multiple outputs, the list is simply extended as shown below:

\begin{Shaded}
\begin{Highlighting}[]
\NormalTok{my.func }\OtherTok{\textless{}{-}} \ControlFlowTok{function}\NormalTok{(}\AttributeTok{a=}\DecValTok{5}\NormalTok{)}
\NormalTok{  \{  number }\OtherTok{\textless{}{-}}\NormalTok{ (a}\SpecialCharTok{+}\DecValTok{3}\NormalTok{)}\SpecialCharTok{\^{}}\DecValTok{2}
     \FunctionTok{list}\NormalTok{(number, number}\SpecialCharTok{/}\NormalTok{a)}
\NormalTok{  \}}
\FunctionTok{my.func}\NormalTok{()}
\CommentTok{\#\textgreater{} [[1]]}
\CommentTok{\#\textgreater{} [1] 64}
\CommentTok{\#\textgreater{} }
\CommentTok{\#\textgreater{} [[2]]}
\CommentTok{\#\textgreater{} [1] 12.8}
\end{Highlighting}
\end{Shaded}

It is good practice to name the output objects in the list, such as:

\begin{Shaded}
\begin{Highlighting}[]
\NormalTok{my.func }\OtherTok{\textless{}{-}} \ControlFlowTok{function}\NormalTok{(}\AttributeTok{a=}\DecValTok{5}\NormalTok{)}
\NormalTok{  \{  number }\OtherTok{\textless{}{-}}\NormalTok{ (a}\SpecialCharTok{+}\DecValTok{3}\NormalTok{)}\SpecialCharTok{\^{}}\DecValTok{2}
     \FunctionTok{list}\NormalTok{(}\AttributeTok{number =}\NormalTok{ number, }\AttributeTok{ratio =}\NormalTok{ number}\SpecialCharTok{/}\NormalTok{a)}
\NormalTok{  \}}
\FunctionTok{my.func}\NormalTok{()}
\CommentTok{\#\textgreater{} $number}
\CommentTok{\#\textgreater{} [1] 64}
\CommentTok{\#\textgreater{} }
\CommentTok{\#\textgreater{} $ratio}
\CommentTok{\#\textgreater{} [1] 12.8}
\end{Highlighting}
\end{Shaded}

Finally, to place the output into an object for further processing, the function is assigned to an object name:

\begin{Shaded}
\begin{Highlighting}[]
\NormalTok{my.func }\OtherTok{\textless{}{-}} \ControlFlowTok{function}\NormalTok{(}\AttributeTok{a=}\DecValTok{5}\NormalTok{)}
\NormalTok{  \{  number }\OtherTok{\textless{}{-}}\NormalTok{ (a}\SpecialCharTok{+}\DecValTok{3}\NormalTok{)}\SpecialCharTok{\^{}}\DecValTok{2}
     \FunctionTok{list}\NormalTok{(}\AttributeTok{number =}\NormalTok{ number, }\AttributeTok{ratio =}\NormalTok{ number}\SpecialCharTok{/}\NormalTok{a)}
\NormalTok{  \}}
\NormalTok{out }\OtherTok{\textless{}{-}} \FunctionTok{my.func}\NormalTok{()}
\NormalTok{out}
\CommentTok{\#\textgreater{} $number}
\CommentTok{\#\textgreater{} [1] 64}
\CommentTok{\#\textgreater{} }
\CommentTok{\#\textgreater{} $ratio}
\CommentTok{\#\textgreater{} [1] 12.8}
\end{Highlighting}
\end{Shaded}

\section{How R finds data}\label{findData}

In order to understand how objects are found by R it is necessary to have some understanding of the concepts

\begin{itemize}
\tightlist
\item
  Environment
\item
  Frame
\item
  Search path
\item
  Parent environment
\item
  Inheritance.
\end{itemize}

The mechanism that R uses to organize objects is based on frames and environments. A \emph{{frame}} is a collection of named objects and an \emph{{environment}} consists of a frame together with a pointer or reference to another environment called the \emph{{parent environment}}. Environments are nested so that the \emph{{parent environment}} is the environment that directly contains the current environment. At the start of an R session a \emph{{workspace}} is created which always has an associate environment, the \emph{{global environment}}. The global environment occupies the first position on the \emph{{search path}} and is accessed by a call to \texttt{globalenv()}. Packages and databases can be added to the search path by a call to \texttt{attach()} and removed from the search path by a call to \texttt{detach()}.

\begin{itemize}
\tightlist
\item
  What is an R \emph{{package}}? What is the difference between \emph{{installing}} and \emph{{loading}} a package?
\item
  Work through the following example:
\end{itemize}

\begin{Shaded}
\begin{Highlighting}[]
\FunctionTok{search}\NormalTok{()}
\CommentTok{\#\textgreater{} [1] ".GlobalEnv"        "package:stats"    }
\CommentTok{\#\textgreater{} [3] "package:graphics"  "package:grDevices"}
\CommentTok{\#\textgreater{} [5] "package:utils"     "package:datasets" }
\CommentTok{\#\textgreater{} [7] "package:methods"   "Autoloads"        }
\CommentTok{\#\textgreater{} [9] "package:base"}
\end{Highlighting}
\end{Shaded}

To attach the package \texttt{MASS}

\begin{Shaded}
\begin{Highlighting}[]
\FunctionTok{library}\NormalTok{ (MASS)}
\end{Highlighting}
\end{Shaded}

By default \texttt{MASS} is attached in the second position in the search path.

\begin{Shaded}
\begin{Highlighting}[]
\FunctionTok{search}\NormalTok{()}
\CommentTok{\#\textgreater{}  [1] ".GlobalEnv"        "package:MASS"     }
\CommentTok{\#\textgreater{}  [3] "package:stats"     "package:graphics" }
\CommentTok{\#\textgreater{}  [5] "package:grDevices" "package:utils"    }
\CommentTok{\#\textgreater{}  [7] "package:datasets"  "package:methods"  }
\CommentTok{\#\textgreater{}  [9] "Autoloads"         "package:base"}
\end{Highlighting}
\end{Shaded}

We use \texttt{detach} to remove \texttt{MASS} from the search path.

\begin{Shaded}
\begin{Highlighting}[]
\FunctionTok{detach}\NormalTok{(}\StringTok{"package:MASS"}\NormalTok{)}
\FunctionTok{search}\NormalTok{()}
\CommentTok{\#\textgreater{} [1] ".GlobalEnv"        "package:stats"    }
\CommentTok{\#\textgreater{} [3] "package:graphics"  "package:grDevices"}
\CommentTok{\#\textgreater{} [5] "package:utils"     "package:datasets" }
\CommentTok{\#\textgreater{} [7] "package:methods"   "Autoloads"        }
\CommentTok{\#\textgreater{} [9] "package:base"}
\end{Highlighting}
\end{Shaded}

To obtain the parent of the global environment

\begin{Shaded}
\begin{Highlighting}[]
\FunctionTok{parent.env}\NormalTok{(.GlobalEnv)}
\CommentTok{\#\textgreater{} \textless{}environment: package:stats\textgreater{}}
\CommentTok{\#\textgreater{} attr(,"name")}
\CommentTok{\#\textgreater{} [1] "package:stats"}
\CommentTok{\#\textgreater{} attr(,"path")}
\CommentTok{\#\textgreater{} [1] "C:/Program Files/R/R{-}4.5.1/library/stats"}
\FunctionTok{parent.env}\NormalTok{(}\FunctionTok{parent.env}\NormalTok{(.GlobalEnv))}
\CommentTok{\#\textgreater{} \textless{}environment: package:graphics\textgreater{}}
\CommentTok{\#\textgreater{} attr(,"name")}
\CommentTok{\#\textgreater{} [1] "package:graphics"}
\CommentTok{\#\textgreater{} attr(,"path")}
\CommentTok{\#\textgreater{} [1] "C:/Program Files/R/R{-}4.5.1/library/graphics"}
\FunctionTok{parent.env}\NormalTok{(}\FunctionTok{parent.env}\NormalTok{(}\FunctionTok{parent.env}\NormalTok{(.GlobalEnv)))}
\CommentTok{\#\textgreater{} \textless{}environment: package:grDevices\textgreater{}}
\CommentTok{\#\textgreater{} attr(,"name")}
\CommentTok{\#\textgreater{} [1] "package:grDevices"}
\CommentTok{\#\textgreater{} attr(,"path")}
\CommentTok{\#\textgreater{} [1] "C:/Program Files/R/R{-}4.5.1/library/grDevices"}
\FunctionTok{environmentName}\NormalTok{(}\FunctionTok{parent.env}\NormalTok{(}\FunctionTok{parent.env}\NormalTok{(}\FunctionTok{parent.env}\NormalTok{(.GlobalEnv))))}
\CommentTok{\#\textgreater{} [1] "package:grDevices"}
\end{Highlighting}
\end{Shaded}

When the R evaluator looks for an object and it cannot find the name in the global environment it will search the parent of the global environment. It will carry on the search along the search path until the first occurrence of the name. If the name is not found it will return the message \texttt{Error:\ object\ \textquotesingle{}xx\textquotesingle{}\ not\ found}. The usage of the double colon \texttt{::} and the triple colon \texttt{:::} is to access the intended object when more than one object with the same name exist on the search path. These two operators use the \emph{{namespace}} facility of R packages. The namespace of a package allow the creator of a package to hide functions and data that are meant only for internal use; it provides a way through the operators \texttt{::} and \texttt{:::} to an object within a particular package. Thus a namespace prevent functions from breaking down when a user selects a name that clashes with one in the package. The double-colon operator \texttt{::} selects objects from a particular namespace. Only functions that are exported from the package can be retrieved in this way. The triple-colon operator \texttt{:::} acts like the double-colon operator but also allows access to hidden objects. Packages are often inter-dependent, and loading one may cause others to be automatically loaded. Such automatically loaded packages are not added to the search list.

We note that the \emph{{function}} call \texttt{getAnywhere()}, which searches multiple packages can be used for finding hidden objects. When a function is called, R creates a new (temporary) environment which is enclosed in the current (calling) environment. Objects created in the new environment are not available in the parent environment and dies with it when the function terminates. Objects in the calling environment are available for use in the new environment created when a function is called.

Similarly, when an \emph{{expression}} is evaluated a hierarchy of environments is created. Search for objects continue up this hierarchy and if necessary to the global environment and from there up onto the search path.

\begin{itemize}
\tightlist
\item
  Study the use of the arguments \texttt{pos}, \texttt{all.names}, and \texttt{pattern} of the function \texttt{objects()}.
\item
  Study the behaviour of the functions \texttt{conflicts()} and \texttt{exists()} in the examples below:
\end{itemize}

\begin{Shaded}
\begin{Highlighting}[]
\FunctionTok{conflicts}\NormalTok{()}
\CommentTok{\#\textgreater{} [1] "body\textless{}{-}"    "kronecker" "plot"}
\FunctionTok{conflicts}\NormalTok{(}\AttributeTok{detail=}\ConstantTok{TRUE}\NormalTok{)}
\CommentTok{\#\textgreater{} $\textasciigrave{}package:graphics\textasciigrave{}}
\CommentTok{\#\textgreater{} [1] "plot"}
\CommentTok{\#\textgreater{} }
\CommentTok{\#\textgreater{} $\textasciigrave{}package:methods\textasciigrave{}}
\CommentTok{\#\textgreater{} [1] "body\textless{}{-}"    "kronecker"}
\CommentTok{\#\textgreater{} }
\CommentTok{\#\textgreater{} $\textasciigrave{}package:base\textasciigrave{}}
\CommentTok{\#\textgreater{} [1] "body\textless{}{-}"    "kronecker" "plot"}
\FunctionTok{exists}\NormalTok{(}\StringTok{"kronecker"}\NormalTok{)}
\CommentTok{\#\textgreater{} [1] TRUE}
\FunctionTok{exists}\NormalTok{(}\StringTok{"kronecker"}\NormalTok{, }\AttributeTok{where =} \DecValTok{1}\NormalTok{)}
\CommentTok{\#\textgreater{} [1] TRUE}
\FunctionTok{exists}\NormalTok{(}\StringTok{"kronecker"}\NormalTok{, }\AttributeTok{where =} \DecValTok{1}\NormalTok{, }\AttributeTok{inherits =} \ConstantTok{FALSE}\NormalTok{)}
\CommentTok{\#\textgreater{} [1] FALSE}
\FunctionTok{exists}\NormalTok{(}\StringTok{"kronecker"}\NormalTok{, }\AttributeTok{where =} \DecValTok{2}\NormalTok{)}
\CommentTok{\#\textgreater{} [1] TRUE}
\FunctionTok{exists}\NormalTok{(}\StringTok{"kronecker"}\NormalTok{, }\AttributeTok{where =} \DecValTok{2}\NormalTok{, }\AttributeTok{inherits =} \ConstantTok{FALSE}\NormalTok{)}
\CommentTok{\#\textgreater{} [1] FALSE}
\FunctionTok{exists}\NormalTok{(}\StringTok{"kronecker"}\NormalTok{, }\AttributeTok{where =} \DecValTok{7}\NormalTok{, }\AttributeTok{inherits =} \ConstantTok{FALSE}\NormalTok{)}
\CommentTok{\#\textgreater{} [1] TRUE}
\FunctionTok{exists}\NormalTok{(}\StringTok{"kronecker"}\NormalTok{, }\AttributeTok{where =} \DecValTok{8}\NormalTok{, }\AttributeTok{inherits =} \ConstantTok{FALSE}\NormalTok{)}
\CommentTok{\#\textgreater{} [1] FALSE}
\FunctionTok{exists}\NormalTok{(}\StringTok{"kronecker"}\NormalTok{, }\AttributeTok{where =} \DecValTok{9}\NormalTok{, }\AttributeTok{inherits =} \ConstantTok{FALSE}\NormalTok{)}
\CommentTok{\#\textgreater{} [1] TRUE}
\end{Highlighting}
\end{Shaded}

\begin{itemize}
\tightlist
\item
  Study the above code carefully and then explain what inheritance does.
\item
  The example below leads to the same conclusion as above but is more complicated at this stage. Its behaviour will become clear as we work through the coming chapters.
\end{itemize}

\begin{Shaded}
\begin{Highlighting}[]
\FunctionTok{sapply}\NormalTok{(}\FunctionTok{search}\NormalTok{(), }\ControlFlowTok{function}\NormalTok{(x) }\FunctionTok{exists}\NormalTok{(}\StringTok{"kronecker"}\NormalTok{, }\AttributeTok{where =}\NormalTok{ x, }\AttributeTok{inherits=}\ConstantTok{FALSE}\NormalTok{))}
\CommentTok{\#\textgreater{}        .GlobalEnv     package:stats  package:graphics }
\CommentTok{\#\textgreater{}             FALSE             FALSE             FALSE }
\CommentTok{\#\textgreater{} package:grDevices     package:utils  package:datasets }
\CommentTok{\#\textgreater{}             FALSE             FALSE             FALSE }
\CommentTok{\#\textgreater{}   package:methods         Autoloads      package:base }
\CommentTok{\#\textgreater{}              TRUE             FALSE              TRUE}
\end{Highlighting}
\end{Shaded}

\begin{itemize}
\tightlist
\item
  Direct access to objects down the search path can be achieved with the function \texttt{get()}.
  The function \texttt{get()} takes as its first argument the name of an object as a character string. The optional argument \texttt{pos} can be used to specify where on the search list to look for the object. As an illustration explain the outcomes of the following function calls:
\end{itemize}

\begin{Shaded}
\begin{Highlighting}[]
\FunctionTok{get}\NormalTok{ (}\StringTok{"\%o\%"}\NormalTok{) }
\CommentTok{\#\textgreater{} function (X, Y) }
\CommentTok{\#\textgreater{} outer(X, Y)}
\CommentTok{\#\textgreater{} \textless{}bytecode: 0x000002c1b064a990\textgreater{}}
\CommentTok{\#\textgreater{} \textless{}environment: namespace:base\textgreater{}}
\NormalTok{mean }\OtherTok{\textless{}{-}} \FunctionTok{mean}\NormalTok{ (}\FunctionTok{rnorm}\NormalTok{ (}\DecValTok{1000}\NormalTok{))}
\FunctionTok{get}\NormalTok{ (mean)}
\CommentTok{\#\textgreater{} Error in get(mean): invalid first argument}
\FunctionTok{get}\NormalTok{ (}\StringTok{"mean"}\NormalTok{) }
\CommentTok{\#\textgreater{} [1] 0.02333831}
\FunctionTok{get}\NormalTok{ (}\StringTok{"mean"}\NormalTok{, }\AttributeTok{pos =} \DecValTok{1}\NormalTok{) }
\CommentTok{\#\textgreater{} [1] 0.02333831}
\FunctionTok{get}\NormalTok{ (}\StringTok{"mean"}\NormalTok{, }\AttributeTok{pos =} \DecValTok{2}\NormalTok{)}
\CommentTok{\#\textgreater{} function (x, ...) }
\CommentTok{\#\textgreater{} UseMethod("mean")}
\CommentTok{\#\textgreater{} \textless{}bytecode: 0x000002c1a7f17530\textgreater{}}
\CommentTok{\#\textgreater{} \textless{}environment: namespace:base\textgreater{}}
\FunctionTok{rm}\NormalTok{ (mean)}
\end{Highlighting}
\end{Shaded}

\begin{itemize}
\tightlist
\item
  Instead of attaching databases the function \texttt{with()} is often to be preferred. Discuss the usage of \texttt{with()} by referring to the instructions:
\end{itemize}

\begin{Shaded}
\begin{Highlighting}[]
\FunctionTok{with}\NormalTok{ (beaver1, }\FunctionTok{mean}\NormalTok{(time))}
\CommentTok{\#\textgreater{} [1] 1312.018}
\FunctionTok{with}\NormalTok{ (beaver2, }\FunctionTok{mean}\NormalTok{(time))}
\CommentTok{\#\textgreater{} [1] 1446.2}
\end{Highlighting}
\end{Shaded}

\section{The organisation of data (data structures)}\label{the-organisation-of-data-data-structures}

Study the help files of \texttt{list()}, \texttt{matrix()}, \texttt{data.frame()} and \texttt{c()} carefully.

A \emph{{list}} is created with the function \texttt{list()}. A list is the basic means of storing a collection of data objects in R when the modes and/or lengths of the objects are different. List elements are accessed using \texttt{{[}{[}\ {]}{]}} or \texttt{\$} when the objects are named. List objects are named using the construction

\begin{Shaded}
\begin{Highlighting}[]
\NormalTok{my.list }\OtherTok{\textless{}{-}} \FunctionTok{list}\NormalTok{(}\AttributeTok{name1 =} \DecValTok{1}\SpecialCharTok{:}\DecValTok{10}\NormalTok{, }\AttributeTok{name2 =}\NormalTok{ mean)}
\NormalTok{my.list}
\CommentTok{\#\textgreater{} $name1}
\CommentTok{\#\textgreater{}  [1]  1  2  3  4  5  6  7  8  9 10}
\CommentTok{\#\textgreater{} }
\CommentTok{\#\textgreater{} $name2}
\CommentTok{\#\textgreater{} function (x, ...) }
\CommentTok{\#\textgreater{} UseMethod("mean")}
\CommentTok{\#\textgreater{} \textless{}bytecode: 0x000002c1a7f17530\textgreater{}}
\CommentTok{\#\textgreater{} \textless{}environment: namespace:base\textgreater{}}
\end{Highlighting}
\end{Shaded}

and elements are retrieved using the instruction

\begin{Shaded}
\begin{Highlighting}[]
\NormalTok{my.list[[}\DecValTok{2}\NormalTok{]]}
\CommentTok{\#\textgreater{} function (x, ...) }
\CommentTok{\#\textgreater{} UseMethod("mean")}
\CommentTok{\#\textgreater{} \textless{}bytecode: 0x000002c1a7f17530\textgreater{}}
\CommentTok{\#\textgreater{} \textless{}environment: namespace:base\textgreater{}}
\NormalTok{my.list}\SpecialCharTok{$}\NormalTok{name2}
\CommentTok{\#\textgreater{} function (x, ...) }
\CommentTok{\#\textgreater{} UseMethod("mean")}
\CommentTok{\#\textgreater{} \textless{}bytecode: 0x000002c1a7f17530\textgreater{}}
\CommentTok{\#\textgreater{} \textless{}environment: namespace:base\textgreater{}}
\end{Highlighting}
\end{Shaded}

A \emph{{matrix}} in R is a rectangular collection of data, all of the same mode (e.g.~numeric, character/text or logical). It is formed with the construction

\begin{Shaded}
\begin{Highlighting}[]
\NormalTok{my.matrix }\OtherTok{\textless{}{-}} \FunctionTok{matrix}\NormalTok{(}\DecValTok{1}\SpecialCharTok{:}\DecValTok{12}\NormalTok{, }\AttributeTok{ncol=}\DecValTok{3}\NormalTok{, }\AttributeTok{nrow=}\DecValTok{4}\NormalTok{, }\AttributeTok{byrow=}\ConstantTok{FALSE}\NormalTok{)}
\NormalTok{my.matrix}
\CommentTok{\#\textgreater{}      [,1] [,2] [,3]}
\CommentTok{\#\textgreater{} [1,]    1    5    9}
\CommentTok{\#\textgreater{} [2,]    2    6   10}
\CommentTok{\#\textgreater{} [3,]    3    7   11}
\CommentTok{\#\textgreater{} [4,]    4    8   12}
\end{Highlighting}
\end{Shaded}

Matrix elements are accessed using \texttt{my.matrix{[}i,j{]}}. The functions \texttt{nrow()}, \texttt{ncol()}, \texttt{dim()}, \texttt{dimnames()}, \texttt{colnames()} and \texttt{rownames()} are useful when working with matrices.

A \emph{{dataframe}} is also a rectangular collection of data but the columns can be of different modes. It can be regarded as a cross between a list and a matrix. Dataframes are constructed with the function \texttt{data.frame()}.

Study the help files of the above functions.

\section{Time series}\label{time-series}

Study the usage of the function \texttt{ts()}.

\section{\texorpdfstring{The functions \texttt{as.xxx()} and \texttt{is.xxx()}}{The functions as.xxx() and is.xxx()}}\label{the-functions-as.xxx-and-is.xxx}

The function \texttt{as.xxx()} transforms an object as best as possible to a specified type e.g.~\texttt{as.matrix(mydata)} transforms the numerical dataframe to a numerical matrix. \texttt{is.xxx()} tests if the argument is of a certain type e.g.~\texttt{is.matrix(mydata)} evaluates to false if \texttt{mydata} does not satisfy all the conditions of a matrix.

\section{Simple manipulations; numbers and vectors}\label{simple-manipulations-numbers-and-vectors}

\begin{itemize}
\tightlist
\item
  Explain vector calculations and the recycling principle by referring to the example below.
\end{itemize}

\begin{Shaded}
\begin{Highlighting}[]
\FunctionTok{c}\NormalTok{(}\DecValTok{1}\NormalTok{,}\DecValTok{3}\NormalTok{,}\DecValTok{5}\NormalTok{,}\DecValTok{9}\NormalTok{) }\SpecialCharTok{+} \FunctionTok{c}\NormalTok{(}\DecValTok{1}\NormalTok{,}\DecValTok{2}\NormalTok{,}\DecValTok{3}\NormalTok{)}
\CommentTok{\#\textgreater{} Warning in c(1, 3, 5, 9) + c(1, 2, 3): longer object length}
\CommentTok{\#\textgreater{} is not a multiple of shorter object length}
\CommentTok{\#\textgreater{} [1]  2  5  8 10}
\end{Highlighting}
\end{Shaded}

\begin{itemize}
\tightlist
\item
  Logical vectors. Explain the behaviour of the instruction below
\end{itemize}

\begin{Shaded}
\begin{Highlighting}[]
\FunctionTok{sum}\NormalTok{ (}\FunctionTok{c}\NormalTok{ (}\ConstantTok{TRUE}\NormalTok{, }\ConstantTok{FALSE}\NormalTok{, }\ConstantTok{TRUE}\NormalTok{, }\ConstantTok{TRUE}\NormalTok{, }\ConstantTok{FALSE}\NormalTok{))}
\CommentTok{\#\textgreater{} [1] 3}
\end{Highlighting}
\end{Shaded}

\begin{itemize}
\tightlist
\item
  Missing values: \texttt{NA} (indicate a missing value in the data), \texttt{NaN} (not a number)
\end{itemize}

\begin{Shaded}
\begin{Highlighting}[]
\DecValTok{10}\SpecialCharTok{/}\DecValTok{0}
\CommentTok{\#\textgreater{} [1] Inf}
\DecValTok{0}\SpecialCharTok{/}\DecValTok{0}
\CommentTok{\#\textgreater{} [1] NaN}
\end{Highlighting}
\end{Shaded}

\begin{itemize}
\item
  Character vectors: see section \ref{character}
\item
  Subscripting vectors: see section \ref{vectorSubscripting}
\end{itemize}

\section{Objects, their modes and attributes}\label{objects-their-modes-and-attributes}

\begin{itemize}
\tightlist
\item
  Vector elements must be of same mode: logical, numeric, complex, character
\item
  Empty object; once created (e.g.~\texttt{xx\ \textless{}-\ numeric()}) components may be added (e.g.~\texttt{xx{[}5{]}\ \textless{}-\ 22})
\item
  Getting and setting attributes: The functions \texttt{attr()} and \texttt{attributes()}
\item
  Class of an object and the function \texttt{unclass()} for removing class.
\end{itemize}

\section{Representation of objects}\label{representation-of-objects}

We have already seen that a representation of an object can be obtained by calling (entering) its name:

\begin{Shaded}
\begin{Highlighting}[]
\NormalTok{cars}
\CommentTok{\#\textgreater{}    speed dist}
\CommentTok{\#\textgreater{} 1      4    2}
\CommentTok{\#\textgreater{} 2      4   10}
\CommentTok{\#\textgreater{} 3      7    4}
\CommentTok{\#\textgreater{} 4      7   22}
\CommentTok{\#\textgreater{} 5      8   16}
\CommentTok{\#\textgreater{} 6      9   10}
\CommentTok{\#\textgreater{} 7     10   18}
\CommentTok{\#\textgreater{} 8     10   26}
\CommentTok{\#\textgreater{} 9     10   34}
\CommentTok{\#\textgreater{} 10    11   17}
\CommentTok{\#\textgreater{} 11    11   28}
\CommentTok{\#\textgreater{} 12    12   14}
\CommentTok{\#\textgreater{} 13    12   20}
\CommentTok{\#\textgreater{} 14    12   24}
\CommentTok{\#\textgreater{} 15    12   28}
\CommentTok{\#\textgreater{} 16    13   26}
\CommentTok{\#\textgreater{} 17    13   34}
\CommentTok{\#\textgreater{} 18    13   34}
\CommentTok{\#\textgreater{} 19    13   46}
\CommentTok{\#\textgreater{} 20    14   26}
\CommentTok{\#\textgreater{} 21    14   36}
\CommentTok{\#\textgreater{} 22    14   60}
\CommentTok{\#\textgreater{} 23    14   80}
\CommentTok{\#\textgreater{} 24    15   20}
\CommentTok{\#\textgreater{} 25    15   26}
\CommentTok{\#\textgreater{} 26    15   54}
\CommentTok{\#\textgreater{} 27    16   32}
\CommentTok{\#\textgreater{} 28    16   40}
\CommentTok{\#\textgreater{} 29    17   32}
\CommentTok{\#\textgreater{} 30    17   40}
\CommentTok{\#\textgreater{} 31    17   50}
\CommentTok{\#\textgreater{} 32    18   42}
\CommentTok{\#\textgreater{} 33    18   56}
\CommentTok{\#\textgreater{} 34    18   76}
\CommentTok{\#\textgreater{} 35    18   84}
\CommentTok{\#\textgreater{} 36    19   36}
\CommentTok{\#\textgreater{} 37    19   46}
\CommentTok{\#\textgreater{} 38    19   68}
\CommentTok{\#\textgreater{} 39    20   32}
\CommentTok{\#\textgreater{} 40    20   48}
\CommentTok{\#\textgreater{} 41    20   52}
\CommentTok{\#\textgreater{} 42    20   56}
\CommentTok{\#\textgreater{} 43    20   64}
\CommentTok{\#\textgreater{} 44    22   66}
\CommentTok{\#\textgreater{} 45    23   54}
\CommentTok{\#\textgreater{} 46    24   70}
\CommentTok{\#\textgreater{} 47    24   92}
\CommentTok{\#\textgreater{} 48    24   93}
\CommentTok{\#\textgreater{} 49    24  120}
\CommentTok{\#\textgreater{} 50    25   85}
\end{Highlighting}
\end{Shaded}

It is often not convenient to have a full representation returned of an object as above. The functions \texttt{head()}, \texttt{str()} and \texttt{summary()} are available for extracting a partial representation of an object:

\begin{Shaded}
\begin{Highlighting}[]
\FunctionTok{head}\NormalTok{(cars)}
\CommentTok{\#\textgreater{}   speed dist}
\CommentTok{\#\textgreater{} 1     4    2}
\CommentTok{\#\textgreater{} 2     4   10}
\CommentTok{\#\textgreater{} 3     7    4}
\CommentTok{\#\textgreater{} 4     7   22}
\CommentTok{\#\textgreater{} 5     8   16}
\CommentTok{\#\textgreater{} 6     9   10}
\FunctionTok{summary}\NormalTok{(cars)}
\CommentTok{\#\textgreater{}      speed           dist       }
\CommentTok{\#\textgreater{}  Min.   : 4.0   Min.   :  2.00  }
\CommentTok{\#\textgreater{}  1st Qu.:12.0   1st Qu.: 26.00  }
\CommentTok{\#\textgreater{}  Median :15.0   Median : 36.00  }
\CommentTok{\#\textgreater{}  Mean   :15.4   Mean   : 42.98  }
\CommentTok{\#\textgreater{}  3rd Qu.:19.0   3rd Qu.: 56.00  }
\CommentTok{\#\textgreater{}  Max.   :25.0   Max.   :120.00}
\FunctionTok{str}\NormalTok{(cars)}
\CommentTok{\#\textgreater{} \textquotesingle{}data.frame\textquotesingle{}:    50 obs. of  2 variables:}
\CommentTok{\#\textgreater{}  $ speed: num  4 4 7 7 8 9 10 10 10 11 ...}
\CommentTok{\#\textgreater{}  $ dist : num  2 10 4 22 16 10 18 26 34 17 ...}
\end{Highlighting}
\end{Shaded}

There are many more R functions provided for getting information of what an R object represents. Some of these functions like \texttt{mode()}, \texttt{class()}, \texttt{length()}, \texttt{levels()}, \texttt{is.xxx()} and \texttt{as.xxx()} have already been encountered and others will be given in the chapters to come.

\begin{Shaded}
\begin{Highlighting}[]
\FunctionTok{length}\NormalTok{(cars) }
\CommentTok{\#\textgreater{} [1] 2}
\FunctionTok{length}\NormalTok{(}\FunctionTok{as.matrix}\NormalTok{(cars))}
\CommentTok{\#\textgreater{} [1] 100}
\FunctionTok{dim}\NormalTok{(cars)}
\CommentTok{\#\textgreater{} [1] 50  2}
\FunctionTok{is.matrix}\NormalTok{(cars)}
\CommentTok{\#\textgreater{} [1] FALSE}
\FunctionTok{is.data.frame}\NormalTok{(cars)}
\CommentTok{\#\textgreater{} [1] TRUE}
\FunctionTok{is.list}\NormalTok{(cars)}
\CommentTok{\#\textgreater{} [1] TRUE}
\FunctionTok{mode}\NormalTok{(cars)}
\CommentTok{\#\textgreater{} [1] "list"}
\FunctionTok{class}\NormalTok{(cars)}
\CommentTok{\#\textgreater{} [1] "data.frame"}
\FunctionTok{levels}\NormalTok{(cars)}
\CommentTok{\#\textgreater{} NULL}
\end{Highlighting}
\end{Shaded}

\section{Exercise}\label{exercise-3}

\subsection{Exercise}\label{exercise-4}

According to the central limit theorem (CLT) the distribution of the sum (or mean) of independently, identically distributed stochastic variables converges to a normal distribution with an increase in the number variables. The binomial distribution can be expressed as the sum of independently, identically distributed Bernoulli stochastic variables and therefore converges in distribution to the normal distribution. The lognormal distribution in contrast cannot be expressed as a sum.

Make use of the function \texttt{rbinom()} to generate a sample of size 10 from a binomial distribution modelling 20 coin flips with a probability of \(0.4\) for returning ``heads''. Use the function \texttt{hist()} to graph the results. Repeat with sample sizes \(50\), \(100\), \(1000\), \(10000\) and \(100000\).
Repeat the whole study with a success probability of \(0.5\), \(0.3\), \(0.1\) and \(0.05\). Discuss your findings.

Now repeat the same exercise using (a) the lognormal distribution with the function \texttt{rlnorm()} and (b) the uniform distribution over the interval \([10; 25]\) with the function \texttt{runif(min\ =\ 10,\ max\ =\ 25)}. Comment on your findings.

\subsection{Exercise}\label{exercise-5}

Assume that a random sample of size \(n\) is available from a certain distribution. A bootstrap sample is obtained by sampling with replacement a sample of size \(n\) from the given sample. One of the uses of the bootstrap is to obtain an estimate of the standard error of a statistic. For example, a bootstrap estimate of the standard error of \(\bar{X}\) can be obtained as follows:

\begin{itemize}
\tightlist
\item
  Generate independently of each other \(B\) bootstrap samples.
\item
  Calculate the mean of the B bootstrap samples, i.e.~calculate \(\bar{x}_1^*, \bar{x}_2^*, \dots, \bar{x}_B^*\).
\item
  Calculate \(\bar{\bar{x}} = \frac{1}{B} \sum_{i=1}^{B}{\bar{x}_i^*}\).
\item
  Calculate \(\hat{se}(b) = \sqrt{\frac{1}{B-1} \sum_{i=1}^{B}{(\bar{x}_i^*-\bar{\bar{x}})^2}}\).
\end{itemize}

\begin{enumerate}
\def\labelenumi{(\alph{enumi})}
\item
  Generate a random sample of size \(25\) from a \(normal (100; 255)\) distribution.
\item
  Use R to obtain graphical representations and statistics of the characteristics of the sample.
\item
  Program the necessary instructions in R to obtain bootstrap estimates of the standard error of the sample mean as well as the sample median. Use \(50\), \(100\), \(500\) and \(1000\) for \(B\) (the number of bootstrap repetitions). How do your answers compare with what is theoretically expected?
\item
  Program the necessary R instructions to obtain graphical representations of the bootstrap distribution in (c).
\end{enumerate}

\subsection{Exercise}\label{exercise-6}

Generate a random sample of size \(50\) from a multivariate normal distribution with mean vector \((118, 396, 118, 400)\) and a covariance matrix so that the variances of the variables are given by \(778\), \(1810\), \(580\) and \(2535\) respectively. Variables 1 and 2 have a covariance of \(-642.5\) and variables 3 and 4 have a covariance of \(-670\). The other variables are uncorrelated. Store the sample as a matrix object and then program the necessary R instructions to calculate the sample covariance matrix and sample mean vector.

\subsection{Exercise}\label{exercise-7}

Execute the instruction \texttt{set.seed(101023)}.

Next, obtain \(400\) random \(normal (0; 1)\) values and arrange them in a matrix with \(20\) rows and \(20\) columns. Finally, write an R function to calculate and return (i) the sum of all the elements in the matrix, (ii) the eigenvalues of the matrix, (iii) the inverse of the matrix as well as (iv) the rank of the matrix {making use of the eigenvalues}. \emph{Hint}: Read the help of the functions \texttt{eigen()} and \texttt{solve()}.)

\chapter{R operators and functions}\label{operators}

After completing Chapters 1 and 2 it is assumed that the following are now familiar:

\begin{itemize}
\tightlist
\item
  How to communicate with R;
\item
  How to manage workspaces;
\item
  How to perform simple tasks using R.
\end{itemize}

In this chapter we take a closer look at the behaviour of some of the most common

\begin{itemize}
\tightlist
\item
  R operators
\item
  R functions.
\end{itemize}

\section{Arithmetic operators}\label{arithmetic-operators}

\begin{enumerate}
\def\labelenumi{(\alph{enumi})}
\tightlist
\item
  Study the use of the operators in Table \ref{tab:ArithOperators}.
\end{enumerate}

\begin{longtable}[]{@{}
  >{\raggedright\arraybackslash}p{(\linewidth - 6\tabcolsep) * \real{0.1429}}
  >{\raggedright\arraybackslash}p{(\linewidth - 6\tabcolsep) * \real{0.3571}}
  >{\raggedright\arraybackslash}p{(\linewidth - 6\tabcolsep) * \real{0.1429}}
  >{\raggedright\arraybackslash}p{(\linewidth - 6\tabcolsep) * \real{0.3571}}@{}}
\caption{\label{tab:ArithOperators} Arithmetic operators.}\tabularnewline
\toprule\noalign{}
\begin{minipage}[b]{\linewidth}\raggedright
\emph{{Operator}}
\end{minipage} & \begin{minipage}[b]{\linewidth}\raggedright
\emph{{Function}}
\end{minipage} & \begin{minipage}[b]{\linewidth}\raggedright
\emph{{Operator}}
\end{minipage} & \begin{minipage}[b]{\linewidth}\raggedright
\emph{{Function}}
\end{minipage} \\
\midrule\noalign{}
\endfirsthead
\toprule\noalign{}
\begin{minipage}[b]{\linewidth}\raggedright
\emph{{Operator}}
\end{minipage} & \begin{minipage}[b]{\linewidth}\raggedright
\emph{{Function}}
\end{minipage} & \begin{minipage}[b]{\linewidth}\raggedright
\emph{{Operator}}
\end{minipage} & \begin{minipage}[b]{\linewidth}\raggedright
\emph{{Function}}
\end{minipage} \\
\midrule\noalign{}
\endhead
\bottomrule\noalign{}
\endlastfoot
\texttt{+} & Addition & \texttt{\^{}} & Exponentiation \\
\texttt{-} & Subtraction & \texttt{\%/\%} & Integer divide \\
\texttt{*} & Multiplication & \texttt{\%\%} & Modulus \\
\texttt{/} & Division & \texttt{:} & Sequence \\
\texttt{\%*\%} & Matrix multiplication & \texttt{-} & Uniry minus \\
\end{longtable}

Note that the arithmetic operators are also functions. That this is so follows by studying the following examples:

\begin{Shaded}
\begin{Highlighting}[]
\DecValTok{3}\SpecialCharTok{+}\DecValTok{7}
\CommentTok{\#\textgreater{} [1] 10}
\StringTok{"+"}\NormalTok{(}\DecValTok{3}\NormalTok{,}\DecValTok{7}\NormalTok{)}
\CommentTok{\#\textgreater{} [1] 10}
\DecValTok{17} \SpecialCharTok{\%\%} \DecValTok{3}
\CommentTok{\#\textgreater{} [1] 2}
\StringTok{"\%\%"}\NormalTok{(}\DecValTok{17}\NormalTok{,}\DecValTok{3}\NormalTok{)}
\CommentTok{\#\textgreater{} [1] 2}
\end{Highlighting}
\end{Shaded}

\begin{enumerate}
\def\labelenumi{(\alph{enumi})}
\setcounter{enumi}{1}
\tightlist
\item
  Rules for operator expressions with vector arguments.
\end{enumerate}

Study the results of the following R instructions.

\begin{Shaded}
\begin{Highlighting}[]
\NormalTok{cars [,}\DecValTok{2}\NormalTok{] }\SpecialCharTok{*} \DecValTok{12} \SpecialCharTok{*} \FloatTok{25.4} \SpecialCharTok{/} \DecValTok{1000}
\CommentTok{\#\textgreater{}  [1]  0.6096  3.0480  1.2192  6.7056  4.8768  3.0480  5.4864}
\CommentTok{\#\textgreater{}  [8]  7.9248 10.3632  5.1816  8.5344  4.2672  6.0960  7.3152}
\CommentTok{\#\textgreater{} [15]  8.5344  7.9248 10.3632 10.3632 14.0208  7.9248 10.9728}
\CommentTok{\#\textgreater{} [22] 18.2880 24.3840  6.0960  7.9248 16.4592  9.7536 12.1920}
\CommentTok{\#\textgreater{} [29]  9.7536 12.1920 15.2400 12.8016 17.0688 23.1648 25.6032}
\CommentTok{\#\textgreater{} [36] 10.9728 14.0208 20.7264  9.7536 14.6304 15.8496 17.0688}
\CommentTok{\#\textgreater{} [43] 19.5072 20.1168 16.4592 21.3360 28.0416 28.3464 36.5760}
\CommentTok{\#\textgreater{} [50] 25.9080}
\DecValTok{7}\SpecialCharTok{\%/\%}\DecValTok{3}
\CommentTok{\#\textgreater{} [1] 2}
\DecValTok{7}\SpecialCharTok{\%\%}\DecValTok{3}
\CommentTok{\#\textgreater{} [1] 1}
\FunctionTok{matrix}\NormalTok{(}\DecValTok{1}\NormalTok{,}\AttributeTok{nrow=}\DecValTok{4}\NormalTok{,}\AttributeTok{ncol=}\DecValTok{4}\NormalTok{) }\SpecialCharTok{*} \FunctionTok{matrix}\NormalTok{(}\DecValTok{3}\NormalTok{,}\AttributeTok{nrow=}\DecValTok{4}\NormalTok{,}\AttributeTok{ncol=}\DecValTok{4}\NormalTok{)}
\CommentTok{\#\textgreater{}      [,1] [,2] [,3] [,4]}
\CommentTok{\#\textgreater{} [1,]    3    3    3    3}
\CommentTok{\#\textgreater{} [2,]    3    3    3    3}
\CommentTok{\#\textgreater{} [3,]    3    3    3    3}
\CommentTok{\#\textgreater{} [4,]    3    3    3    3}
\FunctionTok{matrix}\NormalTok{(}\DecValTok{1}\NormalTok{,}\AttributeTok{nrow=}\DecValTok{4}\NormalTok{,}\AttributeTok{ncol=}\DecValTok{4}\NormalTok{) }\SpecialCharTok{\%*\%} \FunctionTok{matrix}\NormalTok{(}\DecValTok{3}\NormalTok{,}\AttributeTok{nrow=}\DecValTok{4}\NormalTok{,}\AttributeTok{ncol=}\DecValTok{4}\NormalTok{)}
\CommentTok{\#\textgreater{}      [,1] [,2] [,3] [,4]}
\CommentTok{\#\textgreater{} [1,]   12   12   12   12}
\CommentTok{\#\textgreater{} [2,]   12   12   12   12}
\CommentTok{\#\textgreater{} [3,]   12   12   12   12}
\CommentTok{\#\textgreater{} [4,]   12   12   12   12}
\end{Highlighting}
\end{Shaded}

Explain the following instructions and output from R:

\begin{Shaded}
\begin{Highlighting}[]
\DecValTok{1}\SpecialCharTok{:}\DecValTok{12} \SpecialCharTok{+} \DecValTok{1}\SpecialCharTok{:}\DecValTok{3}
\CommentTok{\#\textgreater{}  [1]  2  4  6  5  7  9  8 10 12 11 13 15}
\DecValTok{1}\SpecialCharTok{:}\DecValTok{10} \SpecialCharTok{+} \DecValTok{1}\SpecialCharTok{:}\DecValTok{2}
\CommentTok{\#\textgreater{}  [1]  2  4  4  6  6  8  8 10 10 12}
\DecValTok{1}\SpecialCharTok{:}\DecValTok{10} \SpecialCharTok{+} \DecValTok{1}\SpecialCharTok{:}\DecValTok{3}
\CommentTok{\#\textgreater{} Warning in 1:10 + 1:3: longer object length is not a}
\CommentTok{\#\textgreater{} multiple of shorter object length}
\CommentTok{\#\textgreater{}  [1]  2  4  6  5  7  9  8 10 12 11}
\end{Highlighting}
\end{Shaded}

In the above examples it is illustrated that R uses \emph{{vectorized arithmetic}} i.e.~it operates on vectors as wholes. Sometimes the \emph{{recycling principle}} is applied with or without a warning. It is a good R programming habit to make use of vectorizing calculations where possible. The effect of the recycling principle must be kept in mind since it might lead to unwanted results.

\begin{enumerate}
\def\labelenumi{(\alph{enumi})}
\setcounter{enumi}{2}
\tightlist
\item
  Missing values, infinity and ``not a number''.
\end{enumerate}

A missing value in R is denoted by NA. The result of a computation involving NAs is always NA e.g.

\begin{Shaded}
\begin{Highlighting}[]
\FunctionTok{mean}\NormalTok{(}\FunctionTok{c}\NormalTok{(}\DecValTok{1}\NormalTok{,}\DecValTok{3}\NormalTok{,}\ConstantTok{NA}\NormalTok{,}\DecValTok{12}\NormalTok{,}\DecValTok{5}\NormalTok{))}
\CommentTok{\#\textgreater{} [1] NA}
\DecValTok{0}\SpecialCharTok{/}\DecValTok{0}
\CommentTok{\#\textgreater{} [1] NaN}
\DecValTok{5}\SpecialCharTok{/}\DecValTok{0}
\CommentTok{\#\textgreater{} [1] Inf}
\SpecialCharTok{{-}}\DecValTok{5}\SpecialCharTok{/}\DecValTok{0}
\CommentTok{\#\textgreater{} [1] {-}Inf}
\DecValTok{5}\SpecialCharTok{/}\NormalTok{(}\SpecialCharTok{{-}}\DecValTok{0}\NormalTok{)}
\CommentTok{\#\textgreater{} [1] {-}Inf}
\end{Highlighting}
\end{Shaded}

The result of a computation that cannot be represented as a number e.g.~\texttt{0/0} is denoted by \texttt{NaN}.
Note: some computational results are differently reported by R as the corresponding algebraic equivalents, \texttt{5/0} in R is given by \texttt{Inf} while algebraically it is undefined.

\begin{enumerate}
\def\labelenumi{(\alph{enumi})}
\setcounter{enumi}{3}
\tightlist
\item
  Scientific notation
\end{enumerate}

R uses decimal notation as well as scientific notation for arithmetic calculations. Scientific notation is not to be confused with \(exp()\).

\begin{Shaded}
\begin{Highlighting}[]
\DecValTok{60000000}
\CommentTok{\#\textgreater{} [1] 6e+07}
\DecValTok{1}\SpecialCharTok{/}\DecValTok{6000000}
\CommentTok{\#\textgreater{} [1] 1.666667e{-}07}
\FunctionTok{exp}\NormalTok{(}\DecValTok{15}\NormalTok{)}
\CommentTok{\#\textgreater{} [1] 3269017}
\FunctionTok{exp}\NormalTok{(}\SpecialCharTok{{-}}\DecValTok{15}\NormalTok{)}
\CommentTok{\#\textgreater{} [1] 3.059023e{-}07}
\end{Highlighting}
\end{Shaded}

\begin{enumerate}
\def\labelenumi{(\alph{enumi})}
\setcounter{enumi}{4}
\tightlist
\item
  How are numbers represented in a computer's memory? What are the implications of this?
\end{enumerate}

Computers use ON/OFF (or 1/0) switches for encoding information. A single switch is called a \emph{{bit}} and a group of eight bits is called a \emph{{byte}}. A single integer is represented exactly in a computer by a fixed number of bytes i.e.~32 or 64 bits. There are several schemes according to which integers are represented by bits in a computer. This representation in a computer takes place at a level where R has no control over it but R stores information about the computing environment in an object \texttt{.Machine}. The element \texttt{.Machine\$integer.max} returns the largest integer that can be represented in the computer on which R is running e.g.

\begin{Shaded}
\begin{Highlighting}[]
\NormalTok{.Machine}\SpecialCharTok{$}\NormalTok{integer.max}
\CommentTok{\#\textgreater{} [1] 2147483647}
\end{Highlighting}
\end{Shaded}

Although the above method of representing integers by strings of bits provides a very efficient way of storing integers in a computer R usually treats integers similar to real numbers by using \emph{{floating point representation}}. In binary floating point notation a number x is written as a sequence of zeros and ones (the \emph{{mantissa}}) times two with an exponent say \(m\): \(x=b_0 b_1 b_2…×2^m\) where \(b_0=1\) except when \(x=0\).

In practice there is only a limited number of \(b\)'s available and the exponent is also limited therefore, in general, not all real numbers can be represented exactly in a computer -- they can at most be approximated. The smallest number \(x\) such that \(1 + x\) can be distinguished from \(1\) in a computer is called \emph{{machine epsilon}}. In R this can be obtained from \texttt{.Machine\$double.eps} e.g.

\begin{Shaded}
\begin{Highlighting}[]
\NormalTok{.Machine}\SpecialCharTok{$}\NormalTok{double.eps}
\CommentTok{\#\textgreater{} [1] 2.220446e{-}16}
\end{Highlighting}
\end{Shaded}

Although floating point representation allows computation with very small (in magnitude) and very large numbers the above limitations can lead to \emph{{underflow}} or \emph{{overflow}} which can have disastrous consequences in practice. Writing good code in R must take the above seriously into account.

\section{Logical operators}\label{logical-operators}

Logical operators result in \texttt{TRUE}, \texttt{FALSE} or \texttt{NA}. Study the use of the logical operators in Table \ref{tab:LogicOperators}. \emph{{Warning}}: While it is perfectly legitimate to write

\begin{Shaded}
\begin{Highlighting}[]
\NormalTok{x[x }\SpecialCharTok{==} \SpecialCharTok{{-}}\DecValTok{1}\NormalTok{] }\OtherTok{\textless{}{-}} \DecValTok{0}
\NormalTok{x[x }\SpecialCharTok{==} \DecValTok{1}\NormalTok{] }\OtherTok{\textless{}{-}} \DecValTok{0} 
\end{Highlighting}
\end{Shaded}

it is incorrect to specify

\begin{Shaded}
\begin{Highlighting}[]
\NormalTok{x[x }\SpecialCharTok{==} \ConstantTok{NA}\NormalTok{] }\OtherTok{\textless{}{-}} \DecValTok{0}
\NormalTok{x[x }\OtherTok{=} \ErrorTok{=} \ConstantTok{NaN}\NormalTok{] }\OtherTok{\textless{}{-}} \DecValTok{0} 
\end{Highlighting}
\end{Shaded}

The correct code in the latter case is

\begin{Shaded}
\begin{Highlighting}[]
\NormalTok{x[}\FunctionTok{is.na}\NormalTok{(x)] }\OtherTok{\textless{}{-}} \DecValTok{0}
\NormalTok{x[}\FunctionTok{is.nan}\NormalTok{(x)] }\OtherTok{\textless{}{-}} \DecValTok{0}
\end{Highlighting}
\end{Shaded}

What are the consequences of the above code? Also take note of the functions \texttt{any()} and \texttt{all()}. These two functions are useful when combining logical objects. Give the necessary instructions to carry out the following tasks:

\begin{enumerate}
\def\labelenumi{(\alph{enumi})}
\tightlist
\item
  Check if any of the states in the \texttt{state.x77} data set have populations with an illiteracy rate that is not larger than \(1.6\) and a Murder rate of more than \(10.0\).
\item
  Check if there is at least one state with income greater than \(\$5000\) and life expectancy less than \(70.0\) years.
\item
  Check if all states with an income of more than \(\$5000\) has an illiteracy of below \(2.0\).
\end{enumerate}

What is meant by a control logical operator?

\begin{longtable}[]{@{}
  >{\raggedright\arraybackslash}p{(\linewidth - 2\tabcolsep) * \real{0.2857}}
  >{\raggedright\arraybackslash}p{(\linewidth - 2\tabcolsep) * \real{0.7143}}@{}}
\caption{\label{tab:LogicOperators} Logical operators.}\tabularnewline
\toprule\noalign{}
\begin{minipage}[b]{\linewidth}\raggedright
\emph{{Operator}}
\end{minipage} & \begin{minipage}[b]{\linewidth}\raggedright
\emph{{Function}}
\end{minipage} \\
\midrule\noalign{}
\endfirsthead
\toprule\noalign{}
\begin{minipage}[b]{\linewidth}\raggedright
\emph{{Operator}}
\end{minipage} & \begin{minipage}[b]{\linewidth}\raggedright
\emph{{Function}}
\end{minipage} \\
\midrule\noalign{}
\endhead
\bottomrule\noalign{}
\endlastfoot
\texttt{\textgreater{}} & Greater than \\
\texttt{\textless{}} & Less than \\
\texttt{\textless{}=} & Less than or equal to \\
\texttt{\textgreater{}=} & Greater than or equal to \\
\texttt{==} & Equality \\
\texttt{\&} & Elementwise and \\
\texttt{\textbar{}} & Elementwise or \\
\texttt{\&\&} & Control and \\
\texttt{\textbar{}\textbar{}} & Control or \\
\texttt{!} & Unary not \\
\texttt{!=} & Not equal to \\
\end{longtable}

\begin{enumerate}
\def\labelenumi{(\alph{enumi})}
\setcounter{enumi}{3}
\tightlist
\item
  Carry out the instructions:
\end{enumerate}

\begin{Shaded}
\begin{Highlighting}[]
\NormalTok{mata }\OtherTok{\textless{}{-}} \FunctionTok{matrix}\NormalTok{(}\DecValTok{1}\SpecialCharTok{:}\DecValTok{4}\NormalTok{, }\AttributeTok{ncol =} \DecValTok{2}\NormalTok{)}
\NormalTok{matb }\OtherTok{\textless{}{-}} \FunctionTok{matrix}\NormalTok{(}\FunctionTok{c}\NormalTok{(}\DecValTok{10}\NormalTok{, }\DecValTok{20}\NormalTok{, }\DecValTok{30}\NormalTok{, }\DecValTok{40}\NormalTok{), }\AttributeTok{ncol =} \DecValTok{2}\NormalTok{)}
\NormalTok{mata}
\CommentTok{\#\textgreater{}      [,1] [,2]}
\CommentTok{\#\textgreater{} [1,]    1    3}
\CommentTok{\#\textgreater{} [2,]    2    4}
\NormalTok{matb}
\CommentTok{\#\textgreater{}      [,1] [,2]}
\CommentTok{\#\textgreater{} [1,]   10   30}
\CommentTok{\#\textgreater{} [2,]   20   40}
\NormalTok{mata}\SpecialCharTok{\textgreater{}}\DecValTok{1} \SpecialCharTok{\&}\NormalTok{ matb}\SpecialCharTok{\textgreater{}}\DecValTok{1}
\CommentTok{\#\textgreater{}       [,1] [,2]}
\CommentTok{\#\textgreater{} [1,] FALSE TRUE}
\CommentTok{\#\textgreater{} [2,]  TRUE TRUE}
\NormalTok{mata}\SpecialCharTok{\textgreater{}}\DecValTok{1} \SpecialCharTok{|}\NormalTok{ matb}\SpecialCharTok{\textgreater{}}\DecValTok{1}
\CommentTok{\#\textgreater{}      [,1] [,2]}
\CommentTok{\#\textgreater{} [1,] TRUE TRUE}
\CommentTok{\#\textgreater{} [2,] TRUE TRUE}
\NormalTok{mata}\SpecialCharTok{\textgreater{}}\DecValTok{1} \SpecialCharTok{\&\&}\NormalTok{ matb}\SpecialCharTok{\textgreater{}}\DecValTok{1}
\CommentTok{\#\textgreater{} Error in mata \textgreater{} 1 \&\& matb \textgreater{} 1: \textquotesingle{}length = 4\textquotesingle{} in coercion to \textquotesingle{}logical(1)\textquotesingle{}}
\NormalTok{mata}\SpecialCharTok{\textgreater{}}\DecValTok{1} \SpecialCharTok{||}\NormalTok{ matb}\SpecialCharTok{\textgreater{}}\DecValTok{1}
\CommentTok{\#\textgreater{} Error in mata \textgreater{} 1 || matb \textgreater{} 1: \textquotesingle{}length = 4\textquotesingle{} in coercion to \textquotesingle{}logical(1)\textquotesingle{}}
\end{Highlighting}
\end{Shaded}

Comment on the above.

\begin{enumerate}
\def\labelenumi{(\alph{enumi})}
\setcounter{enumi}{4}
\tightlist
\item
  What is the result of \texttt{sum(c(TRUE,\ !FALSE,\ FALSE,\ TRUE,\ TRUE))}?
\item
  What is the result of \texttt{sum(c(TRUE,\ !FALSE,\ FALSE,\ NA,\ TRUE))} ?
\end{enumerate}

Explain

\section{\texorpdfstring{The operators \texttt{\textless{}-}, \texttt{\textless{}\textless{}-} and \texttt{\textasciitilde{}}}{The operators \textless-, \textless\textless- and \textasciitilde{}}}\label{the-operators-----and}

Before considering the use of these operators answer the following:

\begin{enumerate}
\def\labelenumi{(\alph{enumi})}
\item
  What will happen to an object \texttt{aa} in the working directory if within a function the following assignment is made \texttt{aa\ \textless{}-\ 20}?
\item
  Now, study the help file of \texttt{\textless{}\textless{}-} and then answer (a) if the operator \texttt{\textless{}-} has been replaced with the operator \texttt{\textless{}\textless{}-}. \emph{{Warning}}: use \texttt{\textless{}\textless{}-} very carefully.
\item
  The tilde operator is used in modelling functions, e.g.~\texttt{lm\ (length\ \textasciitilde{}\ age)}.
\end{enumerate}

\section{Operator precedence}\label{operator-precedence}

Study the precedence rules as summarized in Table 3.4.1. The rules followed are shown in Table \ref{tab:Precedence} from top to bottom and left to right. Note the use of

\begin{itemize}
\tightlist
\item
  parentheses \texttt{(\ )} for function arguments and changing precedence,
\item
  braces \texttt{\{\ \}} for demarcating blocks of instructions
\item
  and brackets \texttt{{[}\ {]}} for subscripting.
\end{itemize}

The correct way of extracting the fifth element of a sequence like 1:20 is

\begin{Shaded}
\begin{Highlighting}[]
\NormalTok{(}\DecValTok{1}\SpecialCharTok{:}\DecValTok{20}\NormalTok{)[}\DecValTok{5}\NormalTok{]}
\CommentTok{\#\textgreater{} [1] 5}
\end{Highlighting}
\end{Shaded}

\begin{longtable}[]{@{}
  >{\raggedright\arraybackslash}p{(\linewidth - 2\tabcolsep) * \real{0.2857}}
  >{\raggedright\arraybackslash}p{(\linewidth - 2\tabcolsep) * \real{0.7143}}@{}}
\caption{\label{tab:Precedence} Precedence rules.}\tabularnewline
\toprule\noalign{}
\begin{minipage}[b]{\linewidth}\raggedright
\emph{{Operator}}
\end{minipage} & \begin{minipage}[b]{\linewidth}\raggedright
\emph{{What it does}}
\end{minipage} \\
\midrule\noalign{}
\endfirsthead
\toprule\noalign{}
\begin{minipage}[b]{\linewidth}\raggedright
\emph{{Operator}}
\end{minipage} & \begin{minipage}[b]{\linewidth}\raggedright
\emph{{What it does}}
\end{minipage} \\
\midrule\noalign{}
\endhead
\bottomrule\noalign{}
\endlastfoot
\texttt{\$} & List and dataframe subscripting \\
\texttt{{[}{]}}, \texttt{{[}{[}{]}{]}} & Vector and matrix subscripting; list subscripting \\
\texttt{\^{}} & Exponentiation \\
\texttt{\%*\%}, \texttt{\%/\%}, \texttt{\%\%} & Matrix multiplication; integer divide; modulus \\
\texttt{*}, \texttt{/} & Multiplication and division \\
\texttt{+}, \texttt{-} & Addition and subtraction \\
\texttt{\textless{}}, \texttt{\textgreater{}}, \texttt{\textless{}=}, \texttt{\textgreater{}=}, \texttt{==}, \texttt{!=} & Logical comparisons \\
\texttt{!} & Unary not \\
\texttt{\&}, \texttt{\textbar{}}, \texttt{\&\&}, \texttt{\textbar{}\textbar{}} & Logical and; logical or; control and; control or \\
\texttt{\textless{}-}, \texttt{\textless{}\textless{}-} & Assignment \\
\end{longtable}

Explain the result of the following R instructions:

\begin{Shaded}
\begin{Highlighting}[]
\DecValTok{20} \SpecialCharTok{/} \DecValTok{4} \SpecialCharTok{*} \DecValTok{12} \SpecialCharTok{\^{}}\DecValTok{2} \SpecialCharTok{{-}} \DecValTok{6} \SpecialCharTok{+} \DecValTok{1}
\CommentTok{\#\textgreater{} [1] 715}
\NormalTok{(}\DecValTok{20} \SpecialCharTok{/} \DecValTok{4}\NormalTok{) }\SpecialCharTok{*}\NormalTok{ (}\DecValTok{12} \SpecialCharTok{\^{}}\DecValTok{2}\NormalTok{) }\SpecialCharTok{+}\NormalTok{ (}\SpecialCharTok{{-}}\DecValTok{6} \SpecialCharTok{+} \DecValTok{14}\NormalTok{)}
\CommentTok{\#\textgreater{} [1] 728}
\DecValTok{20} \SpecialCharTok{/} \DecValTok{4} \SpecialCharTok{*} \DecValTok{12} \SpecialCharTok{\^{}}\NormalTok{(}\DecValTok{2} \SpecialCharTok{{-}} \DecValTok{6} \SpecialCharTok{+} \DecValTok{14}\NormalTok{)}
\CommentTok{\#\textgreater{} [1] 309586821120}
\DecValTok{20} \SpecialCharTok{/} \DecValTok{4} \SpecialCharTok{*}\NormalTok{ (}\DecValTok{12} \SpecialCharTok{\^{}}\DecValTok{2} \SpecialCharTok{{-}} \DecValTok{6} \SpecialCharTok{+} \DecValTok{14}\NormalTok{)}
\CommentTok{\#\textgreater{} [1] 760}
\end{Highlighting}
\end{Shaded}

\section{Some mathematical functions}\label{some-mathematical-functions}

\subsection{General mathematical functions}\label{general-mathematical-functions}

\texttt{abs()}, \texttt{exp()}, \texttt{log(x,\ base\ =\ exp(1))}, \texttt{log10()}, \texttt{gamma()}, \texttt{sign()}, \texttt{sqrt()}

\subsection{Trigonometric functions}\label{trigonometric-functions}

See Table \ref{tab:TrigFunc}.

\begin{longtable}[]{@{}
  >{\raggedright\arraybackslash}p{(\linewidth - 6\tabcolsep) * \real{0.1429}}
  >{\raggedright\arraybackslash}p{(\linewidth - 6\tabcolsep) * \real{0.3571}}
  >{\raggedright\arraybackslash}p{(\linewidth - 6\tabcolsep) * \real{0.1429}}
  >{\raggedright\arraybackslash}p{(\linewidth - 6\tabcolsep) * \real{0.3571}}@{}}
\caption{\label{tab:TrigFunc} Trigonometric functions.}\tabularnewline
\toprule\noalign{}
\begin{minipage}[b]{\linewidth}\raggedright
\emph{{Operator}}
\end{minipage} & \begin{minipage}[b]{\linewidth}\raggedright
\emph{{Function}}
\end{minipage} & \begin{minipage}[b]{\linewidth}\raggedright
\end{minipage} & \begin{minipage}[b]{\linewidth}\raggedright
\emph{{Operator}}
\end{minipage} \\
\midrule\noalign{}
\endfirsthead
\toprule\noalign{}
\begin{minipage}[b]{\linewidth}\raggedright
\emph{{Operator}}
\end{minipage} & \begin{minipage}[b]{\linewidth}\raggedright
\emph{{Function}}
\end{minipage} & \begin{minipage}[b]{\linewidth}\raggedright
\end{minipage} & \begin{minipage}[b]{\linewidth}\raggedright
\emph{{Operator}}
\end{minipage} \\
\midrule\noalign{}
\endhead
\bottomrule\noalign{}
\endlastfoot
\texttt{cos()} & cosine & \texttt{acos()} & arc cosine \\
\texttt{sin()} & sine & \texttt{asin()} & arc sine \\
\texttt{tan()} & tangent & \texttt{atan()} & arc tangent \\
\texttt{cosh()} & hyperbolic cosine & \texttt{acosh()} & arc hyperbolic cosine \\
\texttt{sinh()} & hyperbolic sine & \texttt{asinh()} & arc hyperbolic sine \\
\texttt{tanh()} & hyperbolic tangent & \texttt{atanh()} & arc hyperbolic tangent \\
\end{longtable}

\subsection{Complex numbers}\label{complex-numbers}

\texttt{Arg()}, \texttt{Conj()}, \texttt{Mod()}, \texttt{Re()}, \texttt{Im()}

\subsection{Functions for rounding and truncating}\label{functions-for-rounding-and-truncating}

\texttt{round()}, \texttt{ceiling()}, \texttt{floor()}, \texttt{trunc()}

Study the help files of the above functions. Check all arguments.

\subsection{Functions for matrices}\label{functions-for-matrices}

Study Table \ref{tab:MatrixFunc} in detail.

Two other functions that play an important role in matrix calculations are the functions \texttt{rbind()} and \texttt{cbind()} for concatenating matrices row-wise or column-wise. Also revise the functions \texttt{matrix()}, \texttt{dim()}, \texttt{dimnames()}, \texttt{colnames()}, \texttt{rownames()} as well as \texttt{scan()} and \texttt{read.table()}.

\begin{longtable}[]{@{}
  >{\raggedright\arraybackslash}p{(\linewidth - 2\tabcolsep) * \real{0.2857}}
  >{\raggedright\arraybackslash}p{(\linewidth - 2\tabcolsep) * \real{0.7143}}@{}}
\caption{\label{tab:MatrixFunc} Functions for matrices.}\tabularnewline
\toprule\noalign{}
\begin{minipage}[b]{\linewidth}\raggedright
\emph{{Function}}
\end{minipage} & \begin{minipage}[b]{\linewidth}\raggedright
\emph{{What it does}}
\end{minipage} \\
\midrule\noalign{}
\endfirsthead
\toprule\noalign{}
\begin{minipage}[b]{\linewidth}\raggedright
\emph{{Function}}
\end{minipage} & \begin{minipage}[b]{\linewidth}\raggedright
\emph{{What it does}}
\end{minipage} \\
\midrule\noalign{}
\endhead
\bottomrule\noalign{}
\endlastfoot
\texttt{chol()} & Cholesky decomposition \\
\texttt{crossprod()} & Matrix crossproduct \\
\texttt{diag()} & Create identity matrix, diagonal matrix or extract diagonal elements depending on its argument \\
\texttt{eigen()} & Finding eigenvectors and eigenvalues \\
\texttt{kronecker()} & Computing the kronecker product of two matrices \\
\texttt{outer()} & Outer product of two vectors \\
\texttt{scale()} & Centring and scaling a data matrix \\
\texttt{solve()} & Finding the inverse of a nonsingular matrix \\
\texttt{svd()} & Singular value decomposition of a rectangular matrix \\
\texttt{qr()} & QR orthogonalization \\
\texttt{t()} & Transpose of a matrix \\
\end{longtable}

\begin{enumerate}
\def\labelenumi{(\alph{enumi})}
\item
  The function \texttt{chol()} performs a Cholesky decomposition of the square, symmetric, positive definite matrix \(\mathbf{A}=\mathbf{U}'\mathbf{U}\) where \(\mathbf{U}\) is an upper triangular matrix.
\item
  The function \texttt{crossprod\ (A,\ B)} returns the matrix \(\mathbf{A'B}\).
\item
  The function \texttt{diag(arg)} performs various actions depending on its argument: if \texttt{arg} is a positive integer \texttt{diag(arg)} returns an identity matrix of the given size; if \texttt{arg} is a vector \texttt{diag(arg)} returns a diagonal matrix with diagonal elements the respective elements of the given vector; if \texttt{arg} is a matrix then \texttt{diag(arg)} returns a vector containing the diagonal elements of the given matrix.
\item
  What is the difference between \texttt{diag(A)} and \texttt{diag(diag(A))} where \texttt{A} is a square matrix?
\item
  The function \texttt{eigen()} operates on a square matrix and returns a list with named elements \texttt{values} and \texttt{vectors} containing respectively, the eigenvalues and eigenvectors. Study the help file of \texttt{eigen()} carefully.
\item
  The function \texttt{kronecker()} returns the Kronecker product \(\mathbf{A} \otimes \mathbf{B}\) of matrices \(\mathbf{A}\) and \(\mathbf{B}\).
\item
  The function \texttt{outer\ (x,\ y,\ f)} operates on two vectors \(x:n\times 1\) and \(y:p\times 1\) to return a matrix of size \(n \times p\) with \(ij\)th element the result of applying the function \texttt{f} on \texttt{x{[}i{]}} and \texttt{y{[}j{]}}. The default for \texttt{f} is \texttt{*}.
\item
  The function \texttt{scale()} has three arguments: a matrix as first argument; a second argument \texttt{center} and a third argument \texttt{scale}. If \texttt{center\ =\ FALSE}, no centring of the columns of the matrix argument is performed, if set to \texttt{TRUE} (the default), the mean value of each column is subtracted from the respective columns, if given a vector of values these values are subtracted from the respective columns. If \texttt{scale\ =\ FALSE}, no scaling of the columns of the matrix argument is performed, if set to \texttt{TRUE} (the default) each column is divided by its standard deviation, if given a vector of values then each column is divided by the corresponding value.
\item
  The function \texttt{solve\ (A,\ b)} is used for solving the equation \(\mathbf{Ax=b}\) for \(\mathbf{x}\), where \(\mathbf{b}\) can be either a vector or a matrix with \(\mathbf{A}\) being a square matrix. If argument \texttt{b} is missing it is taken to be the identity matrix so that the inverse of argument \texttt{A} is returned.
\item
  The function \texttt{svd()} returns the singular value decomposition of its matrix argument \(\mathbf{A=UDV}'\). It returns a list with three components: \texttt{u} the orthogonal or orthonormal matrix \(\mathbf{U}\); \texttt{d} the vector containing the ordered singular values of the rectangular matrix \(\mathbf{A}\); \texttt{v} the orthogonal or orthonormal matrix \(\mathbf{V}\).
\item
  The function \texttt{qr()} performs a QR decomposition of any arbitrary matrix \(\mathbf{M=QR}\) with \(\mathbf{Q}\) and orthogonal matrix and \(\mathbf{R}\) an upper triangular matrix. Study the help file of \texttt{qr()} for full details and usages of the function. Note that the matrices \(\mathbf{Q}\) and \(\mathbf{R}\) can be obtained directly by calling \texttt{qr.Q(qr())} and \texttt{qr.R(qr())}, respectively.
\end{enumerate}

\begin{enumerate}
\def\labelenumi{(\alph{enumi})}
\setcounter{enumi}{11}
\tightlist
\item
  What is the meaning of each of the following instructions?
\end{enumerate}

\texttt{rbind(a,b)}; \texttt{rbind(1,x)}; \texttt{rbind(a\ =\ 1:5,b\ =\ 10:14,c=20:24)}; \texttt{cbind(\ a=\ 1:5,\ b=10:14,\ c=20:24)}

\begin{enumerate}
\def\labelenumi{(\alph{enumi})}
\setcounter{enumi}{12}
\item
  Write a function to calculate the determinant of a square matrix. Name this function \texttt{det.own()} in order to distinguish it from the built in R function \texttt{det()}.
\item
  When the user is satisfied with a function, it is often necessary to have it available for all R projects. It is useful to assign all such functions to the same data base or folder. Use the function \texttt{assign\ (x,\ object,\ pos\ =\ ,\ envir\ =\ )} to store the function \texttt{det.own()} in your own R functions folder. The argument \texttt{x} in \texttt{assign()} is a character string for assigning a name to the object. The function \texttt{remove\ (list\ of\ objects\ names,\ pos\ =\ ,\ envir\ =\ )} can be used to remove objects from your own or any other database. \emph{Hint}: First create a file and then use \texttt{attach()} to add it to the R search path.
\end{enumerate}

\begin{Shaded}
\begin{Highlighting}[]
\FunctionTok{save}\NormalTok{(}\AttributeTok{file=} \StringTok{" C:}\SpecialCharTok{\textbackslash{}\textbackslash{}}\StringTok{MyFunctions"}\NormalTok{).  }
\end{Highlighting}
\end{Shaded}

Study how \texttt{save()} works.

\begin{Shaded}
\begin{Highlighting}[]
\FunctionTok{attach}\NormalTok{(}\StringTok{"C:}\SpecialCharTok{\textbackslash{}\textbackslash{}}\StringTok{MyFunctions"}\NormalTok{, }\AttributeTok{pos=}\DecValTok{2}\NormalTok{). }
\end{Highlighting}
\end{Shaded}

Study how \texttt{attach()} works.

\begin{Shaded}
\begin{Highlighting}[]
\FunctionTok{assign}\NormalTok{(}\StringTok{"det.own"}\NormalTok{, det.own, }\AttributeTok{pos=}\DecValTok{2}\NormalTok{). }
\end{Highlighting}
\end{Shaded}

Study how \texttt{assign()} works.

\begin{Shaded}
\begin{Highlighting}[]
\FunctionTok{save}\NormalTok{(}\AttributeTok{list=}\FunctionTok{objects}\NormalTok{(}\DecValTok{2}\NormalTok{), }\AttributeTok{file =} \StringTok{"C:}\SpecialCharTok{\textbackslash{}\textbackslash{}}\StringTok{MyFunctions"}\NormalTok{)}
\end{Highlighting}
\end{Shaded}

Explain the use of the argument \texttt{list=objects(2)}. To summarize: The construction \texttt{NAME\ \textless{}-\ object} is a simple way to assign an object to a name. This form of assignment always takes place in the global environment (the workspace). Assignment can also be performed using the functions \texttt{save()} and \texttt{assign()} as illustrated above. The latter form of assignment is more complicated but the assignment is not restricted to the global environment.

\begin{enumerate}
\def\labelenumi{(\alph{enumi})}
\setcounter{enumi}{14}
\item
  The result of the function \texttt{gamma(x)} is \((x-1)!\) if \(x\) is a non-negative whole number. Now write a function \texttt{fact()} to calculate \(x!\). This function must make provision for \(0!\) as well as for a negative number or a fraction that is read in by mistake. \emph{Hint}: First study the usage of the if statement by requesting help \texttt{?Control}, recall Table \ref{tab:HelpQueries}. Store this function in your folder of R functions. How will you go about to make \texttt{fact()} and \texttt{det.own()} available for any R project?
\item
  The function \texttt{lgamma(x)} returns the logarithms of \(\Gamma(x)\). Write a function to calculate the value of \(f(n) = \frac{\Gamma(\frac{n-1}{2})}{\Gamma(\frac{1}{2})\Gamma(\frac{n-2}{2})}\). Calculate the value of \(f(n)\) for \(n = -10, 10, 100, 500, 1000\).
\end{enumerate}

\subsection{Sorting functions}\label{sorting-functions}

Note the use of the functions \texttt{sort()}, \texttt{order()} and \texttt{rank()}. First construct \texttt{MatX} using the functions \texttt{scan()} and \texttt{matrix()}. Explain in detail what \texttt{order()} does by sorting all the columns of \texttt{MatX} according to the values in the first column of the matrix.

\[
MatX = \begin{bmatrix}
         4 & 80 & 12\\
         5 & 70 & 70\\
         6 & 30 & 19\\
         2 & 40 & 80\\
         4 & 90 & 40\\
         1 & 60 & 50\\
         7 & 10 & 20\\
         3 & 30 & 200
       \end{bmatrix}
\]

\subsection{Some functions for data manipulation}\label{some-functions-for-data-manipulation}

Study the functions in Table \ref{tab:DataManipulation}.

\begin{longtable}[]{@{}
  >{\raggedright\arraybackslash}p{(\linewidth - 2\tabcolsep) * \real{0.2857}}
  >{\raggedright\arraybackslash}p{(\linewidth - 2\tabcolsep) * \real{0.7143}}@{}}
\caption{\label{tab:DataManipulation} Functions for data manipulation.}\tabularnewline
\toprule\noalign{}
\begin{minipage}[b]{\linewidth}\raggedright
\emph{{Function}}
\end{minipage} & \begin{minipage}[b]{\linewidth}\raggedright
\emph{{What it does}}
\end{minipage} \\
\midrule\noalign{}
\endfirsthead
\toprule\noalign{}
\begin{minipage}[b]{\linewidth}\raggedright
\emph{{Function}}
\end{minipage} & \begin{minipage}[b]{\linewidth}\raggedright
\emph{{What it does}}
\end{minipage} \\
\midrule\noalign{}
\endhead
\bottomrule\noalign{}
\endlastfoot
\texttt{append()} & Combine vectors; more flexibility than \texttt{c()} \\
\texttt{c()} & Create vectors \\
\texttt{duplicated()} & Extract duplicated values \\
\texttt{match()} & Match values in pairs of vectors \\
\texttt{pmatch()} & Partial matching \\
\texttt{replace()} & Replace specified values in vectors \\
\texttt{unique()} & Extract unique values \\
\end{longtable}

\begin{enumerate}
\def\labelenumi{(\alph{enumi})}
\item
  Insert the vector (101, 102, 103, 104, 105) into the vector (10, 11, 12, 13, 14, 15, 16, 17, 18, 19, 20) after its fifth element by utilising the argument \texttt{after} of the function \texttt{append()}.
\item
  The function \texttt{replace()} requires three arguments \texttt{x}, \texttt{list} and \texttt{vals}. The values in \texttt{x} with indices given in \texttt{list} is replaced by the successive values in \texttt{vals} making use of the recycling principle if needed. Explain this by replacing in the vector (10, 2, 7, 20, 5, 8, 9, 20, 9, 1,1 15), the values 10, 20 and 15 with zeros.
\item
  Find the unique values in the vector (10, 2, 7, 20, 5, 8, 9, 20, 9, 1, 15).
\item
  Find the duplicated values in the vector (10, 2, 7, 20, 5, 8, 9, 20, 9, 1, 15, 20, 20, 15).
\item
  Explain the usage of \texttt{match()} by considering the difference between
\end{enumerate}

\begin{Shaded}
\begin{Highlighting}[]
\FunctionTok{match}\NormalTok{ (}\FunctionTok{c}\NormalTok{(}\DecValTok{10}\NormalTok{,}\DecValTok{2}\NormalTok{,}\DecValTok{7}\NormalTok{,}\DecValTok{20}\NormalTok{,}\DecValTok{5}\NormalTok{,}\DecValTok{8}\NormalTok{,}\DecValTok{9}\NormalTok{,}\DecValTok{20}\NormalTok{,}\DecValTok{9}\NormalTok{,}\DecValTok{1}\NormalTok{,}\DecValTok{15}\NormalTok{), }\FunctionTok{c}\NormalTok{(}\DecValTok{10}\NormalTok{,}\DecValTok{20}\NormalTok{,}\DecValTok{15}\NormalTok{))}
\CommentTok{\#\textgreater{}  [1]  1 NA NA  2 NA NA NA  2 NA NA  3}
\FunctionTok{match}\NormalTok{ (}\FunctionTok{c}\NormalTok{(}\DecValTok{10}\NormalTok{,}\DecValTok{20}\NormalTok{,}\DecValTok{15}\NormalTok{), }\FunctionTok{c}\NormalTok{(}\DecValTok{10}\NormalTok{,}\DecValTok{2}\NormalTok{,}\DecValTok{7}\NormalTok{,}\DecValTok{20}\NormalTok{,}\DecValTok{5}\NormalTok{,}\DecValTok{8}\NormalTok{,}\DecValTok{9}\NormalTok{,}\DecValTok{20}\NormalTok{,}\DecValTok{9}\NormalTok{,}\DecValTok{1}\NormalTok{,}\DecValTok{15}\NormalTok{))}
\CommentTok{\#\textgreater{} [1]  1  4 11}
\end{Highlighting}
\end{Shaded}

\begin{enumerate}
\def\labelenumi{(\alph{enumi})}
\setcounter{enumi}{5}
\tightlist
\item
  Illustrate the difference between \texttt{match()} and \texttt{pmatch()} by considering the names of the days of the week.
\end{enumerate}

\subsection{Basic statistical functions}\label{basic-statistical-functions}

Study the functions in detail in Table \ref{tab:StatFunc}.

\begin{longtable}[]{@{}
  >{\raggedright\arraybackslash}p{(\linewidth - 4\tabcolsep) * \real{0.1667}}
  >{\raggedright\arraybackslash}p{(\linewidth - 4\tabcolsep) * \real{0.4167}}
  >{\raggedright\arraybackslash}p{(\linewidth - 4\tabcolsep) * \real{0.4167}}@{}}
\caption{\label{tab:StatFunc} Basic statistical functions.}\tabularnewline
\toprule\noalign{}
\begin{minipage}[b]{\linewidth}\raggedright
\emph{{Function}}
\end{minipage} & \begin{minipage}[b]{\linewidth}\raggedright
\emph{{What it does}}
\end{minipage} & \begin{minipage}[b]{\linewidth}\raggedright
\emph{{Comments}}
\end{minipage} \\
\midrule\noalign{}
\endfirsthead
\toprule\noalign{}
\begin{minipage}[b]{\linewidth}\raggedright
\emph{{Function}}
\end{minipage} & \begin{minipage}[b]{\linewidth}\raggedright
\emph{{What it does}}
\end{minipage} & \begin{minipage}[b]{\linewidth}\raggedright
\emph{{Comments}}
\end{minipage} \\
\midrule\noalign{}
\endhead
\bottomrule\noalign{}
\endlastfoot
\texttt{cor()} & Correlation & One or two arguments \\
\texttt{cumsum()} & Cumulative sum of elements of a vector & \\
\texttt{mean()} & Arithmetic mean & Optional argument \texttt{trim\ =} \\
\texttt{median()} & Median & Accepts variable number of arguments \\
\texttt{min()} & Minimum value & Accepts variable number of arguments \\
\texttt{max()} & Maximum value & Accepts variable number of arguments \\
\texttt{prod()} & Product of elements of a vector & Accepts variable number of arguments \\
\texttt{cumprod()} & Cumulative product of elements of a vector & \\
\texttt{quantile()} & Returns specified quantiles & \\
\texttt{range()} & Minimum and maximum of a vector & Accepts variable number of arguments \\
\texttt{sample()} & Random sample & With or without replacement \\
\texttt{sum()} & Arithmetic sum & Also used for counting \\
\texttt{var()} & Variance and covariance; uses n-1 as denominator & Accepts vectors or matrices \\
\texttt{sd()} & Standard deviation; uses n-1 as denominator & Accept a vector as argument \\
\end{longtable}

Note also the functions \texttt{pmax()} and \texttt{pmin()}.

\begin{enumerate}
\def\labelenumi{(\alph{enumi})}
\tightlist
\item
  Find the average Life Expectancy of the states in the \texttt{state.x77} data set.
\item
  Find the 5\% trimmed mean for Illiteracy of the states in the \texttt{state.x77} data set.
\item
  Find the correlation between the Illiteracy and the \texttt{Income} of the states in the \texttt{state.x77} data set.
\item
  Find the covariance matrix of all the variables in the \texttt{state.x77} data set.
\item
  Find the range for Murder in the \texttt{state.x77} data set.
\item
  Obtain the details of a random sample of 10 states in the \texttt{state.x77} data set.
\item
  Obtain two independent random permutations of the numbers \(1, 2, \dots, 10\).
\item
  Write a function for computing the coefficient of kurtosis for a random sample. Test your function on the Frost variable in the \texttt{state.x77} data set.
\item
  Write a function for computing the coefficient of skewness for a random sample. Test your function on the Murder variable in the \texttt{state.x77} data set.
\item
  Write a function to compute the harmonic mean of a numeric vector. Test your function on the Life Expectancy of the states in the \texttt{state.x77} data set. Compare your answer to your answer in (a).
\end{enumerate}

\subsection{Probability distributions in R}\label{probability-distributions-in-r}

First, execute the R-instruction

\begin{Shaded}
\begin{Highlighting}[]
\FunctionTok{help.search}\NormalTok{(}\StringTok{"distribution"}\NormalTok{)}
\end{Highlighting}
\end{Shaded}

to obtain a list of available statistical distributions in R. Each distribution has an identifying name preceded by one of the letters \emph{{d}}, \emph{{p}}, \emph{{q}} or \emph{{r}}. In the case of an F-distribution, for example, the identifier is just the letter \texttt{f} and for a normal distribution the identifier is \texttt{norm}. Preceding the distribution's identifier by one of the letters \texttt{d}, \texttt{p}, \texttt{q} or \texttt{r} returns a density value, a probability, a quantile or a random sample for the specified distribution (probability density function or probability mass function). See Figure \ref{fig:Fdist} for an explanation.

\begin{figure}
\includegraphics[width=1\linewidth]{pics/F-distribution} \caption{Meaning of the letters d, p and q when preceding an R distribution identifier.}\label{fig:Fdist}
\end{figure}

\subsection{Functions for categorical variables}\label{areagrp}

Apart from being \emph{{numeric}} or \emph{{logical}}, data in R can also be \emph{{categorical}} (\emph{{factor}} in R) or character strings. Study in detail the functions operating on factor data in Table \ref{tab:CatFunc}.

\begin{enumerate}
\def\labelenumi{(\alph{enumi})}
\item
  Use \texttt{cut()} to create an object \texttt{areagrp} to divide the \texttt{state.x77} data set into three groups representing the states with area within the intervals \((0, 10 000]\),\((10 000, 100 000]\) and \((100 000, Inf]\), respectively. \emph{Hint}: First study the arguments of \texttt{cut()}.
\item
  Repeat (a) with argument \texttt{labels\ =\ ??} to specify each state as being \emph{Small}, \emph{Medium} or \emph{Large} with respect to its area.
\item
  Use \texttt{unclass()} to obtain the numeric codes associated with each level of \texttt{areagrp}.
\item
  Repeat (a) to obtain \texttt{areagrp2} containing five equally spaced categories.
\item
  Repeat (a) to obtain \texttt{areagrp3} containing five groups with each containing \(20\%\) of the data.
\item
  Use \texttt{cut()} to create an object \texttt{illitgrp} to divide the \texttt{state.x77} data set into five groups representing the states with illiteracy within the interval \([0, 0.50)\), \([0.50, 1.00)\), \([1.00, 1.50)\), \([1.50, 2.00)\) and \([2.00, 5.00)\), respectively.
\item
  Obtain a two-way table of the \texttt{state.x77} data set according to \texttt{areagrp} and \texttt{illitgrp}.
\end{enumerate}

\begin{longtable}[]{@{}
  >{\raggedright\arraybackslash}p{(\linewidth - 2\tabcolsep) * \real{0.2857}}
  >{\raggedright\arraybackslash}p{(\linewidth - 2\tabcolsep) * \real{0.7143}}@{}}
\caption{\label{tab:CatFunc} Basic functions for categorical variables.}\tabularnewline
\toprule\noalign{}
\begin{minipage}[b]{\linewidth}\raggedright
\emph{{Function}}
\end{minipage} & \begin{minipage}[b]{\linewidth}\raggedright
\emph{{What it does}}
\end{minipage} \\
\midrule\noalign{}
\endfirsthead
\toprule\noalign{}
\begin{minipage}[b]{\linewidth}\raggedright
\emph{{Function}}
\end{minipage} & \begin{minipage}[b]{\linewidth}\raggedright
\emph{{What it does}}
\end{minipage} \\
\midrule\noalign{}
\endhead
\bottomrule\noalign{}
\endlastfoot
\texttt{cut()} & Creates categories out of a continuous variable \\
\texttt{factor()} & Encodes a vector as a \textbf{\emph{nominal}} categorical variable \\
\texttt{ordered()} & Encodes a vector as a \textbf{\emph{ordinal}} categorical variable when argument ordered is set to TRUE \\
\texttt{levels()} & Displays or sets the levels of a factor variable \\
\texttt{pretty()} & Creates convenient break points for a categorical variable \\
\texttt{split()} & Breaks up an array according to the value of a categorical variable \\
\texttt{table()} & Counts the number of observations cross-classified by categories \\
\texttt{unclass()} & Returns the numeric codes for representing the levels of a factor variable \\
\end{longtable}

\subsection{Functions for character manipulation}\label{character}

Study the functions in Table \ref{tab:CharFunc} in detail.

\begin{longtable}[]{@{}
  >{\raggedright\arraybackslash}p{(\linewidth - 2\tabcolsep) * \real{0.2857}}
  >{\raggedright\arraybackslash}p{(\linewidth - 2\tabcolsep) * \real{0.7143}}@{}}
\caption{\label{tab:CharFunc} Basic functions for character manipulation.}\tabularnewline
\toprule\noalign{}
\begin{minipage}[b]{\linewidth}\raggedright
\emph{{Function}}
\end{minipage} & \begin{minipage}[b]{\linewidth}\raggedright
\emph{{What it does}}
\end{minipage} \\
\midrule\noalign{}
\endfirsthead
\toprule\noalign{}
\begin{minipage}[b]{\linewidth}\raggedright
\emph{{Function}}
\end{minipage} & \begin{minipage}[b]{\linewidth}\raggedright
\emph{{What it does}}
\end{minipage} \\
\midrule\noalign{}
\endhead
\bottomrule\noalign{}
\endlastfoot
\texttt{abbreviate()} & Generates abbreviations of character values \\
\texttt{cat()} & Display,messages and/or values on screen or send to file \\
\texttt{grep()} & Search for patterns in characters \\
\texttt{nchar()} & Number of characters in a string \\
\texttt{paste()} & Combine values into character strings \\
\texttt{strsplit()} & Split the elements of a character vector \(\times\) into substrings \\
\texttt{substring()} & Extracts parts of character strings \\
\end{longtable}

\begin{enumerate}
\def\labelenumi{(\alph{enumi})}
\item
  What is the returned value of \texttt{grep\ ("ia",\ state.name)}?
\item
  Discuss the usage of \texttt{grep\ ("ia",\ state.name)}.
\item
  Discuss the output of \texttt{objects\ (pos\ =\ grep("stats",\ search()))}.
\item
  Use \texttt{paste()} to create variable names: var1, var2, \ldots, var100.
\item
  Repeat (d) to create variable names: var\_1, var\_2, \ldots, var\_100.
\item
  Discuss the output of:
\end{enumerate}

\begin{Shaded}
\begin{Highlighting}[]
\NormalTok{substring (paste (letters, collapse = ""),  }
\NormalTok{             1:nchar (paste (letters, collapse="")), }
\NormalTok{             1:nchar (paste (letters, collapse="")))}
\end{Highlighting}
\end{Shaded}

\begin{enumerate}
\def\labelenumi{(\alph{enumi})}
\setcounter{enumi}{6}
\tightlist
\item
  From the Help menu, select Manuals (in PDF) and open the Introduction to R document. Obtain a copy of the first two paragraphs of the Preface on page 1 of this book in the R commands window. Use this copy to calculate the number of words as well as the total number of characters (including spaces between words) in the passage.
\end{enumerate}

We are going to use several of the functions in Table \ref{tab:CharFunc} to perform this task in steps. Proceed as follows in R after copying the relevant passage to the clipboard:

\begin{Shaded}
\begin{Highlighting}[]
\NormalTok{TextPar }\OtherTok{\textless{}{-}} \FunctionTok{scan}\NormalTok{(}\AttributeTok{file =} \StringTok{"clipboard"}\NormalTok{, }\AttributeTok{what =} \StringTok{""}\NormalTok{)}
\end{Highlighting}
\end{Shaded}

To obtain a vector containing each of the words as a separate element.

\begin{Shaded}
\begin{Highlighting}[]
\NormalTok{TextPar }\OtherTok{\textless{}{-}} \FunctionTok{paste}\NormalTok{ (TextPar, }\AttributeTok{collapse =} \StringTok{" "}\NormalTok{)}
\end{Highlighting}
\end{Shaded}

To convert \texttt{TextPar} into a vector containing one element consisting of all the words concatenated and separated by spaces into a single character string. Add the correct line breaks (``\textbackslash n'') in \texttt{TextPar} using e.g.~\texttt{fix()}.

\begin{Shaded}
\begin{Highlighting}[]
\NormalTok{TextPar }\OtherTok{\textless{}{-}} \FunctionTok{strsplit}\NormalTok{(}\AttributeTok{x =}\NormalTok{ TextPar, }\AttributeTok{split =} \StringTok{\textquotesingle{}}\SpecialCharTok{\textbackslash{}n}\StringTok{\textquotesingle{}}\NormalTok{)}
\end{Highlighting}
\end{Shaded}

\begin{Shaded}
\begin{Highlighting}[]
\NormalTok{mode(TextPar)}
\NormalTok{[1] "list"}

\NormalTok{mode(unlist(TextPar))}
\NormalTok{[1] "character" }
\end{Highlighting}
\end{Shaded}

\begin{Shaded}
\begin{Highlighting}[]
\NormalTok{TextPar }\OtherTok{\textless{}{-}} \FunctionTok{unlist}\NormalTok{(TextPar)}
\end{Highlighting}
\end{Shaded}

To change \texttt{TextPar} into a character vector.

\begin{Shaded}
\begin{Highlighting}[]
\FunctionTok{nchar}\NormalTok{(TextPar)}
\FunctionTok{length}\NormalTok{(TextPar)}
\end{Highlighting}
\end{Shaded}

\section{Differentiation and integration}\label{differentiation-and-integration}

\subsection{Symbolic differentiation}\label{symbolic-differentiation}

Study the help files of \texttt{D()} and \texttt{deriv()}.

\subsection{Integration}\label{integration}

Study the help file of \texttt{integrate()}.

\subsection{Exercise}\label{exercise-8}

\begin{enumerate}
\def\labelenumi{(\arabic{enumi})}
\item
  It is known from elementary statistics that approximately 68\% of data from a normal distribution with a mean of zero and a standard deviation of unity will have an absolute value less than unity. Use the \texttt{sum()} and \texttt{rnorm()} functions to find the proportion of \(n\) random \(normal (0, 1)\) variables whose absolute value is less than \(1.0\). Repeat with different values for \(n\) to investigate how widely the results vary.
\item
  Define: conditional inverse and generalized (Moore-Penrose) inverse for matrix \(\mathbf{X}: p \times q\) and make provision for \(p = q\), \(p > q\) and \(p < q\). First, show how the svd of \(\mathbf{X}\) can be used to obtain a conditional inverse, \(\mathbf{X}^c\) for \(\mathbf{X}\). Now use the above information to write an R function for calculating \(\mathbf{X}^c\) for any given \(\mathbf{X}\). The function must provide a test to check if the calculated conditional inverse is indeed a conditional inverse. Illustrate the usage of your function.
\item
  Give the necessary instructions to:

  \begin{enumerate}
  \def\labelenumii{(\roman{enumii})}
  \tightlist
  \item
    read into R an external text data file consisting of \(10\) sample observations with each consisting of one character variable and two numerical variables.
  \item
    read into R a large external text data file consisting of \(50\) numerical variables but unknown number of records. Each record in this data file takes up 5 lines. The variables in the R object must have the names X1, \ldots, X50.
  \end{enumerate}
\item
  Discuss the meaning of the following R instructions:

  \begin{enumerate}
  \def\labelenumii{(\roman{enumii})}
  \tightlist
  \item
    \texttt{y\ \textless{}-\ x{[}!is.na(x){]}}
  \item
    \texttt{z\ \textless{}-\ (x\ +\ y){[}!is.na(x)\ \&\ x\ \textgreater{}0{]}}
  \item
    \texttt{a\ \textless{}-\ x{[}-(1:5){]}}
  \item
    \texttt{x{[}is.na(x){]}\ \textless{}-\ 0}
  \end{enumerate}
\end{enumerate}

\chapter{Introducing traditional R graphics}\label{graphics}

A basic knowledge of R graphics is needed before directing attention to the art of writing programs (functions) in R. Therefore, in this chapter a brief overview is given of the basics of traditional R graphics. In a later chapter, after studying the principles of R programming, a second round of R graphics will follow.

\section{General}\label{general-1}

Study the graphical parameters by requesting

\begin{Shaded}
\begin{Highlighting}[]
\NormalTok{?par}
\end{Highlighting}
\end{Shaded}

In Figure \ref{fig:figRegion} the main components of a graph window are illustrated. Study this figure in detail. The \emph{{Plot Region}} together with the\emph{{Margins}} is called the \emph{{Figure Region}}.

\begin{figure}
\includegraphics[width=1\linewidth]{pics/figMargins} \caption{The main components of a graph window and the parameters for controlling their sizes.  The parameter mai is a numerical vector of the form c(bottom, left, top, right) specifying the margins in inches while the parameter mar has a similar form specifying the respective margins as the number of lines. The default of mar is c(5, 4, 4, 2) + 0.1.}\label{fig:figRegion}
\end{figure}

\begin{enumerate}
\def\labelenumi{(\alph{enumi})}
\item
  What is the difference between high-level and low-level plotting instructions?
\item
  Take note especially how the functions \texttt{windows()}, \texttt{win.graph()} or \texttt{x11()} are used as well as the different options available for these functions.
\item
  The instruction \texttt{dev.new()} allows opening a new graph window in a platform-independent way.
\item
  In this chapter some high-level plotting instructions are studied. Each of these instructions results in a (new) graph window with a complete graph drawn. The command \texttt{graphics.off()} deletes all open graphic devices.
\item
  Study the use of \texttt{par()}, \texttt{par(mfrow\ =)} and \texttt{par(mfcol\ =)}. Study the use of \texttt{par(new\ =\ TRUE)} to plot more than one figure on the same set of axes.
\item
  Study how the functions \texttt{graphics.off()} and \texttt{dev.off()} work.
\end{enumerate}

\section{High-level plotting instructions}\label{highLevelPlotting}

\begin{enumerate}
\def\labelenumi{(\alph{enumi})}
\tightlist
\item
  Construct a barplot of the illiteracy of the states according to the \texttt{areagrp} (as defined in section \ref{areagrp}) in the \texttt{state.x77} dataframe. \emph{Hint}: The function \texttt{tapply()} operates on a vector given as its first argument. Its second argument groups the first argument into groups so that the function given in its third argument can be applied to each of these groups. Study the following command:
\end{enumerate}

\begin{Shaded}
\begin{Highlighting}[]
\FunctionTok{barplot}\NormalTok{ (}\FunctionTok{tapply}\NormalTok{ (state.x77[, }\StringTok{"Illiteracy"}\NormalTok{], areagrp, mean), }
         \AttributeTok{names=}\FunctionTok{levels}\NormalTok{(areagrp), }\AttributeTok{ylab =} \StringTok{"Illiteracy"}\NormalTok{, }\AttributeTok{xlab =} \StringTok{"Area of State"}\NormalTok{, }
         \AttributeTok{main =} \StringTok{"Barplot of Mean Illiteracy"}\NormalTok{)}
\end{Highlighting}
\end{Shaded}

\begin{enumerate}
\def\labelenumi{(\alph{enumi})}
\setcounter{enumi}{1}
\tightlist
\item
  Construct, for the \texttt{state.x77} data set, box plots of illiteracy broken down by the income of the states. First use \texttt{cut()} to form three categories of state income:
\end{enumerate}

\begin{Shaded}
\begin{Highlighting}[]
\NormalTok{state.income }\OtherTok{\textless{}{-}} \FunctionTok{cut}\NormalTok{ (state.x77[ , }\StringTok{"Income"}\NormalTok{], }\FunctionTok{c}\NormalTok{(}\DecValTok{0}\NormalTok{, }\DecValTok{4000}\NormalTok{, }\DecValTok{5000}\NormalTok{, }\ConstantTok{Inf}\NormalTok{),}
                   \AttributeTok{labels=}\FunctionTok{c}\NormalTok{(}\StringTok{"$4000 or less"}\NormalTok{, }\StringTok{"$4001{-}$5000"}\NormalTok{, }\StringTok{"more than $5001"}\NormalTok{))}
\end{Highlighting}
\end{Shaded}

Then use \texttt{boxplot()} together with \texttt{split()} to produce the desired graph:

\begin{Shaded}
\begin{Highlighting}[]
\FunctionTok{boxplot}\NormalTok{ (}\FunctionTok{split}\NormalTok{ (state.x77[ , }\StringTok{"Income"}\NormalTok{], state.income))}
\end{Highlighting}
\end{Shaded}

Add labels for the axes as well as a title for the figure.

\begin{enumerate}
\def\labelenumi{(\alph{enumi})}
\setcounter{enumi}{2}
\item
  Repeat the previous example but use argument \texttt{notch\ =\ TRUE}.
\item
  Attach the package \texttt{akima}. What is the usage of the function \texttt{interp()}? Discuss by constructing the following contour plot:
\end{enumerate}

\begin{Shaded}
\begin{Highlighting}[]
\FunctionTok{contour}\NormalTok{ (}\FunctionTok{interp}\NormalTok{ (state.center}\SpecialCharTok{$}\NormalTok{x, state.center}\SpecialCharTok{$}\NormalTok{y,  state.x77[,}\StringTok{"Frost"}\NormalTok{])) }
\end{Highlighting}
\end{Shaded}

\begin{enumerate}
\def\labelenumi{(\alph{enumi})}
\setcounter{enumi}{4}
\tightlist
\item
  What is a \emph{{coplot}}? Discuss after giving the following instruction and referring to the role of the tilde (\textasciitilde) operator.
\end{enumerate}

\begin{Shaded}
\begin{Highlighting}[]
\FunctionTok{coplot}\NormalTok{ (state.x77[,}\StringTok{"Illiteracy"}\NormalTok{] }\SpecialCharTok{\textasciitilde{}}\NormalTok{ state.x77[,}\StringTok{"Area"}\NormalTok{] }\SpecialCharTok{|}\NormalTok{ state.x77[,}\StringTok{"Income"}\NormalTok{])}
\end{Highlighting}
\end{Shaded}

\begin{enumerate}
\def\labelenumi{(\alph{enumi})}
\setcounter{enumi}{5}
\tightlist
\item
  A \emph{{dotchart}} is constructed with function \texttt{dotchart()}. First some preparations are necessary:
\end{enumerate}

\begin{Shaded}
\begin{Highlighting}[]
\NormalTok{incgroup }\OtherTok{\textless{}{-}} \FunctionTok{cut}\NormalTok{(state.x77[,}\StringTok{"Income"}\NormalTok{],  }\DecValTok{3}\NormalTok{, }
                \AttributeTok{labels =} \FunctionTok{c}\NormalTok{(}\StringTok{"LowInc"}\NormalTok{, }\StringTok{"MediumInc"}\NormalTok{, }\StringTok{"HighInc"}\NormalTok{))}
\NormalTok{lifgroup }\OtherTok{\textless{}{-}} \FunctionTok{cut}\NormalTok{(state.x77[,}\StringTok{"Life Exp"}\NormalTok{], }\DecValTok{2}\NormalTok{, }
                \AttributeTok{labels =} \FunctionTok{c}\NormalTok{(}\StringTok{"LowExp"}\NormalTok{, }\StringTok{"HighExp"}\NormalTok{))}
\NormalTok{table.out }\OtherTok{\textless{}{-}} \FunctionTok{tapply}\NormalTok{(state.x77 [,}\StringTok{"Income"}\NormalTok{], }\FunctionTok{list}\NormalTok{(lifgroup,incgroup), mean)}
\NormalTok{table.out}
\CommentTok{\#\textgreater{}           LowInc MediumInc HighInc}
\CommentTok{\#\textgreater{} LowExp  3640.917  4698.417    5807}
\CommentTok{\#\textgreater{} HighExp 4039.600  4697.667    5348}
\FunctionTok{dotchart}\NormalTok{ (table.out, }
          \FunctionTok{levels}\NormalTok{ (}\FunctionTok{factor}\NormalTok{ (}\FunctionTok{col}\NormalTok{ (table.out), }
                          \AttributeTok{labels =} \FunctionTok{levels}\NormalTok{ (incgroup)))[}\FunctionTok{col}\NormalTok{(table.out)], }
          \FunctionTok{factor}\NormalTok{(}\FunctionTok{row}\NormalTok{(table.out), }\AttributeTok{labels =} \FunctionTok{levels}\NormalTok{(lifgroup)))}
\end{Highlighting}
\end{Shaded}

\pandocbounded{\includegraphics[keepaspectratio]{04-graphics_files/figure-latex/dotchart-1.pdf}}

Complete the graph by adding a label to the x-axis and a heading for the graph.

\begin{enumerate}
\def\labelenumi{(\alph{enumi})}
\setcounter{enumi}{6}
\item
  Use function \texttt{faces()} available in package \texttt{aplpack} to construct Chernoff faces for the Western states in the data set \texttt{state.x77}. \emph{Hint}: The Western states appear in rows 3, 5, 12, 26, 28, 37, 44, 47 and 50. Explain what is represented by each of the facial features. First set argument \texttt{face.type\ =\ 0} and then \texttt{face.type\ =\ 1}.
\item
  Obtain a histogram of the life expectancy in the states of \texttt{state.x77}.
\item
  Execute the command
\end{enumerate}

\begin{Shaded}
\begin{Highlighting}[]
\FunctionTok{pairs}\NormalTok{ (state.x77)}
\end{Highlighting}
\end{Shaded}

Interpret the graph.

\begin{enumerate}
\def\labelenumi{(\alph{enumi})}
\setcounter{enumi}{9}
\tightlist
\item
  Three-dimensional graphs are constructed with function \texttt{persp()}.
\end{enumerate}

\begin{Shaded}
\begin{Highlighting}[]
\NormalTok{pts }\OtherTok{\textless{}{-}} \FunctionTok{seq}\NormalTok{(}\AttributeTok{from =} \SpecialCharTok{{-}}\NormalTok{pi, }\AttributeTok{to =}\NormalTok{ pi, }\AttributeTok{len =} \DecValTok{20}\NormalTok{)}
\NormalTok{z }\OtherTok{\textless{}{-}} \FunctionTok{outer}\NormalTok{(}\AttributeTok{X =}\NormalTok{ pts, }\AttributeTok{Y =}\NormalTok{ pts, }\ControlFlowTok{function}\NormalTok{(x,y) }\FunctionTok{sin}\NormalTok{(x)}\SpecialCharTok{*}\FunctionTok{cos}\NormalTok{(y))}
\FunctionTok{persp}\NormalTok{(}\AttributeTok{x =}\NormalTok{ pts, }\AttributeTok{y =}\NormalTok{ pts, z, }\AttributeTok{theta =} \DecValTok{10}\NormalTok{, }\AttributeTok{phi =} \DecValTok{60}\NormalTok{, }\AttributeTok{ticktype =} \StringTok{\textquotesingle{}detailed\textquotesingle{}}\NormalTok{)}
\end{Highlighting}
\end{Shaded}

Discuss the meaning of each of the above instructions. Experiment with different values for arguments \texttt{theta} and \texttt{phi}.

\begin{enumerate}
\def\labelenumi{(\alph{enumi})}
\setcounter{enumi}{10}
\item
  Obtain a pie chart of the object \texttt{areagrp} defined in section \ref{areagrp}. \emph{Hint}: function \texttt{table()} may be useful here.
\item
  A cluster plot (dendrogram) can be constructed with function \texttt{plclust()} as follows:
\end{enumerate}

\begin{Shaded}
\begin{Highlighting}[]
\NormalTok{west.rows }\OtherTok{\textless{}{-}} \FunctionTok{c}\NormalTok{(}\DecValTok{3}\NormalTok{, }\DecValTok{5}\NormalTok{, }\DecValTok{12}\NormalTok{, }\DecValTok{26}\NormalTok{, }\DecValTok{28}\NormalTok{, }\DecValTok{37}\NormalTok{, }\DecValTok{44}\NormalTok{, }\DecValTok{47}\NormalTok{, }\DecValTok{50}\NormalTok{)}
\NormalTok{distmat.west }\OtherTok{\textless{}{-}} \FunctionTok{dist}\NormalTok{ (}\FunctionTok{scale}\NormalTok{ (state.x77[west.rows,]))}
\FunctionTok{plot}\NormalTok{(}\FunctionTok{hclust}\NormalTok{(distmat.west), }\AttributeTok{labels =} \FunctionTok{rownames}\NormalTok{(state.x77)[west.rows])}
\end{Highlighting}
\end{Shaded}

Interpret the above instructions and the resulting plot.

\begin{enumerate}
\def\labelenumi{(\alph{enumi})}
\setcounter{enumi}{12}
\item
  Use the function \texttt{plot()} to plot \(sin (\theta)\) as \(\theta\) varies from \(–\pi\) to \(\pi\).
\item
  Could you explain the different graphs resulting from the two calls in (l) and (m) to the \texttt{plot()} function above?
\item
  Obtain the empirical distribution function of variable \texttt{Life\ Exp} in the \texttt{state.x77} data set by using the functions \texttt{cut()}, \texttt{ecdf()} and \texttt{plot()}.
\item
  Check the normality of variable \texttt{Income} in the \texttt{state.x7}7 data set by using function \texttt{qqnorm()}.
\item
  Obtain a \texttt{qqplot} of the income of small states versus the income of large states in the data set \texttt{state.x77} where small and large are defined as below or above the median income, respectively.
\end{enumerate}

\begin{Shaded}
\begin{Highlighting}[]
\NormalTok{state.size }\OtherTok{\textless{}{-}} \FunctionTok{cut}\NormalTok{ (state.x77[,}\StringTok{"Area"}\NormalTok{],  }
                   \FunctionTok{c}\NormalTok{(}\DecValTok{0}\NormalTok{, }\FunctionTok{median}\NormalTok{ (state.x77[,}\StringTok{"Area"}\NormalTok{]), }\FunctionTok{max}\NormalTok{ (state.x77[,}\StringTok{"Area"}\NormalTok{])))}
\NormalTok{state.income }\OtherTok{\textless{}{-}} \FunctionTok{split}\NormalTok{ (state.x77[,}\StringTok{"Income"}\NormalTok{], state.size)}
\FunctionTok{qqplot}\NormalTok{(state.income[[}\DecValTok{1}\NormalTok{]], state.income[[}\DecValTok{2}\NormalTok{]], }\AttributeTok{xlab=}\StringTok{"Income for small states"}\NormalTok{, }
       \AttributeTok{ylab=}\StringTok{"income for large states"}\NormalTok{)}
\end{Highlighting}
\end{Shaded}

\begin{enumerate}
\def\labelenumi{(\alph{enumi})}
\setcounter{enumi}{17}
\tightlist
\item
  Use function \texttt{ts.plot()} to construct a time series plot of the sunspots data set.
\end{enumerate}

\section{Interactive communication with graphs}\label{interactive-communication-with-graphs}

\begin{enumerate}
\def\labelenumi{(\alph{enumi})}
\item
  Study the help files of the functions \texttt{text()}, \texttt{identify()} and \texttt{locator()}.
\item
  Illustrate the usage of \texttt{identify()} on a scatterplot of variables \texttt{Illiteracy} and \texttt{Life\ Exp} in the \texttt{state.x77} data set:
\end{enumerate}

\begin{Shaded}
\begin{Highlighting}[]
\FunctionTok{plot}\NormalTok{ (}\AttributeTok{x =}\NormalTok{ state.x77[,}\StringTok{\textquotesingle{}Life Exp\textquotesingle{}}\NormalTok{], }\AttributeTok{y =}\NormalTok{ state.x77[,}\StringTok{\textquotesingle{}Income\textquotesingle{}}\NormalTok{])}
\end{Highlighting}
\end{Shaded}

To create the scatterplot, then call

\begin{Shaded}
\begin{Highlighting}[]
\FunctionTok{identify}\NormalTok{ (}\AttributeTok{x =}\NormalTok{ state.x77[,}\StringTok{\textquotesingle{}Life Exp\textquotesingle{}}\NormalTok{], }\AttributeTok{y =}\NormalTok{ state.x77[,}\StringTok{\textquotesingle{}Income\textquotesingle{}}\NormalTok{], }
          \FunctionTok{seq}\NormalTok{ (}\AttributeTok{along =} \FunctionTok{rownames}\NormalTok{(state.x77)), }\AttributeTok{n =} \DecValTok{5}\NormalTok{)}
\end{Highlighting}
\end{Shaded}

Notice the change in the cursor; the cursor changes to a cross when moved over the graph. Hover the cursor over a point to identify and click left mouse button. Repeat \(n = 5\) times. Explain the result. Next, create the scatterplot once more and then call

\begin{Shaded}
\begin{Highlighting}[]
\FunctionTok{identify}\NormalTok{ (}\AttributeTok{x =}\NormalTok{ state.x77[,}\StringTok{\textquotesingle{}Life Exp\textquotesingle{}}\NormalTok{],  }\AttributeTok{y =}\NormalTok{ state.x77[,}\StringTok{\textquotesingle{}Income\textquotesingle{}}\NormalTok{], }
          \AttributeTok{labels =} \FunctionTok{rownames}\NormalTok{(state.x77)[}\FunctionTok{seq}\NormalTok{ (}\AttributeTok{along =} 
                                              \FunctionTok{rownames}\NormalTok{(state.x77))] , }\AttributeTok{n =} \DecValTok{5}\NormalTok{) }
\end{Highlighting}
\end{Shaded}

Explain what has happened.

\begin{enumerate}
\def\labelenumi{(\alph{enumi})}
\setcounter{enumi}{2}
\tightlist
\item
  Illustrate the usage of \texttt{locator()} by:
\end{enumerate}

\emph{Joining \(5\) user defined points on a graph interactively with straight lines}

\begin{Shaded}
\begin{Highlighting}[]
\FunctionTok{plot}\NormalTok{ (}\AttributeTok{x =}\NormalTok{ state.x77[,}\StringTok{\textquotesingle{}Life Exp\textquotesingle{}}\NormalTok{], }\AttributeTok{y =}\NormalTok{ state.x77[,}\StringTok{\textquotesingle{}Income\textquotesingle{}}\NormalTok{])}
\FunctionTok{locator}\NormalTok{(}\DecValTok{5}\NormalTok{, }\AttributeTok{type =} \StringTok{"l"}\NormalTok{) }
\end{Highlighting}
\end{Shaded}

Use mouse and select the five points on the graph. What happened on the graph? What happened in the commands window?

\emph{Writing text interactively at a specified position on an existing graph}

\begin{Shaded}
\begin{Highlighting}[]
\FunctionTok{plot}\NormalTok{ (}\AttributeTok{x =}\NormalTok{ state.x77[,}\StringTok{\textquotesingle{}Life Exp\textquotesingle{}}\NormalTok{], }\AttributeTok{y =}\NormalTok{ state.x77[,}\StringTok{\textquotesingle{}Income\textquotesingle{}}\NormalTok{])}
\FunctionTok{text}\NormalTok{ (}\FunctionTok{locator}\NormalTok{ (}\AttributeTok{n =} \DecValTok{1}\NormalTok{, }\AttributeTok{type =} \StringTok{"n"}\NormalTok{), }\AttributeTok{label =} \StringTok{"State with the highest income"}\NormalTok{)}
\end{Highlighting}
\end{Shaded}

\section{3D graphics: package rgl}\label{d-graphics-package-rgl}

Write and execute the following function.

\begin{Shaded}
\begin{Highlighting}[]
\NormalTok{rgl.example }\OtherTok{\textless{}{-}} \ControlFlowTok{function}\NormalTok{ (}\AttributeTok{size =} \FloatTok{0.1}\NormalTok{, }\AttributeTok{col =} \StringTok{"green"}\NormalTok{, }\AttributeTok{alpha.3d =} \FloatTok{0.6}\NormalTok{) }
\NormalTok{\{ }\FunctionTok{require}\NormalTok{(rgl)}
\NormalTok{  datmat }\OtherTok{\textless{}{-}} \FunctionTok{matrix}\NormalTok{ (}\FunctionTok{rnorm}\NormalTok{ (}\DecValTok{30}\NormalTok{), }\AttributeTok{ncol =} \DecValTok{3}\NormalTok{)}
  \FunctionTok{open3d}\NormalTok{()}
  \FunctionTok{spheres3d}\NormalTok{ (datmat,}\AttributeTok{radius =}\NormalTok{ size, }\AttributeTok{color =}\NormalTok{ col, }\AttributeTok{alpha =}\NormalTok{ alpha}\FloatTok{.3}\NormalTok{d)}
  \FunctionTok{axes3d}\NormalTok{(}\AttributeTok{col =} \StringTok{"black"}\NormalTok{)}
\NormalTok{  device.ID }\OtherTok{\textless{}{-}} \FunctionTok{rgl.cur}\NormalTok{()}
\NormalTok{  answer }\OtherTok{\textless{}{-}} \FunctionTok{readline}\NormalTok{ (}\StringTok{"Save 3D graph as a .png file? Y/N}\SpecialCharTok{\textbackslash{}n}\StringTok{"}\NormalTok{)}
  \ControlFlowTok{while}\NormalTok{ (}\SpecialCharTok{!}\NormalTok{(answer }\SpecialCharTok{==} \StringTok{"Y"} \SpecialCharTok{|}\NormalTok{ answer }\SpecialCharTok{==} \StringTok{"y"} \SpecialCharTok{|}\NormalTok{ answer }\SpecialCharTok{==} \StringTok{"N"} \SpecialCharTok{|}\NormalTok{ answer }\SpecialCharTok{==} \StringTok{"n"}\NormalTok{)) }
\NormalTok{    answer }\OtherTok{\textless{}{-}} \FunctionTok{readline}\NormalTok{(}\StringTok{"Save 3D graph as a .png file? Y/N}\SpecialCharTok{\textbackslash{}n}\StringTok{"}\NormalTok{)}
  \ControlFlowTok{if}\NormalTok{ (answer }\SpecialCharTok{==} \StringTok{"Y"} \SpecialCharTok{|}\NormalTok{ answer }\SpecialCharTok{==} \StringTok{"y"}\NormalTok{) }
    \ControlFlowTok{repeat} 
\NormalTok{    \{ file.name }\OtherTok{\textless{}{-}} \FunctionTok{readline}\NormalTok{ (}\StringTok{"Provide file name including full }
\StringTok{                              path NOT in quotes and SINGLE }
\StringTok{                              back slashes!}\SpecialCharTok{\textbackslash{}n}\StringTok{"}\NormalTok{)}
\NormalTok{      file.name }\OtherTok{\textless{}{-}} \FunctionTok{paste}\NormalTok{ (file.name, }\StringTok{".png"}\NormalTok{, }\AttributeTok{sep =} \StringTok{""}\NormalTok{)}
      \FunctionTok{snapshot3d}\NormalTok{ (}\AttributeTok{file =}\NormalTok{ file.name)}
      \FunctionTok{rgl.set}\NormalTok{ (device.ID)}
\NormalTok{      answer2 }\OtherTok{\textless{}{-}} \FunctionTok{readline}\NormalTok{(}\StringTok{"Save another 3D graph as a .png file? Y/N }\SpecialCharTok{\textbackslash{}n}\StringTok{"}\NormalTok{)}
      \ControlFlowTok{if}\NormalTok{ (answer2 }\SpecialCharTok{==} \StringTok{"Y"} \SpecialCharTok{|}\NormalTok{ answer2 }\SpecialCharTok{==} \StringTok{"y"}\NormalTok{) }\ControlFlowTok{next} \ControlFlowTok{else} \ControlFlowTok{break}
\NormalTok{    \}}
  \ControlFlowTok{else} \FunctionTok{rgl.set}\NormalTok{ (device.ID)}
\NormalTok{\}}
\end{Highlighting}
\end{Shaded}

Study the above code and constructions in detail.

\section{Exercise}\label{exercise-9}

\begin{enumerate}
\def\labelenumi{\arabic{enumi}.}
\item
  Obtain a graph of a \(normal(100, 25)\) probability density function (p.d.f.).
\item
  Plot on the same set of axes

  \begin{enumerate}
  \def\labelenumii{(\roman{enumii})}
  \tightlist
  \item
    a central \(beta(9, 5)\) p.d.f.;
  \item
    a non-central \(beta(9 5)\) p.d.f. with non-centrality parameter = \(15\) and
  \item
    a non-central \(beta(9, 5)\) p.d.f. with non-centrality parameter = \(40\).
  \end{enumerate}
\end{enumerate}

Add a suitable legend to the plot.

\begin{enumerate}
\def\labelenumi{\arabic{enumi}.}
\setcounter{enumi}{2}
\tightlist
\item
  Use \texttt{persp()} to obtain a graph of any user specified bivariate function. The challenge is that the function specification must appear as the main title of the graph. In order to address this problem we need information about the arguments of \texttt{persp()}:
\end{enumerate}

\begin{Shaded}
\begin{Highlighting}[]
\FunctionTok{args}\NormalTok{ (persp)}
\CommentTok{\#\textgreater{} function (x, ...) }
\CommentTok{\#\textgreater{} NULL}
\end{Highlighting}
\end{Shaded}

This is not very helpful so we try

\begin{Shaded}
\begin{Highlighting}[]
\FunctionTok{methods}\NormalTok{ (persp)}
\CommentTok{\#\textgreater{} [1] persp.default*}
\CommentTok{\#\textgreater{} see \textquotesingle{}?methods\textquotesingle{} for accessing help and source code}
\FunctionTok{args}\NormalTok{ (persp.default)}
\CommentTok{\#\textgreater{} Error: object \textquotesingle{}persp.default\textquotesingle{} not found}
\end{Highlighting}
\end{Shaded}

The reason for this error message follows from the above as that \texttt{persp.default} is not visible. The immediate visibility of a function is regulated by a package builder through the package's namespace mechanism. Only object names that are exported are immediately visible; object names that are not exported are marked with an asterisk and are not visible. The functions
\texttt{argsAnywhere()} and \texttt{getAnywhere()} are available to get information on asterisked object names:

\begin{Shaded}
\begin{Highlighting}[]
\FunctionTok{argsAnywhere}\NormalTok{ (persp.default)}
\CommentTok{\#\textgreater{} function (x = seq(0, 1, length.out = nrow(z)), y = seq(0, 1, }
\CommentTok{\#\textgreater{}     length.out = ncol(z)), z, xlim = range(x), ylim = range(y), }
\CommentTok{\#\textgreater{}     zlim = range(z, na.rm = TRUE), xlab = NULL, ylab = NULL, }
\CommentTok{\#\textgreater{}     zlab = NULL, main = NULL, sub = NULL, theta = 0, phi = 15, }
\CommentTok{\#\textgreater{}     r = sqrt(3), d = 1, scale = TRUE, expand = 1, col = "white", }
\CommentTok{\#\textgreater{}     border = NULL, ltheta = {-}135, lphi = 0, shade = NA, box = TRUE, }
\CommentTok{\#\textgreater{}     axes = TRUE, nticks = 5, ticktype = "simple", ...) }
\CommentTok{\#\textgreater{} NULL}
\end{Highlighting}
\end{Shaded}

We notice that we can make use of the argument main in a call to \texttt{persp()} to provide our perspective plot with a title. However, main accepts only character strings and not mathematical expressions. Furthermore, we have seen in the \texttt{persp()} example in section \ref{highLevelPlotting} that the values for the argument \texttt{z} are conveniently found by a call to \texttt{outer()} using its argument \texttt{FUN}. However \texttt{FUN} requires a function. So we need the means to convert expressions into character strings and vice versa to convert character strings into expressions.

The following pairs of functions allow these conversions to be made:

Character strings ('' ``) → expressions: \texttt{parse()} and \texttt{eval()}

Expressions (unquoted) → character strings ('' ``): \texttt{deparse()} and \texttt{substitute()}

\begin{Shaded}
\begin{Highlighting}[]
\NormalTok{pts }\OtherTok{\textless{}{-}} \FunctionTok{seq}\NormalTok{ (}\AttributeTok{from =} \SpecialCharTok{{-}}\DecValTok{3}\NormalTok{, }\AttributeTok{to =} \DecValTok{3}\NormalTok{, }\AttributeTok{len =} \DecValTok{50}\NormalTok{)}
\NormalTok{fun1 }\OtherTok{\textless{}{-}} \StringTok{"2 * pi * exp({-}(x\^{}2 + y\^{}2)/2)"}
\NormalTok{fun2 }\OtherTok{\textless{}{-}} \FunctionTok{parse}\NormalTok{ (}\AttributeTok{text =} \FunctionTok{paste}\NormalTok{ (}\StringTok{"function(x,y)"}\NormalTok{, fun1))}
\end{Highlighting}
\end{Shaded}

Explain carefully what \texttt{parse()} is doing.

\begin{Shaded}
\begin{Highlighting}[]
\NormalTok{zz }\OtherTok{\textless{}{-}} \FunctionTok{outer}\NormalTok{ (pts, pts, }\FunctionTok{eval}\NormalTok{(fun2))}
\end{Highlighting}
\end{Shaded}

Explain carefully what \texttt{eval()} is doing.

\begin{Shaded}
\begin{Highlighting}[]
\FunctionTok{persp}\NormalTok{ (}\AttributeTok{x =}\NormalTok{ pts, }\AttributeTok{y =}\NormalTok{ pts, }\AttributeTok{z =}\NormalTok{ zz, }\AttributeTok{theta =} \DecValTok{0}\NormalTok{, }\AttributeTok{phi =} \DecValTok{15}\NormalTok{, }\AttributeTok{ticktype =} \StringTok{"detailed"}\NormalTok{, }
       \AttributeTok{main =} \FunctionTok{paste}\NormalTok{(}\StringTok{"Persp plot of \textasciigrave{}"}\NormalTok{fun2,}\StringTok{"\textasciigrave{}"}\NormalTok{,}\AttributeTok{sep=}\StringTok{""}\NormalTok{))}
\end{Highlighting}
\end{Shaded}

Explain carefully the role of \texttt{paste()}.

\begin{enumerate}
\def\labelenumi{\arabic{enumi}.}
\setcounter{enumi}{3}
\item
  Use the \texttt{volcano} data to:

  \begin{enumerate}
  \def\labelenumii{(\roman{enumii})}
  \item
    Obtain a perspective plot using \texttt{persp()}.
  \item
    Obtain an RGL plot of the \texttt{volcano} data.
  \end{enumerate}
\end{enumerate}

\chapter{Subscripting}\label{subscripting}

Vectorized arithmetic and subscripting are two cornerstones of R programming. Review section \ref{highLevelPlotting} for several examples where subscripting has been used. In this chapter subscripting is studied in detail. Specifically, the following two related topics are studied:

\begin{itemize}
\tightlist
\item
  Extracting parts of an object by using \emph{{subscripting}}.
\item
  The combination and rearranging of data within data structures like matrices, dataframes and lists.
\end{itemize}

\section{Subscripting with vectors}\label{vectorSubscripting}

The different types of subscripting with vectors are summarized in Table \ref{tab:SubscriptVectorTypes}:

\begin{longtable}[]{@{}
  >{\raggedright\arraybackslash}p{(\linewidth - 4\tabcolsep) * \real{0.1622}}
  >{\raggedright\arraybackslash}p{(\linewidth - 4\tabcolsep) * \real{0.5405}}
  >{\raggedright\arraybackslash}p{(\linewidth - 4\tabcolsep) * \real{0.2973}}@{}}
\caption{\label{tab:SubscriptVectorTypes} Different types of subscripting vectors.}\tabularnewline
\toprule\noalign{}
\begin{minipage}[b]{\linewidth}\raggedright
\emph{{Type}}
\end{minipage} & \begin{minipage}[b]{\linewidth}\raggedright
\emph{{Effect}}
\end{minipage} & \begin{minipage}[b]{\linewidth}\raggedright
\emph{{Example}}
\end{minipage} \\
\midrule\noalign{}
\endfirsthead
\toprule\noalign{}
\begin{minipage}[b]{\linewidth}\raggedright
\emph{{Type}}
\end{minipage} & \begin{minipage}[b]{\linewidth}\raggedright
\emph{{Effect}}
\end{minipage} & \begin{minipage}[b]{\linewidth}\raggedright
\emph{{Example}}
\end{minipage} \\
\midrule\noalign{}
\endhead
\bottomrule\noalign{}
\endlastfoot
empty & Extract all values & \texttt{x{[}\ {]}} \\
integer, positive & Extract all values specified by the subscript & \texttt{x{[}c(2:5,8,12)\ {]}} \\
integer, negative & Extract all values except those specified by the subscript & \texttt{x{[}–c(2:5,8,12)\ {]}} \\
logical & Extract those values for which subscript is TRUE & \texttt{x{[}x\ \textgreater{}\ 5\ {]}} \\
character & Extract those values whose names attributes correspond to those specified by the subscript & \texttt{x{[}c("a","d")\ {]}} \\
\end{longtable}

Logical subscripting provides a very powerful operation in R. A logical subscript is a vector of \texttt{TRUE}s and \texttt{FALSE}s that must be of the same length as the object being subscripted e.g.

\begin{Shaded}
\begin{Highlighting}[]
\NormalTok{state.x77[ , }\StringTok{"Area"}\NormalTok{] }\SpecialCharTok{\textgreater{}} \DecValTok{80000}  
\CommentTok{\#\textgreater{}        Alabama         Alaska        Arizona       Arkansas }
\CommentTok{\#\textgreater{}          FALSE           TRUE           TRUE          FALSE }
\CommentTok{\#\textgreater{}     California       Colorado    Connecticut       Delaware }
\CommentTok{\#\textgreater{}           TRUE           TRUE          FALSE          FALSE }
\CommentTok{\#\textgreater{}        Florida        Georgia         Hawaii          Idaho }
\CommentTok{\#\textgreater{}          FALSE          FALSE          FALSE           TRUE }
\CommentTok{\#\textgreater{}       Illinois        Indiana           Iowa         Kansas }
\CommentTok{\#\textgreater{}          FALSE          FALSE          FALSE           TRUE }
\CommentTok{\#\textgreater{}       Kentucky      Louisiana          Maine       Maryland }
\CommentTok{\#\textgreater{}          FALSE          FALSE          FALSE          FALSE }
\CommentTok{\#\textgreater{}  Massachusetts       Michigan      Minnesota    Mississippi }
\CommentTok{\#\textgreater{}          FALSE          FALSE          FALSE          FALSE }
\CommentTok{\#\textgreater{}       Missouri        Montana       Nebraska         Nevada }
\CommentTok{\#\textgreater{}          FALSE           TRUE          FALSE           TRUE }
\CommentTok{\#\textgreater{}  New Hampshire     New Jersey     New Mexico       New York }
\CommentTok{\#\textgreater{}          FALSE          FALSE           TRUE          FALSE }
\CommentTok{\#\textgreater{} North Carolina   North Dakota           Ohio       Oklahoma }
\CommentTok{\#\textgreater{}          FALSE          FALSE          FALSE          FALSE }
\CommentTok{\#\textgreater{}         Oregon   Pennsylvania   Rhode Island South Carolina }
\CommentTok{\#\textgreater{}           TRUE          FALSE          FALSE          FALSE }
\CommentTok{\#\textgreater{}   South Dakota      Tennessee          Texas           Utah }
\CommentTok{\#\textgreater{}          FALSE          FALSE           TRUE           TRUE }
\CommentTok{\#\textgreater{}        Vermont       Virginia     Washington  West Virginia }
\CommentTok{\#\textgreater{}          FALSE          FALSE          FALSE          FALSE }
\CommentTok{\#\textgreater{}      Wisconsin        Wyoming }
\CommentTok{\#\textgreater{}          FALSE           TRUE}
\end{Highlighting}
\end{Shaded}

\includegraphics[width=0.8\linewidth]{pics/matrixSubscripting}

\begin{Shaded}
\begin{Highlighting}[]
\NormalTok{x }\OtherTok{\textless{}{-}} \FunctionTok{c}\NormalTok{(}\DecValTok{10}\NormalTok{, }\DecValTok{15}\NormalTok{, }\DecValTok{12}\NormalTok{, }\ConstantTok{NA}\NormalTok{, }\DecValTok{18}\NormalTok{, }\DecValTok{20}\NormalTok{)}
\FunctionTok{is.na}\NormalTok{ (x)}
\CommentTok{\#\textgreater{} [1] FALSE FALSE FALSE  TRUE FALSE FALSE}
\NormalTok{x[}\FunctionTok{is.na}\NormalTok{ (x)]}
\CommentTok{\#\textgreater{} [1] NA}
\NormalTok{x[}\SpecialCharTok{!}\FunctionTok{is.na}\NormalTok{ (x)]}
\CommentTok{\#\textgreater{} [1] 10 15 12 18 20}
\FunctionTok{mean}\NormalTok{ (x)}
\CommentTok{\#\textgreater{} [1] NA}
\FunctionTok{mean}\NormalTok{ (x[}\SpecialCharTok{!}\FunctionTok{is.na}\NormalTok{ (x)])}
\CommentTok{\#\textgreater{} [1] 15}
\FunctionTok{mean}\NormalTok{ (}\FunctionTok{na.omit}\NormalTok{ (x))}
\CommentTok{\#\textgreater{} [1] 15}
\end{Highlighting}
\end{Shaded}

Logical subscripting allows finding the indices of those elements in a vector that meet a certain condition e.g.

\begin{Shaded}
\begin{Highlighting}[]
\NormalTok{(}\DecValTok{1}\SpecialCharTok{:}\FunctionTok{length}\NormalTok{ (}\FunctionTok{rownames}\NormalTok{ (state.x77)))[state.x77[ ,}\StringTok{"Income"}\NormalTok{] }\SpecialCharTok{\textgreater{}} \DecValTok{5000}\NormalTok{]}
\CommentTok{\#\textgreater{} [1]  2  5  7 13 20 28 30 34}
\end{Highlighting}
\end{Shaded}

and to find the corresponding names of the states

\begin{Shaded}
\begin{Highlighting}[]
\FunctionTok{rownames}\NormalTok{(state.x77)[}
\NormalTok{  (}\DecValTok{1}\SpecialCharTok{:}\FunctionTok{length}\NormalTok{ (}\FunctionTok{rownames}\NormalTok{(state.x77)))[state.x77[ ,}\StringTok{"Income"}\NormalTok{] }\SpecialCharTok{\textgreater{}} \DecValTok{5000}\NormalTok{]]}
\CommentTok{\#\textgreater{} [1] "Alaska"       "California"   "Connecticut" }
\CommentTok{\#\textgreater{} [4] "Illinois"     "Maryland"     "Nevada"      }
\CommentTok{\#\textgreater{} [7] "New Jersey"   "North Dakota"}
\end{Highlighting}
\end{Shaded}

In addition to extracting elements, the above subscripting operations can also be used to modify selected elements of a vector e.g.~changing NA-values to zero:

\begin{Shaded}
\begin{Highlighting}[]
\NormalTok{x}
\CommentTok{\#\textgreater{} [1] 10 15 12 NA 18 20}
\NormalTok{x[}\FunctionTok{is.na}\NormalTok{ (x)] }\OtherTok{\textless{}{-}} \DecValTok{0}
\NormalTok{x}
\CommentTok{\#\textgreater{} [1] 10 15 12  0 18 20}
\end{Highlighting}
\end{Shaded}

When the right-hand side of the assignment above is a scalar value, each of the selected values will be changed to the specified scalar value; if the right-hand side is a vector, the selecting values will be changed in order, \emph{{recycling}} the values if more values were selected on the left-hand side than were available on the right-hand side.

\section{Subscripting with matrices}\label{subscripting-with-matrices}

Element and submatrix extraction of matrices are discussed below.

\begin{enumerate}
\def\labelenumi{(\alph{enumi})}
\item
  Revise the use of \texttt{matrix()}, \texttt{names()}, \texttt{dim()} and \texttt{dimnames()}.
\item
  A matrix in R is an \emph{{array}} with two indices. Arrays of order two and higher can be constructed with the function \texttt{dim()} or \texttt{array()}.
\end{enumerate}

Let, for example, \(\mathbf{a}\) be a vector consisting of \(150\) elements. The instruction

\begin{Shaded}
\begin{Highlighting}[]
\FunctionTok{dim}\NormalTok{(a) }\OtherTok{\textless{}{-}} \FunctionTok{c}\NormalTok{(}\DecValTok{3}\NormalTok{, }\DecValTok{5}\NormalTok{, }\DecValTok{10}\NormalTok{) }
\end{Highlighting}
\end{Shaded}

or the instruction

\begin{Shaded}
\begin{Highlighting}[]
\NormalTok{a }\OtherTok{\textless{}{-}} \FunctionTok{array}\NormalTok{ (a, }\AttributeTok{dim =} \FunctionTok{c}\NormalTok{(}\DecValTok{3}\NormalTok{, }\DecValTok{5}\NormalTok{, }\DecValTok{10}\NormalTok{)) }
\end{Highlighting}
\end{Shaded}

constructs a \(3 \times 5 \times 10\) array.

\begin{itemize}
\tightlist
\item
  Matrices can therefore be formed as above, but the function \texttt{matrix()} is usually easier to use.
\item
  The elements of a \(p\)-dimensional array can also be extracted using the one-index or two-index method as described below.
\end{itemize}

\begin{enumerate}
\def\labelenumi{(\alph{enumi})}
\setcounter{enumi}{2}
\item
  The subscripting methods described in section \ref{vectorSubscripting} can also be applied to both the first or second dimension of a matrix where the first dimension refers to the rows and the second dimension to the columns of the matrix.
\item
  Note that the elements of a matrix can be referred to by the two-index method above or by a one index method. When the one index method is used it is assumed that the matrix has first been strung out \emph{{column}}-wise into a vector.
\end{enumerate}

\begin{Shaded}
\begin{Highlighting}[]
\NormalTok{testmat.a }\OtherTok{\textless{}{-}} \FunctionTok{matrix}\NormalTok{ (}\FunctionTok{c}\NormalTok{ (}\DecValTok{17}\NormalTok{, }\DecValTok{40}\NormalTok{, }\DecValTok{20}\NormalTok{, }\DecValTok{34}\NormalTok{, }\DecValTok{21}\NormalTok{, }\DecValTok{12}\NormalTok{, }\DecValTok{14}\NormalTok{, }\DecValTok{57}\NormalTok{, }
                        \DecValTok{78}\NormalTok{, }\DecValTok{37}\NormalTok{, }\DecValTok{29}\NormalTok{, }\DecValTok{64}\NormalTok{), }\AttributeTok{nrow =} \DecValTok{4}\NormalTok{)}
\NormalTok{testmat.a}
\CommentTok{\#\textgreater{}      [,1] [,2] [,3]}
\CommentTok{\#\textgreater{} [1,]   17   21   78}
\CommentTok{\#\textgreater{} [2,]   40   12   37}
\CommentTok{\#\textgreater{} [3,]   20   14   29}
\CommentTok{\#\textgreater{} [4,]   34   57   64}
\NormalTok{testmat.b }\OtherTok{\textless{}{-}} \FunctionTok{matrix}\NormalTok{ (}\FunctionTok{c}\NormalTok{ (}\DecValTok{17}\NormalTok{, }\DecValTok{40}\NormalTok{, }\DecValTok{20}\NormalTok{, }\DecValTok{34}\NormalTok{, }\DecValTok{21}\NormalTok{, }\DecValTok{12}\NormalTok{, }\DecValTok{14}\NormalTok{, }\DecValTok{57}\NormalTok{, }
                        \DecValTok{78}\NormalTok{, }\DecValTok{37}\NormalTok{, }\DecValTok{29}\NormalTok{, }\DecValTok{64}\NormalTok{), }\AttributeTok{nrow =} \DecValTok{4}\NormalTok{, }\AttributeTok{byrow =} \ConstantTok{TRUE}\NormalTok{)}
\NormalTok{testmat.b}
\CommentTok{\#\textgreater{}      [,1] [,2] [,3]}
\CommentTok{\#\textgreater{} [1,]   17   40   20}
\CommentTok{\#\textgreater{} [2,]   34   21   12}
\CommentTok{\#\textgreater{} [3,]   14   57   78}
\CommentTok{\#\textgreater{} [4,]   37   29   64}
\end{Highlighting}
\end{Shaded}

Comment on the difference between \texttt{testmat.a} and \texttt{testmat.b}.

\begin{Shaded}
\begin{Highlighting}[]
\NormalTok{testmat.a[}\DecValTok{2}\NormalTok{,}\DecValTok{3}\NormalTok{]   }\CommentTok{\# Two index matrix reference}
\CommentTok{\#\textgreater{} [1] 37}
\NormalTok{testmat.a[}\DecValTok{10}\NormalTok{]   }\CommentTok{\# One index matrix reference}
\CommentTok{\#\textgreater{} [1] 37}
\end{Highlighting}
\end{Shaded}

\begin{enumerate}
\def\labelenumi{(\alph{enumi})}
\setcounter{enumi}{4}
\item
  Write a function to convert a one-index to a two-index matrix reference. Give an example of the usage of your function.
\item
  Write a function to convert a two-index to a one-index matrix reference. Give an example of the usage of your function.
\item
  Consider the following example to form submatrices:
\end{enumerate}

\begin{Shaded}
\begin{Highlighting}[]
\NormalTok{testmat }\OtherTok{\textless{}{-}} \FunctionTok{matrix}\NormalTok{(}\DecValTok{1}\SpecialCharTok{:}\DecValTok{50}\NormalTok{, }\AttributeTok{nrow =} \DecValTok{10}\NormalTok{, }\AttributeTok{byrow =} \ConstantTok{TRUE}\NormalTok{)}
\NormalTok{testmat[}\DecValTok{1}\SpecialCharTok{:}\DecValTok{2}\NormalTok{, }\FunctionTok{c}\NormalTok{ (}\DecValTok{3}\NormalTok{, }\DecValTok{5}\NormalTok{)]}
\CommentTok{\#\textgreater{}      [,1] [,2]}
\CommentTok{\#\textgreater{} [1,]    3    5}
\CommentTok{\#\textgreater{} [2,]    8   10}
\NormalTok{testmat[}\DecValTok{1}\SpecialCharTok{:}\DecValTok{2}\NormalTok{, }\DecValTok{3}\NormalTok{]}
\CommentTok{\#\textgreater{} [1] 3 8}
\NormalTok{testmat[}\DecValTok{1}\SpecialCharTok{:}\DecValTok{2}\NormalTok{, }\DecValTok{3}\NormalTok{, drop}\OtherTok{=}\ConstantTok{FALSE}\NormalTok{]}
\CommentTok{\#\textgreater{}      [,1]}
\CommentTok{\#\textgreater{} [1,]    3}
\CommentTok{\#\textgreater{} [2,]    8}
\end{Highlighting}
\end{Shaded}

\begin{enumerate}
\def\labelenumi{(\alph{enumi})}
\setcounter{enumi}{7}
\item
  Notice the difference between \texttt{testmat\ {[}1:2,\ 3{]}} and \texttt{testmat\ {[}1:2,\ 3,\ drop\ =\ FALSE{]}}. The first command results in the output to be given in the form of a vector while the optional \texttt{drop\ =\ FALSE} in the second command retains the matrix structure of the output. This distinction can have serious consequences when a procedure expects a matrix argument and not a vector.
\item
  Notice also that the output of both \texttt{testmat{[}1:2,3{]}} and \texttt{testmat{[}3,\ 1:2{]}} has a similar form: R makes no distinction between column vectors and row vectors; all one-dimensional collections of numbers are treated identically.
\item
  Apart from using vectors as subscripts to a matrix, a matrix can also be used as a subscript to a matrix. There are two cases:

  \begin{enumerate}
  \def\labelenumii{(\Alph{enumii})}
  \tightlist
  \item
    a numeric subscripting matrix and
  \item
    a logical subscripting matrix.
  \end{enumerate}
\end{enumerate}

\subsubsection*{Case A}\label{case-a}
\addcontentsline{toc}{subsubsection}{Case A}

Here the subscripting numeric matrix must have exactly two columns: the first provide row indices and the second column indices.

\begin{enumerate}
\def\labelenumi{(\roman{enumi})}
\item
  If used on the right-hand side of an expression the result of a \emph{case A} subscripting is a vector containing the values specified by the subscripting matrix.
\item
  If used on the left-hand side of an assignment a numeric matrix first selects those elements specified by its row and column indices; then these values are replaced one by one with the objects specified by the right-hand side of the assignment.
\end{enumerate}

Here is an example of \emph{case A} subscripting with the subscript matrix on the right-hand side of the assignment:

\begin{Shaded}
\begin{Highlighting}[]
\NormalTok{xmat }\OtherTok{\textless{}{-}} \FunctionTok{matrix}\NormalTok{ (}\DecValTok{1}\SpecialCharTok{:}\DecValTok{25}\NormalTok{, }\AttributeTok{nrow =} \DecValTok{5}\NormalTok{)}
\NormalTok{xmat}
\CommentTok{\#\textgreater{}      [,1] [,2] [,3] [,4] [,5]}
\CommentTok{\#\textgreater{} [1,]    1    6   11   16   21}
\CommentTok{\#\textgreater{} [2,]    2    7   12   17   22}
\CommentTok{\#\textgreater{} [3,]    3    8   13   18   23}
\CommentTok{\#\textgreater{} [4,]    4    9   14   19   24}
\CommentTok{\#\textgreater{} [5,]    5   10   15   20   25}
\NormalTok{superdiag.index }\OtherTok{\textless{}{-}} \FunctionTok{matrix}\NormalTok{ (}\FunctionTok{c}\NormalTok{ (}\DecValTok{1}\SpecialCharTok{:}\DecValTok{4}\NormalTok{, }\DecValTok{2}\SpecialCharTok{:}\DecValTok{5}\NormalTok{), }\AttributeTok{ncol =} \DecValTok{2}\NormalTok{, }\AttributeTok{byrow =} \ConstantTok{FALSE}\NormalTok{)}
\NormalTok{superdiag.values }\OtherTok{\textless{}{-}}\NormalTok{ xmat[superdiag.index]}
\NormalTok{superdiag.values}
\CommentTok{\#\textgreater{} [1]  6 12 18 24}
\end{Highlighting}
\end{Shaded}

\emph{Case A} subscripting with the numeric subscript matrix on the left-hand side of the assignment:

\begin{Shaded}
\begin{Highlighting}[]
\NormalTok{subscript.mat }\OtherTok{\textless{}{-}} \FunctionTok{matrix}\NormalTok{ (}\FunctionTok{c}\NormalTok{(}\DecValTok{1}\SpecialCharTok{:}\DecValTok{3}\NormalTok{, }\DecValTok{1}\SpecialCharTok{:}\DecValTok{3}\NormalTok{, }\FunctionTok{rep}\NormalTok{(}\DecValTok{1}\NormalTok{,}\DecValTok{3}\NormalTok{), }\FunctionTok{rep}\NormalTok{(}\DecValTok{2}\NormalTok{,}\DecValTok{3}\NormalTok{)), }\AttributeTok{ncol=}\DecValTok{2}\NormalTok{)}
\NormalTok{subscript.mat}
\CommentTok{\#\textgreater{}      [,1] [,2]}
\CommentTok{\#\textgreater{} [1,]    1    1}
\CommentTok{\#\textgreater{} [2,]    2    1}
\CommentTok{\#\textgreater{} [3,]    3    1}
\CommentTok{\#\textgreater{} [4,]    1    2}
\CommentTok{\#\textgreater{} [5,]    2    2}
\CommentTok{\#\textgreater{} [6,]    3    2}
\NormalTok{xx }\OtherTok{\textless{}{-}} \FunctionTok{matrix}\NormalTok{(}\ConstantTok{NA}\NormalTok{, }\AttributeTok{nrow=}\DecValTok{3}\NormalTok{,}\AttributeTok{ncol=}\DecValTok{2}\NormalTok{)}
\NormalTok{xx }
\CommentTok{\#\textgreater{}      [,1] [,2]}
\CommentTok{\#\textgreater{} [1,]   NA   NA}
\CommentTok{\#\textgreater{} [2,]   NA   NA}
\CommentTok{\#\textgreater{} [3,]   NA   NA}
\NormalTok{xx[subscript.mat] }\OtherTok{\textless{}{-}} \FunctionTok{c}\NormalTok{(}\DecValTok{10}\NormalTok{,}\DecValTok{12}\NormalTok{,}\DecValTok{14}\NormalTok{,}\DecValTok{100}\NormalTok{,}\DecValTok{120}\NormalTok{,}\DecValTok{140}\NormalTok{)}
\NormalTok{xx}
\CommentTok{\#\textgreater{}      [,1] [,2]}
\CommentTok{\#\textgreater{} [1,]   10  100}
\CommentTok{\#\textgreater{} [2,]   12  120}
\CommentTok{\#\textgreater{} [3,]   14  140}
\end{Highlighting}
\end{Shaded}

\subsubsection*{Case B}\label{case-b}
\addcontentsline{toc}{subsubsection}{Case B}

The logical subscripting matrix must be in size exactly similar to that matrix it is subscripting and will select those values corresponding to a \texttt{TRUE} in the subscripting matrix.

\emph{Case B} with logical subscripting matrix at right-hand side of assignment:

\begin{Shaded}
\begin{Highlighting}[]
\NormalTok{testmat}
\CommentTok{\#\textgreater{}       [,1] [,2] [,3] [,4] [,5]}
\CommentTok{\#\textgreater{}  [1,]    1    2    3    4    5}
\CommentTok{\#\textgreater{}  [2,]    6    7    8    9   10}
\CommentTok{\#\textgreater{}  [3,]   11   12   13   14   15}
\CommentTok{\#\textgreater{}  [4,]   16   17   18   19   20}
\CommentTok{\#\textgreater{}  [5,]   21   22   23   24   25}
\CommentTok{\#\textgreater{}  [6,]   26   27   28   29   30}
\CommentTok{\#\textgreater{}  [7,]   31   32   33   34   35}
\CommentTok{\#\textgreater{}  [8,]   36   37   38   39   40}
\CommentTok{\#\textgreater{}  [9,]   41   42   43   44   45}
\CommentTok{\#\textgreater{} [10,]   46   47   48   49   50}
\NormalTok{aa }\OtherTok{\textless{}{-}}\NormalTok{ testmat[testmat }\SpecialCharTok{\textless{}} \DecValTok{12}\NormalTok{]}
\NormalTok{aa}
\CommentTok{\#\textgreater{}  [1]  1  6 11  2  7  3  8  4  9  5 10}
\end{Highlighting}
\end{Shaded}

Note that the selected elements are placed column-wise in a vector.

\emph{Case B} with logical subscripting matrix at left-hand side of assignment:

\begin{Shaded}
\begin{Highlighting}[]
\NormalTok{testmat[testmat }\SpecialCharTok{\textless{}} \DecValTok{12}\NormalTok{] }\OtherTok{\textless{}{-}} \DecValTok{12}
\NormalTok{testmat}
\CommentTok{\#\textgreater{}       [,1] [,2] [,3] [,4] [,5]}
\CommentTok{\#\textgreater{}  [1,]   12   12   12   12   12}
\CommentTok{\#\textgreater{}  [2,]   12   12   12   12   12}
\CommentTok{\#\textgreater{}  [3,]   12   12   13   14   15}
\CommentTok{\#\textgreater{}  [4,]   16   17   18   19   20}
\CommentTok{\#\textgreater{}  [5,]   21   22   23   24   25}
\CommentTok{\#\textgreater{}  [6,]   26   27   28   29   30}
\CommentTok{\#\textgreater{}  [7,]   31   32   33   34   35}
\CommentTok{\#\textgreater{}  [8,]   36   37   38   39   40}
\CommentTok{\#\textgreater{}  [9,]   41   42   43   44   45}
\CommentTok{\#\textgreater{} [10,]   46   47   48   49   50}
\end{Highlighting}
\end{Shaded}

In order to restrict assignment to a subset of a matrix two sets of subscripts are needed. See example below:

\begin{Shaded}
\begin{Highlighting}[]
\NormalTok{testmat }\OtherTok{\textless{}{-}} \FunctionTok{matrix}\NormalTok{(}\DecValTok{1}\SpecialCharTok{:}\DecValTok{50}\NormalTok{, }\AttributeTok{nrow=}\DecValTok{10}\NormalTok{, }\AttributeTok{byrow=}\ConstantTok{TRUE}\NormalTok{)}
\NormalTok{testmat[, }\FunctionTok{c}\NormalTok{(}\DecValTok{1}\NormalTok{,}\DecValTok{3}\NormalTok{)][testmat[,}\FunctionTok{c}\NormalTok{(}\DecValTok{1}\NormalTok{,}\DecValTok{3}\NormalTok{)] }\SpecialCharTok{\textless{}}\DecValTok{12}\NormalTok{] }\OtherTok{\textless{}{-}} \DecValTok{12}
\NormalTok{testmat}
\CommentTok{\#\textgreater{}       [,1] [,2] [,3] [,4] [,5]}
\CommentTok{\#\textgreater{}  [1,]   12    2   12    4    5}
\CommentTok{\#\textgreater{}  [2,]   12    7   12    9   10}
\CommentTok{\#\textgreater{}  [3,]   12   12   13   14   15}
\CommentTok{\#\textgreater{}  [4,]   16   17   18   19   20}
\CommentTok{\#\textgreater{}  [5,]   21   22   23   24   25}
\CommentTok{\#\textgreater{}  [6,]   26   27   28   29   30}
\CommentTok{\#\textgreater{}  [7,]   31   32   33   34   35}
\CommentTok{\#\textgreater{}  [8,]   36   37   38   39   40}
\CommentTok{\#\textgreater{}  [9,]   41   42   43   44   45}
\CommentTok{\#\textgreater{} [10,]   46   47   48   49   50}
\end{Highlighting}
\end{Shaded}

Study the use of functions \texttt{row()} and \texttt{col()} in constructing logical matrices.

\section{Extracting elements of lists}\label{extracting-elements-of-lists}

\begin{enumerate}
\def\labelenumi{(\alph{enumi})}
\tightlist
\item
  Note the use of \texttt{list()} to collect objects into a list while elements are extracted with \texttt{\$}
\end{enumerate}

\begin{itemize}
\item
  the function \texttt{names()},
\item
  the single square brackets \texttt{{[}\ {]}} and
\item
  the double square brackets \texttt{{[}{[}\ {]}{]}}.
\end{itemize}

\begin{enumerate}
\def\labelenumi{(\alph{enumi})}
\setcounter{enumi}{1}
\tightlist
\item
  Study the following example carefully:
\end{enumerate}

\begin{Shaded}
\begin{Highlighting}[]
\NormalTok{my.list }\OtherTok{\textless{}{-}} \FunctionTok{list}\NormalTok{(}\AttributeTok{el1 =} \DecValTok{1}\SpecialCharTok{:}\DecValTok{5}\NormalTok{, }
                \AttributeTok{el2 =} \FunctionTok{c}\NormalTok{(}\StringTok{"a"}\NormalTok{, }\StringTok{"b"}\NormalTok{, }\StringTok{"c"}\NormalTok{), }
                \AttributeTok{el3 =} \FunctionTok{matrix}\NormalTok{(}\DecValTok{1}\SpecialCharTok{:}\DecValTok{16}\NormalTok{, }\AttributeTok{ncol =} \DecValTok{4}\NormalTok{), }
                \AttributeTok{el4 =} \FunctionTok{c}\NormalTok{(}\DecValTok{12}\NormalTok{, }\DecValTok{17}\NormalTok{, }\DecValTok{23}\NormalTok{, }\DecValTok{9}\NormalTok{))}
\NormalTok{my.list}
\CommentTok{\#\textgreater{} $el1}
\CommentTok{\#\textgreater{} [1] 1 2 3 4 5}
\CommentTok{\#\textgreater{} }
\CommentTok{\#\textgreater{} $el2}
\CommentTok{\#\textgreater{} [1] "a" "b" "c"}
\CommentTok{\#\textgreater{} }
\CommentTok{\#\textgreater{} $el3}
\CommentTok{\#\textgreater{}      [,1] [,2] [,3] [,4]}
\CommentTok{\#\textgreater{} [1,]    1    5    9   13}
\CommentTok{\#\textgreater{} [2,]    2    6   10   14}
\CommentTok{\#\textgreater{} [3,]    3    7   11   15}
\CommentTok{\#\textgreater{} [4,]    4    8   12   16}
\CommentTok{\#\textgreater{} }
\CommentTok{\#\textgreater{} $el4}
\CommentTok{\#\textgreater{} [1] 12 17 23  9}
\NormalTok{my.list}\SpecialCharTok{$}\NormalTok{el2}
\CommentTok{\#\textgreater{} [1] "a" "b" "c"}
\FunctionTok{mode}\NormalTok{ (my.list}\SpecialCharTok{$}\NormalTok{el2)}
\CommentTok{\#\textgreater{} [1] "character"}
\NormalTok{my.list[el2]}
\CommentTok{\#\textgreater{} Error: object \textquotesingle{}el2\textquotesingle{} not found}
\NormalTok{my.list[}\StringTok{"el2"}\NormalTok{]}
\CommentTok{\#\textgreater{} $el2}
\CommentTok{\#\textgreater{} [1] "a" "b" "c"}
\FunctionTok{mode}\NormalTok{ (my.list[}\StringTok{"el2"}\NormalTok{])}
\CommentTok{\#\textgreater{} [1] "list"}
\NormalTok{my.list[[}\StringTok{"el2"}\NormalTok{]]}
\CommentTok{\#\textgreater{} [1] "a" "b" "c"}
\FunctionTok{mode}\NormalTok{ (my.list[[}\StringTok{"el2"}\NormalTok{]])}
\CommentTok{\#\textgreater{} [1] "character"}
\end{Highlighting}
\end{Shaded}

Note: The above example shows that using the single pair of square brackets for subscripting a list always result in a list object to be returned. This is often the cause of an error message. See the example below.

\begin{Shaded}
\begin{Highlighting}[]
\NormalTok{my.list[}\DecValTok{1}\NormalTok{]}
\CommentTok{\#\textgreater{} $el1}
\CommentTok{\#\textgreater{} [1] 1 2 3 4 5}
\FunctionTok{mode}\NormalTok{ (my.list[}\DecValTok{1}\NormalTok{])}
\CommentTok{\#\textgreater{} [1] "list"}
\NormalTok{my.list[[}\DecValTok{1}\NormalTok{]]}
\CommentTok{\#\textgreater{} [1] 1 2 3 4 5}
\FunctionTok{mode}\NormalTok{ (my.list[[}\DecValTok{1}\NormalTok{]])}
\CommentTok{\#\textgreater{} [1] "numeric"}
\NormalTok{my.list[}\DecValTok{3}\NormalTok{][}\DecValTok{2}\NormalTok{,}\DecValTok{4}\NormalTok{]}
\CommentTok{\#\textgreater{} Error in my.list[3][2, 4]: incorrect number of dimensions}
\NormalTok{my.list[[}\DecValTok{3}\NormalTok{]][}\DecValTok{2}\NormalTok{,}\DecValTok{4}\NormalTok{]}
\CommentTok{\#\textgreater{} [1] 14}
\NormalTok{my.list}\SpecialCharTok{$}\NormalTok{el3[}\DecValTok{2}\NormalTok{,}\DecValTok{4}\NormalTok{]}
\CommentTok{\#\textgreater{} [1] 14}
\FunctionTok{mean}\NormalTok{ (my.list[}\DecValTok{4}\NormalTok{])}
\CommentTok{\#\textgreater{} Warning in mean.default(my.list[4]): argument is not}
\CommentTok{\#\textgreater{} numeric or logical: returning NA}
\CommentTok{\#\textgreater{} [1] NA}
\FunctionTok{mean}\NormalTok{ (my.list[[}\DecValTok{4}\NormalTok{]])}
\CommentTok{\#\textgreater{} [1] 15.25}
\FunctionTok{mean}\NormalTok{ (my.list}\SpecialCharTok{$}\NormalTok{el4)}
\CommentTok{\#\textgreater{} [1] 15.25}
\end{Highlighting}
\end{Shaded}

Explain the differences and similarities between the symbols \texttt{{[}\ {]}}, \texttt{{[}{[}\ {]}{]}} and \texttt{\$} when subscripting lists.

\section{Extracting elements from dataframes}\label{extracting-elements-from-dataframes}

\begin{enumerate}
\def\labelenumi{(\alph{enumi})}
\item
  Note the use of data.frame() for creating dataframes. A dataframe has a rectangular structure similar to a matrix but differs from a matrix in that its columns are not restricted to contain the same type of data. Each of its columns must contain the same sort of data but some columns can be numerical while others are factors for example.
\item
  Explain the difference between the objects created by the following two instructions:
\end{enumerate}

\begin{Shaded}
\begin{Highlighting}[]
\NormalTok{my.matrix }\OtherTok{\textless{}{-}} \FunctionTok{matrix}\NormalTok{ (}\FunctionTok{c}\NormalTok{ (}\DecValTok{17}\NormalTok{, }\DecValTok{40}\NormalTok{, }\DecValTok{20}\NormalTok{, }\DecValTok{34}\NormalTok{, }\DecValTok{21}\NormalTok{, }\DecValTok{12}\NormalTok{, }\DecValTok{14}\NormalTok{, }\DecValTok{57}\NormalTok{,}
                        \DecValTok{78}\NormalTok{, }\DecValTok{37}\NormalTok{, }\DecValTok{29}\NormalTok{, }\DecValTok{64}\NormalTok{), }\AttributeTok{nrow =} \DecValTok{4}\NormalTok{, }\AttributeTok{ncol =} \DecValTok{3}\NormalTok{)}
\NormalTok{my.dataframe }\OtherTok{\textless{}{-}} \FunctionTok{data.frame}\NormalTok{ ( }\FunctionTok{c}\NormalTok{(}\DecValTok{17}\NormalTok{, }\DecValTok{40}\NormalTok{, }\DecValTok{20}\NormalTok{, }\DecValTok{34}\NormalTok{, }\DecValTok{21}\NormalTok{, }\DecValTok{12}\NormalTok{, }\DecValTok{14}\NormalTok{, }\DecValTok{57}\NormalTok{,}
                               \DecValTok{78}\NormalTok{, }\DecValTok{37}\NormalTok{, }\DecValTok{29}\NormalTok{, }\DecValTok{64}\NormalTok{), }\AttributeTok{nrow =} \DecValTok{4}\NormalTok{, }\AttributeTok{ncol =} \DecValTok{3}\NormalTok{)}
\end{Highlighting}
\end{Shaded}

\begin{enumerate}
\def\labelenumi{(\alph{enumi})}
\setcounter{enumi}{2}
\tightlist
\item
  Note the following
\end{enumerate}

\begin{Shaded}
\begin{Highlighting}[]
\FunctionTok{class}\NormalTok{(my.matrix)}
\CommentTok{\#\textgreater{} [1] "matrix" "array"}
\FunctionTok{class}\NormalTok{(my.dataframe)}
\CommentTok{\#\textgreater{} [1] "data.frame"}
\FunctionTok{is.list}\NormalTok{(data.frame)}
\CommentTok{\#\textgreater{} [1] FALSE}
\FunctionTok{mode}\NormalTok{(my.matrix)}
\CommentTok{\#\textgreater{} [1] "numeric"}
\FunctionTok{mode}\NormalTok{(data.frame)}
\CommentTok{\#\textgreater{} [1] "function"}
\end{Highlighting}
\end{Shaded}

\begin{enumerate}
\def\labelenumi{(\alph{enumi})}
\setcounter{enumi}{3}
\tightlist
\item
  A sample of the behaviour of dataframes
\end{enumerate}

\begin{Shaded}
\begin{Highlighting}[]
\NormalTok{my.dataframe}\FloatTok{.2} \OtherTok{\textless{}{-}} \FunctionTok{data.frame}\NormalTok{ (}\AttributeTok{C1 =} \FunctionTok{c}\NormalTok{(}\StringTok{\textquotesingle{}a\textquotesingle{}}\NormalTok{, }\StringTok{\textquotesingle{}b\textquotesingle{}}\NormalTok{, }\StringTok{\textquotesingle{}c\textquotesingle{}}\NormalTok{, }\StringTok{\textquotesingle{}d\textquotesingle{}}\NormalTok{), }
                              \AttributeTok{C2 =} \FunctionTok{c}\NormalTok{(}\DecValTok{5}\NormalTok{, }\DecValTok{9}\NormalTok{, }\DecValTok{23}\NormalTok{, }\DecValTok{17}\NormalTok{), }
                              \AttributeTok{C3 =} \FunctionTok{c}\NormalTok{(}\ConstantTok{TRUE}\NormalTok{, }\ConstantTok{TRUE}\NormalTok{, }\ConstantTok{FALSE}\NormalTok{, }\ConstantTok{TRUE}\NormalTok{))}
\NormalTok{my.dataframe}\FloatTok{.2}
\CommentTok{\#\textgreater{}   C1 C2    C3}
\CommentTok{\#\textgreater{} 1  a  5  TRUE}
\CommentTok{\#\textgreater{} 2  b  9  TRUE}
\CommentTok{\#\textgreater{} 3  c 23 FALSE}
\CommentTok{\#\textgreater{} 4  d 17  TRUE}
\NormalTok{my.dataframe}\FloatTok{.2}\NormalTok{[ ,}\DecValTok{1}\SpecialCharTok{:}\DecValTok{2}\NormalTok{]}
\CommentTok{\#\textgreater{}   C1 C2}
\CommentTok{\#\textgreater{} 1  a  5}
\CommentTok{\#\textgreater{} 2  b  9}
\CommentTok{\#\textgreater{} 3  c 23}
\CommentTok{\#\textgreater{} 4  d 17}
\end{Highlighting}
\end{Shaded}

Dataframe behaves like a matrix

\begin{Shaded}
\begin{Highlighting}[]
\NormalTok{my.dataframe}\FloatTok{.2}\SpecialCharTok{$}\NormalTok{C1}
\CommentTok{\#\textgreater{} [1] "a" "b" "c" "d"}
\end{Highlighting}
\end{Shaded}

Dataframe behaves like a list

\begin{Shaded}
\begin{Highlighting}[]
\FunctionTok{as.matrix}\NormalTok{(my.dataframe}\FloatTok{.2}\NormalTok{)}
\CommentTok{\#\textgreater{}      C1  C2   C3     }
\CommentTok{\#\textgreater{} [1,] "a" " 5" "TRUE" }
\CommentTok{\#\textgreater{} [2,] "b" " 9" "TRUE" }
\CommentTok{\#\textgreater{} [3,] "c" "23" "FALSE"}
\CommentTok{\#\textgreater{} [4,] "d" "17" "TRUE"}
\end{Highlighting}
\end{Shaded}

Explain what has happened above.

\begin{enumerate}
\def\labelenumi{(\alph{enumi})}
\setcounter{enumi}{4}
\item
  The above examples show that a dataframe can be considered as a cross between a matrix and a list. Therefore, subscripting of dataframes generally can be performed using the basic techniques available for matrices and lists.
\item
  An alternative technique is to extract the elements of a list by using the functions \texttt{attach()} and \texttt{names()}. This technique is especially of importance in statistical modelling. What is a potential danger of this technique when attaching dataframes? This danger can be avoided by using \texttt{with()}. Is this also true when modelling is performed?
\item
  Review section \ref{findData}. Study the help file of the function \texttt{with()}. What important usage has \texttt{with()}?
\end{enumerate}

\section{Combining vectors, matrices, lists and dataframes}\label{combining-vectors-matrices-lists-and-dataframes}

\begin{enumerate}
\def\labelenumi{(\alph{enumi})}
\tightlist
\item
  What is the result of the command
\end{enumerate}

\begin{Shaded}
\begin{Highlighting}[]
\NormalTok{my.list }\OtherTok{\textless{}{-}} \FunctionTok{vector}\NormalTok{ (}\StringTok{"list"}\NormalTok{, k)?}
\end{Highlighting}
\end{Shaded}

\begin{enumerate}
\def\labelenumi{(\alph{enumi})}
\setcounter{enumi}{1}
\item
  Recall the function \texttt{c()} for creating vectors. When \texttt{c()} is used to combine a numeric vector and a character vector the result is a vector of mode ``character''. Similarly, using \texttt{c()} to combine a vector with a list results in a list.
\item
  If \texttt{list()} is used to combine two or more vectors or lists the result is a list of all the objects.
\item
  The function \texttt{unlist()} can be used to convert all the elements of a list into a single vector.
\end{enumerate}

\begin{Shaded}
\begin{Highlighting}[]
\NormalTok{my.list}
\CommentTok{\#\textgreater{} $el1}
\CommentTok{\#\textgreater{} [1] 1 2 3 4 5}
\CommentTok{\#\textgreater{} }
\CommentTok{\#\textgreater{} $el2}
\CommentTok{\#\textgreater{} [1] "a" "b" "c"}
\CommentTok{\#\textgreater{} }
\CommentTok{\#\textgreater{} $el3}
\CommentTok{\#\textgreater{}      [,1] [,2] [,3] [,4]}
\CommentTok{\#\textgreater{} [1,]    1    5    9   13}
\CommentTok{\#\textgreater{} [2,]    2    6   10   14}
\CommentTok{\#\textgreater{} [3,]    3    7   11   15}
\CommentTok{\#\textgreater{} [4,]    4    8   12   16}
\CommentTok{\#\textgreater{} }
\CommentTok{\#\textgreater{} $el4}
\CommentTok{\#\textgreater{} [1] 12 17 23  9}
\FunctionTok{unlist}\NormalTok{(my.list)}
\CommentTok{\#\textgreater{}  el11  el12  el13  el14  el15  el21  el22  el23  el31  el32 }
\CommentTok{\#\textgreater{}   "1"   "2"   "3"   "4"   "5"   "a"   "b"   "c"   "1"   "2" }
\CommentTok{\#\textgreater{}  el33  el34  el35  el36  el37  el38  el39 el310 el311 el312 }
\CommentTok{\#\textgreater{}   "3"   "4"   "5"   "6"   "7"   "8"   "9"  "10"  "11"  "12" }
\CommentTok{\#\textgreater{} el313 el314 el315 el316  el41  el42  el43  el44 }
\CommentTok{\#\textgreater{}  "13"  "14"  "15"  "16"  "12"  "17"  "23"   "9"}
\end{Highlighting}
\end{Shaded}

Explain the above output.

\begin{enumerate}
\def\labelenumi{(\alph{enumi})}
\setcounter{enumi}{4}
\tightlist
\item
  Review the functions \texttt{cbind()}, \texttt{rbind()}, \texttt{append()}, \texttt{data.frame()}, \texttt{dim()}, \texttt{dimnames()}, \texttt{names()}, \texttt{colnames()}, \texttt{rownames()}, \texttt{nrow()} and \texttt{ncol()}.
\end{enumerate}

\section{Rearranging the elements in a matrix}\label{rearranging-the-elements-in-a-matrix}

Study the usage of the functions \texttt{matrix()}, \texttt{t()} and \texttt{diag()}. These functions are useful to form submatrices of a matrix or to rearrange matrix elements. Note again the argument \texttt{byrow\ =} of \texttt{matrix()}.

\section{Exercise}\label{exercise-10}

\begin{enumerate}
\def\labelenumi{\arabic{enumi}.}
\item
  Write an R function to check if a given matrix is symmetric.
\item
  Write an R function to extract (i) the row(s) and (ii) the columns containing the maximum value in the matrix. Note that provision must be made that the maximum value can occur in more than one row (column). Furthermore, both the indices and actual values of the rows (columns) must be returned. Illustrate the usage of your function with a suitable example.
\item
  Describe the variables in the built-in data set \texttt{LifeCycleSavings}. Is this data set in the form of a matrix or a dataframe?
\item
  Use subscripting to find the largest proportion of over 75 in those countries with a dpi of less than 1000 in the \texttt{LifeCycleSavings} data set. Also determine the country(ies) having this pop75 value.
\item
  Consider the \texttt{LifeCycleSavings} data set.

  \begin{enumerate}
  \def\labelenumii{(\roman{enumii})}
  \tightlist
  \item
    Use subscripting to find the mean aggregate savings for countries with a percentage of the population younger than 15 at least 10 times the percentage of the population over 75.
  \item
    Also find the mean aggregate savings for countries where the above ratio is less than 10.
  \item
    Use function \texttt{t.test()} to test if mean aggregate savings are different for the above two groups.
  \item
    Use notched box plots for an approximate test.
  \end{enumerate}
\end{enumerate}

\begin{enumerate}
\def\labelenumi{(\alph{enumi})}
\setcounter{enumi}{21}
\tightlist
\item
  First, carefully study the output obtained in (iii) and (iv). Then interpret/discuss this output in detail.
\end{enumerate}

\begin{enumerate}
\def\labelenumi{\arabic{enumi}.}
\setcounter{enumi}{5}
\tightlist
\item
  Consider the \texttt{state.x77} data set and the variable \texttt{state.region}. Find the state with the minimum income in each of the regions defined in state.region.
\end{enumerate}

\chapter{Revision tasks}\label{revision}

In general, the purpose of writing a program in R is to address some practical problem directly or indirectly. To prepare the student for seriously writing R functions (programs) this chapter consists of a mixture of revision tasks. While some of these tasks are straight forward others need more thought and preparation before starting with the writing of R code. In Section \ref{guidelines} some guidelines are considered for writing R code to address a practical problem.

\section{Guidelines for problem solving by writing R code}\label{guidelines}

\begin{enumerate}
\def\labelenumi{(\alph{enumi})}
\item
  Make sure the problem is clearly understood. You cannot write good code for something that is not correctly grasped.
\item
  Break complex problems into simpler components. Formulate these simpler components in terms of specific questions to be answered.
\item
  Think in terms of the way R operates e.g.~vectorized arithmetic, recycling principle, operating on objects as wholes/units, subscripting, R data structures . . .
\item
  Spend time to prepare your data.
\item
  Ask yourself the question what information do you need before attempting to write code for coming up with an answer. Then, what facilities are provided in R to get the necessary information and once the information is available what manipulations are needed to code useful output.
\item
  Write dedicated code for answering the specific questions in (b).
\item
  Do not neglect the debugging/optimizing phase of code that succeeds in providing a first round answer.
\end{enumerate}

\section{Exercise}\label{exercise-11}

\begin{enumerate}
\def\labelenumi{\arabic{enumi}.}
\item
  Use R to obtain a five-point summary of the variable \texttt{dpi} in the \texttt{LifeCycleSavings} data set. Illustrate the difference between the working of \texttt{fivenum()} and \texttt{quantile()}. \emph{Hint}: See \texttt{boxplot.stats()} for the definition of hinges.
\item
  Display the pdf of a \(normal (100, 15)\) distribution graphically. The area under the density bounded by the 70th and 90th percentiles must appear in red.
\item
  Use R to obtain the following graphical representations:

  \begin{enumerate}
  \def\labelenumii{(\roman{enumii})}
  \item
    The pdf as well as the cdf of a \(F (15, 10)\) and a \(F (10, 15)\) stochastic variable. These graphs must be on one graph window with the same set of axes for both F-distributions and be supplied with suitable titles. Furthermore, they must be line graphs that contain no other plotting characters except lines.
  \item
    Obtain representations as line graphs of the inverses of the above cdfs on a single separate graph page.
  \end{enumerate}
\item
  First set the seed to 172389 and then generate a random sample of size 500 from a \(normal (100, 20)\) distribution. Give the necessary R instructions to determine the class frequencies in the class intervals ``Smaller than 50'', ``50 to 75--``, ``75 to 90--``, ``90 to 100'', ``100+ to 110'', ``Larger than 110''.
\item
  Generate a random sample of size 80 from a bivariate normal distribution with mean vector \((50, 100)\). The variances of the two variables are 900 and 2500 respectively with a correlation 0.90. Store the sample in an R matrix object and obtain a scatterplot in the form of

  \begin{enumerate}
  \def\labelenumii{(\roman{enumii})}
  \tightlist
  \item
    a point diagram and
  \item
    a line graph of the sample.
  \end{enumerate}
\item
  Define the harmonic mean for a vector of observations. What conditions must be satisfied by the observations?

  \begin{enumerate}
  \def\labelenumii{(\roman{enumii})}
  \item
    Write your own function for calculating a harmonic mean and use it to calculate the harmonic mean of variable \texttt{dpi} in the \texttt{LifeCycleSavings} data set.
  \item
    Calculate the ordinary mean of variable \texttt{dpi} in the \texttt{LifeCycleSavings} data set. Compare the answer with the answer in (a). Which answer would you use in practice? Motivate.
  \end{enumerate}
\item
  Fisher's linear discriminant function in the case of two groups is defined as follows:
\end{enumerate}

\(LDF = (\mathbf{\bar{x}}_1 - \mathbf{\bar{x}}_2)' \mathbf{S}^{-1} \mathbf{x}\) where \(\mathbf{S} = [(n_1-1)\mathbf{S}_1 + (n_2-1)\mathbf{S}_2]/(n_1 + n_2 - 2)\) with \(\mathbf{\bar{x}}_i\) and \(\mathbf{S}_i\) the vector of means and the covariance matrix of the \(i\)th group (sample), respectively.

The corresponding classification function is written as \(CF =(\mathbf{\bar{x}}_1 - \mathbf{\bar{x}}_2)' \mathbf{S}^{-1} \mathbf{x} - \frac{1}{2} (\mathbf{\bar{x}}_1 - \mathbf{\bar{x}}_2)' \mathbf{S}^{-1} (\mathbf{\bar{x}}_1 + \mathbf{\bar{x}}_2)\). The expression \((\mathbf{\bar{x}}_1 - \mathbf{\bar{x}}_2)' \mathbf{S}^{-1}\) is referred to as the discriminant coefficients.

In agreement with section \ref{guidelines} make sure what an \(LDF\) and a \(CF\) entail. The \texttt{crabs} data set in package \texttt{MASS} consists of 200 rows and 8 columns, describing 5 morphological measurements on 50 crabs each of two colour forms and both sexes, of the species \emph{Leptograpsus variegatus} collected at Fremantle, Western Australia.

\begin{enumerate}
\def\labelenumi{(\roman{enumi})}
\item
  Obtain the covariance matrix for each of the two species of crabs.
\item
  Obtain the vector of means for each of the two species of crabs.
\item
  Use standard R functions operating on matrices to write a function or code that calculates the discriminant coefficients for the given linear discriminant function.
\item
  Write a function that determines the linear discriminant function and return
\end{enumerate}

\begin{itemize}
\item
  the discriminant coefficients;
\item
  The CF for each observation.
\end{itemize}

\begin{enumerate}
\def\labelenumi{(\alph{enumi})}
\setcounter{enumi}{21}
\tightlist
\item
  Repeat the discriminant analysis above, discriminating between male and female crabs, ignoring differences in species.
\end{enumerate}

\begin{enumerate}
\def\labelenumi{(\roman{enumi})}
\setcounter{enumi}{5}
\tightlist
\item
  Compare your results to using the \texttt{lda()} function in the package \texttt{MASS} with the command
\end{enumerate}

\begin{Shaded}
\begin{Highlighting}[]
\FunctionTok{predict}\NormalTok{ (}\FunctionTok{lda}\NormalTok{ (sex }\SpecialCharTok{\textasciitilde{}}\NormalTok{ FL }\SpecialCharTok{+}\NormalTok{ RW }\SpecialCharTok{+}\NormalTok{ CL }\SpecialCharTok{+}\NormalTok{ CW }\SpecialCharTok{+}\NormalTok{ BD, }\AttributeTok{data=}\NormalTok{crabs))}\SpecialCharTok{$}\NormalTok{class}
\end{Highlighting}
\end{Shaded}

\begin{enumerate}
\def\labelenumi{\arabic{enumi}.}
\setcounter{enumi}{7}
\tightlist
\item
  Consider the matrix \(\mathbf{A}:n \times m\). What is understood by the column space \(V(\mathbf{A})\) and the orthogonal complement \(V^⊥(\mathbf{A})\)? The R function \texttt{svd()} can be used to obtain an orthogonal basis for \(V(\mathbf{A})\) when the rank of \(\mathbf{A}\) is \(k\). We also want to determine an orthogonal basis for \(V^⊥(\mathbf{A})\). How can the function \texttt{svd()} be used to simultaneously find a basis for \(V(\mathbf{A})\) and for \(V^⊥(\mathbf{A})\)?
\end{enumerate}

The above propositions can be proved as follows: Assume that \(n≥m\) and that an orthonormal basis for \(V(\mathbf{A})\) as well as for \(V^⊥(\mathbf{A})\) must be found. Append \(n-m\) zero vectors of size \(n\) to the matrix \(\mathbf{A}\). Write \(\mathbf{A}^0\) for the appended matrix and perform the function \texttt{svd()} on \(\mathbf{A}^0\). It follows that \(\mathbf{A}^0 = \mathbf{UDV}'\) so that \(\mathbf{A}^0 \mathbf{V} = \mathbf{UD}\), i.e.~

\[
\begin{bmatrix} 
\mathbf{A}^0 \mathbf{v}_{(1)} & \mathbf{A}^0 \mathbf{v}_{(2)} & \dots & \mathbf{A}^0 \mathbf{v}_{(n)}
\end{bmatrix} = \begin{bmatrix} 
d_1 \mathbf{u}_{(1)} & d_2 \mathbf{u}_{(2)} & \dots & d_n \mathbf{u}_{(n)}
\end{bmatrix}.
\]
Now \(\mathbf{A}^0 \mathbf{v}_{(i)} \in V(\mathbf{A}^0) = V(\mathbf{A})\). (\emph{Motivate in detail}.) It follows that \(\mathbf{u}_{(i)} \in V(\mathbf{A}), i = 1, 2, \dots, k\) . (\emph{Motivate in detail}.) Therefore the columns of \(\mathbf{U}\) that correspond to the non-zero \(d\)s form an orthonormal basis for \(V(\mathbf{A})\) while the columns of \(\mathbf{U}\) that correspond to the zero \(d\)s form an orthonormal basis for the orthogonal complement of \(V(\mathbf{A})\). Motivate the last statement in detail.

\begin{enumerate}
\def\labelenumi{\arabic{enumi}.}
\setcounter{enumi}{8}
\tightlist
\item
  Based on the results in (8) above, write an R function that returns \(rank(\mathbf{A})\), an orthogonal basis for \(V(\mathbf{A})\) and an orthogonal basis for \(V^⊥(\mathbf{A})\). Test your function on the matrix:
\end{enumerate}

\[
\mathbf{A} = \begin{bmatrix} 
                    1 & 1 & 2 \\
                    2 & 2 & 4 \\
                    3 & 2 & 7 \\
                    -1 & -5 & 2 \\
                    2 & 7 & -1
              \end{bmatrix} 
\]

\begin{enumerate}
\def\labelenumi{\arabic{enumi}.}
\setcounter{enumi}{9}
\tightlist
\item
  In many graphical displays whose purpose it is to represent distances in two dimensions, it is essential that the scales of the axes are geometrically accurate. This is called the aspect ratio of the graph and the R graphics parameter \texttt{par} is used for controlling the aspect ratio of graphics in R. The default value of \texttt{par} generally does not ensure that the scales of the horizontal and vertical axes are geometrically accurate. For ensuring geometrically accurate scales the setting \texttt{asp\ =\ 1} must be explicitly specified e.g.~\texttt{plot(x\ =,\ y\ =,\ asp\ =\ 1)}.
\end{enumerate}

We are going to investigate the effect of the aspect ratio on graphs by writing our own function for drawing a circle. In agreement with section \ref{guidelines} we will start our project by reviewing some basic concepts regarding coordinates for graphical purposes. Figure \ref{fig:coordinates} summarizes how to reference a point in geometric space by using (a) Cartesian coordinates and (b) polar coordinates.

\begin{figure}
\includegraphics[width=1\linewidth]{pics/coordinates} \caption{Cartesian and polar coordinates for referencing a point on a graph.}\label{fig:coordinates}
\end{figure}

\begin{enumerate}
\def\labelenumi{(\roman{enumi})}
\tightlist
\item
  Consider the following function for drawing a circle with a specified radius and centred at the origin:
\end{enumerate}

\begin{Shaded}
\begin{Highlighting}[]
\NormalTok{my.circle }\OtherTok{\textless{}{-}} \ControlFlowTok{function}\NormalTok{ (}\AttributeTok{r =} \DecValTok{1}\NormalTok{, }\AttributeTok{xrange =} \SpecialCharTok{{-}}\DecValTok{2}\SpecialCharTok{:}\DecValTok{2}\NormalTok{, }\AttributeTok{yrange =} \SpecialCharTok{{-}}\DecValTok{2}\SpecialCharTok{:}\DecValTok{2}\NormalTok{) }
\NormalTok{\{ }\FunctionTok{plot}\NormalTok{ (}\AttributeTok{x =}\NormalTok{ xrange, }\AttributeTok{y =}\NormalTok{ yrange, }\AttributeTok{type =} \StringTok{\textquotesingle{}n\textquotesingle{}}\NormalTok{, }\AttributeTok{xlab =} \StringTok{\textquotesingle{}\textquotesingle{}}\NormalTok{, }\AttributeTok{ylab =} \StringTok{\textquotesingle{}\textquotesingle{}}\NormalTok{,}
        \AttributeTok{xaxt =} \StringTok{\textquotesingle{}n\textquotesingle{}}\NormalTok{, }\AttributeTok{yaxt =} \StringTok{\textquotesingle{}n\textquotesingle{}}\NormalTok{)}
\NormalTok{  theta }\OtherTok{\textless{}{-}} \FunctionTok{seq}\NormalTok{(}\AttributeTok{from =} \DecValTok{0}\NormalTok{, }\AttributeTok{to =} \DecValTok{2} \SpecialCharTok{*}\NormalTok{ pi, }\AttributeTok{by =} \FloatTok{0.01}\NormalTok{) }
  \CommentTok{\# Notice the use of radians.}
  \FunctionTok{lines}\NormalTok{ (}\AttributeTok{x =}\NormalTok{ r}\SpecialCharTok{*}\FunctionTok{cos}\NormalTok{(theta), }\AttributeTok{y =}\NormalTok{ r}\SpecialCharTok{*}\FunctionTok{sin}\NormalTok{(theta))}
  \FunctionTok{abline}\NormalTok{(}\AttributeTok{h =} \DecValTok{0}\NormalTok{)}
  \FunctionTok{abline}\NormalTok{(}\AttributeTok{v =} \DecValTok{0}\NormalTok{)}
\NormalTok{\}}
\end{Highlighting}
\end{Shaded}

Run the above function and consider the graph window. Increase and decrease the size of the graph window by dragging its edges. Does the figure look like a circle?

\begin{enumerate}
\def\labelenumi{(\roman{enumi})}
\setcounter{enumi}{1}
\item
  Next, add the argument \texttt{asp\ =\ 1} to the call to \texttt{plot} in \texttt{my.circle}. Run the changed function; change the size of the graph window. What happens?
\item
  What changes are necessary for producing a circle centred at any point in a geometrical space? Make the necessary changes in \texttt{my.circle()} for constructing a circle centred at any user specified point on a graph.
\end{enumerate}

\begin{enumerate}
\def\labelenumi{\arabic{enumi}.}
\setcounter{enumi}{10}
\item
  What is understood by a p-dimensional ellipsoid?

  \begin{enumerate}
  \def\labelenumii{(\roman{enumii})}
  \item
    Give a mathematical expression in matrix notation that describes an ellipsoid in p dimensions.
  \item
    Describe the axes of the ellipsoid in terms of eigenvalues and eigenvectors.
  \item
    Let \(p = 2\). Simplify the expression for the ellipse concerned in terms of scalar quantities.
  \item
    Use \texttt{plot()} and write an R-function to draw an ellipse. Make provision for the centre point to be at \((0, 0)\) as well as at an arbitrary \((x_1,x_2)\) point; for no correlation between the two variables as well as for positive and negative correlation.
  \item
    Use your function written in (iv) to illustrate differences between plot (using the default value of argument \texttt{asp}) and plot with \texttt{asp=1}.
  \end{enumerate}
\item
  During experimental design it is often useful to predict the value of the dependent variable at every combination of the levels of the factor variables. Write an R function for this task that makes provision for any number of factor arguments and that also provides a dataframe with the factors as the columns and every combination of levels as the rows. Every levels-combination can only appear once. The function must be user friendly and must test if a given independent variable is a factor variable. \emph{Hint}: Study the help file of \texttt{expand.grid()}.
\item
  Consider the following game. You are given a computer screen containing a rectangle filled at random with evenly spaced letters. Repetitions of the same letter are allowed. The challenge to the user is to sequentially select the first \(n\) letters of the alphabet as quickly as possible. The user must read each line from left to right and from top to bottom. Going backwards is not allowed. The time to complete the task is taken as well as whether the rules have been obeyed. Program an R version of this game.
\end{enumerate}

\chapter{Writing functions in R}\label{functions}

Although we have already written various functions in R, in this chapter the writing of R functions will be approached systematically.

\section{General}\label{general-2}

A good way to learn about functions or to write a new function is to look at existing ones. As an example consider that we would like to write a function to implement a novel plotting procedure. So we start by taking a look at the existing \texttt{plot} function.

\begin{Shaded}
\begin{Highlighting}[]
\NormalTok{plot}
\CommentTok{\#\textgreater{} function (x, y, ...) }
\CommentTok{\#\textgreater{} UseMethod("plot")}
\CommentTok{\#\textgreater{} \textless{}bytecode: 0x00000107d1c9d028\textgreater{}}
\CommentTok{\#\textgreater{} \textless{}environment: namespace:base\textgreater{}}
\end{Highlighting}
\end{Shaded}

This is not very helpful so we give the instruction:

\begin{Shaded}
\begin{Highlighting}[]
\FunctionTok{methods}\NormalTok{(plot)}
\CommentTok{\#\textgreater{}  [1] plot.acf*           plot.data.frame*   }
\CommentTok{\#\textgreater{}  [3] plot.decomposed.ts* plot.default       }
\CommentTok{\#\textgreater{}  [5] plot.dendrogram*    plot.density*      }
\CommentTok{\#\textgreater{}  [7] plot.ecdf           plot.factor*       }
\CommentTok{\#\textgreater{}  [9] plot.formula*       plot.function      }
\CommentTok{\#\textgreater{} [11] plot.hclust*        plot.histogram*    }
\CommentTok{\#\textgreater{} [13] plot.HoltWinters*   plot.isoreg*       }
\CommentTok{\#\textgreater{} [15] plot.lm*            plot.medpolish*    }
\CommentTok{\#\textgreater{} [17] plot.mlm*           plot.ppr*          }
\CommentTok{\#\textgreater{} [19] plot.prcomp*        plot.princomp*     }
\CommentTok{\#\textgreater{} [21] plot.profile*       plot.profile.nls*  }
\CommentTok{\#\textgreater{} [23] plot.raster*        plot.spec*         }
\CommentTok{\#\textgreater{} [25] plot.stepfun        plot.stl*          }
\CommentTok{\#\textgreater{} [27] plot.table*         plot.ts            }
\CommentTok{\#\textgreater{} [29] plot.tskernel*      plot.TukeyHSD*     }
\CommentTok{\#\textgreater{} see \textquotesingle{}?methods\textquotesingle{} for accessing help and source code}
\end{Highlighting}
\end{Shaded}

If we decide to take a look at \texttt{plot.default} we can do so by

\begin{Shaded}
\begin{Highlighting}[]
\NormalTok{plot.default}
\CommentTok{\#\textgreater{} function (x, y = NULL, type = "p", xlim = NULL, ylim = NULL, }
\CommentTok{\#\textgreater{}     log = "", main = NULL, sub = NULL, xlab = NULL, ylab = NULL, }
\CommentTok{\#\textgreater{}     ann = par("ann"), axes = TRUE, frame.plot = axes, panel.first = NULL, }
\CommentTok{\#\textgreater{}     panel.last = NULL, asp = NA, xgap.axis = NA, ygap.axis = NA, }
\CommentTok{\#\textgreater{}     ...) }
\CommentTok{\#\textgreater{} \{}
\CommentTok{\#\textgreater{}     localAxis \textless{}{-} function(..., col, bg, pch, cex, lty, lwd) Axis(...)}
\CommentTok{\#\textgreater{}     localBox \textless{}{-} function(..., col, bg, pch, cex, lty, lwd) box(...)}
\CommentTok{\#\textgreater{}     localWindow \textless{}{-} function(..., col, bg, pch, cex, lty, lwd) plot.window(...)}
\CommentTok{\#\textgreater{}     localTitle \textless{}{-} function(..., col, bg, pch, cex, lty, lwd) title(...)}
\CommentTok{\#\textgreater{}     xlabel \textless{}{-} if (!missing(x)) }
\CommentTok{\#\textgreater{}         deparse1(substitute(x))}
\CommentTok{\#\textgreater{}     ylabel \textless{}{-} if (!missing(y)) }
\CommentTok{\#\textgreater{}         deparse1(substitute(y))}
\CommentTok{\#\textgreater{}     xy \textless{}{-} xy.coords(x, y, xlabel, ylabel, log)}
\CommentTok{\#\textgreater{}     if (is.null(xlab)) }
\CommentTok{\#\textgreater{}         xlab \textless{}{-} xy$xlab}
\CommentTok{\#\textgreater{}     if (is.null(ylab)) }
\CommentTok{\#\textgreater{}         ylab \textless{}{-} xy$ylab}
\CommentTok{\#\textgreater{}     if (is.null(xlim)) }
\CommentTok{\#\textgreater{}         xlim \textless{}{-} range(xy$x[is.finite(xy$x)])}
\CommentTok{\#\textgreater{}     if (is.null(ylim)) }
\CommentTok{\#\textgreater{}         ylim \textless{}{-} range(xy$y[is.finite(xy$y)])}
\CommentTok{\#\textgreater{}     dev.hold()}
\CommentTok{\#\textgreater{}     on.exit(dev.flush())}
\CommentTok{\#\textgreater{}     plot.new()}
\CommentTok{\#\textgreater{}     localWindow(xlim, ylim, log, asp, ...)}
\CommentTok{\#\textgreater{}     panel.first}
\CommentTok{\#\textgreater{}     plot.xy(xy, type, ...)}
\CommentTok{\#\textgreater{}     panel.last}
\CommentTok{\#\textgreater{}     if (axes) \{}
\CommentTok{\#\textgreater{}         localAxis(if (is.null(y)) }
\CommentTok{\#\textgreater{}             xy$x}
\CommentTok{\#\textgreater{}         else x, side = 1, gap.axis = xgap.axis, ...)}
\CommentTok{\#\textgreater{}         localAxis(if (is.null(y)) }
\CommentTok{\#\textgreater{}             x}
\CommentTok{\#\textgreater{}         else y, side = 2, gap.axis = ygap.axis, ...)}
\CommentTok{\#\textgreater{}     \}}
\CommentTok{\#\textgreater{}     if (frame.plot) }
\CommentTok{\#\textgreater{}         localBox(...)}
\CommentTok{\#\textgreater{}     if (ann) }
\CommentTok{\#\textgreater{}         localTitle(main = main, sub = sub, xlab = xlab, ylab = ylab, }
\CommentTok{\#\textgreater{}             ...)}
\CommentTok{\#\textgreater{}     invisible()}
\CommentTok{\#\textgreater{} \}}
\CommentTok{\#\textgreater{} \textless{}bytecode: 0x00000107d26bbe98\textgreater{}}
\CommentTok{\#\textgreater{} \textless{}environment: namespace:graphics\textgreater{}}
\end{Highlighting}
\end{Shaded}

Since our new plotting method is aimed at categorical data we decide rather to take a look at \texttt{plot.factor}. But this is an asterisked function and hence is not visible:

\begin{Shaded}
\begin{Highlighting}[]
\NormalTok{plot.factor}
\CommentTok{\#\textgreater{} Error: object \textquotesingle{}plot.factor\textquotesingle{} not found}
\end{Highlighting}
\end{Shaded}

Asterisked functions can be inspected using the following method:

\begin{Shaded}
\begin{Highlighting}[]
\FunctionTok{getAnywhere}\NormalTok{(plot.factor)}
\CommentTok{\#\textgreater{} A single object matching \textquotesingle{}plot.factor\textquotesingle{} was found}
\CommentTok{\#\textgreater{} It was found in the following places}
\CommentTok{\#\textgreater{}   registered S3 method for plot from namespace graphics}
\CommentTok{\#\textgreater{}   namespace:graphics}
\CommentTok{\#\textgreater{} with value}
\CommentTok{\#\textgreater{} }
\CommentTok{\#\textgreater{} function (x, y, legend.text = NULL, ...) }
\CommentTok{\#\textgreater{} \{}
\CommentTok{\#\textgreater{}     if (missing(y) || is.factor(y)) \{}
\CommentTok{\#\textgreater{}         dargs \textless{}{-} list(...)}
\CommentTok{\#\textgreater{}         axisnames \textless{}{-} dargs$axes \%||\% if (!is.null(dargs$xaxt)) }
\CommentTok{\#\textgreater{}             dargs$xaxt != "n"}
\CommentTok{\#\textgreater{}         else TRUE}
\CommentTok{\#\textgreater{}     \}}
\CommentTok{\#\textgreater{}     if (missing(y)) \{}
\CommentTok{\#\textgreater{}         barplot(table(x), axisnames = axisnames, ...)}
\CommentTok{\#\textgreater{}     \}}
\CommentTok{\#\textgreater{}     else if (is.factor(y)) \{}
\CommentTok{\#\textgreater{}         if (is.null(legend.text)) }
\CommentTok{\#\textgreater{}             spineplot(x, y, ...)}
\CommentTok{\#\textgreater{}         else \{}
\CommentTok{\#\textgreater{}             args \textless{}{-} c(list(x = x, y = y), list(...))}
\CommentTok{\#\textgreater{}             args$yaxlabels \textless{}{-} legend.text}
\CommentTok{\#\textgreater{}             do.call("spineplot", args)}
\CommentTok{\#\textgreater{}         \}}
\CommentTok{\#\textgreater{}     \}}
\CommentTok{\#\textgreater{}     else if (is.numeric(y)) }
\CommentTok{\#\textgreater{}         boxplot(y \textasciitilde{} x, ...)}
\CommentTok{\#\textgreater{}     else NextMethod("plot")}
\CommentTok{\#\textgreater{} \}}
\CommentTok{\#\textgreater{} \textless{}bytecode: 0x00000107d0aee3b0\textgreater{}}
\CommentTok{\#\textgreater{} \textless{}environment: namespace:graphics\textgreater{}}
\end{Highlighting}
\end{Shaded}

\begin{enumerate}
\def\labelenumi{(\alph{enumi})}
\item
  How are default values assigned to arguments of functions?
\item
  What is the default behaviour of \texttt{plot.factor()}?
\item
  What tasks can be achieved with \texttt{pmatch()} and what is understood by partial matching? What will happen if \texttt{plot.factor()} is called with (i) \texttt{legend.text\ =\ \textquotesingle{}AA=Agecat\textquotesingle{}}; (ii) \texttt{leg\ =\ \textquotesingle{}AA=Agecat\textquotesingle{}}? Explain.
\item
  Discuss the usage of \texttt{missing()}.
\item
  Give an example of the usage of the function \texttt{stop(message=\ "\ ")}.
\item
  Give an example of the usage of the function \texttt{warning(message=\ "\ ")}.
\item
  What is the usage of the function \texttt{warnings()}?
\item
  Why can functions be called without specifying any arguments e.g.~\texttt{q()}?
\item
  If the body of a function consists only of a single instruction it is not necessary to enclose it with braces.
\item
  The convention is to use the last evaluated statement as a function's return value. If several objects are to be returned gather them in a list.
\item
  The function \texttt{return()} with a single object or a list of objects is useful to interrupt a function at some intermediate stage and return an object or a list of objects at that particular stage. This is usually done when a function is under development.
\item
  Sometimes there is no meaningful value to return e.g.~when a function is written primarily to produce some plot. In cases like this the function \texttt{invisible()} can be used as the last statement of the function. As an example of the usage of \texttt{invisible()} give the following instructions:
\end{enumerate}

\begin{Shaded}
\begin{Highlighting}[]
\FunctionTok{boxplot}\NormalTok{(}\FunctionTok{rnorm}\NormalTok{(}\DecValTok{100}\NormalTok{), }\AttributeTok{plot =} \ConstantTok{TRUE}\NormalTok{)}
\end{Highlighting}
\end{Shaded}

\pandocbounded{\includegraphics[keepaspectratio]{07-functions_files/figure-latex/invisibleExamples-1.pdf}}

\begin{Shaded}
\begin{Highlighting}[]
\FunctionTok{boxplot}\NormalTok{(}\FunctionTok{rnorm}\NormalTok{(}\DecValTok{100}\NormalTok{), }\AttributeTok{plot =} \ConstantTok{FALSE}\NormalTok{)}
\CommentTok{\#\textgreater{} $stats}
\CommentTok{\#\textgreater{}             [,1]}
\CommentTok{\#\textgreater{} [1,] {-}2.10192730}
\CommentTok{\#\textgreater{} [2,] {-}0.58894488}
\CommentTok{\#\textgreater{} [3,] {-}0.07732177}
\CommentTok{\#\textgreater{} [4,]  0.71833192}
\CommentTok{\#\textgreater{} [5,]  2.39068957}
\CommentTok{\#\textgreater{} }
\CommentTok{\#\textgreater{} $n}
\CommentTok{\#\textgreater{} [1] 100}
\CommentTok{\#\textgreater{} }
\CommentTok{\#\textgreater{} $conf}
\CommentTok{\#\textgreater{}            [,1]}
\CommentTok{\#\textgreater{} [1,] {-}0.2838715}
\CommentTok{\#\textgreater{} [2,]  0.1292280}
\CommentTok{\#\textgreater{} }
\CommentTok{\#\textgreater{} $out}
\CommentTok{\#\textgreater{} numeric(0)}
\CommentTok{\#\textgreater{} }
\CommentTok{\#\textgreater{} $group}
\CommentTok{\#\textgreater{} numeric(0)}
\CommentTok{\#\textgreater{} }
\CommentTok{\#\textgreater{} $names}
\CommentTok{\#\textgreater{} [1] "1"}
\end{Highlighting}
\end{Shaded}

Now look at the end of function \texttt{boxplot.default()} to see how \texttt{invisible()} has been implemented.

\begin{enumerate}
\def\labelenumi{(\alph{enumi})}
\setcounter{enumi}{12}
\item
  Libraries (packages) of R functions. Attaching and detaching libraries to the search path. (Revise Chapter \ref{intro})
\item
  Creating a new function using scripts or \texttt{fix()}. (Revise Chapter \ref{intro})
\item
  Editing an existing function using scripts or \texttt{fix()}. (Revise Chapter \ref{intro})
\item
  Note that when writing a function a line can be interrupted at any place and be continued on a next line. \emph{{Warning: Be careful not to put the break point where it marks the completion of an executable statement.}} Explain.
\end{enumerate}

\section{Writing a new function}\label{writing-a-new-function}

Determining the indices of elements in a vector or matrix that meet a certain condition: the function \texttt{where()}

\begin{enumerate}
\def\labelenumi{(\alph{enumi})}
\tightlist
\item
  Write the following function:
\end{enumerate}

\begin{Shaded}
\begin{Highlighting}[]
\NormalTok{where }\OtherTok{\textless{}{-}} \ControlFlowTok{function}\NormalTok{(x, cond)}
\NormalTok{\{ }\CommentTok{\# Argument cond must evaluate to a logical value}
     \ControlFlowTok{if}\NormalTok{(}\SpecialCharTok{!}\FunctionTok{is.matrix}\NormalTok{(x))}
       \FunctionTok{seq}\NormalTok{(}\AttributeTok{along =}\NormalTok{ x)[cond]}
     \ControlFlowTok{else} \FunctionTok{matrix}\NormalTok{(}\FunctionTok{c}\NormalTok{(}\FunctionTok{row}\NormalTok{(x)[cond], }\FunctionTok{col}\NormalTok{(x)[cond]), }\AttributeTok{ncol =} \DecValTok{2}\NormalTok{)}
\NormalTok{\}}
\end{Highlighting}
\end{Shaded}

\begin{enumerate}
\def\labelenumi{(\alph{enumi})}
\setcounter{enumi}{1}
\item
  Inspect the \emph{airquality} data set using the command \texttt{str(airquality)}.
\item
  Use the \texttt{where()} function to find the indices of (i) the \texttt{NA}s, (ii) the maximum value and (iii) the minimum value in the airquality data set.
\item
  Repeat (b) using the built-in function \texttt{which()}.
\end{enumerate}

\section{Checking for object name clashes}\label{checking-for-object-name-clashes}

\begin{enumerate}
\def\labelenumi{(\alph{enumi})}
\item
  What happens if an R object is given the same name as an existing object?
\item
  Discuss the usages of the functions \texttt{apropos()}, \texttt{conflicts()}, \texttt{find()} and \texttt{match()} for the naming of objects.
\item
  Remember that when a function is called the R evaluator first looks in the \emph{{global environment}} for a function with this name and subsequently in each of the attached packages or date bases in the order shown by \texttt{search()}. The evaluator generally stops searching when the name is found for the first time. If two attached packages have functions with the same name one of them will \emph{{mask}} the object in the other. For example, the function \texttt{gam()} exists in two packages: \texttt{gam} and \texttt{mgcv}. If both were attached the command
\end{enumerate}

\begin{Shaded}
\begin{Highlighting}[]
\FunctionTok{library}\NormalTok{ (mgcv)}
\CommentTok{\#\textgreater{} Loading required package: nlme}
\CommentTok{\#\textgreater{} This is mgcv 1.9{-}3. For overview type \textquotesingle{}help("mgcv{-}package")\textquotesingle{}.}
\FunctionTok{library}\NormalTok{ (gam)}
\CommentTok{\#\textgreater{} Loading required package: splines}
\CommentTok{\#\textgreater{} Loading required package: foreach}
\CommentTok{\#\textgreater{} Loaded gam 1.22{-}6}
\CommentTok{\#\textgreater{} }
\CommentTok{\#\textgreater{} Attaching package: \textquotesingle{}gam\textquotesingle{}}
\CommentTok{\#\textgreater{} The following objects are masked from \textquotesingle{}package:mgcv\textquotesingle{}:}
\CommentTok{\#\textgreater{} }
\CommentTok{\#\textgreater{}     gam, gam.control, gam.fit, s}
\FunctionTok{find}\NormalTok{(}\StringTok{"gam"}\NormalTok{)}
\CommentTok{\#\textgreater{} [1] "package:gam"  "package:mgcv"}
\end{Highlighting}
\end{Shaded}

will return both version.

\begin{enumerate}
\def\labelenumi{(\alph{enumi})}
\setcounter{enumi}{3}
\item
  The operator \texttt{::} can be used to access the intended version of \texttt{gam()} by using the call \texttt{mgcv::gam()} or \texttt{gam::gam()}.
\item
  When writing R packages the \emph{{namespace}} of the package provides another mechanism for ensuring that the correct version of a function is used. Note in this regard that the operator \texttt{:::} can be used to access objects that are not exported.
\end{enumerate}

\section{Returning multiple values}\label{returning-multiple-values}

\subsection{Exercise}\label{exercise-12}

Write an R function that returns the mean, median, variance, minimum, maximum and coefficient of variation of a numeric vector of sample data. The different components must be accessible by name. Test your function with the value of \texttt{rnorm(1000)}. \emph{Hint}: Use the construct \texttt{list\ (mean\ =\ ...,\ median\ =\ ...,\ ...)}.

\section{Local variables and evaluation environments}\label{local-variables-and-evaluation-environments}

\begin{enumerate}
\def\labelenumi{(\alph{enumi})}
\item
  Where is an object stored that is created by a script or \texttt{fix()}?
\item
  Where are local objects (objects that are created during the execution of a function) stored?
\item
  Explain how the evaluation environment works.
\item
  What is understood by the \emph{{global environment}}?
\item
  Study the R help-file w.r.t. the operator \texttt{\textless{}\textless{}-}. When is it useful to use this operator? What are the dangers inherent to this operator?
\item
  What is understood by the scope of an expression or function?
\end{enumerate}

The symbols which occur in the body of a function can be divided into three classes: \emph{{formal parameters}}, \emph{{local variables}} and \emph{{free variables}}. The formal parameters of a function are those appearing within the parentheses denoting the argument list of the function. Their values are determined by the process of \emph{{binding}} the actual function arguments to the formal parameters. Local variables are created by the evaluation of expressions in the body of the functions. Variables which are neither formal parameters nor local variables are called free variables. Free variables become local variables when they are assigned to. Consider the following function definition.

\begin{Shaded}
\begin{Highlighting}[]
\NormalTok{fun }\OtherTok{\textless{}{-}} \ControlFlowTok{function}\NormalTok{(datvec) \{}
\NormalTok{          mean }\OtherTok{\textless{}{-}} \FunctionTok{mean}\NormalTok{(datvec)}
          \FunctionTok{print}\NormalTok{(mean)}
          \FunctionTok{plot}\NormalTok{(datvec)}
          \FunctionTok{plot}\NormalTok{(Traffic)}
\NormalTok{       \}}
\end{Highlighting}
\end{Shaded}

In this function, \texttt{datvec} is a formal parameter, the object \texttt{mean} on the left-hand of the assignment symbol is a local variable (not to be confused with the function \texttt{mean()} on the right-hand side of the assignment symbol) while \texttt{Traffic} is a free variable. In R the free variable bindings are resolved by first looking in the \emph{{environment}} in which the function was created. This is called \emph{{lexical scope}}.

If the following function call is made from the prompt in the working directory \texttt{fun(1:25)} the formal parameter \texttt{datvec} within the body of the function is assigned the value \texttt{1:25} (the actual argument) and its mean is assigned to the local object \texttt{mean}. If the free parameter \texttt{Traffic} is found in the \emph{{global environment}} or in a data base on the search path the required graph will be created else an error message will be sent to the console. Perform the above call.

\section{Cleaning up}\label{cleaning-up}

\begin{enumerate}
\def\labelenumi{(\alph{enumi})}
\item
  Study how the function \texttt{on.exit()} is used. This function can be used to reset options that are changed during an R-session back to their original values when the session is ended or a function terminates with an error message. It is also convenient for removal of temporary files.
\item
  Study the uses of the functions \texttt{.First()} and \texttt{.Last()}.
\item
  Write a function that automatically opens a graph window with a square plot region when an R-session is started.
\end{enumerate}

\section{\texorpdfstring{Variable number of arguments: argument \texttt{...}}{Variable number of arguments: argument ...}}\label{variable-number-of-arguments-argument-...}

\begin{enumerate}
\def\labelenumi{(\alph{enumi})}
\tightlist
\item
  Consider the following situation: You want to write a function for a complex task. At a particular stage a graph of some intermediate results is to be constructed. This requires the calling function to contain a call to the hist function. Here is an example of a chunk of code for executing this task:
\end{enumerate}

\begin{Shaded}
\begin{Highlighting}[]
\NormalTok{complexfun }\OtherTok{\textless{}{-}} \ControlFlowTok{function}\NormalTok{(datmat,colgraph)}
\NormalTok{    \{ datmat }\OtherTok{\textless{}{-}} \FunctionTok{scale}\NormalTok{(datmat) }
       \CommentTok{\# Several lines of complex code here }
      \FunctionTok{hist}\NormalTok{(datmat, }\AttributeTok{col =}\NormalTok{ colgraph)              \}}
\end{Highlighting}
\end{Shaded}

A call like \texttt{complexfun(rnorm(1000),\ \textquotesingle{}yellow\textquotesingle{})} can now be executed for the desired result. The problem is that the hist function has several arguments that you would like to be able to access by passing suitable actual values to them through the calling function \texttt{complexfun}. Instead of having to resort to provide a complete set of arguments in the argument list of \texttt{complexfun} R provides a neat way of addressing this situation: The argument \texttt{...} which acts like any other formal argument except that it can represent a variable number of arguments. To see how the argument \texttt{...} works change the above function to:

\begin{Shaded}
\begin{Highlighting}[]
\NormalTok{complexfun2 }\OtherTok{\textless{}{-}} \ControlFlowTok{function}\NormalTok{(datmat, ... )}
\NormalTok{ \{ datmat }\OtherTok{\textless{}{-}} \FunctionTok{scale}\NormalTok{(datmat) }
       \CommentTok{\# Several lines of complex code here }
   \FunctionTok{hist}\NormalTok{(datmat, ... )    \}}
\end{Highlighting}
\end{Shaded}

Arguments represented by argument \texttt{...} in the argument list of hist are passed to hist through the argument \texttt{...} appearing in the arguments list of function \texttt{complexfun2}:

\begin{Shaded}
\begin{Highlighting}[]
\FunctionTok{complexfun2}\NormalTok{(}\AttributeTok{datmat =} \FunctionTok{rnorm}\NormalTok{(}\DecValTok{1000}\NormalTok{), }\AttributeTok{col =} \StringTok{\textquotesingle{}yellow\textquotesingle{}}\NormalTok{, }
        \AttributeTok{probability =} \ConstantTok{TRUE}\NormalTok{, }\AttributeTok{xlim =} \FunctionTok{c}\NormalTok{(}\SpecialCharTok{{-}}\DecValTok{5}\NormalTok{,}\DecValTok{5}\NormalTok{))}
\end{Highlighting}
\end{Shaded}

\begin{enumerate}
\def\labelenumi{(\alph{enumi})}
\setcounter{enumi}{1}
\tightlist
\item
  Write a function that will retrieve the maximum length of any of an unspecified number of arguments of a specified mode. This is another example of the use of the \texttt{...} argument:
\end{enumerate}

\begin{Shaded}
\begin{Highlighting}[]
\NormalTok{maxlen }\OtherTok{\textless{}{-}} \ControlFlowTok{function}\NormalTok{ (}\AttributeTok{mode.use=}\StringTok{"numeric"}\NormalTok{, ...) }
\NormalTok{  \{ my.list }\OtherTok{\textless{}{-}} \FunctionTok{list}\NormalTok{(...)}
\NormalTok{    out }\OtherTok{\textless{}{-}} \DecValTok{0}
    \ControlFlowTok{for}\NormalTok{(x }\ControlFlowTok{in}\NormalTok{ my.list) }
      \FunctionTok{print}\NormalTok{ (}\FunctionTok{mode}\NormalTok{(x)) }\CommentTok{\#if(mode(x) == mode.use) out \textless{}{-} max(out,length(x))}
\NormalTok{    out}
\NormalTok{  \}}
\end{Highlighting}
\end{Shaded}

Note that the named argument must be specified as such in the function call:

\begin{Shaded}
\begin{Highlighting}[]
\FunctionTok{maxlen}\NormalTok{(}\DecValTok{1}\SpecialCharTok{:}\DecValTok{10}\NormalTok{, }\DecValTok{1}\SpecialCharTok{:}\DecValTok{15}\NormalTok{, }\DecValTok{1}\SpecialCharTok{:}\DecValTok{3}\NormalTok{, letters)}
\CommentTok{\#\textgreater{} [1] "numeric"}
\CommentTok{\#\textgreater{} [1] "numeric"}
\CommentTok{\#\textgreater{} [1] "character"}
\CommentTok{\#\textgreater{} [1] 0}
\FunctionTok{maxlen}\NormalTok{(}\AttributeTok{mode.use=}\StringTok{"numeric"}\NormalTok{, }\DecValTok{1}\SpecialCharTok{:}\DecValTok{10}\NormalTok{, }\DecValTok{1}\SpecialCharTok{:}\DecValTok{15}\NormalTok{, }\DecValTok{1}\SpecialCharTok{:}\DecValTok{3}\NormalTok{, letters)}
\CommentTok{\#\textgreater{} [1] "numeric"}
\CommentTok{\#\textgreater{} [1] "numeric"}
\CommentTok{\#\textgreater{} [1] "numeric"}
\CommentTok{\#\textgreater{} [1] "character"}
\CommentTok{\#\textgreater{} [1] 0}
\FunctionTok{maxlen}\NormalTok{(}\DecValTok{1}\SpecialCharTok{:}\DecValTok{10}\NormalTok{, }\DecValTok{1}\SpecialCharTok{:}\DecValTok{15}\NormalTok{, }\DecValTok{1}\SpecialCharTok{:}\DecValTok{3}\NormalTok{, letters, }\AttributeTok{mode.use=}\StringTok{"character"}\NormalTok{)}
\CommentTok{\#\textgreater{} [1] "numeric"}
\CommentTok{\#\textgreater{} [1] "numeric"}
\CommentTok{\#\textgreater{} [1] "numeric"}
\CommentTok{\#\textgreater{} [1] "character"}
\CommentTok{\#\textgreater{} [1] 0}
\FunctionTok{maxlen}\NormalTok{(}\AttributeTok{mode.use=}\StringTok{"character"}\NormalTok{, }\DecValTok{1}\SpecialCharTok{:}\DecValTok{10}\NormalTok{, }\DecValTok{1}\SpecialCharTok{:}\DecValTok{15}\NormalTok{, }\DecValTok{1}\SpecialCharTok{:}\DecValTok{3}\NormalTok{, letters)}
\CommentTok{\#\textgreater{} [1] "numeric"}
\CommentTok{\#\textgreater{} [1] "numeric"}
\CommentTok{\#\textgreater{} [1] "numeric"}
\CommentTok{\#\textgreater{} [1] "character"}
\CommentTok{\#\textgreater{} [1] 0}
\end{Highlighting}
\end{Shaded}

\section{\texorpdfstring{Retrieving names of arguments: functions \texttt{deparse()} and \texttt{substitute()}}{Retrieving names of arguments: functions deparse() and substitute()}}\label{retrieving-names-of-arguments-functions-deparse-and-substitute}

There are many practical situations requiring the conversion of mathematical expressions into character strings (text) or, conversely, requiring the conversion of text into mathematical expressions. The tools (functions) provided in R for achieving such conversions are summarized in Figure \ref{fig:expression}.

\begin{figure}
\includegraphics[width=0.8\linewidth]{pics/expressions} \caption{Converting text into mathematical expression or mathematical expressions into text.}\label{fig:expression}
\end{figure}

\begin{itemize}
\tightlist
\item
  Task: write an R function that will plot two vectors using as axis labels the names of the objects passed as arguments to the function.
\end{itemize}

It follows from Figure \ref{fig:expression} that the function \texttt{substitute()} takes an expression as argument and returns it unevaluated. In order to evaluate the return value of \texttt{substitute()} the function \texttt{eval()} must be used. The function \texttt{deparse()} takes as argument an unevaluated expression and converts it into a character string. Now we are ready to write the following function:

\begin{Shaded}
\begin{Highlighting}[]
\NormalTok{labplot }\OtherTok{\textless{}{-}} \ControlFlowTok{function}\NormalTok{ (x,y) }
\NormalTok{\{ xname }\OtherTok{\textless{}{-}} \FunctionTok{deparse}\NormalTok{(}\FunctionTok{substitute}\NormalTok{(x))}
\NormalTok{ yname }\OtherTok{\textless{}{-}} \FunctionTok{deparse}\NormalTok{(}\FunctionTok{substitute}\NormalTok{(y))}
 \FunctionTok{plot}\NormalTok{(x,y, }\AttributeTok{xlab=}\NormalTok{xname, }\AttributeTok{ylab=}\NormalTok{yname, }\AttributeTok{main =} \FunctionTok{paste}\NormalTok{(}\StringTok{"Plot of"}\NormalTok{,}
\NormalTok{        yname,}\StringTok{"versus"}\NormalTok{, xname))}
\NormalTok{\}}
\end{Highlighting}
\end{Shaded}

\begin{enumerate}
\def\labelenumi{(\alph{enumi})}
\item
  Study and illustrate the usage of function \texttt{labplot()}.
\item
  From Figure \ref{fig:expression} it also follows that the function \texttt{parse()} does the opposite of \texttt{deparse()} by converting a character string into an unevaluated expression. The latter unevaluated expression can be evaluated when needed using \texttt{eval()}.
\end{enumerate}

\section{Operators}\label{operators-1}

Execute the following instruction

\begin{Shaded}
\begin{Highlighting}[]
\FunctionTok{objects}\NormalTok{(}\StringTok{\textquotesingle{}package:base\textquotesingle{}}\NormalTok{)[}\DecValTok{1}\SpecialCharTok{:}\DecValTok{31}\NormalTok{]}
\CommentTok{\#\textgreater{}  [1] "{-}"                 "{-}.Date"           }
\CommentTok{\#\textgreater{}  [3] "{-}.POSIXt"          "!"                }
\CommentTok{\#\textgreater{}  [5] "!.hexmode"         "!.octmode"        }
\CommentTok{\#\textgreater{}  [7] "!="                "$"                }
\CommentTok{\#\textgreater{}  [9] "$.DLLInfo"         "$.package\_version"}
\CommentTok{\#\textgreater{} [11] "$\textless{}{-}"               "$\textless{}{-}.data.frame"   }
\CommentTok{\#\textgreater{} [13] "$\textless{}{-}.POSIXlt"       "\%\%"               }
\CommentTok{\#\textgreater{} [15] "\%*\%"               "\%/\%"              }
\CommentTok{\#\textgreater{} [17] "\%||\%"              "\%in\%"             }
\CommentTok{\#\textgreater{} [19] "\%o\%"               "\%x\%"              }
\CommentTok{\#\textgreater{} [21] "\&"                 "\&\&"               }
\CommentTok{\#\textgreater{} [23] "\&.hexmode"         "\&.octmode"        }
\CommentTok{\#\textgreater{} [25] "("                 "*"                }
\CommentTok{\#\textgreater{} [27] "*.difftime"        "/"                }
\CommentTok{\#\textgreater{} [29] "/.difftime"        ":"                }
\CommentTok{\#\textgreater{} [31] "::"}
\end{Highlighting}
\end{Shaded}

in order to obtain some examples of operators available in R.

\begin{enumerate}
\def\labelenumi{(\alph{enumi})}
\item
  \emph{{Operators are special R functions.}} Discuss this statement. In what respects do operators differ from ordinary R functions?
\item
  Write an operator \texttt{\%E\%} to determine the Euclidean distance between two vectors and give an example of its usage. \emph{Hint}: when creating operators with \texttt{fix()} or using scripts the name must be given as a character string e.g.~\texttt{fix("\%E\%")}.
\end{enumerate}

\section{Replacement functions}\label{replacement-functions}

Execute the following instruction

\begin{Shaded}
\begin{Highlighting}[]
\FunctionTok{objects}\NormalTok{(}\StringTok{\textquotesingle{}package:base\textquotesingle{}}\NormalTok{)[}\DecValTok{300}\SpecialCharTok{:}\DecValTok{400}\NormalTok{]}
\CommentTok{\#\textgreater{}   [1] "c.factor"                  }
\CommentTok{\#\textgreater{}   [2] "c.noquote"                 }
\CommentTok{\#\textgreater{}   [3] "c.numeric\_version"         }
\CommentTok{\#\textgreater{}   [4] "c.POSIXct"                 }
\CommentTok{\#\textgreater{}   [5] "c.POSIXlt"                 }
\CommentTok{\#\textgreater{}   [6] "c.warnings"                }
\CommentTok{\#\textgreater{}   [7] "call"                      }
\CommentTok{\#\textgreater{}   [8] "callCC"                    }
\CommentTok{\#\textgreater{}   [9] "capabilities"              }
\CommentTok{\#\textgreater{}  [10] "casefold"                  }
\CommentTok{\#\textgreater{}  [11] "cat"                       }
\CommentTok{\#\textgreater{}  [12] "cbind"                     }
\CommentTok{\#\textgreater{}  [13] "cbind.data.frame"          }
\CommentTok{\#\textgreater{}  [14] "ceiling"                   }
\CommentTok{\#\textgreater{}  [15] "char.expand"               }
\CommentTok{\#\textgreater{}  [16] "character"                 }
\CommentTok{\#\textgreater{}  [17] "charmatch"                 }
\CommentTok{\#\textgreater{}  [18] "charToRaw"                 }
\CommentTok{\#\textgreater{}  [19] "chartr"                    }
\CommentTok{\#\textgreater{}  [20] "chkDots"                   }
\CommentTok{\#\textgreater{}  [21] "chol"                      }
\CommentTok{\#\textgreater{}  [22] "chol.default"              }
\CommentTok{\#\textgreater{}  [23] "chol2inv"                  }
\CommentTok{\#\textgreater{}  [24] "choose"                    }
\CommentTok{\#\textgreater{}  [25] "chooseOpsMethod"           }
\CommentTok{\#\textgreater{}  [26] "chooseOpsMethod.default"   }
\CommentTok{\#\textgreater{}  [27] "class"                     }
\CommentTok{\#\textgreater{}  [28] "class\textless{}{-}"                   }
\CommentTok{\#\textgreater{}  [29] "clearPushBack"             }
\CommentTok{\#\textgreater{}  [30] "close"                     }
\CommentTok{\#\textgreater{}  [31] "close.connection"          }
\CommentTok{\#\textgreater{}  [32] "close.srcfile"             }
\CommentTok{\#\textgreater{}  [33] "close.srcfilealias"        }
\CommentTok{\#\textgreater{}  [34] "closeAllConnections"       }
\CommentTok{\#\textgreater{}  [35] "col"                       }
\CommentTok{\#\textgreater{}  [36] "colMeans"                  }
\CommentTok{\#\textgreater{}  [37] "colnames"                  }
\CommentTok{\#\textgreater{}  [38] "colnames\textless{}{-}"                }
\CommentTok{\#\textgreater{}  [39] "colSums"                   }
\CommentTok{\#\textgreater{}  [40] "commandArgs"               }
\CommentTok{\#\textgreater{}  [41] "comment"                   }
\CommentTok{\#\textgreater{}  [42] "comment\textless{}{-}"                 }
\CommentTok{\#\textgreater{}  [43] "complex"                   }
\CommentTok{\#\textgreater{}  [44] "computeRestarts"           }
\CommentTok{\#\textgreater{}  [45] "conditionCall"             }
\CommentTok{\#\textgreater{}  [46] "conditionCall.condition"   }
\CommentTok{\#\textgreater{}  [47] "conditionMessage"          }
\CommentTok{\#\textgreater{}  [48] "conditionMessage.condition"}
\CommentTok{\#\textgreater{}  [49] "conflictRules"             }
\CommentTok{\#\textgreater{}  [50] "conflicts"                 }
\CommentTok{\#\textgreater{}  [51] "Conj"                      }
\CommentTok{\#\textgreater{}  [52] "contributors"              }
\CommentTok{\#\textgreater{}  [53] "cos"                       }
\CommentTok{\#\textgreater{}  [54] "cosh"                      }
\CommentTok{\#\textgreater{}  [55] "cospi"                     }
\CommentTok{\#\textgreater{}  [56] "crossprod"                 }
\CommentTok{\#\textgreater{}  [57] "Cstack\_info"               }
\CommentTok{\#\textgreater{}  [58] "cummax"                    }
\CommentTok{\#\textgreater{}  [59] "cummin"                    }
\CommentTok{\#\textgreater{}  [60] "cumprod"                   }
\CommentTok{\#\textgreater{}  [61] "cumsum"                    }
\CommentTok{\#\textgreater{}  [62] "curlGetHeaders"            }
\CommentTok{\#\textgreater{}  [63] "cut"                       }
\CommentTok{\#\textgreater{}  [64] "cut.Date"                  }
\CommentTok{\#\textgreater{}  [65] "cut.default"               }
\CommentTok{\#\textgreater{}  [66] "cut.POSIXt"                }
\CommentTok{\#\textgreater{}  [67] "data.class"                }
\CommentTok{\#\textgreater{}  [68] "data.frame"                }
\CommentTok{\#\textgreater{}  [69] "data.matrix"               }
\CommentTok{\#\textgreater{}  [70] "date"                      }
\CommentTok{\#\textgreater{}  [71] "debug"                     }
\CommentTok{\#\textgreater{}  [72] "debuggingState"            }
\CommentTok{\#\textgreater{}  [73] "debugonce"                 }
\CommentTok{\#\textgreater{}  [74] "declare"                   }
\CommentTok{\#\textgreater{}  [75] "default.stringsAsFactors"  }
\CommentTok{\#\textgreater{}  [76] "delayedAssign"             }
\CommentTok{\#\textgreater{}  [77] "deparse"                   }
\CommentTok{\#\textgreater{}  [78] "deparse1"                  }
\CommentTok{\#\textgreater{}  [79] "det"                       }
\CommentTok{\#\textgreater{}  [80] "detach"                    }
\CommentTok{\#\textgreater{}  [81] "determinant"               }
\CommentTok{\#\textgreater{}  [82] "determinant.matrix"        }
\CommentTok{\#\textgreater{}  [83] "dget"                      }
\CommentTok{\#\textgreater{}  [84] "diag"                      }
\CommentTok{\#\textgreater{}  [85] "diag\textless{}{-}"                    }
\CommentTok{\#\textgreater{}  [86] "diff"                      }
\CommentTok{\#\textgreater{}  [87] "diff.Date"                 }
\CommentTok{\#\textgreater{}  [88] "diff.default"              }
\CommentTok{\#\textgreater{}  [89] "diff.difftime"             }
\CommentTok{\#\textgreater{}  [90] "diff.POSIXt"               }
\CommentTok{\#\textgreater{}  [91] "difftime"                  }
\CommentTok{\#\textgreater{}  [92] "digamma"                   }
\CommentTok{\#\textgreater{}  [93] "dim"                       }
\CommentTok{\#\textgreater{}  [94] "dim.data.frame"            }
\CommentTok{\#\textgreater{}  [95] "dim\textless{}{-}"                     }
\CommentTok{\#\textgreater{}  [96] "dimnames"                  }
\CommentTok{\#\textgreater{}  [97] "dimnames.data.frame"       }
\CommentTok{\#\textgreater{}  [98] "dimnames\textless{}{-}"                }
\CommentTok{\#\textgreater{}  [99] "dimnames\textless{}{-}.data.frame"     }
\CommentTok{\#\textgreater{} [100] "dir"                       }
\CommentTok{\#\textgreater{} [101] "dir.create"}
\end{Highlighting}
\end{Shaded}

and notice that some object names appear in pairs with the name of one member of the pair ending in \texttt{\textless{}-}. Examples are \texttt{dim\textless{}-}, \texttt{levels\textless{}-}, \texttt{diag\textless{}-}, \texttt{names\textless{}-}, \texttt{rownames\textless{}-}, \texttt{colnames\textless{}-} and \texttt{dimnames\textless{}-}. Functions having names ending in \texttt{\textless{}-} are called \emph{{replacement}} functions. A replacement function appears on the left-hand side of the assignment symbol using the name without the \texttt{\textless{}-} to replace contents of the objects appearing in its argument list by the contents of the object appearing at the right-hand side of the assignment symbol e.g.:

\begin{Shaded}
\begin{Highlighting}[]
\NormalTok{X }\OtherTok{\textless{}{-}} \FunctionTok{matrix}\NormalTok{ (}\DecValTok{1}\SpecialCharTok{:}\DecValTok{12}\NormalTok{, }\AttributeTok{ncol =} \DecValTok{3}\NormalTok{, }\AttributeTok{dimnames =} 
               \FunctionTok{list}\NormalTok{ (}\FunctionTok{paste0}\NormalTok{ (}\StringTok{"Row"}\NormalTok{, }\DecValTok{1}\SpecialCharTok{:}\DecValTok{4}\NormalTok{), }\FunctionTok{paste0}\NormalTok{ (}\StringTok{"X"}\NormalTok{, }\DecValTok{1}\SpecialCharTok{:}\DecValTok{3}\NormalTok{)))}
\NormalTok{a }\OtherTok{\textless{}{-}} \FunctionTok{rownames}\NormalTok{(X) }\CommentTok{\# Function rownames in action.}
\FunctionTok{rownames}\NormalTok{(X) }\OtherTok{\textless{}{-}} \DecValTok{1}\SpecialCharTok{:}\FunctionTok{nrow}\NormalTok{(X) }\CommentTok{\# Replacement function \textquotesingle{}rownames\textless{}{-}\textquotesingle{} in action.}
\end{Highlighting}
\end{Shaded}

How can the object \texttt{diag\textless{}-} be inspected and is it different from the object \texttt{diag}? Compare the result of the following function calls:

\begin{Shaded}
\begin{Highlighting}[]
\FunctionTok{getAnywhere}\NormalTok{(}\StringTok{\textquotesingle{}diag\textquotesingle{}}\NormalTok{)}
\CommentTok{\#\textgreater{} 2 differing objects matching \textquotesingle{}diag\textquotesingle{} were found}
\CommentTok{\#\textgreater{} in the following places}
\CommentTok{\#\textgreater{}   package:base}
\CommentTok{\#\textgreater{}   namespace:Matrix}
\CommentTok{\#\textgreater{}   namespace:base}
\CommentTok{\#\textgreater{} Use [] to view one of them}
\FunctionTok{getAnywhere}\NormalTok{(}\StringTok{\textquotesingle{}diag\textless{}{-}\textquotesingle{}}\NormalTok{)}
\CommentTok{\#\textgreater{} 2 differing objects matching \textquotesingle{}diag\textless{}{-}\textquotesingle{} were found}
\CommentTok{\#\textgreater{} in the following places}
\CommentTok{\#\textgreater{}   package:base}
\CommentTok{\#\textgreater{}   namespace:Matrix}
\CommentTok{\#\textgreater{}   namespace:base}
\CommentTok{\#\textgreater{} Use [] to view one of them}
\end{Highlighting}
\end{Shaded}

In what respects do replacement functions differ from other functions?

In order to write a replacement function the following rules must be met:

\begin{enumerate}
\def\labelenumi{(\roman{enumi})}
\item
  the function name must end in \texttt{\textless{}-}
\item
  the function must return the complete object with suitable changes made
\item
  the final argument of the function corresponding to the replacement data on the right-hand side of the assignment, must be named \texttt{value}
\item
  usually a companion function exists having the same name without the \texttt{\textless{}-}.
\end{enumerate}

As an example, write a replacement function \texttt{undefined()} that will replace missing values in a data object with the values on its right-hand side:

\begin{Shaded}
\begin{Highlighting}[]
\StringTok{"undefined\textless{}{-}"} \OtherTok{\textless{}{-}} \ControlFlowTok{function}\NormalTok{ (x, }\AttributeTok{codes =} \FunctionTok{numeric}\NormalTok{(), value) }
\NormalTok{  \{ }\ControlFlowTok{if}\NormalTok{ (}\FunctionTok{length}\NormalTok{(codes) }\SpecialCharTok{\textgreater{}} \DecValTok{0}\NormalTok{) x[x }\SpecialCharTok{\%in\%}\NormalTok{ codes] }\OtherTok{\textless{}{-}} \ConstantTok{NA}
\NormalTok{    x[}\FunctionTok{is.na}\NormalTok{(x)] }\OtherTok{\textless{}{-}}\NormalTok{ value}
\NormalTok{    x}
\NormalTok{  \}}
\end{Highlighting}
\end{Shaded}

The above function can be created or edited using \texttt{fix("undefined\textless{}-")}. Illustrate the usage of \texttt{undefined()}.

\section{Default values and lazy evaluation}\label{default-values-and-lazy-evaluation}

\begin{enumerate}
\def\labelenumi{(\alph{enumi})}
\tightlist
\item
  The function \texttt{match.arg()} is useful for selecting a default value from one of a set of possible values. Consider the following example:
\end{enumerate}

\begin{Shaded}
\begin{Highlighting}[]
\NormalTok{choice }\OtherTok{\textless{}{-}} \ControlFlowTok{function}\NormalTok{(}\AttributeTok{method=}\FunctionTok{c}\NormalTok{(}\StringTok{"PCA"}\NormalTok{,}\StringTok{"CVA"}\NormalTok{,}\StringTok{"CA"}\NormalTok{,}\StringTok{"NONLIN"}\NormalTok{))}
\NormalTok{   \{ }\FunctionTok{match.arg}\NormalTok{(method)  \}}
\FunctionTok{choice}\NormalTok{()}
\CommentTok{\#\textgreater{} [1] "PCA"}
\FunctionTok{choice}\NormalTok{(}\StringTok{"CVA"}\NormalTok{)}
\CommentTok{\#\textgreater{} [1] "CVA"}
\FunctionTok{choice}\NormalTok{(}\StringTok{"xx"}\NormalTok{)}
\CommentTok{\#\textgreater{} Error in match.arg(method): \textquotesingle{}arg\textquotesingle{} should be one of "PCA", "CVA", "CA", "NONLIN"}
\end{Highlighting}
\end{Shaded}

\begin{enumerate}
\def\labelenumi{(\alph{enumi})}
\setcounter{enumi}{1}
\tightlist
\item
  Functions in the R language are governed by a principle known as \emph{{lazy evaluation}} which means that a default value is not evaluated until it is actually needed within the function body. As a result of lazy evaluation it might happen in a function call that some default values are never evaluated.
\end{enumerate}

\section{The dynamic loading of external routines}\label{the-dynamic-loading-of-external-routines}

Compiled code can run in some instances much faster than corresponding code in R. The functions \texttt{.C()} and \texttt{.Fortran()} allow users to make use of programs written in \emph{{C}} or \emph{{Fortran}} in their R functions. How this is done is illustrated below. Study this example carefully and consult the help files for more details when needed.
First an R function is created to compute the matrix product of two matrices:

\begin{Shaded}
\begin{Highlighting}[]
\NormalTok{matmult }\OtherTok{\textless{}{-}} \ControlFlowTok{function}\NormalTok{ (A,B) }
\NormalTok{ \{ }\ControlFlowTok{if}\NormalTok{(}\FunctionTok{ncol}\NormalTok{(A) }\SpecialCharTok{!=} \FunctionTok{nrow}\NormalTok{(B)) }\FunctionTok{stop}\NormalTok{(}\StringTok{"A and B not conformable with                 }
\StringTok{                       respect to matrix multiplication }\SpecialCharTok{\textbackslash{}n}\StringTok{"}\NormalTok{)}
\NormalTok{   n }\OtherTok{\textless{}{-}} \FunctionTok{nrow}\NormalTok{(A)}
\NormalTok{   q }\OtherTok{\textless{}{-}} \FunctionTok{ncol}\NormalTok{(B)}
\NormalTok{   Cmat }\OtherTok{\textless{}{-}} \FunctionTok{matrix}\NormalTok{(}\ConstantTok{NA}\NormalTok{, }\AttributeTok{nrow=}\NormalTok{n, }\AttributeTok{ncol=}\NormalTok{q)}
   \ControlFlowTok{for}\NormalTok{(i }\ControlFlowTok{in} \DecValTok{1}\SpecialCharTok{:}\NormalTok{n)}
\NormalTok{      \{ }\ControlFlowTok{for}\NormalTok{(j }\ControlFlowTok{in} \DecValTok{1}\SpecialCharTok{:}\NormalTok{q) Cmat[i,j] }\OtherTok{\textless{}{-}} \FunctionTok{sum}\NormalTok{(A[i,] }\SpecialCharTok{*}\NormalTok{ B[,j])}
\NormalTok{      \}}
\NormalTok{  Cmat}
\NormalTok{  \}}
\end{Highlighting}
\end{Shaded}

Next a Fortran subroutine is written for performing matrix multiplication. The Fortran code for this subroutine is given below:

\begin{Shaded}
\begin{Highlighting}[]
\NormalTok{      SUBROUTINE MATM (A1, A2B1, B2, A, B, OUT)}
\NormalTok{C     This subroutine performs matrix multiplication.}
\NormalTok{C     This should be improved with optimized code (such as }
\NormalTok{C     from Linpack, etc.)}
\NormalTok{      IMPLICIT NONE}
\NormalTok{      INTEGER A1, A2B1, B2}
\NormalTok{      DOUBLE PRECISION A(A1,A2B1), B(A2B1,B2), OUT(A1,B2)}
\NormalTok{C     DUMMIES}
\NormalTok{      INTEGER I, J, K}
\NormalTok{      DO 300,J=1,B2}
\NormalTok{        DO 200,I=1,A1}
\NormalTok{          OUT(I,J)=0}
\NormalTok{          DO 100,K=1,A2B1}
\NormalTok{            OUT(I,J)=OUT(I,J)+A(I,K)*B(K,J)}
\NormalTok{100   CONTINUE}
\NormalTok{200   CONTINUE}
\NormalTok{300   CONTINUE}
\NormalTok{      END}
\end{Highlighting}
\end{Shaded}

Next a dynamic link library (\emph{{.dll}}) is made from the Fortran subroutine. The easiest way to do this is to use the command \texttt{R\ CMD\ SHLIB\ matm.f} from the \emph{{Command Prompt}}. The dll is available as \texttt{C:\textbackslash{}matm64.dll}.

Now an R function is to be written where the Fortran code is called:

\begin{Shaded}
\begin{Highlighting}[]
\NormalTok{matmult.Fortran }\OtherTok{\textless{}{-}}\ControlFlowTok{function}\NormalTok{ (A,B) }
\NormalTok{ \{ }\ControlFlowTok{if}\NormalTok{(}\FunctionTok{ncol}\NormalTok{(A) }\SpecialCharTok{!=} \FunctionTok{nrow}\NormalTok{(B)) }\FunctionTok{stop}\NormalTok{(}\StringTok{"A and B not conformable with }
\StringTok{                       respect to matrix multiplication }\SpecialCharTok{\textbackslash{}n}\StringTok{"}\NormalTok{)}
\NormalTok{    n }\OtherTok{\textless{}{-}} \FunctionTok{nrow}\NormalTok{(A)}
\NormalTok{    q }\OtherTok{\textless{}{-}} \FunctionTok{ncol}\NormalTok{(B)}
\NormalTok{    p }\OtherTok{\textless{}{-}} \FunctionTok{ncol}\NormalTok{(A)}
\NormalTok{    Cmat }\OtherTok{\textless{}{-}} \FunctionTok{matrix}\NormalTok{(}\DecValTok{0}\NormalTok{, }\AttributeTok{nrow=}\NormalTok{n, }\AttributeTok{ncol=}\NormalTok{q)}
    \FunctionTok{storage.mode}\NormalTok{(A) }\OtherTok{\textless{}{-}} \StringTok{"double"}
    \FunctionTok{storage.mode}\NormalTok{(B) }\OtherTok{\textless{}{-}} \StringTok{"double"}
    \FunctionTok{storage.mode}\NormalTok{(Cmat) }\OtherTok{\textless{}{-}} \StringTok{"double"}
\NormalTok{    value }\OtherTok{\textless{}{-}} \FunctionTok{.Fortran}\NormalTok{(}\StringTok{"matm"}\NormalTok{, }\FunctionTok{as.integer}\NormalTok{(n), }\FunctionTok{as.integer}\NormalTok{(p), }
                          \FunctionTok{as.integer}\NormalTok{(q), A, B, }\AttributeTok{matprod=}\NormalTok{Cmat)}
\NormalTok{    value}\SpecialCharTok{$}\NormalTok{matprod        \}}
\end{Highlighting}
\end{Shaded}

In order to use \texttt{matmult.Fortran()} the correct dll must be loaded into the current folder using the function \texttt{dyn.load()}:

\begin{Shaded}
\begin{Highlighting}[]
\FunctionTok{dyn.load}\NormalTok{(}\StringTok{"full path}\SpecialCharTok{\textbackslash{}\textbackslash{}}\StringTok{matm64.dll"}\NormalTok{)}
\end{Highlighting}
\end{Shaded}

Compare the answers and execution time of \texttt{matmult()} and \texttt{matmult.Fortran()} for different sized matrices.

The \texttt{Rcpp} package has made the inclusion of \emph{{C++}} code into R considerably easier and more robust. For a detailed description of the package see \href{https://cran.r-project.org/web/packages/Rcpp/vignettes/Rcpp-introduction.pdf}{Rcpp vignette intro}.

\chapter{Vectorized programming and mapping functions}\label{mapping}

In this chapter we continue the study the art of R programming. An important topic is a set of tools operating on objects like matrices, dataframes and lists as wholes.

\section{Mapping functions to a matrix}\label{mapping-functions-to-a-matrix}

\begin{enumerate}
\def\labelenumi{(\alph{enumi})}
\item
  What is understood by a mapping function and of what use are such functions?
\item
  The function \texttt{apply()}.

  \begin{enumerate}
  \def\labelenumii{(\roman{enumii})}
  \item
    What three arguments are required?
  \item
    Suppose the third argument is a function. How are the arguments of this function used within \texttt{apply()}?
  \end{enumerate}
\end{enumerate}

\begin{itemize}
\item
  What is the result of the instruction \texttt{apply(is.na(x),2,all)}?
\item
  What is the result of the instruction \texttt{x{[}\ ,!apply(is.na(x),\ 2,all){]}}?
\item
  What is the result of the instruction \texttt{x{[}\ ,!apply(is.na(x),\ 2,any){]}}?
\item
  Set the random seed to 137921. Obtain a matrix \(\mathbf{A}:10 \times 6\) of random \(n(0, 1)\) values. Use \texttt{apply()} to find the \(10\%\) trimmed mean of each row.
\end{itemize}

\begin{enumerate}
\def\labelenumi{(\alph{enumi})}
\setcounter{enumi}{2}
\item
  The function \texttt{sweep()}.

  \begin{enumerate}
  \def\labelenumii{(\roman{enumii})}
  \item
    What arguments are required?
  \item
    What are the similarities and differences between the arguments of \texttt{sweep()} and \texttt{apply()}?
  \item
    Normalise the columns of a given matrix to have zero means and unit variances using \texttt{scale()}, \texttt{apply()} and \texttt{sweep()}. Which method is the fastest?
  \end{enumerate}
\item
  The function \texttt{ifelse()}.
\end{enumerate}

The usage is illustrated in the following diagram.

\includegraphics[width=1\linewidth]{pics/ifelse}

\begin{enumerate}
\def\labelenumi{(\roman{enumi})}
\item
  Note the difference between the function \texttt{ifelse()} and the control statement: \texttt{if} - \texttt{else}.
\item
  What arguments are required?
\item
  Study the help file in detail.
\end{enumerate}

\begin{enumerate}
\def\labelenumi{(\alph{enumi})}
\setcounter{enumi}{4}
\item
  The function \texttt{outer()}.

  \begin{enumerate}
  \def\labelenumii{(\roman{enumii})}
  \item
    What arguments are required?
  \item
    Revise our previous example of \texttt{outer()} when constructing a perspective plot with \texttt{persp()}.
  \end{enumerate}
\item
  Work through the following examples and note in particular how the above functions are used together:

  \begin{enumerate}
  \def\labelenumii{(\roman{enumii})}
  \item
    Find the maximum value(s) in each column of the \texttt{LifeCycleSavings} data set.
  \item
    Use \texttt{apply()} together with \texttt{cut()} to divide each column of the LifeCycleSaving data set into low, medium and high.
  \item
    Use \texttt{apply()} to plot each column of the \texttt{LifeCycleSaving} data set against the ratio of \texttt{pop75} to \texttt{pop15} on the x-axis.
  \item
    Use \texttt{apply()} to find the coefficient of variation of each column of the \texttt{LifeCycleSaving} data set.
  \item
    Use \texttt{apply()} together with \texttt{cbind()} and \texttt{rbind()} to obtain a table of the minimum and the maximum values of each column of the LifeCycleSaving data set.
  \item
    Repeat (v) using the airquality data set with and without the elimination of the NAs by using an appropriate function definition in the call to \texttt{apply()}.
  \item
    Use \texttt{sweep()} to convert the \texttt{LifeCycleSaving} data set into standardized scores. Could \texttt{apply()} also be used for this task? Discuss.
  \item
    Use \texttt{ifelse()} to convert negative values in a given vector to zero leaving positive values and missing values unchanged. Illustrate.
  \end{enumerate}
\end{enumerate}

\section{Mapping functions to vectors, dataframes and lists}\label{mapping-functions-to-vectors-dataframes-and-lists}

\begin{enumerate}
\def\labelenumi{(\alph{enumi})}
\item
  Study the functions \texttt{lapply()}, \texttt{sapply()} and \texttt{split()}.
\item
  Carefully study what is produced by the command
\end{enumerate}

\begin{Shaded}
\begin{Highlighting}[]
\FunctionTok{lapply}\NormalTok{ (}\FunctionTok{split}\NormalTok{ (}\FunctionTok{data.frame}\NormalTok{ (state.x77),   }
               \FunctionTok{cut}\NormalTok{ (}\FunctionTok{data.frame}\NormalTok{ (state.x77)}\SpecialCharTok{$}\NormalTok{Illiteracy, }\DecValTok{3}\NormalTok{)), pairs)}
\end{Highlighting}
\end{Shaded}

\pandocbounded{\includegraphics[keepaspectratio]{08-mapping_files/figure-latex/splitExample-1.pdf}} \pandocbounded{\includegraphics[keepaspectratio]{08-mapping_files/figure-latex/splitExample-2.pdf}} \pandocbounded{\includegraphics[keepaspectratio]{08-mapping_files/figure-latex/splitExample-3.pdf}}

\begin{verbatim}
#> $`(0.498,1.27]`
#> NULL
#> 
#> $`(1.27,2.03]`
#> NULL
#> 
#> $`(2.03,2.8]`
#> NULL
\end{verbatim}

Note: in order to see all graphs in the R-GUI it is necessary to issue the command

\begin{Shaded}
\begin{Highlighting}[]
\FunctionTok{par}\NormalTok{(}\AttributeTok{ask=}\ConstantTok{TRUE}\NormalTok{) }
\end{Highlighting}
\end{Shaded}

before calling the function \texttt{lapply()}.

\begin{enumerate}
\def\labelenumi{(\alph{enumi})}
\setcounter{enumi}{2}
\tightlist
\item
  Use \texttt{lapply()} to produce histograms of each of the variables in the \texttt{state.x77} data set such that each histogram has as title the correct variable name. The \(x\)- and \(y\)-axis must also be labelled correctly.
\end{enumerate}

\section{\texorpdfstring{The functions: \texttt{mapply()}, \texttt{rapply()} and \texttt{Vectorize()}}{The functions: mapply(), rapply() and Vectorize()}}\label{the-functions-mapply-rapply-and-vectorize}

\begin{enumerate}
\def\labelenumi{(\alph{enumi})}
\tightlist
\item
  To apply a function to more than one list, \texttt{mapply()} is a multivariate version of \texttt{sapply()}. The first argument to \texttt{mapply()} is a function followed by the arguments for that function. The first argument function is applied to each of the elements in the following arguments.
\end{enumerate}

\begin{Shaded}
\begin{Highlighting}[]
\FunctionTok{mapply}\NormalTok{ (}\ControlFlowTok{function}\NormalTok{ (x,y,z) \{x}\SpecialCharTok{+}\NormalTok{y}\SpecialCharTok{+}\NormalTok{z\}, }\AttributeTok{x =} \FunctionTok{c}\NormalTok{(}\DecValTok{2}\NormalTok{, }\DecValTok{3}\NormalTok{), }\AttributeTok{y =} \FunctionTok{c}\NormalTok{(}\DecValTok{4}\NormalTok{,}\DecValTok{5}\NormalTok{), }\AttributeTok{z =} \FunctionTok{c}\NormalTok{(}\DecValTok{1}\NormalTok{,}\DecValTok{8}\NormalTok{))}
\CommentTok{\#\textgreater{} [1]  7 16}
\FunctionTok{mapply}\NormalTok{ (}\ControlFlowTok{function}\NormalTok{(x,y,z) \{ }\FunctionTok{list}\NormalTok{ (}\FunctionTok{min}\NormalTok{ (}\FunctionTok{c}\NormalTok{(x,y,z)), }\FunctionTok{max}\NormalTok{ (}\FunctionTok{c}\NormalTok{(x,y,z))) \}, }
        \AttributeTok{x =} \FunctionTok{c}\NormalTok{(}\DecValTok{2}\NormalTok{, }\DecValTok{3}\NormalTok{), }\AttributeTok{y =} \FunctionTok{c}\NormalTok{(}\DecValTok{4}\NormalTok{, }\DecValTok{5}\NormalTok{), }\AttributeTok{z =} \FunctionTok{c}\NormalTok{(}\DecValTok{1}\NormalTok{, }\DecValTok{8}\NormalTok{))}
\CommentTok{\#\textgreater{}      [,1] [,2]}
\CommentTok{\#\textgreater{} [1,] 1    3   }
\CommentTok{\#\textgreater{} [2,] 4    8}
\end{Highlighting}
\end{Shaded}

\begin{enumerate}
\def\labelenumi{(\alph{enumi})}
\setcounter{enumi}{1}
\tightlist
\item
  Study the help-files of \texttt{rapply()} and \texttt{Vectorize()}.
\end{enumerate}

\section{The mapping function tapply() for grouped data}\label{the-mapping-function-tapply-for-grouped-data}

\begin{enumerate}
\def\labelenumi{(\alph{enumi})}
\item
  Study the arguments of \texttt{tapply()}.
\item
  Consider the \texttt{LifeCycleSavings} data set. Create an object \texttt{ddpigrp} that groups the \texttt{LifeCycleSavings} data into four groups G1, G2, G3 and G4 such that G1 members have \texttt{ddpi} within \((0, 2.0]\), G2 members have \texttt{ddpi} within \((2.0, 3.5]\), G3 members have \texttt{ddpi} within \((3.5, 5.0]\), and G4 members have \texttt{ddpi} larger than \(5.0\). Use \texttt{tapply()} to obtain the mean aggregate personal savings of each of the groups defined by \texttt{ddpigrp}.
\item
  If it is needed to break down a vector by more than one categorical variable, a list containing the grouping variables is used as the second argument to \texttt{tapply()}. Illustrate this by finding the mean aggregate personal savings of the groups in \texttt{ddpigrp} broken down by the \texttt{pop15} rating.
\item
  In order to use \texttt{tapply()} on more than one variable simultaneously \texttt{apply()} can be used to map \texttt{tapply()} to each of the variables in turn. Study the following command and its output carefully:
\end{enumerate}

\begin{Shaded}
\begin{Highlighting}[]
\NormalTok{ddpigrp }\OtherTok{\textless{}{-}} \FunctionTok{cut}\NormalTok{ (LifeCycleSavings}\SpecialCharTok{$}\NormalTok{ddpi, }
                \AttributeTok{breaks =} \FunctionTok{c}\NormalTok{(}\DecValTok{0}\NormalTok{, }\DecValTok{2}\NormalTok{, }\FloatTok{3.5}\NormalTok{, }\DecValTok{5}\NormalTok{, }\FunctionTok{max}\NormalTok{(LifeCycleSavings}\SpecialCharTok{$}\NormalTok{ddpi)),}
                \AttributeTok{labels =} \FunctionTok{paste0}\NormalTok{ (}\StringTok{"G"}\NormalTok{, }\DecValTok{1}\SpecialCharTok{:}\DecValTok{4}\NormalTok{))}
\FunctionTok{apply}\NormalTok{ (LifeCycleSavings [,}\FunctionTok{c}\NormalTok{ (}\DecValTok{1}\NormalTok{, }\DecValTok{3}\NormalTok{, }\DecValTok{4}\NormalTok{)], }\DecValTok{2}\NormalTok{, }\ControlFlowTok{function}\NormalTok{(x) }
                                           \FunctionTok{tapply}\NormalTok{ (x, ddpigrp, mean)) }
\CommentTok{\#\textgreater{}           sr    pop75       dpi}
\CommentTok{\#\textgreater{} G1  7.855385 1.790769  712.1677}
\CommentTok{\#\textgreater{} G2  8.230625 2.456250 1497.0731}
\CommentTok{\#\textgreater{} G3 11.959000 3.189000 1569.4910}
\CommentTok{\#\textgreater{} G4 11.831818 1.834545  584.6964}
\end{Highlighting}
\end{Shaded}

\begin{enumerate}
\def\labelenumi{(\alph{enumi})}
\setcounter{enumi}{4}
\tightlist
\item
  If \texttt{tapply()} is called without a third argument it returns a vector of the same length than its first argument containing an index into the output that normally would be produced. Illustrate this behaviour and discuss its usage.
\end{enumerate}

\section{\texorpdfstring{The control of execution flow statement if-else and the control functions \texttt{ifelse()} and \texttt{switch()}}{The control of execution flow statement if-else and the control functions ifelse() and switch()}}\label{the-control-of-execution-flow-statement-if-else-and-the-control-functions-ifelse-and-switch}

\begin{enumerate}
\def\labelenumi{(\alph{enumi})}
\tightlist
\item
  The primary tool for conditional computations is the \texttt{if} statement. It takes the form:
\end{enumerate}

\begin{Shaded}
\begin{Highlighting}[]
\NormalTok{if (logical condition evaluating to either TRUE or FALSE)}
\NormalTok{    \{}
\NormalTok{     First set consisting of one or more R expressions}
\NormalTok{    \}}
\NormalTok{else}
\NormalTok{    \{}
\NormalTok{     Second set consisting of one or more R expressions}
\NormalTok{    \} }
\NormalTok{Expression3}
\end{Highlighting}
\end{Shaded}

\begin{enumerate}
\def\labelenumi{(\alph{enumi})}
\setcounter{enumi}{1}
\item
  In the above the \texttt{else} and its accompanying expression(s) are optional.
\item
  If-else statements can be nested.
\item
  Remember that the function \texttt{ifelse()} operates on objects as wholes as illustrated below:
\end{enumerate}

\begin{Shaded}
\begin{Highlighting}[]
\NormalTok{xx }\OtherTok{\textless{}{-}} \FunctionTok{matrix}\NormalTok{(}\DecValTok{1}\SpecialCharTok{:}\DecValTok{25}\NormalTok{, }\AttributeTok{ncol=}\DecValTok{5}\NormalTok{)}
\NormalTok{xx}
\CommentTok{\#\textgreater{}      [,1] [,2] [,3] [,4] [,5]}
\CommentTok{\#\textgreater{} [1,]    1    6   11   16   21}
\CommentTok{\#\textgreater{} [2,]    2    7   12   17   22}
\CommentTok{\#\textgreater{} [3,]    3    8   13   18   23}
\CommentTok{\#\textgreater{} [4,]    4    9   14   19   24}
\CommentTok{\#\textgreater{} [5,]    5   10   15   20   25}
\FunctionTok{ifelse}\NormalTok{(xx }\SpecialCharTok{\textless{}} \DecValTok{10}\NormalTok{, }\DecValTok{0}\NormalTok{, }\DecValTok{1}\NormalTok{)}
\CommentTok{\#\textgreater{}      [,1] [,2] [,3] [,4] [,5]}
\CommentTok{\#\textgreater{} [1,]    0    0    1    1    1}
\CommentTok{\#\textgreater{} [2,]    0    0    1    1    1}
\CommentTok{\#\textgreater{} [3,]    0    0    1    1    1}
\CommentTok{\#\textgreater{} [4,]    0    0    1    1    1}
\CommentTok{\#\textgreater{} [5,]    0    1    1    1    1}
\end{Highlighting}
\end{Shaded}

\begin{enumerate}
\def\labelenumi{(\alph{enumi})}
\setcounter{enumi}{4}
\tightlist
\item
  Note that the function \texttt{match()} can be used as an alternative to multiple if-else statements in certain cases. The function \texttt{match()} takes as first argument a vector, \texttt{x}, of values to be matched and as second argument, \texttt{table}, a vector of possible values to be matched against. A third argument \texttt{nomatch\ =\ NA} specifies the return value if no match occurs. See the example below:
\end{enumerate}

\begin{Shaded}
\begin{Highlighting}[]
\FunctionTok{match}\NormalTok{ (}\FunctionTok{c}\NormalTok{ (}\DecValTok{1}\SpecialCharTok{:}\DecValTok{5}\NormalTok{, }\DecValTok{3}\NormalTok{), }\FunctionTok{c}\NormalTok{ (}\DecValTok{2}\NormalTok{, }\DecValTok{3}\NormalTok{))}
\CommentTok{\#\textgreater{} [1] NA  1  2 NA NA  2}
\FunctionTok{match}\NormalTok{ (}\FunctionTok{c}\NormalTok{ (}\DecValTok{1}\SpecialCharTok{:}\DecValTok{5}\NormalTok{, }\DecValTok{3}\NormalTok{), }\FunctionTok{c}\NormalTok{ (}\DecValTok{2}\NormalTok{, }\DecValTok{3}\NormalTok{), }\AttributeTok{nomatch =} \DecValTok{0}\NormalTok{)}
\CommentTok{\#\textgreater{} [1] 0 1 2 0 0 2}
\FunctionTok{match}\NormalTok{ (}\FunctionTok{c}\NormalTok{ (}\DecValTok{1}\SpecialCharTok{:}\DecValTok{5}\NormalTok{, }\DecValTok{3}\NormalTok{), }\FunctionTok{c}\NormalTok{ (}\DecValTok{3}\NormalTok{, }\DecValTok{2}\NormalTok{), }\AttributeTok{nomatch =} \DecValTok{0}\NormalTok{)}
\CommentTok{\#\textgreater{} [1] 0 2 1 0 0 1}
\end{Highlighting}
\end{Shaded}

\begin{enumerate}
\def\labelenumi{(\alph{enumi})}
\setcounter{enumi}{5}
\tightlist
\item
  The following example provides an illustration of the usage of \texttt{match()}:
\end{enumerate}

\begin{Shaded}
\begin{Highlighting}[]
\NormalTok{month.num }\OtherTok{\textless{}{-}} \DecValTok{5}\SpecialCharTok{:}\DecValTok{9}
\NormalTok{month.name }\OtherTok{\textless{}{-}} \FunctionTok{c}\NormalTok{(}\StringTok{"May"}\NormalTok{, }\StringTok{"June"}\NormalTok{, }\StringTok{"July"}\NormalTok{, }\StringTok{"Aug"}\NormalTok{, }\StringTok{"Sept"}\NormalTok{)}
\NormalTok{new.vec }\OtherTok{\textless{}{-}}\NormalTok{  month.name [}\FunctionTok{match}\NormalTok{ (airquality [, }\StringTok{"Month"}\NormalTok{], month.num)]}
\NormalTok{out }\OtherTok{\textless{}{-}} \FunctionTok{data.frame}\NormalTok{ (airquality [ ,}\DecValTok{1}\SpecialCharTok{:}\DecValTok{5}\NormalTok{], }\AttributeTok{MonthName =}\NormalTok{ new.vec, }
                   \AttributeTok{Day =}\NormalTok{ airquality}\SpecialCharTok{$}\NormalTok{Day)}
\NormalTok{out[}\FunctionTok{c}\NormalTok{(}\DecValTok{1}\SpecialCharTok{:}\DecValTok{5}\NormalTok{,}\DecValTok{148}\SpecialCharTok{:}\DecValTok{153}\NormalTok{), ]}
\CommentTok{\#\textgreater{}     Ozone Solar.R Wind Temp Month MonthName Day}
\CommentTok{\#\textgreater{} 1      41     190  7.4   67     5       May   1}
\CommentTok{\#\textgreater{} 2      36     118  8.0   72     5       May   2}
\CommentTok{\#\textgreater{} 3      12     149 12.6   74     5       May   3}
\CommentTok{\#\textgreater{} 4      18     313 11.5   62     5       May   4}
\CommentTok{\#\textgreater{} 5      NA      NA 14.3   56     5       May   5}
\CommentTok{\#\textgreater{} 148    14      20 16.6   63     9      Sept  25}
\CommentTok{\#\textgreater{} 149    30     193  6.9   70     9      Sept  26}
\CommentTok{\#\textgreater{} 150    NA     145 13.2   77     9      Sept  27}
\CommentTok{\#\textgreater{} 151    14     191 14.3   75     9      Sept  28}
\CommentTok{\#\textgreater{} 152    18     131  8.0   76     9      Sept  29}
\CommentTok{\#\textgreater{} 153    20     223 11.5   68     9      Sept  30}
\end{Highlighting}
\end{Shaded}

\begin{enumerate}
\def\labelenumi{(\alph{enumi})}
\setcounter{enumi}{6}
\tightlist
\item
  The function \texttt{switch()} provides an alternative to a set of nested if-else statements. It takes as first argument, \texttt{EXPR}, an integer value or a character string and as second argument, \texttt{...}, the list of alternatives. As an illustration:
\end{enumerate}

\begin{Shaded}
\begin{Highlighting}[]
\NormalTok{centre }\OtherTok{\textless{}{-}} \ControlFlowTok{function}\NormalTok{(x, type) }
\NormalTok{  \{ }\ControlFlowTok{switch}\NormalTok{(type,}
           \AttributeTok{mean =} \FunctionTok{mean}\NormalTok{(x),}
           \AttributeTok{median =} \FunctionTok{median}\NormalTok{(x),}
           \AttributeTok{trimmed =} \FunctionTok{mean}\NormalTok{(x, }\AttributeTok{trim =} \FloatTok{0.1}\NormalTok{))}
\NormalTok{  \}}

\NormalTok{x }\OtherTok{\textless{}{-}} \FunctionTok{rcauchy}\NormalTok{(}\DecValTok{10}\NormalTok{)}
\NormalTok{x}
\CommentTok{\#\textgreater{}  [1] {-}0.6897862  0.9203964 {-}3.4916787  8.3234230  0.1589843}
\CommentTok{\#\textgreater{}  [6] {-}5.1391375 {-}2.1279538 {-}9.5079710  7.2078543 {-}0.1195617}
\FunctionTok{centre}\NormalTok{(x,}\StringTok{"mean"}\NormalTok{)}
\CommentTok{\#\textgreater{} [1] {-}0.4465431}
\FunctionTok{centre}\NormalTok{(x,}\StringTok{"median"}\NormalTok{)}
\CommentTok{\#\textgreater{} [1] {-}0.4046739}
\FunctionTok{centre}\NormalTok{(x,}\StringTok{"trimmed"}\NormalTok{)}
\CommentTok{\#\textgreater{} [1] {-}0.4101104}
\end{Highlighting}
\end{Shaded}

\begin{enumerate}
\def\labelenumi{(\alph{enumi})}
\setcounter{enumi}{7}
\tightlist
\item
  The two logical control operators \texttt{\&\&} and \texttt{\textbar{}\textbar{}} are useful when using if-else statements. These two operators operate on logical expressions in contrast to the operators \texttt{\&} and \texttt{\textbar{}} which operate on vectors/matrices.
\end{enumerate}

\section{Loops in R}\label{loops-in-r}

\begin{enumerate}
\def\labelenumi{(\alph{enumi})}
\tightlist
\item
  \texttt{for} loops: The general form is
\end{enumerate}

\begin{Shaded}
\begin{Highlighting}[]
\NormalTok{for (name in values)}
\NormalTok{      \{ expression(s)}
\NormalTok{      \}}
\end{Highlighting}
\end{Shaded}

This type of loop is useful if it is known in advance \emph{{how many times}} the statements in the loop are to be performed. In the above definition values can be either a vector or a list with elements not restricted to be numeric:

\begin{Shaded}
\begin{Highlighting}[]
\ControlFlowTok{for}\NormalTok{ (i }\ControlFlowTok{in} \DecValTok{1}\SpecialCharTok{:}\DecValTok{26}\NormalTok{) }\FunctionTok{cat}\NormalTok{(i, letters[i],}\StringTok{"}\SpecialCharTok{\textbackslash{}n}\StringTok{"}\NormalTok{)}
\CommentTok{\#\textgreater{} 1 a }
\CommentTok{\#\textgreater{} 2 b }
\CommentTok{\#\textgreater{} 3 c }
\CommentTok{\#\textgreater{} 4 d }
\CommentTok{\#\textgreater{} 5 e }
\CommentTok{\#\textgreater{} 6 f }
\CommentTok{\#\textgreater{} 7 g }
\CommentTok{\#\textgreater{} 8 h }
\CommentTok{\#\textgreater{} 9 i }
\CommentTok{\#\textgreater{} 10 j }
\CommentTok{\#\textgreater{} 11 k }
\CommentTok{\#\textgreater{} 12 l }
\CommentTok{\#\textgreater{} 13 m }
\CommentTok{\#\textgreater{} 14 n }
\CommentTok{\#\textgreater{} 15 o }
\CommentTok{\#\textgreater{} 16 p }
\CommentTok{\#\textgreater{} 17 q }
\CommentTok{\#\textgreater{} 18 r }
\CommentTok{\#\textgreater{} 19 s }
\CommentTok{\#\textgreater{} 20 t }
\CommentTok{\#\textgreater{} 21 u }
\CommentTok{\#\textgreater{} 22 v }
\CommentTok{\#\textgreater{} 23 w }
\CommentTok{\#\textgreater{} 24 x }
\CommentTok{\#\textgreater{} 25 y }
\CommentTok{\#\textgreater{} 26 z}
\ControlFlowTok{for}\NormalTok{ (letter }\ControlFlowTok{in}\NormalTok{ letters) }\FunctionTok{cat}\NormalTok{(letter, }\StringTok{"}\SpecialCharTok{\textbackslash{}n}\StringTok{"}\NormalTok{)}
\CommentTok{\#\textgreater{} a }
\CommentTok{\#\textgreater{} b }
\CommentTok{\#\textgreater{} c }
\CommentTok{\#\textgreater{} d }
\CommentTok{\#\textgreater{} e }
\CommentTok{\#\textgreater{} f }
\CommentTok{\#\textgreater{} g }
\CommentTok{\#\textgreater{} h }
\CommentTok{\#\textgreater{} i }
\CommentTok{\#\textgreater{} j }
\CommentTok{\#\textgreater{} k }
\CommentTok{\#\textgreater{} l }
\CommentTok{\#\textgreater{} m }
\CommentTok{\#\textgreater{} n }
\CommentTok{\#\textgreater{} o }
\CommentTok{\#\textgreater{} p }
\CommentTok{\#\textgreater{} q }
\CommentTok{\#\textgreater{} r }
\CommentTok{\#\textgreater{} s }
\CommentTok{\#\textgreater{} t }
\CommentTok{\#\textgreater{} u }
\CommentTok{\#\textgreater{} v }
\CommentTok{\#\textgreater{} w }
\CommentTok{\#\textgreater{} x }
\CommentTok{\#\textgreater{} y }
\CommentTok{\#\textgreater{} z}
\end{Highlighting}
\end{Shaded}

Consider a list consisting of several matrices, each with different numbers of rows but the same number of columns. Write an R function that will create a single matrix consisting of all the elements of the given list concatenated by rows.

\begin{enumerate}
\def\labelenumi{(\alph{enumi})}
\setcounter{enumi}{1}
\tightlist
\item
  \texttt{while} loops: The general form is
\end{enumerate}

\begin{Shaded}
\begin{Highlighting}[]
\NormalTok{while (condition)}
\NormalTok{        \{ expression(s)}
\NormalTok{        \}}
\end{Highlighting}
\end{Shaded}

This type of loop continues while condition evaluates to TRUE.

\begin{enumerate}
\def\labelenumi{(\alph{enumi})}
\setcounter{enumi}{2}
\tightlist
\item
  Control inside loops: \texttt{next} and \texttt{break}
\end{enumerate}

The command \texttt{next} is used to skip over any remaining statements in the loop and continue executing. The command \texttt{break} causes the immediate exit from the loop. In nested loops these commands apply to the most recently opened loop.

\begin{enumerate}
\def\labelenumi{(\alph{enumi})}
\setcounter{enumi}{3}
\tightlist
\item
  \texttt{repeat} loops: The general form is
\end{enumerate}

\begin{Shaded}
\begin{Highlighting}[]
\NormalTok{repeat \{ expression(s)}
\NormalTok{       \}}
\end{Highlighting}
\end{Shaded}

This type of loop continues until a break command is encountered.

\begin{enumerate}
\def\labelenumi{(\alph{enumi})}
\setcounter{enumi}{4}
\item
  Remember that many operations that might be handled by loops can be more efficiently performed in R by using the subscripting tools discussed earlier.
\item
  As a further example we will consider the calculation of the Pearson chi-squared statistic for the test of independence in a two-way classification table:
\end{enumerate}

\[
\chi^2_p = \sum_{i=1}^r \sum_{j=1}^c \frac{(f_{ij}-e_{ij})^2}{e_{ij}}
\]

with \(e_{ij} = \frac{f_{i.}f_{.j}}{f_{..}}\) the expected frequencies. This statistic can be calculated in R without using loops as follows:

\begin{Shaded}
\begin{Highlighting}[]
\NormalTok{fi. }\OtherTok{\textless{}{-}}\NormalTok{ ftable }\SpecialCharTok{\%*\%} \FunctionTok{rep}\NormalTok{ (}\DecValTok{1}\NormalTok{, }\FunctionTok{ncol}\NormalTok{ (ftable))}
\NormalTok{f.j }\OtherTok{\textless{}{-}} \FunctionTok{rep}\NormalTok{ (}\DecValTok{1}\NormalTok{, }\FunctionTok{nrow}\NormalTok{ (ftable)) }\SpecialCharTok{\%*\%}\NormalTok{ ftable}
\NormalTok{e }\OtherTok{\textless{}{-}}\NormalTok{ (fi. }\SpecialCharTok{\%*\%}\NormalTok{ f.j)}\SpecialCharTok{/}\FunctionTok{sum}\NormalTok{(fi.)}
\NormalTok{X2p }\OtherTok{\textless{}{-}} \FunctionTok{sum}\NormalTok{ ( (ftable}\SpecialCharTok{{-}}\NormalTok{e)}\SpecialCharTok{\^{}}\DecValTok{2} \SpecialCharTok{/}\NormalTok{e)}
\end{Highlighting}
\end{Shaded}

Explicit loops in R can potentially be expensive in terms of time and memory. The functions \texttt{apply()}, \texttt{tapply()}, \texttt{sapply()} and \texttt{lapply()} should be used instead if possible. The expected frequencies in the previous example can, for example, be obtained as follows:

\begin{Shaded}
\begin{Highlighting}[]
\NormalTok{e.freq }\OtherTok{\textless{}{-}} \FunctionTok{outer}\NormalTok{ (}\FunctionTok{apply}\NormalTok{ (ftable, }\DecValTok{1}\NormalTok{, sum),  }\FunctionTok{apply}\NormalTok{ (ftable, }\DecValTok{2}\NormalTok{, sum)) }\SpecialCharTok{/} \FunctionTok{sum}\NormalTok{(ftable)}
\end{Highlighting}
\end{Shaded}

\section{The execution time of R tasks}\label{the-execution-time-of-r-tasks}

The functions \texttt{system.time()} and \texttt{proc.time()} provide information regarding the execution of R tasks.

\begin{enumerate}
\def\labelenumi{(\alph{enumi})}
\tightlist
\item
  \texttt{proc.time} determines how much real and CPU time (in seconds) the currently running R process has already take:
\end{enumerate}

\begin{Shaded}
\begin{Highlighting}[]
\FunctionTok{proc.time}\NormalTok{()   }\CommentTok{\# called with no arguments}
\CommentTok{\#\textgreater{}    user  system elapsed }
\CommentTok{\#\textgreater{}    0.25    0.03    3.42}
\end{Highlighting}
\end{Shaded}

\begin{enumerate}
\def\labelenumi{(\alph{enumi})}
\setcounter{enumi}{1}
\tightlist
\item
  \texttt{system.time(expr)} calls the function \texttt{proc.time()}, evaluates \texttt{expr}, and then calls \texttt{proc.time()} once more, returning the difference between the two \texttt{proc.time()} calls:
\end{enumerate}

\begin{Shaded}
\begin{Highlighting}[]
\FunctionTok{system.time}\NormalTok{ (}\FunctionTok{hist}\NormalTok{ (}\FunctionTok{rev}\NormalTok{ (}\FunctionTok{sort}\NormalTok{ (}\FunctionTok{rnorm}\NormalTok{ (}\DecValTok{1000000}\NormalTok{)))))}
\end{Highlighting}
\end{Shaded}

\pandocbounded{\includegraphics[keepaspectratio]{08-mapping_files/figure-latex/systemtimeExample-1.pdf}}

\begin{verbatim}
#>    user  system elapsed 
#>    0.09    0.03    0.21
\end{verbatim}

Note that user and system times do not necessarily add up to elapsed time exactly.

\begin{enumerate}
\def\labelenumi{(\alph{enumi})}
\setcounter{enumi}{2}
\item
  Write the necessary code using \texttt{proc.time()} directly to obtain the execution time of \texttt{hist\ (rev\ (sort\ (rnorm\ (1000000))))}.
\item
  As an application of \texttt{system.time()} and \texttt{proc.time()} perform the following simulation study: Given a covariance matrix \(\mathbf{S}:p \times p\) the task is to compute the corresponding correlation matrix. The execution times of the following three methods are to be compared:

  \begin{enumerate}
  \def\labelenumii{(\roman{enumii})}
  \item
    Direct elementwise calculation of \(r_{ij} = \frac{s_{ij}}{\sqrt{s_{ii}s_{jj}}}\) using two nested for loops;
  \item
    Two applications of \texttt{sweep()};
  \item
    Matrix multiplication where \(\mathbf{R}:p \times p = [diag(\mathbf{S})]^{-\frac{1}{2}} \mathbf{S} [diag(\mathbf{S})]^{-\frac{1}{2}}\) where \(diag(\mathbf{A})\) denotes the diagonal matrix formed from \(\mathbf{A}:p \times p\) by setting all its off-diagonal elements equal to zero.
  \end{enumerate}
\end{enumerate}

Use \texttt{var()} and \texttt{rnorm()} to compute covariance matrices of different sizes \(p\) from samples varying in size \(n\). Study the role of \(n\) and \(p\) in the effectiveness (economy in execution time) of the above three methods. Display the results graphically. Remember that for valid comparisons the three methods must be executed with identical samples.

\section{The calling of functions with argument lists}\label{the-calling-of-functions-with-argument-lists}

\begin{enumerate}
\def\labelenumi{(\alph{enumi})}
\tightlist
\item
  The function \texttt{do.call()} provides an alternative to the usual method of calling functions by name. It allows specifying the name of the function with its arguments in the form of a list:
\end{enumerate}

\begin{Shaded}
\begin{Highlighting}[]
\FunctionTok{mean}\NormalTok{ ( }\FunctionTok{c}\NormalTok{ (}\DecValTok{1}\SpecialCharTok{:}\DecValTok{100}\NormalTok{, }\DecValTok{500}\NormalTok{), }\AttributeTok{trim=}\FloatTok{0.1}\NormalTok{)}
\CommentTok{\#\textgreater{} [1] 51}
\FunctionTok{do.call}\NormalTok{ (}\StringTok{"mean"}\NormalTok{, }\FunctionTok{list}\NormalTok{( }\FunctionTok{c}\NormalTok{ (}\DecValTok{1}\SpecialCharTok{:}\DecValTok{100}\NormalTok{, }\DecValTok{500}\NormalTok{), }\AttributeTok{trim=}\FloatTok{0.1}\NormalTok{))}
\CommentTok{\#\textgreater{} [1] 51}
\end{Highlighting}
\end{Shaded}

\begin{enumerate}
\def\labelenumi{(\alph{enumi})}
\setcounter{enumi}{1}
\item
  How does \texttt{do.call()} differ from the function \texttt{call()}?
\item
  As an illustration of the usage of \texttt{do.call()} study the following example:
\end{enumerate}

\begin{Shaded}
\begin{Highlighting}[]
\NormalTok{na.pattern }\OtherTok{\textless{}{-}} \ControlFlowTok{function}\NormalTok{(frame)}
\NormalTok{\{ nas }\OtherTok{\textless{}{-}} \FunctionTok{is.na}\NormalTok{ (frame)}
  \FunctionTok{storage.mode}\NormalTok{ (nas) }\OtherTok{\textless{}{-}} \StringTok{"integer"}
  \FunctionTok{table}\NormalTok{ (}\FunctionTok{do.call}\NormalTok{ (}\StringTok{"paste"}\NormalTok{, }\FunctionTok{c}\NormalTok{(}\FunctionTok{as.data.frame}\NormalTok{(nas), }\AttributeTok{sep =} \StringTok{""}\NormalTok{)))}
\NormalTok{\}}
\FunctionTok{na.pattern}\NormalTok{(}\FunctionTok{as.data.frame}\NormalTok{(airquality))}
\CommentTok{\#\textgreater{} }
\CommentTok{\#\textgreater{} 000000 010000 100000 110000 }
\CommentTok{\#\textgreater{}    111      5     35      2}
\end{Highlighting}
\end{Shaded}

What can be learned from the above output?

\begin{enumerate}
\def\labelenumi{(\alph{enumi})}
\setcounter{enumi}{3}
\tightlist
\item
  What is the difference between \texttt{as.integer()}, \texttt{storage.mode()\ \textless{}–\ "integer"}, \texttt{storage.mode()} and \texttt{mode()}?
\end{enumerate}

\section{Evaluating R strings a commands}\label{evaluating-r-strings-a-commands}

Recall from Figure \ref{fig:expression} that the function \texttt{parse(text\ =\ "3\ +\ 4")} returns the unevaluated expression \texttt{3\ +\ 4}. In order to evaluate the expression use function \texttt{eval()}: \texttt{eval\ (parse\ (text\ =\ "3\ +\ 4"))} returns \texttt{7}.

\section{Object oriented programming in R}\label{object-oriented-programming-in-r}

Suppose we would like to investigate the body of function \texttt{plot()}. We know that this can be done by entering the function's name at the R prompt:

\begin{Shaded}
\begin{Highlighting}[]
\NormalTok{plot}
\CommentTok{\#\textgreater{} function (x, y, ...) }
\CommentTok{\#\textgreater{} UseMethod("plot")}
\CommentTok{\#\textgreater{} \textless{}bytecode: 0x00000262a8958d80\textgreater{}}
\CommentTok{\#\textgreater{} \textless{}environment: namespace:base\textgreater{}}
\end{Highlighting}
\end{Shaded}

The presence of \texttt{UseMethod("plot")} shows that \texttt{plot()} is a \emph{{generic}} function. The \emph{{class}} of an object determines how it will be treated by a generic function i.e.~what \emph{{method}} will be applied to it. Function \texttt{setClass()} is used for setting the class attribute of an object. Function \texttt{methods()} is used to find out (a) what is the repertoire of methods of a generic function and (b) what methods are available for a certain class:

\begin{Shaded}
\begin{Highlighting}[]
\FunctionTok{methods}\NormalTok{(plot) }\CommentTok{\# repertoire of methods for FUNCTION plot()}
\CommentTok{\#\textgreater{}  [1] plot.acf*           plot.data.frame*   }
\CommentTok{\#\textgreater{}  [3] plot.decomposed.ts* plot.default       }
\CommentTok{\#\textgreater{}  [5] plot.dendrogram*    plot.density*      }
\CommentTok{\#\textgreater{}  [7] plot.ecdf           plot.factor*       }
\CommentTok{\#\textgreater{}  [9] plot.formula*       plot.function      }
\CommentTok{\#\textgreater{} [11] plot.hclust*        plot.histogram*    }
\CommentTok{\#\textgreater{} [13] plot.HoltWinters*   plot.isoreg*       }
\CommentTok{\#\textgreater{} [15] plot.lm*            plot.medpolish*    }
\CommentTok{\#\textgreater{} [17] plot.mlm*           plot.ppr*          }
\CommentTok{\#\textgreater{} [19] plot.prcomp*        plot.princomp*     }
\CommentTok{\#\textgreater{} [21] plot.profile*       plot.profile.nls*  }
\CommentTok{\#\textgreater{} [23] plot.raster*        plot.spec*         }
\CommentTok{\#\textgreater{} [25] plot.stepfun        plot.stl*          }
\CommentTok{\#\textgreater{} [27] plot.table*         plot.ts            }
\CommentTok{\#\textgreater{} [29] plot.tskernel*      plot.TukeyHSD*     }
\CommentTok{\#\textgreater{} see \textquotesingle{}?methods\textquotesingle{} for accessing help and source code}
\FunctionTok{methods}\NormalTok{(}\AttributeTok{class=}\StringTok{"lm"}\NormalTok{)  }\CommentTok{\# what methods are available for CLASS lm}
\CommentTok{\#\textgreater{}  [1] add1           alias          anova         }
\CommentTok{\#\textgreater{}  [4] case.names     coerce         confint       }
\CommentTok{\#\textgreater{}  [7] cooks.distance deviance       dfbeta        }
\CommentTok{\#\textgreater{} [10] dfbetas        drop1          dummy.coef    }
\CommentTok{\#\textgreater{} [13] effects        extractAIC     family        }
\CommentTok{\#\textgreater{} [16] formula        hatvalues      influence     }
\CommentTok{\#\textgreater{} [19] initialize     kappa          labels        }
\CommentTok{\#\textgreater{} [22] logLik         model.frame    model.matrix  }
\CommentTok{\#\textgreater{} [25] nobs           plot           predict       }
\CommentTok{\#\textgreater{} [28] print          proj           qr            }
\CommentTok{\#\textgreater{} [31] residuals      rstandard      rstudent      }
\CommentTok{\#\textgreater{} [34] show           simulate       slotsFromS3   }
\CommentTok{\#\textgreater{} [37] summary        variable.names vcov          }
\CommentTok{\#\textgreater{} see \textquotesingle{}?methods\textquotesingle{} for accessing help and source code}
\end{Highlighting}
\end{Shaded}

In broad terms there are currently three types of classes in use in R: The old classes or S3 classes and the newer S4 and S5 (also called \emph{{reference classes}}) classes. The newer classes can contain one or more \emph{{slots}} which can be accessed using the operator \texttt{@}. Central to the concept of object oriented programming is that a method can inherit from another method. The function \texttt{NextMethod()} provides a mechanism for \emph{{inheritance}}.

\begin{enumerate}
\def\labelenumi{(\alph{enumi})}
\item
  As an example of a generic function study the example in the help file of the function \texttt{all.equal()}.
\item
  R provides many more facilities for writing object oriented functions. Consult the \href{https://cran.r-project.org/doc/manuals/r-release/R-lang.pdf}{R Language Definition Manual} Chapter 5: Object-Oriented Programming for further details.
\item
  A statistical investigation is often concerned with survey or questionnaire data where respondents must select one of several categorical alternatives. The \texttt{questdata} below shows the responses made by 10 respondents on four questions. The alternatives for each question were measured on a five point categorical scale. We can refer to the \texttt{questdata\ dataframe} as the full data. This form of representing the data is not an effective way of storing the data when the number of respondents is large. A more compact way of saving the data without any loss in information is to store the data in the form of a \emph{{response pattern}} matrix or dataframe. The first row of \texttt{questdata} constitutes one particular response pattern namely \texttt{("b"\ "c"\ "a"\ "d")}. A response pattern matrix (dataframe) shows all the unique response patterns together with the frequency with which each of the different response patterns has occurred. Your challenge is to provide the necessary R functions to convert the full data into a response pattern representation, and conversely to recover the full data from its response pattern representation.
\end{enumerate}

\begin{Shaded}
\begin{Highlighting}[]
\NormalTok{questdata }\OtherTok{\textless{}{-}} \FunctionTok{rbind}\NormalTok{ (}\FunctionTok{c}\NormalTok{(}\StringTok{"b"}\NormalTok{, }\StringTok{"c"}\NormalTok{, }\StringTok{"a"}\NormalTok{, }\StringTok{"d"}\NormalTok{),}
                    \FunctionTok{c}\NormalTok{(}\StringTok{"d"}\NormalTok{, }\StringTok{"d"}\NormalTok{, }\StringTok{"c"}\NormalTok{, }\StringTok{"a"}\NormalTok{),}
                    \FunctionTok{c}\NormalTok{(}\StringTok{"a"}\NormalTok{, }\StringTok{"d"}\NormalTok{, }\StringTok{"c"}\NormalTok{, }\StringTok{"e"}\NormalTok{),}
                    \FunctionTok{c}\NormalTok{(}\StringTok{"a"}\NormalTok{, }\StringTok{"d"}\NormalTok{, }\StringTok{"c"}\NormalTok{, }\StringTok{"e"}\NormalTok{),}
                    \FunctionTok{c}\NormalTok{(}\StringTok{"b"}\NormalTok{, }\StringTok{"c"}\NormalTok{, }\StringTok{"a"}\NormalTok{, }\StringTok{"d"}\NormalTok{),}
                    \FunctionTok{c}\NormalTok{(}\StringTok{"a"}\NormalTok{, }\StringTok{"d"}\NormalTok{, }\StringTok{"c"}\NormalTok{, }\StringTok{"e"}\NormalTok{),}
                    \FunctionTok{c}\NormalTok{(}\StringTok{"b"}\NormalTok{, }\StringTok{"c"}\NormalTok{, }\StringTok{"a"}\NormalTok{, }\StringTok{"d"}\NormalTok{),}
                    \FunctionTok{c}\NormalTok{(}\StringTok{"d"}\NormalTok{, }\StringTok{"d"}\NormalTok{, }\StringTok{"c"}\NormalTok{, }\StringTok{"a"}\NormalTok{),}
                    \FunctionTok{c}\NormalTok{(}\StringTok{"c"}\NormalTok{, }\StringTok{"b"}\NormalTok{, }\StringTok{"a"}\NormalTok{, }\StringTok{"e"}\NormalTok{),}
                    \FunctionTok{c}\NormalTok{(}\StringTok{"b"}\NormalTok{, }\StringTok{"c"}\NormalTok{, }\StringTok{"a"}\NormalTok{, }\StringTok{"d"}\NormalTok{))}
\FunctionTok{colnames}\NormalTok{(questdata) }\OtherTok{\textless{}{-}} \FunctionTok{c}\NormalTok{(}\StringTok{"Q1"}\NormalTok{, }\StringTok{"Q2"}\NormalTok{, }\StringTok{"Q3"}\NormalTok{, }\StringTok{"Q4"}\NormalTok{)}
\end{Highlighting}
\end{Shaded}

\begin{enumerate}
\def\labelenumi{(\roman{enumi})}
\tightlist
\item
  Create the R object \texttt{questdata} and then give the following instructions:
\end{enumerate}

\begin{Shaded}
\begin{Highlighting}[]
\FunctionTok{unique}\NormalTok{ (questdata [,}\DecValTok{1}\NormalTok{])}
\CommentTok{\#\textgreater{} [1] "b" "d" "a" "c"}
\FunctionTok{duplicated}\NormalTok{ (questdata)}
\CommentTok{\#\textgreater{}  [1] FALSE FALSE FALSE  TRUE  TRUE  TRUE  TRUE  TRUE FALSE}
\CommentTok{\#\textgreater{} [10]  TRUE}
\FunctionTok{duplicated}\NormalTok{ (questdata, }\AttributeTok{MARGIN =} \DecValTok{1}\NormalTok{)}
\CommentTok{\#\textgreater{}  [1] FALSE FALSE FALSE  TRUE  TRUE  TRUE  TRUE  TRUE FALSE}
\CommentTok{\#\textgreater{} [10]  TRUE}
\FunctionTok{duplicated}\NormalTok{ (questdata, }\AttributeTok{MARGIN =} \DecValTok{2}\NormalTok{)}
\CommentTok{\#\textgreater{}    Q1    Q2    Q3    Q4 }
\CommentTok{\#\textgreater{} FALSE FALSE FALSE FALSE}
\FunctionTok{unique}\NormalTok{ (questdata)}
\CommentTok{\#\textgreater{}      Q1  Q2  Q3  Q4 }
\CommentTok{\#\textgreater{} [1,] "b" "c" "a" "d"}
\CommentTok{\#\textgreater{} [2,] "d" "d" "c" "a"}
\CommentTok{\#\textgreater{} [3,] "a" "d" "c" "e"}
\CommentTok{\#\textgreater{} [4,] "c" "b" "a" "e"}
\FunctionTok{unique}\NormalTok{ (questdata, }\AttributeTok{MARGIN =} \DecValTok{1}\NormalTok{)}
\CommentTok{\#\textgreater{}      Q1  Q2  Q3  Q4 }
\CommentTok{\#\textgreater{} [1,] "b" "c" "a" "d"}
\CommentTok{\#\textgreater{} [2,] "d" "d" "c" "a"}
\CommentTok{\#\textgreater{} [3,] "a" "d" "c" "e"}
\CommentTok{\#\textgreater{} [4,] "c" "b" "a" "e"}
\FunctionTok{unique}\NormalTok{ (questdata, }\AttributeTok{MARGIN =} \DecValTok{2}\NormalTok{)}
\CommentTok{\#\textgreater{}       Q1  Q2  Q3  Q4 }
\CommentTok{\#\textgreater{}  [1,] "b" "c" "a" "d"}
\CommentTok{\#\textgreater{}  [2,] "d" "d" "c" "a"}
\CommentTok{\#\textgreater{}  [3,] "a" "d" "c" "e"}
\CommentTok{\#\textgreater{}  [4,] "a" "d" "c" "e"}
\CommentTok{\#\textgreater{}  [5,] "b" "c" "a" "d"}
\CommentTok{\#\textgreater{}  [6,] "a" "d" "c" "e"}
\CommentTok{\#\textgreater{}  [7,] "b" "c" "a" "d"}
\CommentTok{\#\textgreater{}  [8,] "d" "d" "c" "a"}
\CommentTok{\#\textgreater{}  [9,] "c" "b" "a" "e"}
\CommentTok{\#\textgreater{} [10,] "b" "c" "a" "d"}
\end{Highlighting}
\end{Shaded}

\begin{enumerate}
\def\labelenumi{(\roman{enumi})}
\setcounter{enumi}{1}
\item
  Examine Table \ref{tab:MatrixFunc} and carefully describe the behaviour of the functions \texttt{duplicated()} and \texttt{unique()}.
\item
  Write an R function, say \texttt{full2resp} to obtain the response pattern representation of questionnaire data like those given above. Test your function on \texttt{questdata}.
\item
  Write an R function, say \texttt{resp2full} to obtain the full data set given its response pattern representation. Test your function on the response pattern representation of the \texttt{questdata}.
\end{enumerate}

\section{Recursion}\label{recursion}

Functions in R can call themselves. This process is called \emph{{recursion}} and it is implemented in R programming by the function \texttt{Recall()}.

\begin{enumerate}
\def\labelenumi{(\alph{enumi})}
\tightlist
\item
  As an example we will use recursion to calculate \(x(x+1)(x+2)\dots(x+k)\) with \(k\) a positive integer:
\end{enumerate}

\begin{Shaded}
\begin{Highlighting}[]
\NormalTok{recurs.example }\OtherTok{\textless{}{-}} \ControlFlowTok{function}\NormalTok{ (x, k) }
\NormalTok{\{ }\CommentTok{\# Function to calculate x(x+1)(x+2).....(x+k)}
  \CommentTok{\# where k is a positive integer.}
     \ControlFlowTok{if}\NormalTok{ (k }\SpecialCharTok{\textless{}} \DecValTok{0}\NormalTok{ ) }
      \FunctionTok{stop}\NormalTok{(}\StringTok{"k not allowed to be negative or non{-}integer"}\NormalTok{)}
    \ControlFlowTok{else} \ControlFlowTok{if}\NormalTok{( k }\SpecialCharTok{==} \DecValTok{0}\NormalTok{) x}
       \ControlFlowTok{else}\NormalTok{(x}\SpecialCharTok{+}\NormalTok{k) }\SpecialCharTok{*} \FunctionTok{Recall}\NormalTok{(x,k}\DecValTok{{-}1}\NormalTok{)}
\NormalTok{   \}}
\end{Highlighting}
\end{Shaded}

Investigate if \texttt{recurs.example()} works correctly.

\begin{enumerate}
\def\labelenumi{(\alph{enumi})}
\setcounter{enumi}{1}
\tightlist
\item
  Explain how recursion works by studying the output of the following function for values of \(r = 1, 2, 3, 4, 5, 6\):
\end{enumerate}

\begin{Shaded}
\begin{Highlighting}[]
\NormalTok{Recursiontest }\OtherTok{\textless{}{-}} \ControlFlowTok{function}\NormalTok{ (r)}
\NormalTok{\{ }\ControlFlowTok{if}\NormalTok{ (r }\SpecialCharTok{\textless{}=} \DecValTok{0}\NormalTok{) }\ConstantTok{NULL}
  \ControlFlowTok{else}\NormalTok{ \{ }\FunctionTok{cat}\NormalTok{(}\StringTok{"Write = "}\NormalTok{, r, }\StringTok{"}\SpecialCharTok{\textbackslash{}n}\StringTok{"}\NormalTok{)}
         \FunctionTok{Recall}\NormalTok{ (r }\SpecialCharTok{{-}} \DecValTok{1}\NormalTok{)}
         \FunctionTok{Recall}\NormalTok{ (r }\SpecialCharTok{{-}} \DecValTok{2}\NormalTok{)}
\NormalTok{       \}}
\NormalTok{\}}
\FunctionTok{Recursiontest}\NormalTok{(}\DecValTok{1}\NormalTok{)}
\CommentTok{\#\textgreater{} Write =  1}
\CommentTok{\#\textgreater{} NULL}
\FunctionTok{Recursiontest}\NormalTok{(}\DecValTok{2}\NormalTok{)}
\CommentTok{\#\textgreater{} Write =  2 }
\CommentTok{\#\textgreater{} Write =  1}
\CommentTok{\#\textgreater{} NULL}
\FunctionTok{Recursiontest}\NormalTok{(}\DecValTok{3}\NormalTok{)}
\CommentTok{\#\textgreater{} Write =  3 }
\CommentTok{\#\textgreater{} Write =  2 }
\CommentTok{\#\textgreater{} Write =  1 }
\CommentTok{\#\textgreater{} Write =  1}
\CommentTok{\#\textgreater{} NULL}
\FunctionTok{Recursiontest}\NormalTok{(}\DecValTok{4}\NormalTok{)}
\CommentTok{\#\textgreater{} Write =  4 }
\CommentTok{\#\textgreater{} Write =  3 }
\CommentTok{\#\textgreater{} Write =  2 }
\CommentTok{\#\textgreater{} Write =  1 }
\CommentTok{\#\textgreater{} Write =  1 }
\CommentTok{\#\textgreater{} Write =  2 }
\CommentTok{\#\textgreater{} Write =  1}
\CommentTok{\#\textgreater{} NULL}
\FunctionTok{Recursiontest}\NormalTok{(}\DecValTok{5}\NormalTok{)}
\CommentTok{\#\textgreater{} Write =  5 }
\CommentTok{\#\textgreater{} Write =  4 }
\CommentTok{\#\textgreater{} Write =  3 }
\CommentTok{\#\textgreater{} Write =  2 }
\CommentTok{\#\textgreater{} Write =  1 }
\CommentTok{\#\textgreater{} Write =  1 }
\CommentTok{\#\textgreater{} Write =  2 }
\CommentTok{\#\textgreater{} Write =  1 }
\CommentTok{\#\textgreater{} Write =  3 }
\CommentTok{\#\textgreater{} Write =  2 }
\CommentTok{\#\textgreater{} Write =  1 }
\CommentTok{\#\textgreater{} Write =  1}
\CommentTok{\#\textgreater{} NULL}
\FunctionTok{Recursiontest}\NormalTok{(}\DecValTok{6}\NormalTok{)}
\CommentTok{\#\textgreater{} Write =  6 }
\CommentTok{\#\textgreater{} Write =  5 }
\CommentTok{\#\textgreater{} Write =  4 }
\CommentTok{\#\textgreater{} Write =  3 }
\CommentTok{\#\textgreater{} Write =  2 }
\CommentTok{\#\textgreater{} Write =  1 }
\CommentTok{\#\textgreater{} Write =  1 }
\CommentTok{\#\textgreater{} Write =  2 }
\CommentTok{\#\textgreater{} Write =  1 }
\CommentTok{\#\textgreater{} Write =  3 }
\CommentTok{\#\textgreater{} Write =  2 }
\CommentTok{\#\textgreater{} Write =  1 }
\CommentTok{\#\textgreater{} Write =  1 }
\CommentTok{\#\textgreater{} Write =  4 }
\CommentTok{\#\textgreater{} Write =  3 }
\CommentTok{\#\textgreater{} Write =  2 }
\CommentTok{\#\textgreater{} Write =  1 }
\CommentTok{\#\textgreater{} Write =  1 }
\CommentTok{\#\textgreater{} Write =  2 }
\CommentTok{\#\textgreater{} Write =  1}
\CommentTok{\#\textgreater{} NULL}
\end{Highlighting}
\end{Shaded}

\begin{enumerate}
\def\labelenumi{(\alph{enumi})}
\setcounter{enumi}{2}
\item
  Use recursion and the function \texttt{Recall()} to write an R function to calculate \(x!\).
\item
  Use recursion to write an R function that generates a matrix whose rows contain subsets of size \(r\) of the first \(n\) elements of the vector \texttt{v}. Ignore the possibility of repeated values in \texttt{v} and give this vector the default value of \texttt{1:n}.
\end{enumerate}

\section{Environments in R}\label{environments-in-r}

Study the following parts from the \emph{R Language definition Manual}: § 3.5 Scope of variables; Chapter 4: \emph{Functions}.

Consider an R function \texttt{xx(argument)}. Write an R function to add a constant to the correct object (i.e.~the object in the correct environment) that corresponds to \texttt{argument}. In order to answer this question, you must determine in which environment \texttt{argument} exists and evaluation must take place in this environment. Possible candidates to consider are the \emph{{parent frame}}, the \emph{{global environment}} and the search list. Assume that only the first data basis on the search list is not read-only so that in cases where argument can be found anywhere in the search list it can be assigned to the first data basis. \emph{Hint}: Study how the following functions work: \texttt{assign()}, \texttt{deparse()}, \texttt{invisible()}, \texttt{exists()}, \texttt{substitute()}, \texttt{sys.parent()}.

\section{``Computing on the language''}\label{computing-on-the-language}

Read \emph{R Language Definition Manual Chapter 6: Computing on the language}.

\section{Writing user friendly applications: the package shiny}\label{writing-user-friendly-applications-the-package-shiny}

The \texttt{shiny} package in R allows one to create an interactive environment inside R. As an example, the code below generates data from a bivariate normal distribution and makes a scatter plot of the two variables. With shiny a sliding bar is added where the user can adjust the correlation between the two variables.

A shiny app consists of a user interface (\texttt{ui}) a \texttt{server} function and the \texttt{shinyApp} function that uses the \texttt{ui} object and the \texttt{server} function to build a Shiny app object. For the sliding bar, the function \texttt{sliderInput()} is used. Table \ref{tab:InputElements} provides a list of different input elements.

The \texttt{server} function uses the \texttt{inputs} -- the \texttt{cor.val} in this example -- to produce an \texttt{output} -- the scatter plot in this example -- using a reactive expression -- the \texttt{plot} command in this example. The \texttt{server} function and thus the reactive expression is called with every change in the \texttt{input}, i.e.~the plot is executed with the updated \texttt{cor.val}. The \texttt{output} produced by die \texttt{server} function -- \texttt{scatter} in this example -- is plotted in the \texttt{mainPanel} with the function \texttt{plotOutput}.

Table: \label{tab:InputElements} Input elements for shiny apps.

------ \textbar{} ------ \textbar{} ------ \textbar{}\\
\texttt{actionButton()} \textbar{} \texttt{fileInput()} \textbar{} \texttt{sliderInput()} \textbar{}\\
\texttt{checkboxGroupInput()} \textbar{} \texttt{numericInput()} \textbar{} \texttt{submitButton()} \textbar{}\\
\texttt{checkboxInput()} \textbar{} \texttt{passwordInput()} \textbar{} \texttt{textAreaInput()} \textbar{}\\
\texttt{dateInput()} \textbar{} \texttt{radioButtons()} \textbar{} \texttt{textInput()} \textbar{}\\
\texttt{dateRangeInput()} \textbar{} \texttt{selectInput()} \textbar{} \texttt{varSelectInput()} \textbar{}

\begin{Shaded}
\begin{Highlighting}[]
\FunctionTok{library}\NormalTok{(shiny)}

\NormalTok{ui }\OtherTok{\textless{}{-}} \FunctionTok{pageWithSidebar}\NormalTok{(}
      \FunctionTok{headerPanel}\NormalTok{(}\StringTok{"Bivariate normal plot"}\NormalTok{),}
      \CommentTok{\# App title}

      \FunctionTok{sidebarPanel}\NormalTok{(}
      \CommentTok{\# Sidebar panel for inputs}

          \FunctionTok{sliderInput}\NormalTok{(}\AttributeTok{inputId =} \StringTok{"cor.val"}\NormalTok{,}
                      \AttributeTok{label =} \StringTok{"Correlation"}\NormalTok{,}
                      \AttributeTok{min =} \SpecialCharTok{{-}}\DecValTok{1}\NormalTok{,}
                      \AttributeTok{max =} \DecValTok{1}\NormalTok{,}
                      \AttributeTok{value =} \DecValTok{0}\NormalTok{,}
                      \AttributeTok{step =} \FloatTok{0.01}
\NormalTok{          )}
\NormalTok{      ),}

      \FunctionTok{mainPanel}\NormalTok{(}
      \CommentTok{\# Main panel for scatter plot}

          \FunctionTok{textOutput}\NormalTok{(}\StringTok{"caption"}\NormalTok{),}
          \FunctionTok{plotOutput}\NormalTok{(}\StringTok{"scatter"}\NormalTok{)}
\NormalTok{      )}
\NormalTok{   )}

\NormalTok{server }\OtherTok{\textless{}{-}} \ControlFlowTok{function}\NormalTok{(input, output) \{}
         \FunctionTok{require}\NormalTok{(MASS)}
\NormalTok{         sigma }\OtherTok{\textless{}{-}} \FunctionTok{diag}\NormalTok{(}\DecValTok{2}\NormalTok{)}

\NormalTok{         output}\SpecialCharTok{$}\NormalTok{caption }\OtherTok{\textless{}{-}} \FunctionTok{renderText}\NormalTok{(\{ }\FunctionTok{paste}\NormalTok{ (}\StringTok{"Bivariate normal data with }
\StringTok{                                correlation"}\NormalTok{, input}\SpecialCharTok{$}\NormalTok{cor.val)}
\NormalTok{                           \})}
\NormalTok{         output}\SpecialCharTok{$}\NormalTok{scatter }\OtherTok{\textless{}{-}} \FunctionTok{renderPlot}\NormalTok{(\{  }
\NormalTok{                              sigma[}\DecValTok{1}\NormalTok{,}\DecValTok{2}\NormalTok{] }\OtherTok{\textless{}{-}}\NormalTok{ sigma[}\DecValTok{2}\NormalTok{,}\DecValTok{1}\NormalTok{] }\OtherTok{\textless{}{-}}\NormalTok{ input}\SpecialCharTok{$}\NormalTok{cor.val}
\NormalTok{                              X }\OtherTok{\textless{}{-}} \FunctionTok{mvrnorm}\NormalTok{(}\DecValTok{1000}\NormalTok{, }\AttributeTok{mu=}\FunctionTok{c}\NormalTok{(}\DecValTok{0}\NormalTok{,}\DecValTok{0}\NormalTok{), sigma)}
                              \FunctionTok{plot}\NormalTok{(X,}\AttributeTok{asp=}\DecValTok{1}\NormalTok{,}\AttributeTok{col=}\StringTok{"red"}\NormalTok{,}\AttributeTok{pch=}\DecValTok{15}\NormalTok{)}
\NormalTok{                           \})}
\NormalTok{      \}}

\FunctionTok{shinyApp}\NormalTok{(ui, server)}
\end{Highlighting}
\end{Shaded}

Adjust the shiny app above by adding three more input sources:

\begin{enumerate}
\def\labelenumi{\roman{enumi}.}
\item
  The number of observations to be generated.
\item
  Selecting the mean vector for the bivariate normal from the following options
\end{enumerate}

\begin{itemize}
\tightlist
\item
  \(\mathbf{\mu}' = [0, 0]\)
\item
  \(\mathbf{\mu}' = [10, 2]\)
\item
  \(\mathbf{\mu}' = [-3, -3]\)
\item
  \(\mathbf{\mu}' = [8, 207]\)
\end{itemize}

\begin{enumerate}
\def\labelenumi{\roman{enumi}.}
\setcounter{enumi}{2}
\tightlist
\item
  Having a series of radio buttons to choose the colour for the observations in the plot.
\end{enumerate}

\section{Exercise}\label{exercise-13}

\begin{enumerate}
\def\labelenumi{(\alph{enumi})}
\item
  Write an R function to determine which positive whole number elements \(≤10^{10}\) of a given vector are prime and to return these primes. Test this function with randomly generated vectors.
\item
  Repeat (a) using recursion.
\item
  Write a Shiny App that allows the user to choose between one of the data sets:\texttt{LifeCycleSavings} and \texttt{state.x77} as a data matrix \(\mathbf{X}:n \times p\). The unweighted Minkowski metric for the pairwise distance between observation \(i\) and observation \(j\) is defined as \(d_{ij} = \left( \sum_{k=1}^p{|x_{ik}-x_{jk}|^λ} \right)^{(1/λ)}\), \(λ≥1\). Make provision for the user to choose the value of \(\lambda\) to be used to calculate the pairwise distances between all the rows of the data matrix. Note that \(λ=1\) is the Manhattan distance and \(λ=2\) is the Euclidean distance. Use \(λ=2\) as your default value.
\end{enumerate}

\section{The function on.exit()}\label{the-function-on.exit}

What does the function \texttt{on.exit()} do?

One use of the special argument \texttt{...} together with the \texttt{on.exit()} function is to allow a user to make temporary changes to graphical parameters of a graphical display within a function. This can be done as follows:

\begin{Shaded}
\begin{Highlighting}[]
\ControlFlowTok{function}\NormalTok{(...)}
\NormalTok{ \{ oldpar }\OtherTok{\textless{}{-}} \FunctionTok{par}\NormalTok{(...)}
   \FunctionTok{on.exit}\NormalTok{(}\FunctionTok{par}\NormalTok{(oldpar))  }
\NormalTok{   or }\FunctionTok{on.exit}\NormalTok{(}\FunctionTok{par}\NormalTok{(}\FunctionTok{c}\NormalTok{(}\FunctionTok{par}\NormalTok{(oldpar),}\FunctionTok{par}\NormalTok{(}\AttributeTok{mfrow =} \FunctionTok{c}\NormalTok{(}\DecValTok{1}\NormalTok{,}\DecValTok{1}\NormalTok{)))))}
\NormalTok{   new plot instructions}
\NormalTok{   ..............................}
\NormalTok{  \}}
\end{Highlighting}
\end{Shaded}

In the above it is assumed that only arguments of \texttt{par()} can be substituted when the function concerned is called. A further use of \texttt{on.exit()} is for temporarily changing \emph{{options}}.

\section{Error tracing}\label{error-tracing}

Any error that is generated during the execution of a function will record details of the calls that were being executed at the time. These details can be shown by using the function \texttt{traceback()}. The function \texttt{dump.frames()} gives more detailed information, but it must be used sparingly because it can create very large objects in the \emph{{workspace}}. The function \texttt{options\ (error\ =\ xx)} can be used to specify the action taken when an error occurs. The recommended option during program development is \texttt{options(error\ =\ recover)}. This ensures that an error during an interactive session will call \texttt{recover()} from the lowest relevant function call, usually the call that produced the error. You can then browse in this or any of the currently active calls to recover arbitrary information about the state of computation at the time of the error. An alternative is to set \texttt{options(error\ =\ dump.frames)}. This will save all the data in the calls that were active when an error occurred. Calling \texttt{debugger()} later on produce a similar result to \texttt{recover()}.

The following is a summary of the most common error tracing facilities in R:

------ \textbar{} ---------------- \textbar{}\\
\texttt{print()}, \texttt{cat()} \textbar{} The printing of key values within a function is often all that is needed. \textbar{}\\
\texttt{traceback()} \textbar{} Must be used together with \texttt{dump.frames()}. \textbar{}\\
\texttt{options(warn=2)} \textbar{} Changes warning to an error that causes a dump. \textbar{}\\
\texttt{options(error=)} \textbar{} Changes the function that is used for the dump action. \textbar{}\\
\texttt{last.dump()} \textbar{} The object in the \emph{{.RData}} that contains a list of calls to dump. \textbar{}\\
\texttt{debugger()} \textbar{} Function to inspect last.dump for an error. \textbar{}\\
\texttt{browser()} \textbar{} Function that can be used within a function to interrupt the latter's execution so that variables within the local frame concerned can be inspected. \textbar{}\\
\texttt{trace()} \textbar{} Places tracing information before or within functions. Can be used to place calls to the browser at given positions within a function. \textbar{}\\
\texttt{untrace()} \textbar{} Switches all or some of the functions of \texttt{trace()} off. \textbar{}

\begin{enumerate}
\def\labelenumi{(\alph{enumi})}
\item
  Study the \emph{R Language Manual Definition Chapter 9: Debugging} for a summary of error tracing facilities in R . Note especially how the functions \texttt{print()}, \texttt{cat()}, \texttt{traceback()}, \texttt{browser()}, \texttt{trace()}, \texttt{untrace()}, \texttt{debug()}, \texttt{undebug()} and \texttt{options(warn=2\ or\ error=)} work.
\item
  Study usage of: \texttt{options(error\ =\ dump.frames);\ \ debugger()}
\item
  Study usage of: \texttt{options(error\ =\ dump.frames)}
\item
  Study usage of the objects \texttt{last.dump} and \texttt{.Traceback}.
\end{enumerate}

\section{\texorpdfstring{Error handling: The function \texttt{try()}}{Error handling: The function try()}}\label{error-handling-the-function-try}

As an example of the need to be able to handle errors properly consider a simulation study involving a large number of repetitive calculations.

\begin{Shaded}
\begin{Highlighting}[]
\NormalTok{Example.}\DecValTok{8}\NormalTok{.}\FloatTok{18.}\NormalTok{a }\OtherTok{\textless{}{-}} \ControlFlowTok{function}\NormalTok{ (}\AttributeTok{iter =} \DecValTok{500}\NormalTok{)}
\NormalTok{\{ select.sample }\OtherTok{\textless{}{-}} \ControlFlowTok{function}\NormalTok{ (x) }
\NormalTok{  \{ temp }\OtherTok{\textless{}{-}} \FunctionTok{rnorm}\NormalTok{ (}\DecValTok{100}\NormalTok{, }\AttributeTok{m =} \DecValTok{50}\NormalTok{, }\AttributeTok{s =} \DecValTok{20}\NormalTok{)}
    \ControlFlowTok{if}\NormalTok{ (}\FunctionTok{any}\NormalTok{ (temp }\SpecialCharTok{\textless{}} \DecValTok{0}\NormalTok{)) }\FunctionTok{stop}\NormalTok{(}\StringTok{"Negative numbers not allowed"}\NormalTok{)}
    \FunctionTok{mean}\NormalTok{(}\FunctionTok{log}\NormalTok{(temp))                                                         \}}
\NormalTok{  out }\OtherTok{\textless{}{-}} \FunctionTok{lapply}\NormalTok{(}\DecValTok{1}\SpecialCharTok{:}\NormalTok{iter, }\ControlFlowTok{function}\NormalTok{(i) }\FunctionTok{select.sample}\NormalTok{(i))}
\NormalTok{  out}
\NormalTok{\}}
\end{Highlighting}
\end{Shaded}

With \texttt{iter} set to a large value, inevitably a call to \texttt{Example.8.18.a()} will result in an error message:

\begin{Shaded}
\begin{Highlighting}[]
\NormalTok{\textgreater{} Example.8.18.a()}
\NormalTok{Error in select.sample(i) : Negative numbers not allowed.}
\end{Highlighting}
\end{Shaded}

To see how \texttt{try()} can be used make the following change in \texttt{Example.8.18.a()}:

\begin{Shaded}
\begin{Highlighting}[]
\NormalTok{Example.}\DecValTok{8}\NormalTok{.}\FloatTok{18.}\NormalTok{b }\OtherTok{\textless{}{-}} \ControlFlowTok{function}\NormalTok{ (}\AttributeTok{iter =} \DecValTok{500}\NormalTok{)}
\NormalTok{\{ select.sample }\OtherTok{\textless{}{-}} \ControlFlowTok{function}\NormalTok{ (x) }
\NormalTok{  \{ temp }\OtherTok{\textless{}{-}} \FunctionTok{rnorm}\NormalTok{ (}\DecValTok{100}\NormalTok{, }\AttributeTok{m =} \DecValTok{50}\NormalTok{, }\AttributeTok{s =} \DecValTok{20}\NormalTok{)}
    \ControlFlowTok{if}\NormalTok{ (}\FunctionTok{any}\NormalTok{ (temp }\SpecialCharTok{\textless{}} \DecValTok{0}\NormalTok{)) }\FunctionTok{stop}\NormalTok{(}\StringTok{"Negative numbers not allowed"}\NormalTok{)}
    \FunctionTok{mean}\NormalTok{(}\FunctionTok{log}\NormalTok{(temp))                                                         \}}
\NormalTok{  out }\OtherTok{\textless{}{-}} \FunctionTok{lapply}\NormalTok{(}\DecValTok{1}\SpecialCharTok{:}\NormalTok{iter, }\ControlFlowTok{function}\NormalTok{(i) }
                        \FunctionTok{try}\NormalTok{(}\FunctionTok{select.sample}\NormalTok{(i), }\AttributeTok{silent =} \ConstantTok{TRUE}\NormalTok{))}
\NormalTok{  out}
\NormalTok{\}}
\end{Highlighting}
\end{Shaded}

A typical chunk of output from a call to \texttt{Example.8.18.b()} is

\begin{Shaded}
\begin{Highlighting}[]
\NormalTok{\textgreater{} Example.8.18.b(2)}
\NormalTok{[[1]]}
\NormalTok{[1] 3.804975}
\NormalTok{[[2]]}
\NormalTok{[1] "Error in select.sample(i) : Negative numbers not allowed\textbackslash{}n"}
\NormalTok{attr(,"class")}
\NormalTok{[1] "try{-}error"}
\NormalTok{attr(,"condition")}
\NormalTok{\textless{}simpleError in select.sample(i): Negative numbers not allowed\textgreater{}}
\end{Highlighting}
\end{Shaded}

Notice that execution of \texttt{Example.8.18.b} was not halted prematurely. From the above output we can make some final changes to our example function:

\begin{Shaded}
\begin{Highlighting}[]
\NormalTok{Example.}\DecValTok{8}\NormalTok{.}\FloatTok{18.}\NormalTok{c }\OtherTok{\textless{}{-}} \ControlFlowTok{function}\NormalTok{ (}\AttributeTok{iter =} \DecValTok{500}\NormalTok{)}
\NormalTok{\{ select.sample }\OtherTok{\textless{}{-}} \ControlFlowTok{function}\NormalTok{ (x) }
\NormalTok{  \{ temp }\OtherTok{\textless{}{-}} \FunctionTok{rnorm}\NormalTok{ (}\DecValTok{100}\NormalTok{, }\AttributeTok{m =} \DecValTok{50}\NormalTok{, }\AttributeTok{s =} \DecValTok{20}\NormalTok{)}
    \ControlFlowTok{if}\NormalTok{ (}\FunctionTok{any}\NormalTok{ (temp }\SpecialCharTok{\textless{}} \DecValTok{0}\NormalTok{)) }\FunctionTok{stop}\NormalTok{(}\StringTok{"Negative numbers not allowed"}\NormalTok{)}
    \FunctionTok{mean}\NormalTok{(}\FunctionTok{log}\NormalTok{(temp))                                                         \}}
\NormalTok{  out }\OtherTok{\textless{}{-}} \FunctionTok{lapply}\NormalTok{(}\DecValTok{1}\SpecialCharTok{:}\NormalTok{iter, }\ControlFlowTok{function}\NormalTok{(i) }
                        \FunctionTok{try}\NormalTok{(}\FunctionTok{select.sample}\NormalTok{(i), }\AttributeTok{silent =} \ConstantTok{TRUE}\NormalTok{))}
\NormalTok{  out }\OtherTok{\textless{}{-}} \FunctionTok{lapply}\NormalTok{(out, }\ControlFlowTok{function}\NormalTok{(x)}
\NormalTok{                     \{ }\ControlFlowTok{if}\NormalTok{ (}\FunctionTok{is.null}\NormalTok{ (}\FunctionTok{attr}\NormalTok{ (x,}\StringTok{"condition"}\NormalTok{))) x }\OtherTok{\textless{}{-}}\NormalTok{ x}
                       \ControlFlowTok{else}\NormalTok{ x }\OtherTok{\textless{}{-}} \FunctionTok{attr}\NormalTok{(x, }\StringTok{"condition"}\NormalTok{)}
\NormalTok{                     \})}
\NormalTok{  Error.report }\OtherTok{\textless{}{-}} \FunctionTok{lapply}\NormalTok{(out, }\ControlFlowTok{function}\NormalTok{(x) }
                              \FunctionTok{ifelse}\NormalTok{(}\SpecialCharTok{!}\FunctionTok{is.numeric}\NormalTok{(x), x, }\StringTok{"No Error"}\NormalTok{))}
\NormalTok{  Numeric.results }\OtherTok{\textless{}{-}} \FunctionTok{unlist}\NormalTok{(}\FunctionTok{lapply}\NormalTok{(out, }\ControlFlowTok{function}\NormalTok{(x)   }
                                        \FunctionTok{ifelse}\NormalTok{ (}\FunctionTok{is.numeric}\NormalTok{(x), x, }\ConstantTok{NA}\NormalTok{)))}
  \FunctionTok{list}\NormalTok{ (}\AttributeTok{Error.report =}\NormalTok{ Error.report, }\AttributeTok{Numeric.results =}\NormalTok{ Numeric.results) }
\NormalTok{\}}
\end{Highlighting}
\end{Shaded}

Study the output of a call to \texttt{Example.8.18.c} and comment on the merits of \texttt{try()} in this example.

\chapter{Reading data files into R, formatting and printing}\label{data}

\section{Reading Microsoft Excel files into R}\label{reading-microsoft-excel-files-into-r}

The following three ways can be used to read an Excel file into R as an object:

\begin{enumerate}
\def\labelenumi{(\alph{enumi})}
\item
  The file can be stored as a \emph{{.txt}} or \emph{{.csv}} file and then \texttt{read.table()}, \texttt{scan()} or \texttt{read.csv()} can be used to read the file into R.
\item
  Directly read the \emph{{.xlsx}} file into R with the \texttt{readxl} package. List the sheet names with \texttt{excel\_sheets()}. Specify a worksheet by name or number with a command like \texttt{objectname\ \textless{}-\ read\_excel(xlsx\_example,\ sheet\ =\ "Sheet1")}.
\item
  The \emph{{.xlsx}} file can also be read into R with the \texttt{xlsx} package. The R functions \texttt{read.xlsx()} and \texttt{read.xlsx2()} can be used to read the contents of an Excel worksheet into an R data.frame. The difference between these two functions is that \texttt{read.xlsx()} preserves the data type. It tries to guess the class type of the variable corresponding to each column in the worksheet. Note that, the \texttt{read.xlsx()} function is slow for large data sets (worksheet with more than 100 000 cells). The \texttt{read.xlsx2()} function is faster on big files compared to \texttt{read.xlsx()} function. The commands have the following format: \texttt{objectname\ \textless{}-\ read.xlsx\ (file,\ sheetIndex,\ header\ =\ TRUE,\ \ \ \ \ \ \ \ \ \ \ \ \ colClasses=NA)} and \texttt{objectname\ \textless{}-\ read.xlsx2\ (file,\ sheetIndex,\ header\ =\ TRUE,\ colClasses="character")}.
\item
  Select the data in Excel (Data can also be selected in any other application such as Word or a text editor). Copy the selected range. In R:
  \texttt{objectname\ \textless{}-\ read.table\ (file\ =\ "clipboard")}. \emph{Hint}: Be careful with empty cells in Excel: some preparation of the Excel file might be needed.
\item
  To avoid problems with end-of-file characters that can occur when using the method in (d), the package \texttt{clipr} can be used.
\end{enumerate}

\begin{Shaded}
\begin{Highlighting}[]
\FunctionTok{library}\NormalTok{ (clipr)}
\NormalTok{objectname }\OtherTok{\textless{}{-}} \FunctionTok{read\_clip\_tbl}\NormalTok{ (}\AttributeTok{header =} \ConstantTok{TRUE}\NormalTok{, }\AttributeTok{row.names =} \DecValTok{1}\NormalTok{)}
\end{Highlighting}
\end{Shaded}

The functions \texttt{clear\_clip()} and \texttt{write\_clip()} can also be very useful.

\section{Reading other data files into R}\label{reading-other-data-files-into-r}

The R package \texttt{foreign()} provides functions for reading data from other packages into R:

\begin{Shaded}
\begin{Highlighting}[]
\FunctionTok{library}\NormalTok{(foreign)}
\FunctionTok{objects}\NormalTok{(}\AttributeTok{name=}\StringTok{"package:foreign"}\NormalTok{)}
\CommentTok{\#\textgreater{}  [1] "data.restore"  "lookup.xport"  "read.arff"    }
\CommentTok{\#\textgreater{}  [4] "read.dbf"      "read.dta"      "read.epiinfo" }
\CommentTok{\#\textgreater{}  [7] "read.mtp"      "read.octave"   "read.S"       }
\CommentTok{\#\textgreater{} [10] "read.spss"     "read.ssd"      "read.systat"  }
\CommentTok{\#\textgreater{} [13] "read.xport"    "write.arff"    "write.dbf"    }
\CommentTok{\#\textgreater{} [16] "write.dta"     "write.foreign"}
\end{Highlighting}
\end{Shaded}

Study the helpfiles of these functions for reading into R binary data, SAS XPORT format, Weka Attribute-Relation File Format, the Xbase family of database languages dBase, Clipper and FoxPro, Stata, Epi Info and EpiData files, Minitab portable worksheets, Octave text files, data.dump files that were produced in S version 3, SPSS save or export files, SAS data sets to be converted to ssd format and Systat files.

\subsection{ssd format footnote}\label{ssd-format-footnote}

\section{Sending output to a file}\label{sending-output-to-a-file}

The function \texttt{sink("filename")} can be used to divert output that normally appears in the console to a file. The option \texttt{options\ (echo\ =\ TRUE)} ensures that the R instructions will also be included in the file. The instruction \texttt{sink()} makes output to appear in the console again.

How do the functions \texttt{write(x)} and \texttt{sink("filename")} differ? Study the arguments of \texttt{write()} thoroughly.

\section{Writing R objects for transport}\label{writing-r-objects-for-transport}

The R function \texttt{save(...,\ file\ =\ )} writes an external representation of R objects to the specified file. The names of the objects to be saved should appear either as symbols (or character strings) in \texttt{...} or as a character vector in list. These objects can be read back from the file using the function \texttt{load\ (file\ =\ )}. Study how these two functions work by consulting the help files. The functions \texttt{save()} and \texttt{load()} are very useful for transporting R objects between computers.

The functions \texttt{saveRDS\ (object\ =\ ,\ file\ =\ )} and \texttt{object.name\ \textless{}-\ readRDS\ (file\ =\ )} write a single R object to a file, and restore it named \texttt{object.name}. Care has to be taken with the deprecated functions \texttt{dump()} and \texttt{source()}. If R objects were saved to a file using \texttt{dump()}, it should be restored to an R workspace with \texttt{source()}, not \texttt{load()}.

\section{\texorpdfstring{The use of the file .Rhistory and the function \texttt{history()}}{The use of the file .Rhistory and the function history()}}\label{the-use-of-the-file-.rhistory-and-the-function-history}

The file \emph{{.Rhistory}} is created in the same folder where the \emph{{.Rdata}} exists. It can be inspected with any text editor or with MS Word and as such provides an exact record of all activity in the R console (commands window).

Study the help file of the function \texttt{history()}.

\section{Command re-editing}\label{command-re-editing}

\begin{enumerate}
\def\labelenumi{(\alph{enumi})}
\item
  Use of the up and down arrows to recall previous commands. Delete, Backspace, Home and End keys for editing.
\item
  Note the use of the script window to execute entire functions or selected instructions only.
\end{enumerate}

\section{Customized printing}\label{customized-printing}

The basic tool for customized printing is the function \texttt{cat()}. This function can be used to output messages to the console or to a file. Note the different arguments that are available for \texttt{cat()}:

\begin{enumerate}
\def\labelenumi{(\roman{enumi})}
\item
  By default output is display on the screen; for output to be directed to a file, use argument \texttt{file\ =\ "file\ name\ including\ path"}.
\item
  By default output directed to a file replaces previous contents of the file; use argument \texttt{append\ =\ TRUE} to append new output to previous contents.
\item
  Use \texttt{sep\ =\ "xx"} to automatically insert characters between the unnamed arguments to \texttt{cat()} in the output.
\item
  To automatically insert new lines in the output use \texttt{fill\ =\ TRUE}.
\item
  The \texttt{labels\ =} argument allows insertion of a character string at the beginning of each output line. If labels is a vector its values are used cyclically.
\end{enumerate}

Write today's date as given by the function date() in the form \texttt{“The\ date\ today\ is:\ \ \ Day\ of\ the\ week,\ \ xx,\ month,\ \ 20xx.”} as an heading to a file. \emph{Hint}: recall functions \texttt{cat()}, \texttt{match()}, \texttt{substring()}, \texttt{paste()}, \texttt{replace()}.

\section{Formatting numbers}\label{formatting-numbers}

\begin{enumerate}
\def\labelenumi{(\alph{enumi})}
\item
  Study how the functions \texttt{round()} and \texttt{signif()} together with \texttt{cat()} can be used to set the number of decimals that are printed.
\item
  Study the use of \texttt{options(digits=xx)}.
\item
  Study how the function \texttt{format()} works. Note the use of \texttt{format()} together with \texttt{paste()} and \texttt{cat()}.
\item
  What does \texttt{print()} do?
\item
  Study the help file of \texttt{write.table()}.
\item
  The functions \texttt{prmatrix()} or \texttt{print()} can be used to output matrices to the console during execution of a function. This is very convenient for inspecting intermediate results. Determine how the latter function differs from \texttt{cat()}.
\item
  Note the difference between the following statements:
\end{enumerate}

\begin{Shaded}
\begin{Highlighting}[]
\FunctionTok{colnames}\NormalTok{(state.x77)}
\CommentTok{\#\textgreater{} [1] "Population" "Income"     "Illiteracy" "Life Exp"  }
\CommentTok{\#\textgreater{} [5] "Murder"     "HS Grad"    "Frost"      "Area"}
\FunctionTok{format}\NormalTok{(}\FunctionTok{colnames}\NormalTok{(state.x77))}
\CommentTok{\#\textgreater{} [1] "Population" "Income    " "Illiteracy" "Life Exp  "}
\CommentTok{\#\textgreater{} [5] "Murder    " "HS Grad   " "Frost     " "Area      "}
\end{Highlighting}
\end{Shaded}

\begin{enumerate}
\def\labelenumi{(\alph{enumi})}
\setcounter{enumi}{7}
\tightlist
\item
  Study the following example carefully:
\end{enumerate}

\begin{Shaded}
\begin{Highlighting}[]
\NormalTok{format.mns }\OtherTok{\textless{}{-}} \FunctionTok{format}\NormalTok{ (}\FunctionTok{apply}\NormalTok{ (state.x77, }\DecValTok{2}\NormalTok{, mean))}
\NormalTok{format.names }\OtherTok{\textless{}{-}} \FunctionTok{format}\NormalTok{ (}\FunctionTok{colnames}\NormalTok{ (state.x77))}
\NormalTok{descrip.mns }\OtherTok{\textless{}{-}} \FunctionTok{paste}\NormalTok{(}\StringTok{"Mean for variable"}\NormalTok{, format.names, }\StringTok{" = "}\NormalTok{, format.mns)}
\FunctionTok{cat}\NormalTok{(descrip.mns, }\AttributeTok{fill =} \FunctionTok{max}\NormalTok{(}\FunctionTok{nchar}\NormalTok{(descrip.mns)))}
\CommentTok{\#\textgreater{} Mean for variable Population  =   4246.4200 }
\CommentTok{\#\textgreater{} Mean for variable Income      =   4435.8000 }
\CommentTok{\#\textgreater{} Mean for variable Illiteracy  =      1.1700 }
\CommentTok{\#\textgreater{} Mean for variable Life Exp    =     70.8786 }
\CommentTok{\#\textgreater{} Mean for variable Murder      =      7.3780 }
\CommentTok{\#\textgreater{} Mean for variable HS Grad     =     53.1080 }
\CommentTok{\#\textgreater{} Mean for variable Frost       =    104.4600 }
\CommentTok{\#\textgreater{} Mean for variable Area        =  70735.8800}
\end{Highlighting}
\end{Shaded}

\chapter{R graphics: Round II}\label{graphics2}

\chapter{Statistical modelling with R}\label{modelling}

\chapter{Introduction to Optimisation}\label{optimisation}

\bibliography{book.bib}

\end{document}
