% Options for packages loaded elsewhere
\PassOptionsToPackage{unicode}{hyperref}
\PassOptionsToPackage{hyphens}{url}
\documentclass[
]{book}
\usepackage{xcolor}
\usepackage{amsmath,amssymb}
\setcounter{secnumdepth}{5}
\usepackage{iftex}
\ifPDFTeX
  \usepackage[T1]{fontenc}
  \usepackage[utf8]{inputenc}
  \usepackage{textcomp} % provide euro and other symbols
\else % if luatex or xetex
  \usepackage{unicode-math} % this also loads fontspec
  \defaultfontfeatures{Scale=MatchLowercase}
  \defaultfontfeatures[\rmfamily]{Ligatures=TeX,Scale=1}
\fi
\usepackage{lmodern}
\ifPDFTeX\else
  % xetex/luatex font selection
\fi
% Use upquote if available, for straight quotes in verbatim environments
\IfFileExists{upquote.sty}{\usepackage{upquote}}{}
\IfFileExists{microtype.sty}{% use microtype if available
  \usepackage[]{microtype}
  \UseMicrotypeSet[protrusion]{basicmath} % disable protrusion for tt fonts
}{}
\makeatletter
\@ifundefined{KOMAClassName}{% if non-KOMA class
  \IfFileExists{parskip.sty}{%
    \usepackage{parskip}
  }{% else
    \setlength{\parindent}{0pt}
    \setlength{\parskip}{6pt plus 2pt minus 1pt}}
}{% if KOMA class
  \KOMAoptions{parskip=half}}
\makeatother
\usepackage{longtable,booktabs,array}
\usepackage{calc} % for calculating minipage widths
% Correct order of tables after \paragraph or \subparagraph
\usepackage{etoolbox}
\makeatletter
\patchcmd\longtable{\par}{\if@noskipsec\mbox{}\fi\par}{}{}
\makeatother
% Allow footnotes in longtable head/foot
\IfFileExists{footnotehyper.sty}{\usepackage{footnotehyper}}{\usepackage{footnote}}
\makesavenoteenv{longtable}
\usepackage{graphicx}
\makeatletter
\newsavebox\pandoc@box
\newcommand*\pandocbounded[1]{% scales image to fit in text height/width
  \sbox\pandoc@box{#1}%
  \Gscale@div\@tempa{\textheight}{\dimexpr\ht\pandoc@box+\dp\pandoc@box\relax}%
  \Gscale@div\@tempb{\linewidth}{\wd\pandoc@box}%
  \ifdim\@tempb\p@<\@tempa\p@\let\@tempa\@tempb\fi% select the smaller of both
  \ifdim\@tempa\p@<\p@\scalebox{\@tempa}{\usebox\pandoc@box}%
  \else\usebox{\pandoc@box}%
  \fi%
}
% Set default figure placement to htbp
\def\fps@figure{htbp}
\makeatother
\setlength{\emergencystretch}{3em} % prevent overfull lines
\providecommand{\tightlist}{%
  \setlength{\itemsep}{0pt}\setlength{\parskip}{0pt}}
\usepackage[]{natbib}
\bibliographystyle{apalike}
\usepackage{booktabs}
\usepackage{xcolor}

\usepackage{bookmark}
\IfFileExists{xurl.sty}{\usepackage{xurl}}{} % add URL line breaks if available
\urlstyle{same}
\hypersetup{
  pdftitle={Introduction into R Applications and Programming: A Tutorial},
  pdfauthor={Niël J le Roux and Sugnet Lubbe},
  hidelinks,
  pdfcreator={LaTeX via pandoc}}

\title{Introduction into R Applications and Programming: A Tutorial}
\author{Niël J le Roux and Sugnet Lubbe}
\date{2025}

\begin{document}
\maketitle

{
\setcounter{tocdepth}{1}
\tableofcontents
}
\chapter*{Preface}\label{preface}
\addcontentsline{toc}{chapter}{Preface}

\includegraphics[width=1\linewidth,height=\textheight,keepaspectratio]{frontcover.jpg}

This book is an updated version of \citep{StepbyStep}.

\section*{Preface to A Step-by-Step R Tutorial (2013)}\label{preface-to-a-step-by-step-r-tutorial-2013}
\addcontentsline{toc}{section}{Preface to A Step-by-Step R Tutorial (2013)}

The R system is an open-source software project for analyzing data and constructing graphics. It provides a general computer language for performing tasks like organizing data, statistical analyses, simulation studies, model fitting, building of complex graphics and many more.

Central to the R system is the high-level R computer language. Its roots date back to the birth of the computer language S on May 5, 1976 at Bell Labs, Murray Hill, New Jersey \citep{Chambers2008}. In its early days S underwent several revisions and extensions mainly for implementation on the UNIX operating system. Eventually an enhanced version of S was licensed under the name S-PLUS and became available for the Windows operating system under the name S-PLUS for Windows. The earlier versions of R adhered to the principles of functional programming and with the release of version S3 in the middle eighties its building blocks were dynamically generated, self-describing objects. The publication The New S Language \citep{BeckerChambersWilks1988} provides a detailed description of S3. The next major development of S was the release of Statistical Models in S \citep{ChambersHastie1993} which involved the merging of the functional style of S with object-oriented programming concepts of classes and methods. However, S3 has only limited formal support for classes and methods. The introduction of S4 objects \citep{Chambers1998} introduced a new class and method system but retains S3 compatibility. In the meantime several versions of S-PLUS based upon S3 at first and later on S4 were released in the commercial market.

The R language itself was introduced in a paper published by Ross Ihaka and Robert Gentleman of Auckland, New Zealand in 1996 \citep{IhakaGentleman1996}. This proposal was to a large extent compatible with S but included features from the Lisp/Scheme family of languages. An important aspect of R was its availability as an open-source system.

Both R and S-PLUS can be considered to be clones of the same underlying S. That means that if you are able to program in the one you can quite easily program in the other but be warned: there are also fundamental differences between the two systems.

In the first two decades of the twenty-first century interest in R has exceeded all possible expectations. Apart from a well-maintained core system with new releases every few months there are currently literally thousands of researchers contributing add-on packages on cutting-edge developments in statistics and data analysis.

This book is a tutorial with a twofold aim; learning the basics of the R system and how to program efficiently in R. It is the result of an introductory course in S-PLUS taught at the University of Stellenbosch since 1995. The initial course was based on the book An Introduction to S and S-Plus \citep{Spector1994}. Since 2002 increasingly more emphasis was put on R to such an extent that it is currently exclusively devoted to R. This change necessitated the preparation of class notes for a ten-day (eight hours a day) tutorial course in R. The result is A Step-by-Step R Tutorial: An introduction into R applications and programming.

\section*{Preface to A Step-by-Step R Tutorial (2021)}\label{preface-to-a-step-by-step-r-tutorial-2021}
\addcontentsline{toc}{section}{Preface to A Step-by-Step R Tutorial (2021)}

Since the first publication of A Step-by-Step R Tutorial: An introduction into R applications and programming the R system has experienced a dramatic evolutionary process. This edition still maintains the twofold aim of the first edition while adapting its contents to the needs of the modernization that has been happening within the R system itself. Deprecated or outdated material has been omitted and new developments included. What follows is a brief description of these changes.

Chapter 1 contains a new section explaining how to use R Markdown for creating PDF and HTML documents from R output. Chapters 2, 3, 4 and 5 see only minor changes. In Chapter 6 changes are made in the data sets used as well as in some exercises being borrowed from later chapters in the first edition. In Chapter 7, `Writing R Functions', a notable reference is made to the \texttt{Rcpp} package for the inclusion of C++ code into R. This package allows compiled code to be included considerably easier and more robust. Vectorized programming and mapping functions are enhanced in Chapter 8 by a discussion of the function \texttt{mapply()}. A major addition is a discussion in section 8.14 for writing user-friendly applications using the package \texttt{shiny}. This replaces the usage of the function \texttt{menu()}. An exercise to create a simple shiny App is also included.

In the first part of Chapter 9, `Reading data files into R, formatting and printing', methods for reading Microsoft Excel files have been updated; functions like \texttt{readRDS()} and \texttt{writeRDS()} for transporting R objects are introduced; and the \texttt{clipr} package is discussed. A major addition to this chapter is the section devoted to the functionality provided by the \texttt{tidyverse} collection of R packages for data manipulation and exploration; \texttt{tibbles} are discussed in detail as well as the pipe operator \texttt{\%\textgreater{}\%}, tidy data is illustrated and the data manipulation functions of \texttt{dplyr} illustrated in detail.

Chapter 10, `R graphics: Round II', has been considerably extended by the inclusion of a section on how to specify colours; a rewritten section on quantile plots and inclusion of material previously in Chapter 11. There is now a section on density estimation, which includes a discussion of density histograms and average shifted histograms. In the new section 10.14 the package \texttt{ggplot2} is discussed with many examples of its capabilities.

The chapter on `Modelling in R' (Chapter 11) and the extensive discussion of the Analysis of Variance and Covariance (Chapter 12) in the previous edition have been rewritten completely and consolidated into a new Chapter 11. The final chapter is now Chapter 12, `Introduction to Optimization'. Apart from a new data set the material is similar to that in Chapter 13 of the previous edition.

\chapter{Introducing the R System}\label{intro}

\section{Introduction}\label{introduction}

This chapter introduces the R system to the new R user. The Windows operating system is emphasized but most of the material covered also applies to other operating systems after allowing for the requirements of the particular operating system in use. Users with some experience with R should quickly glance through this chapter making sure they have mastered all topics covered here before proceeding with the main tutorial starting with Chapter 2.

In the computer age statistics has become inseparable from being able to write computer programs. Therefore, let us start with a reminder of the Fundamental Goal of S:

\textbf{\emph{Conversion of an idea into useful software}}

The challenge is to pursue this goal keeping in mind the Mission of R \citep{Chambers2008}:

\textbf{\emph{\ldots{} to enable the best and most thorough exploration of data possible}}

and its Prime Directive \citep{Chambers2008}:

\textbf{\emph{\ldots{} places and obligation on all creators of software to program in such a way that the computations can be understood and trusted}}.

\section{Downloading the R system}\label{downloading-the-r-system}

\href{http://www.R-project.org}{Website for downloading R}.

To download R to your own computer: Navigate to \emph{\ldots/bin/windows/base} and save the file \emph{\textcolor[HTML]{BE99FF}{R-x.y.z.-win.exe}} on your computer. Click this file to start the installation procedure and select the defaults unless you have a good reason not to do so. If you select `Create desktop icon' during the installation phase, an icon similar to the one below should appear on the desktop. Alternatively, you can find R under \emph{All Applications}.

\pandocbounded{\includegraphics[keepaspectratio]{pics/Rlogo.jpg}}

The core R system that is installed includes several \emph{\textcolor[HTML]{FF9966}{packages}}. Apart from these installed packages several thousands of dedicated \emph{\textcolor[HTML]{FF9966}{contributed}} are available to be downloaded by users in need of any of them.

\bibliography{book.bib}

\end{document}
